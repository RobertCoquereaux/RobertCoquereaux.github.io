
\chapter{Un saut dans l'infini}

\subsection{Action du groupe de jauge sur l'espace des connexions}


Soit  $Ad\; P = P \times_{Ad} G$ le fibr\'e adjoint en groupes et $ad\; P = P \times_{ad} {\frak g}$ le 
fibr\'e adjoint en alg\`ebres de Lie (voir le chapitre ``Espaces fibr\'es''). On 
se souvient que l'ensemble des sections globales de $Ad\; P$ n'est autre 
que le groupe de jauge ${\frak G}$ (automorphismes verticaux de $P$)
et que l'ensemble des sections globales de $ad\; P$ est 
l'alg\`ebre de Lie ${\scr G}$ du groupe de jauge. La fibre type 
de $ad\; P$ est une alg\`ebre de Lie, et donc, en particulier, un espace 
vectoriel. Puisque $ad\; P$ est un fibr\'e vectoriel, toute connexion 
$\omega$ sur $P$ donne naissance \`a une diff\'erentielle covariante 
$\nabla^\omega$ agissant sur les sections de $ad\; P$ (qui sont des 
transformations de jauge infinit\'esimales) et plus 
g\'en\'eralement \`a une diff\'erentielle ext\'erieure 
covariante $d^{\nabla^\omega}$ agissant sur ${\scr G}^p  =  \Omega^p(M,ad\; P)$.
$$
d^{\nabla^\omega} : {\scr G}^p  =  \Omega^p(M,ad\; P) \mapsto 
{\scr G}^{p+1}  =  \Omega^p(M,ad\; P)
$$
Noter que $\Omega^p(M,ad\; P) \simeq \Omega^p_{ad}(P,Lie\, {\scr G})$ (voir section 3.3.10 (2)).
On supposera aussi qu'il est possible de plonger le groupe structural $G$ dans une 
alg\`ebre de matrices $M(n,\CC)$, pour $n$ assez grand et donc consid\'erer 
$Ad\; P$ comme sous-fibr\'es du fibr\'e vectoriel 
$P \times_{Ad} M(n,\CC) $. On peut donc \'egalement consid\'erer la 
diff\'erentielle covariante $\nabla^\omega$ agissant sur une transformation de jauge 
finie, $\Phi$.

Soit ${\scr A}$ l'ensemble de toutes les connexions qu'on peut 
d\'efinir sur un fibr\'e principal donn\'e $P=P(M,G)$. Il est facile de voir que le groupe 
de jauge ${\frak G}$ agit sur ${\scr A}$ par ``pull-back''.
Soit $\Phi \in {\frak G}$, $\omega \in {\scr A},$ $z \in P$ et $V \in 
T(P,z)$. On d\'efinit 
$\omega^\Phi \equiv \omega.\Phi$
par 
$${\omega^\Phi}_z (V)  =  \omega_{\Phi(z)}(\Phi_*(V))$$

L'\'equivariance d'une connexion peut s'\'ecrire, comme on le sait, en terme du potentiel de jauge,
sous la forme $A \rightarrow A' = g^{-1}Ag + g^{-1} dg$. Le second membre de cette \'egalit\'e
peut encore s'\'ecrire $A + g^{-1}(dg + [A,g])$ et la quantit\'e $dg + [A,g]$ appara\^it comme
une diff\'erentielle covariante $\nabla g$ relative \`a la connexion choisie.
De la m\^eme fa\c con, l'action du groupe de jauge ${\frak G}$ sur l'espace ${\scr A}$ des
formes de connexion s'\'ecrit 
$$
(\omega, \Phi)\in {\scr A} \times {\frak G} \rightarrow \omega^\Phi  =  \omega.\Phi = \omega + 
\Phi^{-1}\nabla^\omega \Phi
$$

Cette loi de transformation nous montre que l'ensemble ${\scr A}$ de 
toutes les connexions n'est pas un espace vectoriel mais un espace 
affine. En g\'eom\'etrie 
\'el\'ementaire, la diff\'erence de deux points est un vecteur de l'espace 
vectoriel sous-jacent. Il en est de m\^eme ici. L'objet 
$\Phi^{-1}\nabla^\omega \Phi$ est une $1$-forme (\'equivariante) sur $P$, \`a 
valeurs dans ${\frak g}=Lie(G)$. L'espace vectoriel sous-jacent \`a l'espace affine ${\scr A}$ est
donc ${\scr G}^1$. En particulier, notons que si $\omega_1$ et 
$\omega_2$ d\'esignent deux connexions, alors
$(\omega_2^\Phi - \omega_1^\Phi) = ( \omega_2 - \omega_1) + 
\Phi^{-1}(\nabla^{\omega_{2}}-\nabla^{\omega_{1}}) \Phi$.

\subsection{L'espace des orbites}

Puisqu'on a une action de groupe sur un espace, on peut \'etudier le quotient par cette
action, c'est \`a dire l'espace des orbites ${\scr I}  =  {\scr A}/{\frak G}$.
Il faut tenir compte, en fait, de quelques subtilit\'es car l'action de ${\frak G}$ n'est pas libre,
en g\'en\'eral. En d'autres termes, certaines connexions peuvent avoir des sym\'etries : c'est
le cas lorsque le sous-groupe de ${\frak G}$ qui stabilise une connexion donn\'ee 
ne se r\'eduit pas \`a l'identit\'e. Pour rendre cette action libre, il faut, soit consid\'erer
un espace des connexions plus petit (c'est \`a dire ne consid\'erer que les connexions ``irr\'eductibles''
pour lesquelles le stabilisateur est trivial), soit diminuer la taille du groupe de jauge (on
consid\`ere le {\sl groupe de jauge point\'e\/} ${\frak G}_x$ obtenu en se fixant arbitrairement un point
$x$ de $M$ et en ne consid\'erant que les transformations de jauge $\Phi$ telles que $\Phi(z)=z$, pour
tout $z$ appartenant \`a la fibre de $P$ au dessus de $x$)\index{groupe 
de jauge point\'e}.
Moyennant ces quelques pr\'ecautions, par exemple celle qui revient \`a ne consid\'erer que l'action du
groupe de jauge point\'e, on montre que l'espace des connexions est lui-m\^eme un espace fibr\'e principal, de
groupe structural ${\frak G}_x$ au dessus de l'espace des orbites, qu'on note
${\scr I}$.

Lorsque la vari\'et\'e $M$ est compacte,  munie d'une m\'etrique et qu'on a choisi \'egalement
une m\'etrique bi-invariante sur le groupe de structure $G$, on peut construire 
un produit scalaire global sur les espaces vectoriels ${\scr G}^p$ ainsi qu'un
laplacien g\'en\'eralis\'e. Le produit scalaire sur l'espace vectoriel ${\scr G}^1$ (identifi\'e
avec l'espace vectoriel sous-jacent \`a l'espace affine ${\scr A}$ en un point quelconque $\omega$)
fait de ${\scr A}$ un espace affine euclidien (de dimension infinie, bien s\^ur~!). On a donc
une m\'etrique sur ${\scr A}$. Cette m\'etrique permet de d\'ecomposer l'espace tangent
$T({\scr A},\omega)$ en un sous-espace vertical \'evident (celui qui est tangent \`a l'action
de ${\frak G}$) et un sous-espace horizontal d\'efini comme \'etant perpendiculaire au sous-espace
vertical pour cette m\'etrique. On obtient ainsi une connexion sur le fibr\'e 
${\scr A = A(I},{\frak G)}$.
La m\'etrique sur ${\scr A}$, \'etant  ${\frak G}$-invariante, 
permet \'egalement de d\'efinir une nouvelle m\'etrique sur la base du fibr\'e
 ${\scr A = A(I},{\scr G)}$, c'est \`a dire sur  l'espace des orbites ${\scr I}$.

\subsection{Conclusion}

La g\'eom\'etrie riemannienne de l'espace des orbites des connexions, modulo l'action du groupe de jauge,
est un sujet \`a la fois complexe et
fascinant. Notre but, dans ce dernier chapitre n'avait d'autre but 
que d'entrebailler une porte$\ldots$
 Nous allons arr\^eter l\`a notre escapade en dimension infinie, non sans
faire un clin d'\oe il \`a la physique...bouclant ainsi la boucle.
En effet, on sait que dans les th\'eories de jauge, deux connexions qui diff\`erent par
une transformation de jauge d\'ecrivent la m\^eme situation physique. L'espace des
champs de Yang-Mills possibles, celui sur lequel on doit int\'egrer lorsqu'on fait de la th\'eorie
quantique des champs ``\`a la Feynman'' est donc l'espace des orbites ${\scr I}$.
Par ailleurs, les champs de mati\`ere sont, comme on le sait, d\'ecrits par des sections
de fibr\'es $E$ associ\'es \`a un fibr\'e principal $P=P(M,G)$, mais le groupe de jauge
${\frak G}$ agit sur l'espace $\Gamma E$ de ces sections et deux sections qui diff\`erent par
l'action du groupe de jauge sont \'egalement physiquement \'equivalentes.
On en d\'eduit que la physique des champs de Yang-Mills et des champs de mati\`ere
qui leur sont coupl\'es
est en d\'efinitive d\'ecrite par la g\'eom\'etrie
de l'espace fibr\'e $$\aleph  =  {\scr A(I},{\frak G)} \times_{\frak G} \Gamma E$$
 Une structure analogue existe lorsqu'on s'int\'eresse \`a la gravitation quantique et qu'on
veut \'etudier l'espace des m\'etriques qu'il est possible de
d\'efinir sur une vari\'et\'e diff\'erentiable donn\'ee. Le groupe de jauge ${\frak G}$ est
alors remplac\'e par le groupe des diff\'eomorphismes de $M$. La situation est d'ailleurs
plus complexe dans ce cas. 

C'est donc de la g\'eom\'etrie diff\'erentielle en dimension infinie qu'il faut faire
pour comprendre, du point de vue quantique, 
la structure des th\'eories physiques d\'ecrivant les interactions
fondamentales. Il n'est pas exclu que le traitement math\'ematique le plus
adapt\'e \`a cette \'etude de la g\'eom\'etrie en dimension infinie passe par une
``alg\'ebra\"isation''
compl\`ete des techniques de la g\'eom\'etrie diff\'erentielle et au remplacement de celle-ci 
par la g\'eom\'etrie non commutative (voir chapitre suivant).

