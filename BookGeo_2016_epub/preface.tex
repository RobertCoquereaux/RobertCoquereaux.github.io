\chapter*{Pr\'eface}
\addcontentsline{toc}{chapter}{Pr\'eface}

Il est quelquefois plus facile de pr\'esenter un livre en disant ce 
qu'il n'est pas et en dressant une liste des motivations de
l'auteur qu'en essayant d'en expliciter le contenu.

Ce livre, bien qu'il contienne un expos\'e  de
g\'eom\'etrie diff\'erentielle (avec un accent particulier mis sur
les groupes de Lie, la th\'eorie des espaces fibr\'es, la th\'eorie
des connexions et la g\'eom\'etrie riemannienne) n'est certainement
pas un cours de math\'ematiques traditionnel. En g\'en\'eral les math\'ematiciens
cultivent
%d'ordinaire et \`a loisir,
 la pr\'ecision du
style et la concision du discours, alors que l'expos\'e qui suit  essaye
% au contraire 
de pr\'esenter les id\'ees importantes en faisant
souvent appel \`a l'intuition, en effectuant de nombreux retours en
arri\`ere et en ne n\'egligeant pas les clins d'oeil \`a la physique.
On peut donc esp\'erer que la lecture de l'ouvrage pr\'esent sera un
peu moins aride que celle d'un trait\'e traditionnel.
\par
Ce livre n'est pas non plus un cours de physique
th\'eorique. Il y manque beaucoup trop d'informations~! Celui ou
celle qui souhaite se lancer \`a la d\'ecouverte de l'Espace-Temps et
d\'echiffrer certains des myst\`eres de notre univers devrait
s'attaquer \`a de saines lectures (par exemple \cite{MTW}). 
L'ouvrage pr\'esent ressemble plus
\`a un cours de math\'ematiques qu'\`a un cours de physique; la
physique n'est cependant pas absente, au contraire: des id\'ees
physiques sont cach\'ees derri\`ere chaque paragraphe, et ce sont elles
qui sont, la plupart du temps, \`a l'origine des notions
``abstraites'' que nous allons pr\'esenter.
\par 

Bien qu'il ne s'agisse pas l\`a d'un ouvrage de vulgarisation sur la
physique th\'eorique ou les math\'ematiques, j'ai pourtant
r\'edig\'e de nombreux paragraphes en pensant \`a certains de mes amis
%poss\'edant
ayant une culture math\'ematique relativement modeste mais
n\'eanmoins dot\'es d'un esprit curieux et aimant vagabonder de temps
\`a autres sur des terrains situ\'es au confluent de l'infiniment
petit, de l'infiniment grand, des math\'ematiques et de la
m\'etaphysique. Je dois dire, en relisant l'ouvrage apr\`es coup,
que, de ce point de vue, j'ai peur d'avoir echou\'e: le contenu
pr\'esent\'e ressemble plus \`a un cours de troisi\`eme cycle
sp\'ecialis\'e qu`\`a un ouvrage de vulgarisation$\ldots$
 Cela dit, je pense --- et j'esp\`ere --- qu'\`a la condition de commencer
la lecture \`a la premi\`ere page sans essayer de d\'emarrer en plein milieu,
l'ouvrage reste accessible \`a tout lecteur disposant d'un bagage math\'ematique
\'equivalent \`a celui qu'on est cens\'e acqu\'erir \`a l'issue d'un premier cycle universitaire,  ou d'une 
classe de Math\'ematiques Sp\'eciales.
A propos de motivations, je dois aussi
signaler que d'autres de mes amis, dot\'es d'une culture
math\'ematique plus que respectable n'ont malheureusement jamais eu
le temps ou la patience de traduire le jargon quelquefois flou des
physiciens dans la langue bourbakiste qu'ils affectionnent. Le
pr\'esent ouvrage, bien que r\'esolument peu bourbakiste dans le
style, est \'egalement \'ecrit pour eux. Finalement, ce livre est
\'egalement ---et probablement surtout--- \'ecrit pour les
\'etudiants en math\'ematiques ou en physique, mais qu'on ne vienne pas
me demander ``De quelle ann\'ee?''~!
En effet, certains des th\`emes qui seront abord\'es peuvent \^etre rencontr\'es dans un
cours de ma\^itrise de math\'ematiques (ou de DEA) et on les
trouvera souvent incorpor\'es \`a un enseignement de troisi\`eme cycle
de physique th\'eorique ou de g\'eom\'etrie diff\'erentielle, 
mais d'autres th\`emes, probablement aussi int\'eressants, et quelquefois m\^eme fondamentaux, risquent fort de ne figurer dans le programme d'aucun enseignement universitaire.
L'\'etudiant, physicien ou
math\'ematicien, trouvera peut-\^etre, dans cet ouvrage, ce qu'il
cherche (en utilisant l'index et la table des mati\`eres) et le
non-sp\'ecialiste y trouvera peut-\^etre ce qu'il ne cherchait pas$\ldots$ \par

Enfin, ce livre n'est pas un ouvrage de philosophie ou de
m\'etaphysique (Dieu m'en garde!) bien que certaines r\'eflexions de
nature \'eminemment philosophiques ne soient pas absentes des pages
qui suivent, surtout dans la section Introduction.\par

La partie ``g\'eom\'etrie diff\'erentielle''  de ce travail est issue d'un
cours de troisi\`eme cycle que j'ai eu l'occasion de donner pendant plusieurs
ann\'ees au sein du Dipl\^ome d'Etudes Approfondies (DEA) de Physique
Th\'eorique, organis\'e au Centre de Physique Th\'eorique, \`a Luminy (Marseille) ainsi qu'en 1997, dans le DEA de Physique Th\'eorique organis\'e \`a 
l'Ecole Normale Sup\'erieure de Lyon. La partie ``non commutative'' (la derni\`ere section) est un court extrait d'une s\'erie de cours que j'ai  donn\'es dans les universit\'es de Rio de Janeiro (URJ, UFRJ et CBPF), de Saragosse et de La Plata ainsi qu'\`a San Carlos de  Bariloche, en 1996 et 1997. L'ouvrage a \'egalement servi de support \`a un cours semestriel de l'IMPA, Rio de Janeiro, en 2012 (P\'os-Gradua\c c\~ao em Matem\'atica, curso de Doutorado).


