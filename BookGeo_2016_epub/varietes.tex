

\chapter {Vari\'et\'es diff\'erentiables}
\minitoc

\section {Vari\'et\'es topologiques}

\subsection  {D\'efinition}
\par
Une {\sl vari\'et\'e topologique}\index{vari\'et\'e topologique} est tout d'abord un espace
topologique, mais on suppose, de surcro\^{i}t, que chacun de
ses points poss\`ede un voisinage hom\'eomorphe \`a un ouvert
de $\RR^n$. On dit alors que cet espace est une vari\'et\'e
topologique de dimension~$n$.\par

Intuitivement, une vari\'et\'e topologique de dimension $2$
est un espace qui, localement, c'est \`a dire si on ne
regarde pas trop loin, ressemble \`a un petit morceau de
feuille de papier qu'on aurait pu d\'ecouper avec des
ciseaux apr\`es en avoir trac\'e le pourtour au crayon (on
peut d'ailleurs froisser le bout de papier en question).
La structure globale de cet espace peut \^etre \'evidemment
assez diff\'erente puisque la vari\'et\'e elle-m\^eme est obtenue
par recollement de tous ces petits morceaux de papier.
Ainsi, un pneu de bicyclette, \'eventuellement d\'egonfl\'e,
pli\'e et ``froiss\'e'' fournit un exemple d'objet physique
qu'on peut mod\'eliser \`a l'aide d'une vari\'et\'e topologique de
dimension $2$: un tore.
\smallskip
\subsection  {Vari\'et\'es \`a bord}
\par
 Les vari\'et\'es dont il vient d'\^etre question n'ont pas de
bord (au sens intuitif du terme). En effet, si nous nous
transformons en \^etres plats, rampant sur la surface d'un
ballon -- ou d'un pneu -- nous ne sommes jamais arr\^et\'es
par une quelconque barri\`ere. Cela ne serait pas le cas si
nous nous d\'eplacions sur la surface d'un quartier d'orange
ou d'un pneu crev\'e (nous nous arr\^eterions au bord du
trou!). Sans se transformer en \^etres plats, cela ne serait
pas le cas non plus si nous nous d\'eplacions \`a l'int\'erieur
d'une boule ferm\'ee. De fa\c con g\'en\'erale, il est possible de
fabriquer des ``vari\'et\'es \`a bord'' en effectuant un ou
plusieurs trous dans une vari\'et\'e sans bord (\`a l'aide d'une
petite cuill\`ere multi-dimensionelle!); la partie enlev\'ee,
comme la partie qui reste, devient une {\sl vari\'et\'e topologique \`a 
bord\/}.\par

Pour pr\'eciser cette notion, il nous faut \'elargir la
d\'efinition de vari\'et\'e que nous avons donn\'e plus
haut puisque certains des points (ceux du bord) ont un
voisinage non pas hom\'eomorphe \`a un ouvert de $\RR^n$ mais \`a\
un voisinage de $\RR^n_+$ (le ferm\'e de $\RR^n$  form\'e des points dont
la derni\`ere composante est positive ou nulle).
\par

Attention: si nous nous promenons dans une boule
ouverte, nous ne pourrons jamais atteindre aucun bord... par
d\'efinition d'une boule ouverte! Une boule ouverte est une
vari\'et\'e sans bord de dimension $3$ qui est d'ailleurs
hom\'eomorphe \`a $\RR^3$. Par contre, une boule ferm\'ee est une
vari\'et\'e \`a bord de dimension $3$, les points du bords sont
ceux de la sph\`ere (une vari\'et\'e de dimension $2$) et ils
poss\`edent -- dans la boule ferm\'ee -- des voisinages
particuliers. Le disque ouvert (la
boule de dimension $2$) est aussi une vari\'et\'e sans bord
et le disque ferm\'e est une vari\'et\'e \`a bord (son bord est
constitu\'e d'un cercle qu'on peut appeler \'egalement
``sph\`ere de dimension $1$''. Dans le m\^eme genre, un
intervalle ouvert est une vari\'et\'e sans bord (la boule de
dimension $1$) et un intervalle ferm\'e est une vari\'et\'e \`a
bord (son bord est constitu\'e de deux points dont la
r\'eunion constitue ce qu'on peut appeler la sph\`ere de
dimension $0$. Les exemples qui pr\'ec\`edent sont
g\'en\'eralisables en toutes dimensions.\par

Terminologie: Si on ne pr\'ecise pas davantage, une vari\'et\'e
topologique est cens\'ee \^etre une vari\'et\'e sans bord.\par
\smallskip
\subsection  {Contre-exemples}
\par
 La plupart des objets math\'ematiques auxquels nous avons
tendance \`a penser de prime abord sont des exemples de
vari\'et\'es topologiques (avec ou sans bord), et, pour cette
raison, il est bon de donner quelques exemples d'espaces
topologiques qui ne sont pas des vari\'et\'es.
 Consid\'erez par exemple une croix  (r\'eunion de
deux segments d'intersection r\'eduite \`a un point); ce n'est pas une vari\'et\'e 
car le point
situ\'e \`a l'intersection des deux segments poss\`ede des
voisinages en forme de croix, et une croix n'est jamais
hom\'eomorphe \`a un ouvert de $\RR = \RR^1$.  Le globe imp\'erial
est un objet qu'on pourrait penser \`a
mod\'eliser math\'ematiquement par une sph\`ere
(vari\'et\'e de dimension $2$) sur laquelle on aurait coll\'e
une croix (r\'eunion de deux segments) Cet espace n'est pas
une vari\'et\'e pour deux raisons. La premi\`ere vient du point
d'intersection des deux branches de la croix (d\'ej\`a vu) et
la deuxi\`eme est analogue puisque le point ou on a coll\'e
la croix sur la sph\`ere poss\`ede des voisinages qui ne sont
hom\'eomorphes ni \`a des ouverts de $\RR^1$ ni \`a des ouverts
de $\RR^2$. \par

Ces derniers exemples ne sont pas des vari\'et\'es mais sont
n\'eanmoins obtenus par recollement de vari\'et\'es... (CW
complexes) Ils ne poss\`edent pas une dimension
d\'etermin\'ee mais ont n\'eanmoins une structure assez
simple. On peut cependant faire bien pire... Les exemples
d'espaces topologiques qui ne sont pas des vari\'et\'es
abondent (prenez par exemple des espaces topologiques
qui ne sont pas de Haussdorf, c'est \`a dire qui poss\`edent
des points qu'on ne peut pas s\'eparer \`a l'aide d'ouverts
disjoints). Il ne faudrait pas croire que les espaces
qui ne sont pas des vari\'et\'es n'ont pas d'int\'er\^et
math\'ematique ou physique, bien au contraire. En fait, la
g\'eom\'etrie non commutative (dont nous ne parlerons pratiquement pas dans 
cet ouvrage) s'est d\'evelopp\'ee en grande partie pour forger des outils 
permettant de ``calculer'' dans de tels espaces, espaces qui sont 
en fait compl\`etement d\'ecrits par des alg\`ebres associatives mais 
g\'en\'eralement non commutatives$\ldots$ Par ailleurs, on sait que la 
description 
math\'ematique de la m\'ecanique quantique repose sur l'utilisation des 
alg\`ebres d'op\'erateurs, ce qui explique la raison pour laquelle les 
ph\'enom\`enes physiques relevant de cette m\'ecanique soient si peu 
intuitifs puisqu'il nous faut, dans ce cas, abandonner nos notions 
famili\`eres de 
g\'eom\'etrie ``commutative''. C'est \`a l'\'etude de cette g\'eom\'etrie commutative 
qu'est consacr\'ee le pr\'esent ouvrage. Attention \`a la terminologie (mise en 
garde destin\'ee au lecteur trop savant)~: l'expression classique des 
th\'eories 
de jauge non ab\'eliennes ainsi que l'\'etude des groupes de Lie (en g\'en\'eral 
non commutatifs), 
rel\`event de la g\'eom\'etrie commutative!
Le calcul diff\'erentiel -- et la physique classique -- se
sont d\'evelopp\'es dans le cadre des vari\'et\'es et c'est
pourquoi nous commen\c cons par l\`a. La structure de vari\'et\'e
topologique est d'ailleurs elle-m\^eme insuffisante pour
pouvoir travailler dans de bonnes conditions: Il nous
faudra pouvoir diff\'erentier les fonctions un nombre de
fois suffisant. Pour ce faire il nous faudra supposer que
les vari\'et\'es (en anglais {\it manifolds}) consid\'er\'ees ne
sont pas ``froiss\'ees'': elles doivent \^etre ``lisses'' (bien
repass\'ees!). Ce sont les vari\'et\'es diff\'erentiables (en
anglais {\it  smooth manifolds}). \par
\bigskip
\section { Vari\'et\'es diff\'erentiables}
\smallskip
\subsection{  Vari\'et\'es, cartes, atlas}
\par
Intuitivement, on peut consid\'erer une vari\'et\'e
diff\'erentiable comme une vari\'et\'e topologique (voir
exemples {\it supra}) qui soit ``lisse'', c'est \`a dire sans
plis, sans coins etc.
Une vari\'et\'e diff\'erentiable $M$ de dimension $n$ est donc avant
tout une vari\'et\'e topologique. Nous d\'efinissons tout s'abord
la notion de carte qui g\'en\'eralise la notion usuelle de
carte g\'eographique. Une {\sl carte}\index{carte} consiste en la donn\'ee
d'un ouvert $U_i$ de $M$ ainsi que d'une application $x:
{\cal P} \in U_i \subset M \stackrel{x}{\longrightarrow} (x^\mu({\cal P}))
 \in
\RR^n $ avec $\mu \in \{1 \dots n\}$. Il importe de bien
\'etablir une distinction entre le point ${\cal P}$ lui-m\^eme
et ce qu'on appelle ses coordonn\'ees $x^\mu({\cal P})$ dans la
carte choisie. On suppose, de plus, que l'application
$x$ est bijective et bi-continue de $U_i$ sur son
image.\par

Mis \`a part le cas relativement trivial o\`u $M$ est
hom\'eomorphe \`a $\RR^n$, il nous faut plusieurs cartes pour
recouvrir la vari\'et\'e $M$. On appellera {\sl atlas}\index{atlas}
(sous-entendu diff\'erentiable) la donn\'ee d'un ensemble de
cartes $(U_i, x)$ qui recouvrent $M$ c'est \`a dire telles
que $\cup_i U_i = M$ et telles que les changements de
cartes $\phi_{ij}$ soient des bijections
diff\'erentiables, ainsi que leurs inverses. Pr\'ecisons ce
dernier point. Supposons que ${\cal P} \in U_i \cap U_j
\subset M$, on peut donc repr\'esenter ${\cal P}$ par un
point $x^\mu({\cal P})$ de $\RR^n$ dans la carte $(U_i,x)$ ou par
un autre point $y^\mu({\cal P})$ de $\RR^n$ dans la carte
$(U_j,y)$. On note $\phi_{ij}$ le changement de cartes (encore
appel\'e transformation de coordonn\'ees); c'est une
application de l'ouvert $x(U_i)$ de $\RR^n$ dans l'ouvert
$y(U_j)$ de $\RR^n$. On sait ce que signifie
``diff\'erentiable'' pour une application de $\RR^n$ dans
$\RR^n$: les d\'eriv\'ees partielles, par rapport \`a chacune
des variables, doivent exister. On impose donc \`a
$\phi_{ij}$ d'\^etre une application diff\'erentiable. On
lui impose \'egalement d'\^etre bijective (donc inversible)
et on impose \`a son inverse $\phi_{ji}  = 
\phi_{ij}^{-1}$ d'\^etre \'egalement diff\'erentiable. Bien
entendu, il faut pr\'eciser un peu plus ce qu'on entend par
diff\'erentiable: suivant qu'on impose aux applications
$\phi_{ij}$ d'\^etre une seule fois diff\'erentiable, $r$
fois diff\'erentiables ou infiniment diff\'erentiables, 
on parle d'atlas de classe $C^1$, $C^r$ ou $C^\infty$.
Dans la suite de l'ouvrage et sauf mention explicite du
contraire, c'est de classe $C^\infty$ qu'il s'agira. La
premi\`ere fa\c con de d\'efinir une {\sl vari\'et\'e
diff\'erentiable}\index{vari\'et\'e diff\'erentiable} est de se donner une vari\'et\'e
topologique ainsi qu'un atlas diff\'erentiable. Du pont de vue des
notations, il n'est pas tr\`es commode de faire figurer
l'indice $i$ qui se rapporte \`a la carte, sur le syst\`eme
de coordonn\'ees $x$; dans le cas o\`u on en consid\`ere
deux (par exemple $x$ et $y$) on \'ecrira les formules de
changement de carte (l'application $\phi_{ij}$) sans
introduire de nouvelle notation en \'ecrivant simplement
$y^\mu$ comme une fonction de $x^\nu$, c'est \`a dire $ y^\mu
=  y^\mu (x^\nu)$.\par

\smallskip
\subsection{  Atlas maximal}
\par
En g\'eographie ordinaire (celle du globe terrestre) il est
bien connu qu'il nous faut au moins deux cartes pour
d\'ecrire la Terre. Par contre, rien ne nous interdit d'en
utiliser trois ou plus $\ldots$. Si on r\'eunit les cartes
d'un atlas avec celles d'un atlas diff\'erent (concernant la
m\^eme vari\'et\'e topologique), on peut s'attendre \`a fabriquer
ainsi un atlas plus grand, un peu redondant, certes, mais
 n\'eanmoins utile. Il faut cependant prendre la pr\'ecaution
d'imposer aux cartes d'\^etre compatibles, c'est \`a dire
telles que les formules de changements
de cartes, d'un atlas \`a l'autre,  puissent s'exprimer
en terme de transformations diff\'erentiables de $\RR^n$.
Cette pr\'ecaution n'est pas
inutile et peut conduire \`a des surprises. Rien ne nous
emp\^eche alors  de consid\'erer l'ensemble (assez gros il est vrai!)
de tous les atlas compatibles possibles d'une vari\'et\'e
donn\'ee et de les r\'eunir en un unique {\sl atlas maximal}\index{atlas maximal}.
Bien qu'un seul atlas suffise \`a caract\'eriser compl\`etement
la vari\'et\'e, il peut \^etre tr\`es utile de consid\'erer la
vari\'et\'e $M$ \'equip\'ee d'un tel atlas maximal contenant
toutes les cartes compatibles possibles. En d'autres termes, on peut
compl\`etement caract\'eriser une vari\'et\'e diff\'erentiable par la 
donn\'ee d'une vari\'et\'e topologique et d'un atlas maximal. Il se trouve 
que, dans
certains cas, une vari\'et\'e topologique donn\'ee poss\`ede plusieurs 
structures diff\'erentiables (plusieurs atlas maximaux distincts). C'est 
le cas pour $\RR^4$ (le seul, parmi les espaces num\'eriques $\RR^n$ \`a 
poss\'eder des structures diff\'erentiables ``exotiques'') et c'est aussi 
le cas pour les sph\`eres $S^n$ lorsque $n \geq 7$. Nous ne nous 
int\'eresserons pas \`a ces ph\'enom\`enes dans le cadre de cet ouvrage.
\par 

\smallskip
\subsection{  Vari\'et\'es et calcul diff\'erentiel ``intrins\`eque''}
\par
En math\'ematiques \'el\'ementaires, on d\'efinit souvent les
espaces g\'eom\'etriques int\'eressants (par exemple une sph\`ere)
comme sous espace d'un espace affine $\RR^n$. L'id\'ee
fondamentale du calcul sur les vari\'et\'es (calcul
diff\'erentiel intrins\`eque comme on l'appelait autrefois)
est de faire abstraction du fait que la vari\'et\'e qui nous
int\'eresse est, ou non, plong\'ee dans un espace $\RR^n$ ``plus
grand'' et de d\'evelopper un calcul qui soit totalement
ind\'ependant du plongement en question. Les motivations
physiques sont analogues. Par exemple, l'exp\'erience
quotidienne nous montre que tout \'ev\'enement de
l'univers sensible (whatever it means) peut se d\'ecrire \`a
l'aide de quatre nombres sp\'ecifiant sa position (trois
nombres) et sa date (un nombre). Mais pourquoi supposer,
{\it a priori} que l'ensemble de ces \'ev\'enements doive
\^etre d\'ecrit \`a l'aide d'un un espace $\RR^4$~? Pourquoi pas
une hyper-sph\`ere (ou n'importe quoi d'autre?) Mais alors,
si on d\'ecide d'utiliser une hyper-sph\`ere de dimension
$4$ pour d\'ecrire notre espace-temps, ou, comme dans
certains mod\`eles cosmologiques, comme le produit d'une
hyper-sph\`ere (gonflable) de dimension $3$ par une droite
ou une demi-droite, pourquoi supposer que notre vari\'et\'e est
plong\'ee dans un espace de dimension $5$ ou plus dont les
points sont sans signification physique? Puisque c'est
possible, autant travailler dans la vari\'et\'e qui nous
int\'eresse sans chercher \`a en ``sortir''.\par

L'id\'ee la plus fondamentale et la plus simple du calcul
diff\'erentiel sur les vari\'et\'es est la suivante. Gr\^ace \`a
l'existence locale des cartes, on peut toujours faire
``comme si'' on \'etait sur $\RR^n$ et d\'evelopper des outils
et des m\'ethodes de calcul sans se soucier -- dans un
premier temps -- de leur globalisation, quitte \`a
v\'erifier, par la suite, que tout se recolle comme il faut
lorsqu'on passe d'une carte \`a l'autre. C'est ainsi que
l'essentiel des notions qui suivent sont en fait des
notions qui peuvent \^etre d\'efinies dans un espace $\RR^n$
et dont la g\'en\'eralisation, au cas des vari\'et\'es, est
quasi-imm\'ediate. Nous ne supposons pas que le lecteur est
d\'ej\`a familier des notions en question; c'est la raison
d'\^etre des paragraphes qui suivent.
\bigskip
\section {Applications diff\'erentiables,  diff\'eomorphismes} 
\smallskip
\subsection{ D\'efinition}
\par
Soient $M$ et $N$ deux vari\'et\'es diff\'erentiables de
dimensions respectives $m$ et $n$. Une {\sl application diff\'erentiable\/}
\index{application diff\'erentiable} $\phi$ de $M$ dans $N$ est une application
qui peut s'\'ecrire localement \`a l'aide d'une application
diff\'erentiable (encore not\'ee $\phi$) de $\RR^m$ dans $\RR^n$.
En d'autres termes, si on a ${  Q}\in N = \phi({ 
P})$ avec ${  P} \in M$,  alors,  gr\^ace au choix
de cartes ${  P} \in U_i \subset M \rightarrow
x^\mu({  P})\in \RR^m$ et ${  Q} \in V_i \subset N
\rightarrow y^\nu({  Q}) \in \RR^n$, on pourra \'ecrire (et on
\'ecrira!) $ y = \phi (x)$ ce qui signifie, en fait $
y^\nu({  Q}) = \phi (x^\mu({  P}))$. L'ensemble
des applications diff\'erentiables de $M$ dans $N$ se
note $ C^\infty(M,N).$\par

 Petite parenth\`ese sur le probl\`eme des notations en
math\'ematiques: Il est important de comprendre la
signification de ce qu'on \'ecrit, mais il est (de l'avis
de l'auteur) absurde de vouloir que la notation utilis\'ee
nous rappelle \`a tout moment les diff\'erents abus d'\'ecriture
commis depuis le chapitre 1 du tome 1 de Bourbaki et sans
lesquels il n'est pas de calcul possible!\par

L'application $\phi$ (celle qui va de $M$ dans $N$) est
donc caract\'eris\'ee -- les cartes \'etant choisies -- par $n$
fonctions diff\'erentiables $y^\nu$ de $m$ variables
$x^\mu$. Il est alors naturel de consid\'erer la matrice
jacobienne de cette application, c'est \`a dire la matrice
rectangulaire $m \times n$ des d\'eriv\'ees partielles
$\partial y^\nu \over \partial x^\mu .$ Nous en
reparlerons un peu plus tard.\par
\smallskip
\subsection{ Diff\'eomorphismes et changements de coordonn\'ees}
\par
Il existe deux cas particuliers particuli\`erement
int\'eressants.\par

 Le premier est celui o\`u $M$ et $N$
co\"incident. Dans ce cas, il peut se faire que
l'application diff\'erentiable $\phi$ soit non seulement
diff\'erentiable mais encore bijective et que son inverse
soit \'egalement diff\'erentiable. On dit alors que $\phi$
est un {\sl diff\'eomorphisme}\index{diff\'eomorphisme}. Notons qu'une application
diff\'erentiable est automatiquement continue et que, par
cons\'equent, un diff\'eomorphisme est automatiquement un
hom\'eomorphisme. Il est facile de v\'erifier que l'ensemble
des diff\'eomorphismes d'une vari\'et\'e diff\'erentiable $M$
constitue un groupe pour la composition des applications.
On note ce groupe $Diff(M) \subset C^\infty(M,M)$; c'est un
sous groupe de l'ensemble $Hom(M) \subset C^0(M,M)$ des
hom\'eomorphismes de $M$. Notons qu'il existe une
correspondance assez subtile entre diff\'eomorphismes d'une
part -- qui sont des transformations que l'on appelait
autrefois ``actives'' car elles transforment les points
de $M$ en d'autres points de $M$ -- et changements de
coordonn\'ees -- qui sont des transformations que l'on appelait
autrefois ``passives'' car elles ne transforment pas les
points de $M$ mais r\'esultent seulement d'un changement de
carte.

Il est \`a peu pr\`es \'evident que ces deux notions
co\"incident dans le cas o\`u $M$ est l'espace $\RR^n$
lui-m\^eme (muni de la structure diff\'erentiable d\'efinie
par une unique carte canonique, l'application identique).
Examinons de plus pr\`es le cas g\'en\'eral. Les cartes \'etant
elles-m\^emes des diff\'eomorphismes locaux entre ouverts de
$M$ et ouverts de $\RR^n$, effectuer un changement de carte
(changement de syst\`eme de coordonn\'ees) se traduit par un
diff\'eomorphisme local $y(x)$ de $\RR^n$. Par contre, un
diff\'eomorphisme de $M$ est, par d\'efinition, une notion
globale qui se traduit elle-aussi, apr\`es choix de cartes,
par un diff\'eomorphisme local de $\RR^n$. L'\'equivalence des
points de vue ``actifs'' et ``passifs'' n'existe donc que
pour $\RR^n$ et il semble pr\'ef\'erable d'\'eviter cette
terminologie. Une id\'ee physique fondamentale, \`a la base
de la th\'eorie de la relativit\'e g\'en\'erale est que les
\'equations de la physique doivent pouvoir s'\'ecrire de fa\c con
tout \`a fait ind\'ependante de l'observateur, quelle que soit
l'\'etat de mouvement de ce dernier. Traduite en termes de
coordonn\'ees, ce ``Principe de Relativit\'e G\'en\'erale'' a
souvent \'et\'e exprim\'e de par le pass\'e comme affirmant
l'ind\'ependance des lois de la physique par rapport aux
changements de syst\`emes de coordonn\'ees. Une telle
affirmation manque de pr\'ecision, d\`es lors qu'on travaille
sur une vari\'et\'e quelconque et non sur un espace
num\'erique. Il semble d'ailleurs qu'A. Einstein lui-m\^eme
n'ait jamais pu exprimer correctement ce principe de
fa\c con vraiment pr\'ecise et moderne (cela n'enl\`eve rien \`a son
g\'enie!). Le principe en question  peut s'\'enoncer ainsi:
l'espace-temps \'etant d\'ecrit par une vari\'et\'e
diff\'erentiable, les lois de la physique doivent \^etre
invariantes sous l'action du groupe des diff\'eomorphismes
de cette vari\'et\'e.\par
\smallskip
\subsection{  Fonctions diff\'erentiables}
\par
 La deuxi\`eme classe de cas particuliers int\'eressants est
celle o\`u l'application diff\'erentiable consid\'er\'ee $\phi$,
de $M$ dans $N$  est d\'efinie sur une vari\'et\'e
quelconque $M$, mais ou $N$ co\"incide avec
l'ensemble $\RR$ des nombres r\'eels. Les applications
diff\'erentiables en question sont d\'esign\'ees sous le nom de
{\sl fonctions diff\'erentiables sur $M$} \index{fonctions diff\'erentiables}; l'utilisation du mot ``fonction''
 est en accord
avec les habitudes terminologiques anglaises, o\`u les
applications quelconques sont des ``{\it maps\/}'' , mais o\`u les
applications \`a valeurs r\'eelles (ou complexes) sont des
``{\it functions\/}''.  L'ensemble des fonctions diff\'erentiables
sur $M$ se note $C^\infty(M)  = 
C^\infty(M,\RR).$\par

Remarque: l'ensemble des fonctions
diff\'erentiables $C^\infty(M)$ est une alg\`ebre pour
l'addition des fonctions $[f+g](x) =  f(x) + g(x)$, la
multiplication des fonctions d\'efinie (ponctuellement)
par $[fg](x) =  f(x) g(x)$ et l'op\'eration externe de
multiplication par un nombre r\'eel. C'est une sous-alg\`ebre
de l'alg\`ebre commutative $C^0(M)$.\par

Le lecteur peut s'\'etonner de la pr\'esence et de la
signification de l'indice sup\'erieur $0$ ou $\infty$
dans les notations $C^0(M)$ ou $C^\infty(M)$. Cet
indice se r\'ef\`ere \`a l'ordre de diff\'erentiabilit\'e suppos\'e
des fonctions appartenant \`a l'ensemble consid\'er\'e. On
pourrait bien entendu consid\'erer des ensembles tels que
$C^p(M)$ constitu\'es de fonctions qui sont, au moins, $p$
fois diff\'erentiables. Dans la suite de cet ouvrage,
cependant, nous nous limiterons aux cas $p=0$, c'est \`a
dire les fonctions continues (qui peuvent \'evidemment \^etre
diff\'erentiables ou non) et $p=\infty$, c'est \`a dire
les fonctions infiniment
diff\'erentiables.\par

\bigskip

\section { Champs de vecteurs}
\index{champs de vecteurs}
\smallskip
\subsection{  Notions \'el\'ementaires et intuitives}
\par
Avant de donner une d\'efinition g\'en\'erale des vecteurs et
champs de vecteurs, d\'efinition qui pourrait sembler
assez abstraite de prime abord, nous souhaitons motiver
quelque peu cette d\'efinition. Le lecteur est d\'ej\`a suppos\'e
\^etre familier de la notion \'el\'ementaire de vecteur, \`a savoir
une classe d'\'equivalence de bi-points parall\`eles et de
m\^eme sens, dans l'espace affine $\RR^n$. Un champ de
vecteurs de $\RR^n$, au sens \'el\'ementaire du terme, est donc
une application qui, \`a tout point de $\RR^n$ -- consid\'er\'e
comme espace affine --  associe un vecteur de $\RR^n$
--consid\'er\'e comme espace vectoriel. Intuitivement, on a
une ``fl\`eche'' en tout point; on peut penser \`a l'exemple du
champ des vitesses d'un solide en mouvement, mais on peut
aussi penser au champ magn\'etique en tout point de
l'espace, etc. En physique -- mais aussi, comme nous
allons le voir, en math\'ematiques --  un vecteur peut \^etre
consid\'er\'e comme un ``petit d\'eplacement''. Soit $M$ une
vari\'et\'e diff\'erentiable, $f$ une fonction diff\'erentiable
ainsi que $P$ et $Q$ deux points de $M$. Si $M$ \'etait un
espace affine (comme $\RR^n$), cela aurait un sens de
consid\'erer la diff\'erence de $Q$ et de $P$, puisque
cette diff\'erence d\'efinirait simplement le vecteur
$ \overrightarrow{PQ} =  Q-P$. On pourrait aussi (mais on
peut de toutes fa\c cons) consid\'erer la diff\'erence $f(Q) -
f(P)$ des valeurs prises par $f$ en $Q$ et $P$. Dans le cas
de $M= \RR^n$ et lorsque $Q$ (coordonn\'ees $x'$) tend vers $P$
(coordonn\'ees $x$), le th\'eor\`eme des accroissement finis
(ou celui de Taylor) nous dit que
\footnote{Nous utilisons la convention d'Einstein : il existe toujours
une somme sous-entendue sur les indices r\'ep\'et\'es, par exemple $v^\mu w_\mu 
\equiv \sum_\mu v^\mu w_\mu$. En r\`egle g\'en\'erale, les indices r\'ep\'et\'es ne 
sont jamais en m\^eme position : l'un est en position haute et l'autre en
position basse, \`a moins qu'une m\'etrique ne soit utilis\'ee.} 
 $f(x') - f(x) = (x'-x)^i
\partial / \partial x^i f(x) + \ldots = v^i \partial / \partial x^i f(x) + 
\ldots
$ o\`u les nombres $ v^i  =  (x'-x)^i $ ne sont autres que
les composantes du vecteur $\overrightarrow{ PQ} =  Q-P$
dans le rep\`ere o\`u $P$ et $Q$ ont des composantes $x^i$ et
${x'}^i$. Dans le cas des vari\'et\'es, l'expression $(x'-x)^i
\partial / \partial x^i f(x)$ a encore un sens. En effet,
choisissons tout d'abord une carte, et notons $v$ la
quantit\'e  $v  =  v^i \partial / \partial x^i$.
 Si $x(P)$ sont les coordonn\'ees de $P$ dans le
domaine de la carte $x$, on pourra consid\'erer la quantit\'e
$$
\mbox{\fbox{$
v[f]  =  v^i {\partial \over \partial x^i} f(x)
$}}
$$ qui nous
d\'ecrit la variation - au premier ordre -- de $f$ dans ce
que nous avons envie d'appeler la direction $v$. La
quantit\'e pr\'ec\'edente $v[f]$ est elle-m\^eme une fonction, qui,
lorsqu'elle est \'evalu\'ee au point $P$ nous fournit un
nombre $v[f](P)$.

\par
\smallskip
\subsection{  Vecteurs, espace tangent et champs de vecteurs}
\par
Dans le cas des vari\'et\'es, il est clair que les vecteurs
ne peuvent pas \^etre d\'efinis comme des bi-points (ou des
classes d'\'equivalences de bi-points), par contre, rien ne
nous emp\^eche d'utiliser leur propri\'et\'e de
machine-\`a-fabriquer-des-d\'eriv\'ees-partielles pour les
d\'efinir de fa\c con g\'en\'erale. Dans le domaine d'une carte
$x$, un champ de vecteurs sera donc d\'efini comme un op\'erateur de
diff\'erentiation d'ordre $1$ \`a
savoir  $$v = v^i {\partial \over \partial x^i}$$ Cet op\'erateur
agit sur les fonctions $f\in C^\infty(M)$ pour donner
d'autres fonctions (puisque $v[f] \in C^\infty(M)$).
Le champ de vecteurs $v$ ainsi
d\'efini est ind\'ependant de la carte choisie.


 L'op\'erateur
diff\'erentiel d'ordre $1$ not\'e $v=v^i \partial / \partial
x^i$ est un champ de vecteurs car les $v^i$ sont des
fonctions sur $M$ alors que $v(P)=v^i(P) \partial /
\partial x^i$ est un vecteur au point $P$, de composantes
$v^i(P)$.


En g\'eom\'etrie \'el\'ementaire des courbes, la
tangente en $P$ \`a une courbe (diff\'erentiable) est d\'efinie
comme limite des s\'ecantes $PQ$ lorsque $Q$ tends vers $P$;
cela signifie que les vecteurs $\overrightarrow{ PQ}$
tendent vers un vecteur tangent \`a la courbe. En g\'eom\'etrie
des vari\'et\'es diff\'erentiables, on pourrait faire de m\^eme,
\`a condition de plonger notre vari\'et\'e (par exemple la
sph\`ere usuelle $S^2$) dans un espace plus grand (par
exemple $\RR^3$) et voir ainsi, un vecteur de $S^2$ comme
un vecteur tangent \`a la sph\`ere (et donc ``sortant'' de
celle-ci); mais une telle contrainte serait pr\'ecis\'ement
contraire \`a l'id\'ee m\^eme du calcul intrins\`eque sur les
vari\'et\'es, calcul qui se veut, justement, ind\'ependant de
l'existence de plongements possibles. La d\'efinition
adopt\'ee pr\'ec\'edemment est bien ind\'ependante de la
pr\'esence d'un espace affine ambiant, mais il est
n\'eanmoins commode, pour l'intuition, de visualiser nos
vecteurs de fa\c con \'el\'ementaire et d'adopter une
terminologie qui nous rappelle des situations bien
connues. Pour ces raisons, un vecteur de la vari\'et\'e $M$
en un point $P$ est souvent appel\'e {\sl vecteur tangent \/}\index{vecteur 
tangent} en
$P$, l'ensemble de ces vecteurs se note $T(M,P)$ ou
encore $T_P M$ et est d\'esign\'e sous le nom de {\sl  espace
tangent \/}\index{espace tangent} \`a $M$ en $P$~; on a donc un espace tangent en
chaque point de la vari\'et\'e. L'ensemble des vecteurs eux-m\^emes 
(tous les vecteurs), se note $T(M)$ ou simplement $TM$
et est appel\'e l'{\sl espace tangent} \`a $M$ ou encore, pour une
raison qu'on expliquera ult\'erieurement le {\sl fibr\'e tangent}\index{fibr\'e 
tangent}
\`a M (``{\it tangent bundle\/}''). Un \'el\'ement de $TM$ est donc la donn\'ee 
$(P,u)$ d'un
point de $M$ et d'un vecteur en ce point. Attention, il
faut bien distinguer les notions de vecteur en un point et
de champs de vecteurs (mais nous allons tr\`es souvent
oublier cette distinction). L'ensemble des champs de vecteurs se note
$\Gamma TM$. Notons que cet espace est un espace
vectoriel (de dimension infinie), et $T(M,P)$ est
un espace vectoriel de dimension $n$ (supposant que $M$ est
elle-m\^eme de dimension $n$), alors que $TM$ n'est pas un
espace vectoriel du tout (on ne peut pas additionner un
vecteur en $P$ avec un vecteur en $Q$!). On verra que $TM$, que l'on peut 
consid\'erer comme une
collection d'espaces vectoriels param\`etris\'es par les
points de $M$, poss\`ede la structure d'espace fibr\'e
vectoriel (cette structure sera d\'efinie et \'etudi\'ee plus
loin). Notons que l'espace $TM$ est lui-m\^eme une
vari\'et\'e diff\'erentiable. Supposons que $M$ soit une vari\'et\'e
de dimension $n$, un point $P$ de $M$ est en effet caract\'eris\'e
(dans une certaine carte) par $n$ composantes $x^\mu$ et
 un ``point'' (c'est \`a dire un \'el\'ement) de $TM$
consistera en la donn\'ee d'un couple $(P,u)\in M \times
T(M,P)$ c'est \`a dire $2n$ nombres ($n$ nombres $x^\mu$ et
$n$ composantes du vecteur $u$ dans une base
choisie de l'espace vectoriel $T(M,P)$. Ainsi $TM$ est
une vari\'et\'e de dimension $2n$. Intuitivement, on peut se
repr\'esenter par exemple $TS^2$ comme la donn\'ee d'une
infinit\'e de plans tangents coll\'es \`a la sph\`ere; il s'agit,
dans ce cas d'une vari\'et\'e de dimension $4$.\par
\smallskip
\subsection{  R\`egle de Leibniz}
\par
Soit $v$ un champ de vecteurs. Il pourra donc s'\'ecrire
localement (c'est \`a dire dans une certaine carte)
$v=v^\mu \partial / \partial x^\mu$. Si $f$ et $g$
d\'esignent deux fonctions sur $M$, il est clair que $$
{ \partial \over \partial x^\mu}
[f g] = {\partial \over \partial x^\mu} [f] g + f {\partial \over \partial 
x^\mu}
[g]$$ Par cons\'equent on aura plus g\'en\'eralement: $$
\mbox{\fbox{$
 v[f g] = v[f] g + f v[g]
 $}}
$$
On retrouve la r\`egle usuelle de d\'erivation d'un produit.
De fa\c con g\'en\'erale, si ${\cal A}$ est une alg\`ebre
associative, on dit que $v$ est une d\'erivation, lorsque $v$
est  une application lin\'eaire (un ``op\'erateur'') de $A$
dans $A$ telle que $v[f g] = v[f] g + f v[g]$ avec $f,g \in
{\cal A}$. Les champs de vecteurs sont des
d\'erivations de l'alg\`ebre associative (et commutative)
$C^\infty(M)$. On pourrait d'ailleurs les
d\'efinir directement par cette propri\'et\'e. En d'autres
termes, $\Gamma T M  = Der \, C^\infty(M)$. \par
\smallskip
\subsection{ Crochet de deux champs de vecteurs}
\par
Notons que le produit de deux vecteurs n'est pas un
vecteur (produit d\'efini par composition de l'action des
vecteurs sur les fonctions) mais un op\'erateur
diff\'erentiel d'ordre 2. En effet, soient $v=v^\mu
\partial / \partial x^\mu$ et $w=w^\nu
\partial / \partial x^\nu$ deux champs de vecteurs
(attention les $v^\mu$ est les $w^\nu$ n'ont aucune
raison d'\^etre constants dans la carte choisie). Alors,
$(v w)[f]  =  v[w[f]] = v[w^\nu
\partial / \partial x^\nu [f]] = v^\mu
\partial / \partial x^\mu [w^\nu \partial / \partial x^\nu
[f]] = v^\mu \partial / \partial x^\mu [w^\nu]\partial / \partial x^\nu
[f] + v^\mu w^\nu \partial^2 / \partial x^\mu \partial x^\nu [f]$
Par contre, le commutateur (notation crochet) de deux
champs de vecteurs, d\'efini par $$
\mbox{\fbox{$
[v,w] =  v w - w v
$}} $$
est un champ de vecteurs. Pour s'en convaincre, il suffit
de v\'erifier que c'est bien un op\'erateur diff\'erentiel
d'ordre un. Le petit calcul pr\'ec\'edent montre
imm\'ediatement que les d\'eriv\'ees secondes disparaissent
lorsqu'on calcule la diff\'erence et qu'il reste $$
[v,w][f] = (v w)[f] - (w v)[f] = (
v^\mu {\partial \over \partial x^\mu} [w^\nu]{\partial \over \partial
x^\nu} -
w^\mu {\partial \over \partial x^\mu} [v^\nu]{\partial \over \partial
x^\nu})[f] $$
La d\'efinition pr\'ec\'edente du crochet $[v,w] = v w - w v$ de
deux champs de vecteurs \index{crochet de Lie} implique de fa\c con imm\'ediate les
deux propri\'et\'es suivantes:\par
Antisym\'etrie $$\mbox{\fbox{$[u,v]=-[v,u]$}}$$
Identit\'e  de Jacobi
$$\mbox{\fbox{$ [u,[v,w]] + [v,[w,u]] + [w,[u,v]] = 0 $}}$$
Une alg\`ebre (\'evidemment non associative) o\`u les
\'el\'ements v\'erifient ces deux identit\'es est appel\'ee une
alg\`ebre de Lie. Notons qu'une alg\`ebre de Lie est, en
particulier, un espace vectoriel. Nous pouvons donc
conclure ce paragraphe en disant ``l'ensemble des champs
de vecteurs est une alg\`ebre de Lie (de dimension
infinie)''.
\smallskip
\subsection{ Rep\`ere naturel associ\'e \`a une carte}
\par
On appelle rep\`ere sur $U\subset M$, la donn\'ee, en chaque
point $P\in U$, d'une base de l'espace vectoriel tangent
en $P$. Un rep\`ere est en
g\'en\'eral ``local'', c'est \`a dire qu'on n'essaye pas, ou
qu'on ne peut pas choisir $U=M$. Si $x^\mu (P) $ d\'esignent les composantes 
de $P$
dans une carte locale $(U,x)$, on a d\'ej\`a vu que des vecteurs
quelconques en $P$ ou dans un voisinage de $P$ peuvent se
d\'ecomposer sur les vecteurs $\partial / \partial x^\mu$.
En d'autres termes, l'ensemble des $$\mbox{\fbox{$ e_\mu  = 
\{\partial / \partial x^\mu \}$}}$$ fournit un rep\`ere. Ce
rep\`ere est appel\'e {\sl rep\`ere naturel associ\'e \`a la carte}\index{rep\`ere 
naturel}
$x$ ou aux coordonn\'ees $x^\mu$ (``{\it coordinate frame\/}''). Par suite 
de la
propri\'et\'e de commutativit\'e des d\'eriv\'ees partielles,
 il est \'evident que
$\partial^2 /\partial x^\mu \partial x^\nu - \partial^2 /\partial
x^\nu \partial x^\mu = 0$. En d'autres termes, si $\{e_\mu\}$
d\'esigne le rep\`ere naturel associ\'e \`a la carte $x^\mu$, on
a $$ [e_\mu, e_\nu] = 0$$
Une telle propri\'et\'e
caract\'erise, en fait, les rep\`eres naturels.
\smallskip
\subsection{ Changement de carte}
\par
Soit $P\in M \rightarrow y(P) \in \RR^n $ un nouveau syst\`eme
de coordonn\'ees. Si $x$ d\'esigne l'ancien syst\`eme, on notera
\'egalement $y: \RR^n \mapsto \RR^n$ les fonctions de
changement de carte, on \'ecrira donc $y(P) = y(x(P))$. Le
rep\`ere naturel associ\'e aux coordonn\'ees $x$ est $e_\mu
 =  \{\partial / \partial x^\mu \}$, celui associ\'e aux
coordonn\'ees $y$ est $e^\prime_\mu
 =  \{\partial / \partial y^\mu \}$. Nous savons
(depuis le secondaire) comment calculer la d\'eriv\'ee d'une
fonction compos\'ee, et donc $$
\mbox{\fbox{$
{\partial \over \partial x^\mu}  = 
{ \partial y^\nu \over \partial x^\mu  } 
{ \partial \over \partial y^\nu }  $}} $$
ce qui, avec d'autres notations, s'\'ecrit $$
e_\mu =  {\partial y^\nu \over \partial x^\mu }
e^\prime_\nu$$
{\it Notation} 

Il est souvent commode de noter tout simplement
$\partial_\mu$ les vecteurs du rep\`ere naturel $\{e_\mu =
\partial / \partial x^\mu \}$ associ\'es \`a la carte
$x^\mu$. La d\'ecomposition d'un vecteur quelconque $v$
suivant ce rep\`ere s'\'ecrit $v = v^\mu \partial_\mu $,
o\`u les $v^\mu$ sont des nombres r\'eels.
\smallskip
\subsection{ Rep\`eres mobiles (rep\`eres quelconques)}
\par
Dans un espace vectoriel, nous savons que les changements
de base sont d\'ecrits par des matrices ``de passage'' qui
ne sont autres que des matrices inversibles
$\Lambda^\mu_\alpha$ quelconques. En g\'eom\'etrie
diff\'erentielle, nous pouvons bien entendu faire de
m\^eme, \`a ceci pr\`es que la matrice
$\Lambda^\mu_\alpha$ peut maintenant d\'ependre du point
de la vari\'et\'e. En d'autres termes, on a des matrices
de passage dont les \'el\'ements sont des fonctions sur
la vari\'et\'e. Supposons que nous nous trouvons
dans le domaine d'une carte et que $\{\partial_\mu\}$
d\'esigne le rep\`ere naturel associ\'e. Ce
rep\`ere, au point $P$, constitue une base de
l'espace tangent en $P$. Mais rien ne nous emp\^eche
de choisir une autre base au m\^eme point. Si
$\Lambda^\mu_\alpha$ d\'esigne une matrice inversible en
$P$, alors la famille de vecteurs $\{e_\alpha  = 
\Lambda^\mu_\alpha \partial_\mu \}$ est une autre base
de l'espace tangent $T_pM$, c'est \`a dire un rep\`ere au
point $P$. Un tel rep\`ere est couramment d\'esign\'e
sous le nom de {\sl rep\`ere mobile\/}\index{rep\`ere mobile}. Notons qu'il n'y a
aucune raison, {\it a priori}, pour que ce rep\`ere
co\"incide avec le rep\`ere naturellement associ\'e \`a une
autre carte que celle des $x^\mu$; pour que cela soit le
cas, il faudrait qu'on puisse trouver une solution locale
$y^\alpha$ au   syst\`eme d'\'equations $\partial y^\alpha / \partial
x^\mu = ({\Lambda^{-1}})^\alpha_\mu)$ o\`u $\Lambda^{-1}$
d\'esigne la matrice inverse de la matrice $\Lambda$. Le th\'eor\`eme
garantissant l'existence de solutions pour une \'equation
diff\'erentielle aux d\'eriv\'ees partielles nous assure
seulement l'existence d'une telle solution
$y^\mu(x^\nu))$ le long d'une ligne, mais pas dans un
voisinage ouvert de la vari\'et\'e.
\par
Soit $\{e_\alpha\}$ un rep\`ere mobile. Nous avons d\'ej\`a vu
que le crochet (commutateur) de deux champs de vecteurs
est un champ de vecteurs. En particulier
$[e_\alpha,e_\beta]$ est un champ de vecteurs qui, \'evalu\'e au
point $P$, appartient \`a l'espace tangent en ce point
et peut donc se d\'ecomposer sur une base de l'espace tangent
en $P$. On \'ecrira donc 
$$[e_\alpha,e_\beta](P) = f_{\alpha \beta}^\gamma (P)
e_\gamma (P)$$ ou, plus simplement
$$
\mbox{\fbox{$
[e_\alpha,e_\beta] = f_{\alpha \beta}^\gamma e_\gamma 
$}}
$$
o\`u les $f_{\alpha \beta}^\gamma$ sont des fonctions sur
la vari\'et\'e qu'on appelle {\sl fonctions de structure du
rep\`ere mobile \/}\index{fonctions de structure} (on ne doit pas les appeler {\sl constantes
de structure}\index{constantes de structure} car, pr\'ecis\'ement, elles ne sont pas
constantes en g\'en\'eral!).

Par ailleurs, on posera souvent $\partial_\alpha  =  e_\alpha$ m\^eme s'il
n'existe pas de syst\`eme de coordonn\'ees $\{y^\alpha\}$ tel que 
$\partial_\alpha$ soit
le rep\`ere naturel associ\'e. Le lecteur doit donc se m\'efier de cet abus 
d'\'ecriture pourtant commode: il est des cas o\`u
$\partial_\alpha$ et $\partial_\beta$ ne commutent pas!
\par


\bigskip
\section [Tenseurs et formes ext\'erieures]{Tenseurs et formes ext\'erieures 
sur les espaces vectoriels}
\smallskip
Avant de passer au cas des vari\'et\'es, il convient d'effectuer quelques
rappels d'alg\`ebre lin\'eaire puisque le passage du cas vectoriel
au cas des vari\'et\'es s'effectue essentiellement en rempla\c cant
un espace vectoriel unique par une famille d'espaces
vectoriels ``de m\^eme nature'', param\`etris\'ee par les points de la
vari\'et\'e. 
\smallskip
\subsection{ Alg\`ebre tensorielle d'un espace vectoriel}
\par
Soit $E$ un espace vectoriel de dimension finie $n$ sur un corps $K$. On
note $E^*$ son dual, c'est \`a dire l'ensemble des formes $K$-lin\'eaires
sur $E$ (applications lin\'eaires sur $E$ \`a valeurs dans le
corps de base, qu'on suppose commutatif). En terme de composantes, soit 
$\{e_\mu\}$ une base de
$E$, et $\{e^\mu\}$ la base duale correspondante de $E^*$, on a
 $$e^\mu [e_\nu] = \delta^\mu_\nu$$ o\`u $\delta^\mu_\nu$
d\'esigne le symbole de Kronecker ($1$ si $\mu = \nu$ et $0$ si $\mu
\neq \nu$).

L'espace vectoriel $E$ de r\'ef\'erence \'etant choisi, on \'ecrira (comme le font 
toujours 
les physiciens) les vecteurs de base avec des ``indices en bas'' et les 
composantes avec des ``indices en haut''. Bien entendu, la convention est 
oppos\'ee pour ce qui concerne l'espace vectoriel dual. Par ailleurs nous 
adoptons \'egalement la ``convention d'Einstein'', c'est \`a dire que nous 
effectuons toujours une sommation (le signe somme \'etant sous-entendu) sur 
les indices r\'ep\'et\'es, lorsque l'un des indices est en position haute et 
l'autre en position basse. Nous avons d\'ej\`a utilis\'e cette convention dans 
les sous-sections pr\'ec\'edentes. Cette convention all\`ege consid\'erablement 
l'\'ecriture des formules.

Nous n'adopterons pas, dans cet ouvrage, la notation dyadique ch\`ere \`a 
Dirac utilisant des {\sl bra}\index{bra et ket} et des {\sl ket} car elle est peu usuelle en 
g\'eom\'etrie mais il est peut-\^etre utile d'y consacrer quelques lignes. Avec 
cette notation, les \'el\'ements d'un certain espace vectoriel $E$ choisi une 
fois pour toutes sont not\'es avec des ``kets'', par exemple $\vert v 
\rangle$ et les \'el\'ements du dual avec des ``bras'', par exemple $\langle 
\sigma \vert$. L'\'evaluation d'une forme sur un vecteur se note ainsi 
naturellement sous forme de ``bracket'' $\langle \sigma \vert v \rangle$.
La relation pr\'ec\'edente caract\'erisant la dualit\'e entre une base de $E$ et 
une base de 
$E^*$ s'\'ecrira
donc $$\langle e^\mu \vert e_\nu \rangle = \delta^\mu_\nu$$
On \'evalue ici une forme sur un vecteur et on obtient donc un nombre.

Par contre, la quantit\'e $\vert e_\nu
\rangle \langle  e^\mu  \vert $  d\'esigne une application lin\'eaire de $E$ 
dans $E$ puisque
$\vert e_\nu \rangle \langle  e^\mu  \vert e_\rho\rangle = \vert 
e_\nu\rangle
\delta^\mu_\rho = \vert e_\rho\rangle$. Ainsi, en prenant $\vert v 
\rangle = v^\rho \vert e_\rho \rangle$, on obtient $\vert e_\nu \rangle 
\langle  
e^\mu \vert v \rangle = v^\mu \vert e_\nu \rangle$. Pour les m\^emes 
raisons, 
l'\'ecriture $\vert v \rangle \langle  \sigma \vert $ d\'esigne un op\'erateur 
(alors que  $\langle  \sigma \vert v \rangle$ d\'esigne un nombre).

 L'identification des vecteurs
de $E$ avec des applications de $K$ dans $E$ (\`a $v \in E $ on associe 
l'application
$\lambda \in K \rightarrow \lambda v \in E$)  permet de bien comprendre 
cette dualit\'e
et l'int\'er\^et de la notation dyadique.

Si on se souvient ``qui est qui'', et si on fait attention \`a l'ordre des 
termes, on peut simplifier les notations \`a 
l'extr\^eme et ne noter ni les produits tensoriels, ni les symboles 
$\langle {\,} \vert$ ou $\vert {\,} \rangle$.
On \'ecrira ainsi parfois de fa\c con un peu
provocante les \'el\'ements de $E$ sous la forme $$v= e_\mu v^\mu$$ (avec 
$v^\mu \in \RR$)
et les composantes {\it \`a droite} des vecteurs.  On \'ecrira parfois de m\^eme
les \'el\'ements du dual $E^*$ sous la forme $$\sigma = \sigma_\mu
e^\mu$$ (avec  $\sigma_\mu \in \RR$) et les composantes {\it \`a
gauche} des formes lin\'eaires. Si on ne note explicitement ni les produits 
tensoriels ni les \'evaluations des formes sur les vecteurs, on voit que
$\sigma v = \sigma_\mu (e^\mu e_\nu) v^\nu =
\sigma_\mu \delta^\mu_\nu v^\nu = \sigma_\mu v^\mu$ est un nombre. Par 
contre $v \sigma$
est un op\'erateur (on pourrait l'\'ecrire $v \otimes \sigma \in E \otimes 
E^*$), plus
pr\'ecis\'ement $v \sigma = e_\mu (v^\mu \sigma_\nu) e^\nu$.

 L'ordre adopt\'e ci-dessus
(le fait d'\'ecrire les composantes ---qui sont pourtant des nombres!--- \`a 
droite des vecteurs, etc) est particuli\`erement
adapt\'e aux g\'en\'eralisations non commutatives de la g\'eom\'etrie diff\'erentielle
-- cela vient du fait qu'en Occident, nous \'ecrivons de gauche \`a droite!-- 
mais
rappelons nous que, bien entendu, en g\'eom\'etrie ordinaire ``commutative'' 
(celle
qui nous int\'eresse ici), on peut toujours \'ecrire $v= e_\mu v^\mu = v^\mu 
e_\mu$. Un
dernier mot de mise en garde: lorsqu'on veut insister sur le fait que
le vecteur  $e_\mu$ d\'esigne une d\'erivation $\partial_\mu$,
il est pr\'ef\'erable -- pour ne pas se tromper! -- d'\'ecrire les composantes du
c\^ot\'e gauche. Il en va de m\^eme en g\'eom\'etrie non commutative o\`u champs
de vecteurs et d\'erivations d'alg\`ebre sont de toute fa\c con des concepts
diff\'erents puisque les premiers forment un module sur l'alg\`ebre 
associative des
``fonctions'' alors que les d\'erivations ne forment un module que sur le 
centre de
cette alg\`ebre. Aucune ambigu\"it\'e n'est donc possible dans ce cadre plus 
g\'en\'eral.

\par

On note $\bigotimes E$ l'alg\`ebre tensorielle sur $E$ c'est \`a dire
la somme directe $\oplus_{p=0}^{\infty} E^{\otimes p}$ o\`u $E^{\otimes
p}$ d\'esigne la puissance tensorielle d'ordre $p$ de $E$, c'est \`a dire
encore l'ensemble des applications multilin\'eaires d'ordre $p$ sur
$E^*$. Soit $T\in E^{\otimes p}$ alors on peut \'ecrire $$
T = T^{\mu_1 \mu_2 \ldots \mu_p} \; e_{\mu_1}\otimes e_{\mu_2}\otimes \ldots
\otimes e_{\mu_p}
$$
Les \'el\'ements de $\bigotimes E$ sont encore appel\'es tenseurs
contravariants (d'ordre $p$ s'ils appartiennent \`a $E^{\otimes p}$).
Bien entendu, cet ensemble est non
seulement un espace vectoriel (de dimension infinie, les $E^{\otimes
p}$ \'etant de dimension $(dim \, E)^p$ mais encore une alg\`ebre pour le
produit tensoriel. On peut ne pas \'ecrire le symbole $\otimes$ 
explicitement dans
l'expression pr\'ec\'edente du tenseur $T$, car$\ldots$ ``what else could it 
be?'', auquel cas,
 $$
T = T^{\mu_1 \mu_2 \ldots \mu_p} \; e_{\mu_1}e_{\mu_2}\ldots e_{\mu_p}
$$
De la m\^eme fa\c con, on note  $\bigotimes E^*$ l'{\sl alg\`ebre
tensorielle \/}\index{alg\`ebre tensorielle} sur $E^*$ c'est \`a dire la somme directe
$\oplus_{p=0}^{\infty} {E^*}^{\otimes p}$ o\`u ${E^*}^{\otimes p}$
d\'esigne la puissance tensorielle d'ordre $p$ de ${E^*}$, c'est \`a
dire encore l'ensemble des applications multilin\'eaires d'ordre $p$
sur $E$. Soit $T\in {E^*}^{\otimes p}$ alors on peut \'ecrire $$ T =
T_{\mu_1 \mu_2 \ldots \mu_p} e^{\mu_1}e^{\mu_2}\ldots e^{\mu_p} $$
Les \'el\'ements de $\bigotimes E^*$ sont encore appel\'es tenseurs covariants 
(d'ordre $p$ s'ils appartiennent \`a ${E^*}^{\otimes p}$).\par

Bien entendu, nous pourrons consid\'erer des tenseurs $p$-fois
contravariants et $q$-fois covariants (\'el\'ements $T$ de ${E}^{\otimes
p} \otimes {E^*}^{\otimes q})$ et pour rester coh\'erents avec nos
notations, nous \'ecrirons les produits tensoriels des vecteurs de $E$ \`a
gauche de ceux de $E^*$, c'est \`a dire $$
T = T^{\mu_1 \mu_2 \ldots \mu_p}_{\nu_1 \nu_2 \ldots \nu_q}
\, e_{\mu_1}e_{\mu_2}\ldots e_{\mu_p} \, e^{\nu_1}e^{\nu_2}\ldots
e^{\nu_q}$$ ou m\^eme encore $$
T = e_{\mu_1}e_{\mu_2}\ldots e_{\mu_p}\,  T^{\mu_1 \mu_2 \ldots 
\mu_p}_{\nu_1
\nu_2 \ldots \nu_q}\,  e^{\nu_1}e^{\nu_2}\ldots e^{\nu_q}$$
On pose ${E^*}^{\otimes 0} = {E}^{\otimes 0} = \RR$. 
\smallskip
\subsection{ Alg\`ebre ext\'erieure d'un espace vectoriel. Produit ext\'erieur}
\par
On notera $\Lambda^k(E^*)$ l'espace vectoriel des formes $k$-lin\'eaires
altern\'ees sur $E$. Rappelons que $
T \in {E^*}^{\otimes k}$ est altern\'ee lorsque
$T(v_1,\ldots,v_i\ldots,v_j,\ldots, v_k) = 0$ d\`es que $v_i = v_j, i\neq j$.
Il est \'equivalent de dire (si le corps de base n'est pas de
caract\'eristique $2$) que $T$ est antisym\'etrique, c'est \`a dire que 
$T(v_1,\ldots,v_i\ldots,v_j,\ldots, v_k)  = -
T(v_1,\ldots,v_j\ldots,v_i,\ldots, v_k).$
On dit aussi que $T$ est une {\sl forme ext\'erieure}\index{forme ext\'erieure} d'ordre $k$ et que 
$\Lambda(E^*)$ est l'alg\`ebre ext\'erieure construite sur $E^*$.
\begin{description}
	\item[L'antisym\'etriseur ${\bf Alt}$]  

	Le groupe sym\'etrique ${\cal S}_k$ des substitutions sur $k$ \'el\'ements op\`ere
de fa\c con \'evidente sur les $k$-uplets de vecteurs.
Soit $s \in {\cal S}_k$
$$
s . (v_1,v_2,\ldots,v_k)  =  (v_{s(1)},v_{s(2)},\ldots,v_{s(k)})$$
Gr\^ace \`a cette action, on peut d\'efinir un op\'erateur $Alt$ qui projette
les tenseurs covariants d'ordre $k$ sur les formes $k$-lin\'eaires
antisym\'etriques $$
\mbox{\fbox{$
Alt \, T  =  {1 \over k!} \sum_{s\in S_k} \epsilon_s Tos
$}}
$$
o\`u $\epsilon$ d\'esigne la parit\'e de la substitution $s$.
On peut v\'erifier les propri\'et\'es suivantes du projecteur $Alt$. Tout
d'abord, c'est effectivement un projecteur de ${E^*}^{\otimes p}$ sur
 $\Lambda^p(E^*)$, par ailleurs, si $\omega, \eta$ et
$\theta$ d\'esignent trois tenseurs de $\bigotimes E$,
alors $Alt(Alt(\omega \otimes \eta)\otimes \theta) = Alt(\omega \otimes
Alt(\eta \otimes \theta)$ et on peut donc \'ecrire cette quantit\'e sous
la forme $Alt(\omega \otimes \eta \otimes \theta).$ La pr\'esence du $k!$
dans la d\'efinition de $Alt$ est indispensable pour que la propri\'et\'e
pr\'ec\'edente d'associativit\'e soit v\'erifi\'ee.\par


	\item[Le  produit ext\'erieur $\land$]

Soient $\omega \in \Lambda^k(E^*)$ et $\eta \in \Lambda^p(E^*)$. On
d\'efinit {\sl Le  produit ext\'erieur \/}\index{produit ext\'erieur} $\land$,
 $$
 \mbox{\fbox{$
\omega \land \eta  =  {(k+p)! \over k! p!}\, Alt(\omega \otimes
\eta) \; \in \;  \Lambda^{k+p}(E^*)
$}}
$$

Propri\'et\'es:

$\land$ est associatif et distributif \`a droite et \`a gauche sur $+$

$a \omega \land \eta = \omega \land a \eta = a(\omega \land \eta)$
avec $a \in \RR$

$\omega \land \eta = (-1)^{(pk)} \eta \land \omega$ En particulier,
si $\omega$ est impaire, $\omega \land \omega = 0$

Ces propri\'et\'es font de $\Lambda(E^*) = \bigoplus_{k=0}^n \Lambda^k(E^*)$ 
une alg\`ebre
super-commutative (une alg\`ebre commutative $\ZZ_2$-gradu\'ee).

De plus, si $\omega \in \Lambda^k(E^*)$, $\eta \in \Lambda^p(E^*)$ et
$\theta \in \Lambda^q(E^*)$, alors $$
\omega \land \eta \land \theta = {(k+p+q)! \over k! p! q!}
Alt(\omega \otimes \eta \otimes \theta)$$

La pr\'esence des diverses factorielles dans les expressions ci-dessus,
aussi bien dans la d\'efinition de $Alt$ que dans celle du produit
ext\'erieur, dispara\^it dans bien des cas; par exemple, le lecteur
pourra se convaincre que si $\{\theta^\mu\}$ d\'esigne une base de
$1-formes$, les d\'efinitions pr\'ec\'edentes conduisent aux expressions
suivantes:$$ 
\theta^1 \land \theta^2 = \theta^1 \otimes \theta^2 - \theta^2
\otimes \theta^1$$ et
\begin{eqnarray*}
\theta^1 \land \theta^2 \land \theta^3 & = &
\theta^1 \otimes \theta^2 \otimes \theta^3 +
\theta^2 \otimes \theta^3 \otimes \theta^1 +
\theta^3 \otimes \theta^1 \otimes \theta^2 \\
{} & {} &  -
\theta^2 \otimes \theta^1 \otimes \theta^3 -
\theta^1 \otimes \theta^3 \otimes \theta^2 -
\theta^3 \otimes \theta^2 \otimes \theta^1  
\end{eqnarray*}

Il faut signaler ici qu'il existe une autre d\'efinition du produit
ext\'erieur o\`u les membres de droite des expressions pr\'ec\'edentes
sont respectivement multipli\'es par $1/2!$ et $1/3!$ La
d\'efinition adopt\'ee ici est telle que si $\{e_\mu\}$ d\'esigne une base
de l'espace vectoriel consid\'er\'e et $\{\theta^\mu\}$ la base duale
correspondante, nous avons $$
\theta^{1}\land \theta^{2}\land \ldots \land
\theta^{n} (e_1,e_2,\ldots,e_n) = 1$$

	\item[D\'ependance et ind\'ependance lin\'eaire des formes ext\'erieures] 

	D\'esignons par  $\{\theta^\mu\}_{\mu \in \{1,2\ldots n\}}$ une base de 
${E^*}$.
Consid\'erons un mon\^ome tel que $\theta^{\mu_1} \land \theta^{\mu_2}
\land \ldots \land \theta^{\mu_k}$. Par suite de l'antisym\'etrie du
produit ext\'erieur, il est clair qu'une telle expression est nulle d\`es
qu'un vecteur de base est r\'ep\'et\'e deux fois (c'est une autre fa\c con de
dire qu'un tenseur compl\`etement antisym\'etrique est nul d\`es que deux
indices sont r\'ep\'et\'es). Par ailleurs, deux mon\^omes de ce type qui ne
diff\`erent que par l'ordre des termes sont soit \'egaux, soit oppos\'es.
On peut donc supposer que les indices sont ordonn\'es de la fa\c con
suivante: $1 \leq \mu_1 < \mu_2 < \ldots <\mu_k \leq n$. Enfin, il
est facile de voir que toute forme ext\'erieure d'ordre $k$, c'est \`a
dire tout \'el\'ement de $\Lambda^k(E^*)$ peut se d\'ecomposer sur des
mon\^omes de ce type. La dimension de l'espace vectoriel
$\Lambda^k(E^*)$ est donc $({}^n_k)$. Bien entendu,
lorsque $k>n$, toute forme ext\'erieure est nulle (deux indices sont
alors automatiquement r\'ep\'et\'es!). La dimension de l'alg\`ebre ext\'erieure
est donc $\Sigma_{k=0}^n {}^n_k= 2^n$.
Pour conclure ce paragraphe, citons sans d\'emonstration
(mais elle est facile) le petit r\'esultat bien utile suivant: Les formes
lin\'eaires $\omega^1, \omega^2, \ldots, \omega^p$ sont ind\'ependantes si
et seulement si leur produit ext\'erieur $\omega^1 \land \omega^2  \land 
\ldots  \land  \omega^p$ est non nul.\par

	\item[Ecriture des formes ext\'erieures]

	Une forme ext\'erieure $\omega$ d'ordre $k$ peut s'\'ecrire de trois
fa\c cons possibles. Tout d'abord, on peut la consid\'erer comme un
tenseur $k$ fois covariant, et , \`a ce titre, on peut la d\'ecomposer
(existence et unicit\'e) sur la base des tenseurs d'ordre $k$. On peut
donc \'ecrire $$
\mbox{\fbox{$
\omega = \omega_{\mu_1\mu_2\dots\mu_k}
\theta^{\mu_1} \otimes \theta^{\mu_1} \otimes \ldots \otimes
\theta^{\mu_k}
$}}
$$
On peut aussi la d\'ecomposer sur la base des formes ext\'erieures 
$\theta^{\mu_1} \land \theta^{\mu_2}
\land \ldots \land \theta^{\mu_k}$, \`a condition d'ordonner les
indices (sinon, la famille pr\'ec\'edente est g\'en\'eratrice mais n'est pas
libre et donc n'est pas une base!). 
\begin{eqnarray*}
\omega & = & \sum_{\mu_1 < \mu_2 < \ldots <\mu_k}
\omega_{\mu_1\mu_2\dots\mu_k}
\theta^{\mu_1} \land \theta^{\mu_2} \land \ldots \land
\theta^{\mu_k}\\
{} & = & \omega_{| \mu_1\mu_2\dots\mu_k |}
\theta^{\mu_1} \land \theta^{\mu_2} \land \ldots \land
\theta^{\mu_k}
\end{eqnarray*}
La deuxi\`eme \'egalit\'e utilise une notation $|\ldots|$ qui
signifie que non seulement on utilise la convention d'Einstein (sommation 
sur les
indices r\'ep\'et\'es) mais qu'on d\'ecide d'ordonner les indices.

La troisi\`eme \'ecriture --- de loin, la plus utilis\'ee --- est
celle o\`u on d\'ecompose la forme $\omega$ (toujours la m\^eme) sur la
famille g\'en\'eratrice des formes ext\'erieures $\theta^{\mu_1} \land 
\theta^{\mu_1}
\land \ldots \land \theta^{\mu_k}$ mais sans ordonner les indices!
Bien entendu, pour un ensemble d'indices donn\'es (pour un ensemble de
vecteurs de base donn\'e), $k!$ des mon\^omes pr\'ec\'edents vont \^etre \'egaux
(ou oppos\'es) et il faudra ``corriger'' le d\'eveloppement de $\omega$ en
rajoutant un $1/k!$ devant l'expression. Ainsi donc, 
$$
\mbox{\fbox{$
\omega = {1\over k!} \quad
\omega_{\mu_1\mu_2\dots\mu_k}
\quad
\theta^{\mu_1} \land \theta^{\mu_2} \land \ldots \land
\theta^{\mu_k}
$}} $$
Notons que la premi\`ere \'ecriture contient $n^k$ termes (et il y a
unicit\'e de la d\'ecomposition), la seconde contient ${n! \over
k! (n-k)!}$ termes (et il y a unicit\'e de la
d\'ecomposition), la troisi\`eme contient $n^k$ termes (mais il n'y a pas
unicit\'e de la d\'ecomposition). 
Il est quelquefois utile, pour all\'eger les notations, d'introduire
des multi-indices $M = ({\mu_1\mu_2\dots\mu_k})$. Alors, les deux
 d\'ecompositions pr\'ec\'edentes s'\'ecrivent
$$ \omega = \omega_{|M|} \, \theta^M = {\omega_M \over k!} \theta^M $$

\end{description}

\subsection{ Produit int\'erieur d'une forme par un vecteur}\index{produit 
int\'erieur}
\par
Soit $E$ un espace vectoriel et $\Lambda(E^*)$ l'alg\`ebre ext\'erieure sur 
son dual.
Nous avons d\'efini pr\'ec\'edemment le produit {\em ext\'erieur}, qui est une
loi de composition interne \`a l'alg\`ebre ext\'erieure. Au contraire,
l'op\'eration que nous allons maintenant d\'efinir, le produit {\em
int\'erieur} n'est pas un produit au sens usuel du terme, en effet, il
associe, \`a la donn\'ee d'une forme ext\'erieure $\omega$ d'ordre $k$ (un
\'el\'ement de $\Lambda^k(E^*)$) et d'un vecteur $v$ (un \'el\'ement de $E$) une
autre forme diff\'erentielle, mais maintenant d'ordre $k-1$, c'est \`a
dire un \'el\'ement de $\Lambda^{k-1}(E^*)$. Cette nouvelle forme est
simplement obtenue en ``contractant'' $\omega$ et $v$, plus
pr\'ecis\'ement, en \'ecrivant $$
\mbox{\fbox{$
[i_v \omega](v_1,v_2,\ldots, v_{k-1} )  = 
\omega(v,v_1,v_2,\ldots,v_{k-1}) 
$}}
$$
c'est \`a dire encore, en terme de composantes et en notant $\alpha = 
i_v \omega $, $$\alpha_{\mu_1,\mu_2,\ldots,\mu_{k-1}} =
v^\mu \; \omega_{\mu,\mu_1,\mu_2,\ldots,\mu_{k-1}}$$
 Cette op\'eration est quelquefois not\'ee $v {\lfloor} \omega $ au lieu de
$i_v \omega $.\par
Il r\'esulte de l'antisym\'etrie des formes ext\'erieures que deux
op\'erations $i_v$ et $i_w$ anticommutent, en particulier, le carr\'e de
l'op\'eration $i_v$ est nul:$
i_v i_w \omega = - i_w i_v \omega $
et
$i_v i_v \omega = 0$, ce qu'on \'ecrit simplement $$
\mbox{\fbox{$
i_v i_w = - i_w i_v  \quad
\hbox {et} \quad
i_v i_v = 0 $}} $$
Le produit int\'erieur est
une antid\'erivation de l'alg\`ebre ext\'erieure, c'est \`a dire que pour
$\omega_1 \in \Lambda^{k_1}(E^*)$ et $\omega_2 \in \Lambda^{k_2}(E^*)$,
nous avons un analogue $\ZZ_2$-gradu\'e de la r\`egle de Leibniz $$
\mbox{\fbox{$
i_v(\omega_1\land\omega_2) = 
i_v(\omega_1)\land\omega_2 + (-1)^{k_1}\omega_1\land i_v(\omega_2)
$}}$$
En particulier, si $v = e_{\mu_i}$ est un vecteur de base et si
$\omega$ est \'egal au produit ext\'erieur d'un certain nombre de
vecteurs de la base duale, l'expression pr\'ec\'edente donne simplement:
\begin{eqnarray*}
i(e_{\mu_i}) \, \sigma^{\mu_1} \land \ldots
 \land \sigma^{\mu_j} \land \dots \land \sigma^{\mu_k} & = & 
(-1)^{j - 1} \sigma^{\mu_1}\land\ldots\land
\widehat{\sigma^{\mu_j}}\land\ldots \sigma^{\mu_k} \quad \hbox{si}
\quad i = j \\   {} & = & 0 \quad \hbox{si} \quad i \not \in
\{\mu_1,\mu_2,\ldots,\mu_k\}
\end{eqnarray*}
 o\`u le symbole $\; \widehat{} \;$  d\'esigne l'omission du symbole au dessus duquel
il est situ\'e.
\smallskip

\subsection{ Transformation du produit ext\'erieur et du produit int\'erieur 
par
endomorphismes}
\par

%\def\fsubtilde{\mathop{\oalign{f\cr\hidewidth \tilde {} \hidewidth\cr}}}

%\def\fsubtilde{{\oalign{f\cr\hidewidth \tilde {} \hidewidth\cr}}}

Soit $f_\sim$ un endomorphisme de l'espace vectoriel $E$ et soit
$f^\sim $ l'endomorphisme dual (aussi appel\'e transpos\'e). Rappelons ce
que cela signifie: $f_\sim$ est une application lin\'eaire de $E$ dans $E$ et
$f^\sim $ est une application lin\'eaire du dual $E^*$ dans lui-m\^eme
 d\'efinie comme suit: soient $v \in E$ et $\theta \in E^*$, alors $
f^\sim (\theta) (v)  =  \theta (f_\sim  (v))$, c'est \`a dire encore
$f^\sim (\theta) = \theta o f_\sim $.  On peut alors v\'erifier ais\'ement
que
$$
f^\sim (\omega \land \eta) = f^\sim (\omega) \land f^\sim (\eta) 
$$ 
et que
$$
i_{f_\sim (v)}(\omega) = i_v f^\sim (\omega)
$$
Remarque sur les notations: celle utilisant $*$ et non ${\sim {}}$ est 
beaucoup
plus utilis\'ee; cela dit, le symbole $*$ est ``trop'' utilis\'e et
peut quelquefois pr\^eter \`a confusion puisqu'il peut aussi bien d\'esigner
la conjugaison complexe, la dualit\'e de Hodge, qu'une involution
quelconque... Par ailleurs le ``tilde'' est sous-employ\'e (il d\'esigne
traditionnellement l'expression matricielle de $f^\sim $), il en va de 
m\^eme de la ``fl\`eche''.
Dans le cadre de cet ouvrage, nous \'ecrirons indiff\'eremment
 $f_*=f_\sim=\vr f$ et $f^*=f^\sim=\vl f$.
Notons pour finir que la notation $f_\sim$ est en g\'en\'eral inutile  dans le 
cas des espaces
vectoriels puisqu'on peut \'ecrire tout simplement $f = f_\sim $, mais dans 
le cas des
vari\'et\'es, nous verrons que $f \neq f_\sim \neq f^\sim$!
\smallskip
%%%\subsection{ Alg\`ebre sym\'etrique d'un espace vectoriel. Alg\`ebre de Weyl}
%%%\subsection{ Op\'erateurs (pseudo) diff\'erentiels}
\bigskip


\section{Formes diff\'erentielles}

\subsection{D\'efinition}

Nous avons d\'ej\`a d\'efini la notion de vecteur au point 
$P$ d'une vari\'et\'e diff\'e\-ren\-tia\-ble 
$M$ ainsi que la notion de champ de vecteurs. L'ensemble des vecteurs
au point 
$P$ se notant 
$T_P M$, l'ensemble de tous les vecteurs (le fibr\'e tangent) se notant
$T M$ et l'ensemble des champs de vecteurs se notant
$\Gamma T M$, on obtient, par dualit\'e, les notions qui suivent.
Tout d'abord l'espace vectoriel dual de 
$T_P M$ se note
$T_P^* M$~; ses \'el\'ements sont donc des formes ext\'erieures de
degr\'e~1, ou plus simplement, des ``1-formes''~. L'ensemble 
$T^* M$, baptis\'e fibr\'e cotangent, est l'ensemble de toutes les 
1-formes, lorsque le point
$P$ d\'ecrit 
$M$, c'est-\`a-dire 
$T^* M ={\cup}_{p \in M} T_P^* M$.
%$T^* M = \build{\cup}_{p \in M}^{} T_P^* M$.  % OK for latex but not for html

Une forme diff\'erentielle (en degr\'e un) est tout simplement un champ de
formes ext\'erieures, c'est-\`a-dire une application qui \`a tout
point 
$P \in M$ associe une forme ext\'erieure en ce point. Nous verrons
un peu plus loin la raison d'\^etre de cette terminologie.
L'ensemble des formes diff\'erentielles de degr\'e $1$ peut se noter
$\Gamma T^* M$ ou $\Omega^1M$.

Toutes les constructions alg\'ebriques du paragraphe pr\'ec\'edent
(tenseurs et formes ext\'erieures sur un espace vectoriel) sont en
particulier valables ici puisqu'on peut choisir comme espace
vectoriel, l'espace vectoriel tangent au point
$P$, c'est-\`a-dire 
$T_P M$. Les tenseurs 
$p$ fois contravariants, 
$q$ fois cova\-riants au point
$P$ sont donc des \'el\'ements de 
$(T_P M)^{\otimes p} \otimes (T_P^* M)^{\otimes q}$. Si on
consid\`ere tous les tenseurs de ce type (c'est-\`a-dire qu'on
effectue la r\'eunion de ces espaces lorsque 
$P$ d\'ecrit 
$M$) on obtient
$(T M)^{\otimes p} \otimes (T^* M)^{\otimes q}$ et on peut bien
entendu consid\'erer des champs de tenseurs de ce type, dont
l'ensemble constitue
$\Gamma 
\Bigl(
(T M)^{\otimes p} \otimes (T^* M)^{\otimes q}
\Bigr)
$.

Le cas particulier des tenseurs compl\`etement antisym\'etriques est
parti\-cu\-li\`erement int\'eres\-sant. On notera
$\Lambda (T_P M)^* = \bigoplus_k \Lambda^k(T_P M)^*$ l'alg\`ebre ext\'erieure 
sur le dual de l'espace vectoriel
$T_P M$ et $\Lambda (T^* M) = \cup_{p \in M} \Lambda(T_P M)^*$.

Les {\sl formes diff\'erentielles\/} \index{formes diff\'erentielles} de degr\'e 
$q$ sont des ``sections'' de 
$\Lambda^q (T^* M)$ c'est-\`a-dire des champs de formes ext\'erieures de
degr\'e 
$q$. Leur ensemble peut se noter, bien entendu, 
$\Gamma \Lambda^q (T^* M)$. Lorsque 
$q = 1$, on a 
$\Lambda^1 (T^* M) = T^* M$. Pour all\'eger la notation, on d\'ecide
de poser
$\Omega^q (M)  =  \Gamma \Lambda^q (T^*M)$. On sait que 
$q$ ne peut pas \^etre trop grand~; plus pr\'ecis\'ement
$0 \le q \le n$ avec
$n = \dim M$. Attention, ne pas confondre la dimensionalit\'e de 
$\Lambda (T_P M)^*$ -- qui est 
$2^n$ -- et celle de 
%$\Omega M  =  \build{\oplus}_{q = 0}^n \Omega^q M$, 
$\Omega M  =  {\oplus}_{q = 0}^n \Omega^q M$,
qui est infinie. Notons que les \'el\'ements de 
$\Omega^\circ M$ sont simplement les fonctions sur
$M$ c'est-\`a-dire 
$\Omega^0 M = C^\infty (M)$. Nous avons d\'ej\`a \'etudi\'e les
propri\'et\'es du produit ext\'erieur et il n'y a rien \`a rajouter
ici~: le produit ext\'erieur
$\alpha \land \beta$ de deux formes diff\'erentielles 
$\alpha$ et 
$\beta$ est obtenu en ``globalisant'' la d\'efinition d\'ej\`a connue
pour chaque point 
$P$ de 
$M$. 


$\Omega M$, munie des op\'erations de multiplication par un
scalaire, d'addition et de produit ext\'erieur, devient ainsi une
alg\`ebre. Cette alg\`ebre n'est pas commutative mais elle est
commutative gradu\'ee puisque
$\alpha \land \beta = (- 1)^{\#\alpha \#\beta}
\beta \land \alpha$ o\`u $\#\alpha$ d\'esigne le degr\'e de $\alpha$.
On appelle cette alg\`ebre {\sl alg\`ebre de De Rham}\index{alg\`ebre de De 
Rham} des formes 
diff\'erentielles.

Pour ce qui est de l'\'ecriture locale d'une forme diff\'erentielle,
il n'y a pas grand-chose \`a rajouter non plus puisque nous savons
d\'ej\`a d\'ecomposer une forme ext\'erieure sur une base de
l'espace vectoriel
$T_P^* M$. Le seul probl\`eme qui se pose est de savoir comment la
base en question varie avec le point
$P$.

Soient
$x^\mu (P)$ les coordonn\'ees de 
$P$ dans une carte locale. On sait que l'ensemble des vecteurs
$e_\mu = {\partial \over \partial x^\mu}$ fournit le rep\`ere
naturel associ\'e \`a cette carte, c'est-\`a-dire que 
$\{ e_\mu \}$ est une base de l'espace tangent en tout point d'un
voisinage de
$P$.

On d\'esignera par
$\{ d x^\mu \}$ la base duale correspondante et on \'ecrira avec des
indices ``en haut'' 
$\{ e^\mu  =  d x^\mu \}$. On peut, si on veut, ``visualiser'' 
$d x^\mu$ par ``un petit accroissement'' , mais ceci pr\'esente un
int\'er\^et purement psychologique~; en effet
$d x^\mu$ est {\it d\'efini\/} par dualit\'e et donc par la relation
$\langle d x^\mu,\ {\partial \over \partial x^\nu} \rangle
= \delta_\nu^\mu$. De la m\^eme fa\c con qu'on avait un rep\`ere
naturel
$\left( {\partial \over \partial x^\mu} \right)$ associ\'e aux
coordonn\'ees 
$x^\mu$, on a donc aussi un {\sl corep\`ere naturel \/}\index{corep\`ere 
naturel}
$\{ d x^\mu \}$.

Dans le cas de l'espace tangent, nous avons d\'efini la notion
de rep\`ere mobile
$\{ e_\alpha \}$ (qui \'etait issu de
${\partial \over \partial x^\mu}$ par changement de base
arbitraire), nous aurons donc aussi un {\sl corep\`ere 
mobile\/}\index{corep\`ere mobile}
$\{ e^\alpha \}$ d\'efini, en chaque point
$P$ de la carte, comme la base duale de
$\{ e_\alpha \}$, c'est-\`a-dire
$\langle e^\alpha,\ e_\beta \rangle
= \delta_\beta^\alpha$.

Venons-en maintenant \`a la notion de diff\'erentielle proprement
dite. Pour ce qui est des fonctions (0-formes), on pose bien entendu
$$
\mbox{\fbox{$
d f  =  {\partial f \over \partial x^\mu} d x^\mu
$}}
$$

\medskip

La $1$-forme $df$ peut \^etre \'evalu\'ee sur le champ de vecteurs 
$v=v^\mu {\partial \over \partial x^\mu}$.
On obtient
$$
\langle df, v \rangle = \langle {\partial f \over \partial x^\mu} d x^\mu,
v=v^\nu {\partial \over \partial x^\nu} \rangle =  {\partial f \over 
\partial x^\mu} v^\nu
\langle dx^\mu,  {\partial \over \partial x^\nu} \rangle =
 {\partial f \over \partial x^\mu} v^\nu
\delta^\mu_\nu = v^\mu {\partial f \over \partial x^\mu}
$$
On voit que le membre de droite n'est autre que $v[f]$. Ainsi donc,
$$
\mbox{\fbox{$
\langle df, v \rangle = v[f] 
$}}
$$
On notera souvent $df(v)$ au lieu de $\langle df, v\rangle$, 
l'\'evaluation de
la forme $df$ sur le vecteur $v$.

La r\`egle de Leibniz usuelle pour la diff\'erentielle d'un produit de 
deux fonctions, \`a savoir
$$
\mbox{\fbox{$
d(fg)=df \, g + f \, dg
$}}
$$
r\'esulte imm\'ediatement de la propri\'et\'e, pour les champs de vecteurs, 
d'\^etre des d\'erivations de l'alg\`ebre des fonctions.

Nous allons g\'en\'eraliser aussi bien la d\'efinition de $d$ que la 
r\`egle de Leibniz \`a des formes diff\'erentielles de degr\'e sup\'erieur.


\subsection{La diff\'erentielle ext\'erieure $d$}\label{sec:diffext}
\index{diff\'erentielle ext\'erieure}\index{diff\'erentielle de De Rham}
Soit
$\omega$ une 
$k$-forme diff\'erentielle~; on va d\'efinir un op\'erateur 
$d$ qui, appliqu\'e \`a
$\omega$, cr\'ee une 
$(k + 1)$-forme. Cet op\'erateur est d\'esign\'e sous le nom de
 {\sl diff\'erentielle ext\'erieure\/} ou  {\sl diff\'erentielle de De Rham\/}.


\begin{all}{\bigskip}{D\'efinition}{\it} La forme diff\'erentielle 
$d \omega$ peut se d\'efinir directement par son action sur tout
$(k+1)$-uplet
$\{ v_1, v_2, \dots, v_{k + 1} \}$ de champs de vecteurs, en posant 
\begin{eqnarray*}
 d \omega (v_1, v_2, \dots, v_{k + 1}) 
& = & \sum_{i = 1}^{k + 1} (- 1)^{i + 1} v_i
\Bigl[
\omega (v_1, \dots, \widehat v_i, \dots, v_{k + 1})
\Bigr]                                               \nonumber \\
& & + 
\sum_{i \le i \le j \le k+1} (- 1)^{i + j} \omega
\Bigl(
[v_i, v_j], v_1, \dots, 
\widehat v_i, \dots, 
\widehat v_j, \dots, v_{k + 1}
\Bigr) 
\end{eqnarray*}
o\`u le symbole
 $\widehat{}$ d\'esigne l'omission de l'argument correspondant.
\end{all}

\bigskip

Cette d\'efinition poss\`ede un int\'er\^et pratique
certain. Pour se rappeler des signes, on peut signaler le moyen
mn\'emotechnique suivant~: le premier type de termes s'obtient en
faisant passer les vecteurs
$v_i$ devant 
$\omega$ et en comptant un signe 
``$-$'' chaque fois que 
$v_i$ ``traverse'' un des autres vecteurs~; le second type de terme
s'obtient en choisissant une paire
$v_i, v_j$ et en la faisant passer en position 1 et 2 de la forme
$\omega$, tout en utilisant l'antisym\'etrie de 
$\omega$ lorsqu'on effectue des transpositions. On remplace alors la
paire 
$(v_i, v_j)$ par son crochet
$[v_i, v_j]$ et on multiplie le tout par un signe 
$- 1$.

\medskip

\noindent Exemple 1~: Soit $f$ une $0$-forme, c'est \`a dire une fonction
sur $M$. La d\'efinition ci-dessus conduit \`a
$$ df(v) = v[f] $$
ce qu'on savait d\'ej\`a.

\noindent Exemple 2~: Soit 
$\omega$, une 1-forme, alors 
$$
d \omega (u, v) 
= u (\omega (v)) - v (\omega (u)) - \omega ([u, v])
$$
Si $u = {\partial \over \partial x^\mu}$ et  $v= {\partial \over \partial 
x^\nu}$, on trouve simplement que les composantes de $F =  d\omega$ sont
donn\'ees par
 $F_{\mu\nu}=\partial_\mu \omega_\nu - \partial_\nu \omega_\mu$.

Le lecteur aura reconnu,
dans le
cas de la dimension $4$,
 l'expression du champ \'electromagn\'etique (le
tenseur $F$ ) en terme du (quadri) potentiel vecteur $\omega$.
Soit dit en passant, il faut incorporer le troisi\`eme terme
(l'\'evaluation de $\omega$ sur
le commutateur $ [u,v]$) lorsqu'on veut exprimer le champ $F=d\omega$ dans 
un
rep\`ere quelconque.

\medskip

\noindent Exemple 3~: Soit 
$\omega$, une 2-forme, alors

\begin{eqnarray*}
 d \omega (x, y, z)
& = & x (\omega (y, z)) - y (\omega (x, z)) + z (\omega (x, y))      \\
{} & {} & - \omega ([x, y], z) - \omega ([y, z], x) + \omega ([x, z], y)
\end{eqnarray*}


En utilisant la d\'efinition de $d$, donn\'ee ci-dessus,
on montre imm\'ediatement que, si 
$\omega_1 \in \Omega^{k_1}$ et 
$\omega_2 \in \Omega^{k_2}$, alors
$$
\mbox{\fbox{$
d (\omega_1 \land \omega_2) = d \omega_1 \land \omega_2
+ (- 1)^{k_1} \omega_1 \land d \omega_2
$}}
$$

De la m\^eme fa\c con, on montre que $$\mbox{\fbox{$d^2 = 0$}}$$

Les deux propri\'et\'es ci-dessus sont absolument fondamentales et peuvent
m\^eme servir \`a d\'efinir l'op\'erateur $d$ lui m\^eme.


\begin{all}{\bigskip}{D\'efinition}{\it} 
$d$ est l'unique op\'erateur (application lin\'eaire) de 
$\Omega^k M$ dans 
$\Omega^{k + 1} M$ tel que, pour tout 
$k$,
$\omega_1 \in \Omega^{k_1}$,
$\omega_2 \in \Omega^{k_2}$,
$k=k_1 + k_2$, on ait
$d (\omega_1 \land \omega_2) = d \omega_1 \land \omega_2
+ (- 1)^{k_1} \omega_1 \land d \omega_2$ et
$d^2 = 0$.
En d'autre terme 
$d$ \'etend la d\'efinition usuelle de diff\'erentiation des
fonctions en une d\'eriva\-tion gradu\'ee de carr\'e nul de
l'alg\`ebre
$\Omega M$.
\end{all}
\index{champ electromagn\'etique\/}
En physique, si $\omega$ d\'esigne le quadri-potentiel vecteur, alors,
$F = d\omega$ ob\'eit automatiquement \`a l'\'equation $d\, F = 0$,
puisque $d^2 =0$. Ceci
nous donne donc la moiti\'e des \'equations de Maxwell (les \'equations 
sans
source).

Il existe une troisi\`eme d\'efinition possible de l'op\'erateur $d$, 
d\'efinition qui est \'egalement
d'un int\'er\^et pratique certain. La voici :


\begin{all}{\bigskip}{D\'efinition}{\it} Relativement \`a un choix
de coordonn\'ees on peut \'ecrire 
$\omega = \omega_I d x^I$, o\`u 
$I$ est un multi-indice et 
$\omega_I$ est une 0-forme, c'est-\`a-dire une fonction. On
d\'efinit d'abord 
$d$ sur les fonctions 
$d \omega_I  =  {\partial \omega_I \over \partial x^\mu} 
d x^\mu$. Ensuite, plus g\'en\'eralement, on pose 
$d \omega  =  d \omega_I \land d x^I$.
\end{all}

Nous venons de voir
trois d\'efinitions \'equivalentes
possibles de l'op\'erateur $d$. Toutes les trois sont utiles et nous 
laissons au lecteur le
soin de d\'emontrer l'\'equivalence des d\'efinitions. 

Terminons par un 
petit calcul \'el\'ementaire (clin d'\oe{il} au cours d'\'electromagn\'etisme). \index{champ electromagn\'etique\/}
Soit
$A = A_\mu dx^\mu$ une $1$-forme (le quadri-potentiel vecteur). Le champ 
de Maxwell est d\'efini par $$
\mbox{\fbox{$
F=dA=
{1\over 2}F_{\mu\nu} dx^\mu\land dx^\nu = F_{\mu\nu}dx^\mu\otimes dx^\nu
$}}
$$
 Or 
 \begin{eqnarray*}
 dA & = &
d( A_\mu)\land dx^\mu =  {\partial A_\mu \over  \partial x^\nu}
dx^\nu \land dx^\mu = {1\over 2}(
 {\partial A_\mu \over  \partial x^\nu}
dx^\nu \land dx^\mu + 
 {\partial A_\mu \over  \partial x^\nu}
dx^\nu \land dx^\mu) \\
{} & = & {1\over 2}(
 {\partial A_\nu \over  \partial x^\mu}
-
 {\partial A_\mu \over  \partial x^\nu})
dx^\mu \land dx^\nu
 \end{eqnarray*}
 
 
Ainsi $$
\mbox{\fbox{$
F_{\mu\nu} =  
 {\partial A_\nu \over  \partial x^\mu}
-
 {\partial A_\mu \over  \partial x^\nu}$}}$$

\subsection{L'\'equation de Maurer-Cartan pour un rep\`ere mobile}
\label{sec:mobileMaurerCartan}
Soit
$\{ e_\alpha \}$ un rep\`ere mobile et 
$f_{\beta \gamma}^\alpha$ les fonctions de structure correspondantes,
c'est-\`a-dire que ce rep\`ere v\'erifie l'\'equation de structure~:
$[e_\beta, e_\gamma] = f^\alpha_{\beta \gamma} e_\alpha$.

Soit
$\{ e^\alpha \}$ le co-rep\`ere mobile correspondant d\'efini, comme
on l'a vu, par dualit\'e. Le co-rep\`ere v\'erifie \'egalement une
\'equation de structure (souvent d\'esign\'ee sous le nom
d'{\sl \'equation de Maurer-Cartan\/})\index{equation de Maurer-Cartan}
$$
\mbox{\fbox{$
d e^\alpha + {1 \over 2} f_{\beta \gamma}^\alpha 
e^\beta \land e^\gamma = 0
$}}
$$ La fa\c con la plus simple de
d\'emontrer cette identit\'e est de la v\'erifier en l'\'evaluant
sur un couple 
$(e_\delta, e_\epsilon)$ de vecteurs du rep\`ere mobile. D'une part, en 
effet,
\begin{eqnarray*}
d e^\alpha (e_\delta, e_\epsilon) 
& = & e_\delta (e^\alpha (e_\epsilon)) 
- e_\epsilon (e^\alpha (e_\delta))
- e^\alpha ([e_\delta, e_\epsilon])                \nonumber \\
& = & e_\delta (\delta_\epsilon^\alpha)  
- e_\epsilon (\delta_\delta^\alpha)
- f_{\delta \epsilon}^\gamma e^\alpha (e_\gamma)   \nonumber \\
& = & 0 - 0
- f_{\delta \epsilon}^\gamma \delta_\gamma^\alpha 
= - f_{\delta \epsilon}^\alpha
\end{eqnarray*}
Les deux premiers termes sont nuls puisqu'on
 d\'erive des
constantes~!

D'autre part
\begin{eqnarray*}
{1 \over 2} f_{\beta \gamma}^\alpha 
e^\beta \land e^\gamma (e_\delta, e_\epsilon)
& = & {1 \over 2} f_{\beta \gamma}^\alpha
(e^\beta \otimes e^\gamma - e^\gamma \otimes e^\beta)
(e_\delta, e_\epsilon)
= {1 \over 2} f_{\beta \gamma}^\alpha
\left(
\delta_\delta^\beta \delta_\epsilon^\gamma
- \delta_\delta^\gamma \delta_\epsilon^\beta
\right)                                           \nonumber \\
& = & {1 \over 2} f_{\delta \epsilon}^\alpha
- {1 \over 2} f_{\epsilon \delta}^\alpha
= {1 \over 2} f_{\delta \epsilon}^\alpha
+ {1 \over 2} f_{\delta \epsilon}^\alpha
= f_{\delta \epsilon}^\alpha
\end{eqnarray*}
D'o\`u le r\'esultat.

\subsection{Produit int\'erieur d'une forme par un champ ou vecteurs}

 Cette
op\'eration g\'en\'eralise celle \'etudi\'ee pr\'ec\'edemment (produit int\'erieur d'une 
forme ext\'erieure par un vecteur). On associe, \`a une 
$k$ forme 
$\omega$ et un vecteur 
$v$ une 
$k - 1$ forme not\'ee 
$i_v \omega$. La d\'efinition en est tr\`es simple~: pour une 1-forme,
c'est tout simplement l'\'evaluation. C'est-\`a-dire
$i_v \omega = \omega (v) = \langle \omega, v \rangle$. Pour une 
$k$-forme, on g\'en\'eralise simplement en contractant l'indice du
vecteur
$v$ avec le premier indice de la forme 
$\omega$~; en d'autres termes (et sans utiliser d'indices) 
$i_v \omega$ est la 
$k - 1$ forme d\'efinie par
$$
\mbox{\fbox{$
i_v \omega (v_1, v_2, \dots, v_{k - 1})
 =  \omega (v, v_1, v_2, \dots, v_{k - 1})
$}}
$$
Si
$\alpha = i_v \omega$ on a \'evidemment
$\alpha_{\mu_1, \mu_2, \dots, \mu_{k - 1}} 
= v^\nu \omega_{\nu \mu_1, \mu_2, \dots, \mu_{k - 1}}$. Il s'agit d'une
 op\'eration tr\`es \'el\'ementaire
g\'en\'eralisant l'\'evaluation d'une forme sur un vecteur. L'op\'eration 
$i_v$, de 
$\Omega^k M$ dans 
$\Omega^{k - 1} M$ est, comme l'op\'eration
$d$, une d\'erivation gradu\'ee et de carr\'e nul de l'alg\`ebre
ext\'erieure.
La propri\'et\'e d'\^etre de carr\'e nul est une cons\'equence
imm\'ediate du fait que les formes diff\'erentielles sont des objets
antisym\'etriques et donc s'annulent d\`es que deux arguments sont
\'egaux. La propri\'et\'e d'anti-d\'erivation, c'est-\`a-dire 
$$
\mbox{\fbox{$
i_v (\omega_1 \land \omega_2) 
= (i_v \omega_1) \land \omega_2 
+ (- 1)^k \omega_1 \land (i_v \omega_2)
$}}
$$ o\`u 
$k$ est le degr\'e de 
$\omega_1$ est \'egalement une cons\'equence imm\'ediate de la
d\'efinition. La diff\'erence essentielle entre l'op\'eration 
$d$ et l'op\'eration
$i_v$ est que ces deux op\'erations vont dans des sens diff\'erents 
($i_v$ fait baisser le degr\'e d'une unit\'e alors que
$d$ l'\'el\`eve). Notons que la d\'efinition de 
$i_v$ d\'epend du choix du vecteur
$v$.


\section{Application tangente et cotangente}

Soit
$f$ un diff\'eomorphisme de la vari\'et\'e
$M$, ou, plus g\'en\'eralement, une application diff\'erentiable de
$M$ (de dimension
$m$) dans
$N$ (de dimension
$n$). En coordonn\'ees locales, 
$f$ s'\'ecrit \`a l'aide de 
$n$ fonctions
$f^\alpha$ de 
$m$ variables
$y^\alpha = f^\alpha (x^\mu)$. La matrice jacobienne de cette
application est la matrice
$(n, m)$ des \'el\'ements
${\partial y^\alpha /\partial x^\mu}$. Une telle matrice
d\'efinit une application lin\'eaire de l'espace vectoriel tangent
\`a
$M$ au point
$P$ dans l'espace tangent \`a
$N$ au point
$f (P).$
Soit
$\{ \partial_\mu \}$ un rep\`ere naturel de 
$M$ d\'efini dans un voisinage de 
$P$ et
$\{ \partial_\alpha \}$ un rep\`ere naturel de 
$N$ d\'efini dans un voisinage de 
$f (P)$. Soit 
$v \in T_P (M)$, on peut \'ecrire
$v = v^\mu \partial_\mu$. On obtient un vecteur 
$w \in T_{f (P)} (N)$ en \'ecrivant
$w = w^\alpha \partial_\alpha$ avec
$$
\mbox{\fbox{$
(w^\alpha)  =  
\left( {\partial y^\alpha \over \partial x^\mu} \right) 
(v^\mu)
$}}
$$


Cette application, dite application lin\'eaire tangente (ou ``push 
forward'') se note, suivant les auteurs
$f_*$, $Tf$,
$f_\sim$, ou m\^eme
$\vr f$ et on dit que
$w = \vr f (v)$ est l'{\sl image directe \/}\index{image directe} de
$v$. 
On peut bien entendu d\'efinir directement
$\vr f$ sans utiliser de syst\`emes coordonn\'es. De fa\c con 
g\'en\'erale, � toute application diff\'erentiable
$f : M \to N$, on associe une application lin\'eaire tangente 
$\vr f : T M \to T N$, et si  $v \in T_{P} M$, alors 
$\vr f [v] \in T_{f(P)} M$.

Remarque: on peut toujours prendre l'image d'un vecteur tangent par 
l'application tangente, mais l'image d'un champ de vecteurs 
$v$  sur $M$ ne 
d\'efinit pas n\'ecessairement un champ de vecteur sur $N$;
d'une part, en effet, rien ne prouve qu'un point $Q$ quelconque de $N$ soit 
n\'ecessairement dans l'image de $f$, et par ailleurs, m\^eme si $f$ 
est surjective, rien ne dit, dans le cas o\`u deux points distincts $P_{1}$ et 
$P_{2}$ seraient tels que $f(P_{1}) = f(P_{2})$ que l'image par $\vr 
f$ du vecteur $v(P_{1})$ co\"incide avec l'image par $\vr 
f$ du vecteur $v(P_{2})$. En fait, pour une application 
differentiable surjective $f: M \to N$ donn\'ee, 
il est commode d'introduire la notion de 
{\sl champ de vecteurs projetable} \index{champ de vecteurs projetable}: $v \in \Gamma(TM)$ est dit 
projetable (par $f$) si, pour tout $Q \in N$ et pour toute paire 
$(P_{1},P_{2})$ de points de $M$ tels que $Q=f(P_{1}) = f(P_{2})$ on 
ait $\vr f [v(P_{1})] = \vr f [v(P_{2})]$; dans ce cas on obtient 
bien un champ de vecteur sur $N$. 


La m\^eme matrice jacobienne
${(\partial y^\alpha / \partial x^\mu)}$ d\'efinit \'egalement une
application lin\'eaire de l'espace cotangent \`a
$N$ au point
$f (P)$ dans l'espace cotangent \`a
$M$ au point
$P$. En effet, soit
$\tau \in T_{f (P)}^* N$, alors
$\tau = \tau_\alpha d y^\alpha$. L'image de la forme
$\tau$ est la forme
$\sigma \in T_P^* M$, avec
$\sigma = \sigma_\mu d x^\mu$ et 
$$
\mbox{\fbox{$
(\sigma_\mu)  =  
\left( {\partial y^\alpha \over \partial x^\mu} \right) 
(\tau_\alpha)
$}}
$$


Cette application, qu'on pourrait appeler {\sl  application
lin\'eaire cotangente \/}\index{application cotangente}, (ou ``pull 
back'') et noter $f^*$, $T^{*}f$, $f^\sim$, ou m\^eme $\vl f$
n'est donc autre  que la transpos\'ee de l'application lin\'eaire tangente $\vr f$ {\sl au 
point $P \in M$}: elle envoie les co-vecteurs de 
$N$ au point $f(P)$ (i.e. les 1-formes de
$N$ au point $f(P)$ ) dans les co-vecteurs de $M$ au point $P$.
Si $\tau \in T_{f(P)}^* N$, alors $\vl f (\tau) = \tau \circ \vr f \in T_{P}^* M$.
Cette application de $T_{f(P)}^* N$ dans $T_{P}^* M$ ne peut 
manifestement pas, en g\'en\'eral, se g\'en\'eraliser \`a une 
application de $T^{*}N$ dans $T^{*}M$; la situation n'est donc pas 
tout \`a fait analogue \`a celle de l'application tangente, qui, elle, 
est bien d\'efinie, comme application de $TM$ dans $TN$.


Par contre, si $\omega$ est une $1$-forme differentielle sur $N$, c'est 
\`a dire {\sl un champ} de co-vecteurs, on peut toujours consid\'erer
son image par $\vl f$;  en effet, dans ce cas, si $v$ est un vecteur 
quelconque en  $P\in M$, alors $\vr f (P)$ est un vecteur en $Q=f(P) \in N$ et 
le nombre $\omega_{Q}[\vr f (P)]$ est bien d\'efini. On obtient ainsi 
une $1$-forme differentielle sur $M$ qu'on notera $f^{*} \omega$ ou 
$\vl f (\omega)$. On l'appelle en g\'en\'eral ``pull back'' de 
$\omega$ par $f$.

Quelques remarques sur les notations :
on peut trouver commode d'utiliser de nouveau le
symbole
$d f$ et d'\'ecrire tout simplement (en un point $P$ donn\'e, 
non explicitement indiqu\'e par la notation)
$$
\mbox{\fbox{$
d f =
\left( {\partial f^\alpha \over \partial x^\mu} \right) 
{\partial \over \partial y^\alpha} \otimes d x^\mu
$}}
$$ et donc de
consid\'erer 
$d f$ comme un \'el\'ement de
$T_{f (P)} N \otimes T_P^* M$.
La matrice $(\partial f^\alpha / \partial x^\mu)$ est la matrice jacobienne
de l'application $f$.
Attention, $df$, ici, n'est pas une $1$-forme puisque $f$ n'est pas une 
fonction \`a valeurs dans le corps des scalaires mais une application entre 
deux vari\'et\'es. Il ne semble pas n\'ecessairement utile
de vouloir \`a tout crin introduire de nouvelles notations chaque fois qu'une
fonction de plusieurs variables donne naissance \`a des applications 
diff\'erentes lorsqu'on d\'ecide de geler l'un ou l'autre de ses arguments~! La
notation ``diff\'erentielle''  pr\'ec\'edente est une g\'en\'eralisation directe
de la notation d\'esignant la diff\'erentielle d'une fonction \`a valeurs 
r\'eelles. Ici, $d f$ doit
\^etre consid\'er\'ee comme une application bilin\'eaire
qu'on peut noter $(.,df,.)$
dont l' une des restrictions co\"incide avec $\vr f$ et l'autre
avec $\vl f$.
Si on choisit $\tau \in T_{f(P)}^*N$ et $v \in T_{P}M$, on voit que
$$ \sigma = \vl f (\tau) = (\tau,df,\cdot) \in T_{P}^*M 
\qquad \text{et} \qquad
 w = \vr f (v) = (\cdot, df, v) \in T_{f(P)}N $$

La notation suivante est \'egalement tr\`es commode : 
$$
\langle \sigma \vert  = \langle \tau \vert df 
\qquad \text{et} \qquad
\vert w \rangle = df \vert v \rangle
$$ 

Si on choisit un rep\`ere mobile
$e_{\widehat \alpha} 
= \Lambda_{\widehat \alpha}^\beta \, \partial_\beta$ dans
$N$ et un co-rep\`ere mobile
$e^{\widehat \mu} = L^{\widehat \mu}_\nu dx^\nu$ dans 
$M$, on pourra \'ecrire \'egalement
$$
d f
= (\Lambda^{-1})_\alpha^{\widehat \beta} \,  
{\partial f^\alpha \over \partial x^\mu} \, 
 (L^{-1})_{\widehat \nu}^\mu \qquad 
e_{\widehat \beta} \otimes e^{\widehat \nu} 
$$
et consid\'erer la quantit\'e
$f_{\widehat \nu}^{\widehat \beta}  = 
(\Lambda^{-1})_\alpha^{\widehat \beta} 
\,({\partial f^\alpha / \partial x^\mu}) \,
(L^{-1})_{\widehat \nu}^\mu$ comme les \'el\'ements de la matrice
jacobienne de
$f$ par rapport au choix de deux rep\`eres mobiles.

Nous venons de voir que les 1-formes de
$N$ peuvent \^etre ``rappel\'ees'' sur 
$M$ \`a l'aide de
$\vl f$:
$$\vl f \omega = \omega \vr f$$
Il en va de m\^eme des 
$p$-formes et on d\'efinit, pour
$\omega \in \Omega^p N$, la 
$p$-forme
$\vl f (\omega) \in \Omega^p M$ par l'\'egalit\'e
$$\vl f \omega (v_1, v_2, \dots, v_p)  = 
\omega (\vr f v_1, \vr f v_2, \dots, \vr f v_p)$$ avec
$v_1, v_2, \dots, v_p \in T M$.

Nous laissons au lecteur le soin de d\'emontrer les propri\'et\'es
suivantes~:
\begin{eqnarray*}
\vl f (\omega \land \eta)
& = & \vl f (\omega) \land \vl f (n)              \nonumber \\[4mm]
i_v \vl f (\omega)
& = & \vl f \biggl( i_{\vr f (v)} \omega \biggr)  \nonumber \\
\vl f d \omega
& = & d \vl f \omega
\end{eqnarray*}


\section{D\'eriv\'ees de Lie}

La notion usuelle de d\'eriv\'ee d'une fonction num\'erique 
$f$ nous permet de pr\'eciser la notion de variation locale de cette
fonction lorsque son argument cro\^{\i}t ou d\'ecro\^{\i}t. Lorsque
l'argument se d\'eplace sur une vari\'et\'e de dimension
sup\'erieure \`a~1, la variation ne sera d\'efinie que si on
pr\'ecise dans quelle direction se d\'eplace le point (l'argument).
En d'autres termes, la g\'en\'eralisation de la notion de
d\'eriv\'ee invoque obligatoirement la notion de vecteur tangent. La
d\'eriv\'ee d'une fonction
$f : M \to \RR$ par rapport \`a un champ de vecteurs
$X$ se note
$L_X f$ et est tout simplement d\'efinie par l'\'egalit\'e
$$
L_X f  =  d f [X] \quad = X [f]
$$
Ainsi, la diff\'erentielle de
$f$ ``code'' toutes les variations possibles, alors que la
d\'eriv\'ee de 
$f$ dans la direction 
$X$ est obtenue en \'evaluant la diff\'erentielle de 
$f$ sur le vecteur
$X$.

Cette notion de d\'eriv\'ee se g\'en\'eralise au cas o\`u 
$f$ n'est plus une fonction sur 
$M$ \`a valeurs r\'eelles mais un champ $t$ de tenseurs quelconques (ou
m\^eme, comme on le verra plus tard, une section d'un fibr\'e
quelconque au-dessus de
$M$). On a envie de donner un sens \`a la limite de
${t (x + \epsilon X) - t (x) \over \epsilon}$ lorsque 
$\epsilon$ tend vers 0. La quantit\'e correspondante se note toujours
$L_X t$ et s'appelle {\sl d\'eriv\'ee de Lie \/}\index{d\'eriv\'ee de Lie} du tenseur 
$t$ par rapport au champ de vecteurs 
$X$. C'est un tenseur de m\^eme type que 
$t$. On veut que 
$L_X$ soit une d\'erivation de l'alg\`ebre tensorielle,
c'est-\`a-dire qu'on impose
 $$
 \mbox{\fbox{$
L_X (t_1 \otimes t_2) 
= L_X t_1 \otimes t_2 + t_1 \otimes L_X t_2
$}}
$$
o\`u 
$t_1$ et 
$t_2$ sont des tenseurs quelconques. Pour d\'efinir compl\`etement 
$L_X$ il suffit de pr\'eciser la valeur de
$L_X Y$ lorsque
$Y$ est un champ de vecteurs (contravariants) et de
$L_X \omega$ lorsque
$\omega$ est une 1-forme (champ de vecteurs covariants). Dans le
premier cas on pose
$$
L_X Y  =  [X, Y]
  $$
  Par exemple,
  $$L_{e_\rho} e_\mu = f^\nu_{\rho \mu} e_\nu $$
Dans le second cas (action de $L$ sur les $1$-formes), on d\'efinit
$L_X \omega$ par la relation
$$[L_X \omega] (Y) = X (\omega (Y)) - \omega ([X, Y])$$ o\`u 
$Y$ est, comme 
$X$, un champ de vecteurs sur 
$M$. 
Nous laissons le soin au lecteur de d\'emontrer que cette d\'efinition, ainsi 
que la propri\'et\'e de d\'erivation, conduit \`a la relation suivante 
caract\'erisant l'action de $L_X$ sur les formes diff\'erentielles:
$$ 
\mbox{\fbox{$
L_X = d \, i_X + i_X \, d 
$}}
$$
Cette derni\`ere propri\'et\'e peut d'ailleurs servir de d\'efinition.
Il est alors imm\'ediat de v\'erifier que 
$L_X$ est une d\'erivation de l'alg\`ebre des formes
diff\'erentielles puisque 
$d$ et 
$i_X$ sont des d\'erivations gradu\'ees. 
$L_X$ est une ``vraie'' d\'erivation et non une d\'erivation
gradu\'ee~: le signe
$(- 1)$ potentiel, dans la r\`egle de Leibniz, dispara\^{\i}t.
$$
\mbox{\fbox{$
L_X(\alpha \land \beta) = L_X(\alpha)\land\beta + \alpha \land L_X(\beta)
$}}
$$
 De
plus 
$L_X$ ne modifie pas le degr\'e des formes puisque 
$d$ et
$i_X$ agissent en sens contraires. La relation pr\'ec\'edente conduit
imm\'ediatement \`a la formule explicite
$$
L_X \omega (X_1, \dots, X_k)
= X (\omega (X_1, \dots, X_k))
- \sum_{i = 1}^k \omega (X_1, \dots, [X, X_i], \dots, X_k)
$$

\medskip


Le cas particulier o\`u 
$\omega$ est une 1-forme se retrouve aussi ais\'ement. Notons que,
dans ce dernier cas, si 
$e_\mu$ et 
$e^\mu$ d\'esignent deux rep\`eres duaux l'un de l'autre (un rep\`ere mobile et 
le co-rep\`ere mobile dual), on a
$$L_X e^\mu = e^\mu ([e_\nu, X]) e^\nu$$ et en particulier
$$
\mbox{\fbox{$
L_{e_\rho} e^\mu = f_{\nu \rho}^\mu \; e^\nu
$}}
$$
En coordonn\'ees locales, lorqu'on choisit un rep\`ere naturel, on peut \'ecrire $\omega = 1/k! \, \omega_{\mu_1\ldots\mu_k} \, dx^{\mu_1}\wedge\ldots\wedge dx^{\mu_k}$ et on obtient
$$
\mbox{\fbox{$
L_X\omega ={1\over k!} (X^\rho {\partial \over \partial x^\rho} \omega_{\mu_1\ldots\mu_k}+
\omega_{\rho\mu_2\ldots\mu_k} {\partial X^\rho \over \partial x^{\mu_1}}+
\ldots + \omega_{\mu_1\ldots\mu_{k-1}\rho} {\partial X^\rho \over \partial x^{\mu_k}}) \, dx^{\mu_1}\wedge \ldots\wedge dx^{\mu_k} 
$}}
$$
\bigskip

Dans le cas des formes, la d\'efinition de la d\'eriv\'ee de Lie implique
imm\'ediatement que
$L_X$ commute avec
$d$ (car
$d^2 = 0$), ainsi qu'avec
$i_X$ (puisque
$i_v i_v = 0$), et que, par ailleurs
\begin{eqnarray*}
i_{[X, Y]}
& = & L_X i_Y - i_Y L_X   \nonumber \\[1mm]
[L_X, L_Y]
& = & L_{[X, Y]}     \nonumber 
\end{eqnarray*}

\bigskip

Enfin, si 
$f$ est une application diff\'erentiable et 
$\omega$ une forme diff\'erentielle, on voit que
$$L_X \vl f \omega = \vl f L_{\vr f X} \omega$$

Nous terminons ce paragraphe en montrant que la d\'eriv\'ee de Lie du
tenseur de Kronecker
$\delta = e_\mu \otimes e^\mu$ est nulle dans toutes les directions.
En effet
\begin{eqnarray*}
L_{e_\rho} \delta
& = & L_{e_\rho} e_\mu \otimes e^\mu
+ e_\mu \otimes L_{e_\rho} e^\mu                 \nonumber \\[1mm]
& = & [e_\rho, e_\mu] \otimes e^\mu
+ e_\mu \otimes f_{\nu \rho }^\mu e^\nu          \nonumber \\[1mm]
& = & f_{\rho \mu}^\sigma e_\sigma \otimes e^\mu
+ f_{\nu \rho }^\mu e_\mu \otimes e^\nu 
= 0          \nonumber
\end{eqnarray*}

\section{Flots}

Un {\sl flot} sur la vari\'et\'e $M$ est un sous-groupe differentiable \`a un param\`etre $\{\phi_t\}$ de diff\'eomorphismes de $M$: on suppose que pour tout $t$ r\'eel, l'application
$t\in \RR \mapsto \phi_t \in Diff(M)$ est un homomorphisme de groupe et que l'application $(t,P) \in \RR \times M \mapsto \phi_t(P) \in  M$ est diff\'erentiable.

La {\sl trajectoire du flot} aussi appel\'ee {\sl courbe int\'egrale du flot} passant par le point $P \in M$ est la courbe $t \mapsto \phi_t(P)$.
L'application lin\'eaire tangente \`a cette courbe associe au vecteur unit\'e $1$ de $R$ (identifi\'e avec son espace tangent en $0$) un vecteur tangent en $Q=\phi_t(P) \in T(M,Q)$.
Ce vecteur tangent en $Q$ ne d\'epend que du flot. En effet,  les propri\'et\'es d'homomorphisme $\phi_{t_1+t_2} = \phi_{t_1} \circ \phi_{t_2}$, et de bijectivit\'e de $\phi_f$ , montrent que si deux trajectoires passent par le m\^eme point $Q$, c'est \`a dire si $\phi_{t}(P_1)=\phi_{t_2}(P_2)$, avec $t \geq t_2$ par exemple, on peut poser $t_1=t-t_2$ et $P_2=\phi_{t_1}(P_1)$, ce qui montre que  les vecteurs tangents en $Q$ \`a ces deux trajectoires co\"incident.
Le vecteur tangent obtenu, notons le $X(Q)$, d\'efinit un champ de vecteurs $X$ quelquefois d\'esign\'e sous le nom de {\sl champ des vitesses du flot}.

Puisque le flot choisi d\'efinit un champ des vitesses, il d\'efinit \'egalement une d\'eriv\'ee de Lie par rapport \`a ce champ de vecteurs.

On peut d\'emontrer qu'inversement, un champ de vecteurs sur une vari\'et\'e d\'efinit un {\sl flot local}, c'est \`a dire qu'il ne sera en g\'en\'eral d\'efini que sur un ouvert strictement inclus dans $\RR \times M$; si cette inclusion devient une \'egalit\'e, le champ de vecteur est dit  ``complet'' et il engendre un flot: les diffeomorphismes $\phi_t$ sont alors d\'efinis quel que soit $t$.

\section[Orientation et int\'egration]{Orientation -- El\'ement de volume \\ 
D\'eterminant -- Int\'egration}

En g\'eom\'etrie \'el\'ementaire, l'orientation d'un espace
vectoriel r\'eel est sp\'ecifi\'ee par le choix d'une base
${\cal B}$ (choix {\it ordonn\'e\/} d'un syst\`eme libre et g\'en\'erateur).
Le choix d'une autre base
${\cal B'}$ d\'etermine un isomorphisme 
$g$ qui envoie les vecteurs de
${\cal B}$ sur les vecteurs de
${\cal B'}$. On dit que 
$g$ pr\'eserve l'orientation si
$\det g > 0$ et renverse l'orientation si
$\det g < 0$. Dans le premier cas on dit que 
${\cal B}$ et 
${\cal B'}$ ont la m\^eme orientation~; dans le second cas, 
${\cal B}$ et
${\cal B'}$ ont des orientations oppos\'ees. On peut alors r\'epartir les
bases de l'espace vectoriel en question en deux classes
d'\'equivalence correspondant aux deux orientations possibles. Afin
de g\'en\'eraliser cette discussion au cadre des vari\'et\'es, il
est utile de reformuler ce qui pr\'ec\`ede en terme de formes
ext\'erieures. Nous allons donc travailler avec les bases duales et poser
$e^{\mu'} = f(e^{\mu})$.

Soit 
${\cal B}^* = \{ e^1, e^2, \dots, e^n \}$ et 
$\omega  =  e^1 \land e^2 \land \dots \land e^n$.

Soit
${\cal B'}^* = \{ e^{1'}, e^{2'}, \dots, e^{n'} \}$ avec
$e^{\mu'} = f (e^\mu)$ et 
$\omega' =  f (e^1) \land f (e^2) \land \dots \land f (e^n)$.
L'espace des formes ext\'erieures de degr\'e
$n$ sur un espace vectoriel de dimension
$n$ est un espace vectoriel de dimension 1. Les formes
$\omega$ et 
$\omega'$ sont donc proportionnelles et
 le coefficient de proportionnalit\'e n'est autre que le d\'eterminant 
de $f$ : $\omega'= (\det f) \, \omega$. Nous laissons au lecteur le soin
de retrouver la d\'efinition \'el\'ementaire des d\'eterminants en \'ecrivant
$e^{\mu'}= \Lambda^{\mu'}_\mu e^\mu$.


 L'orientation de l'espace vectoriel qui
\'etait d\'efinie par le choix de
${\cal B}$ peut tout aussi bien se d\'efinir par le choix de la 
$n$-forme
$\omega$. Deux 
$n$-formes 
$\omega$ et 
$\omega'$ (obligatoirement proportionnelles) d\'efinissent la m\^eme
orientation si le coefficient de proportionnalit\'e est positif et
deux orientations de sens contraire si le coefficient en question
est n\'egatif. Nous pouvons maintenant passer au cas des
vari\'et\'es. Nous venons de voir que l'orientation, en chaque point 
$P$ de 
$M$, de l'espace tangent
$T_P M$, est \'equivalente au (ou d\'efinie par le) choix d'une 
$n$-forme ext\'erieure en ce point. On pourrait donc na\"{\i}vement
penser que, pour d\'efinir une orientation globale de la vari\'et\'e 
$M$, il suffit de choisir une 
$n$-forme diff\'erentiable 
$\omega$. Le probl\`eme est que, si 
$\omega$ s'annule en un point, l'orientation cesse d'\^etre
d\'efinie en ce point~! Pour pouvoir parler d'orientation
de fa\c con globale, il faut
donc qu'il soit possible de choisir une 
$n$-forme diff\'erentielle sur 
$M$ qui ne s'annule nulle part. Ceci n'est pas toujours possible~:
on dit que la vari\'et\'e est orientable ou non orientable suivant
les cas. Tout le monde conna\^{\i}t l'exemple fameux du ruban de
Moebius ou de la bouteille de Klein.

 On appelle ``\'el\'ement de volume'' sur 
$M$ le choix d'une 
$n$-forme 
$\omega$ sur 
$M$ qui ne s'annule nulle part (ce qui suppose, par d\'efinition, que
$M$ soit orientable). On note
$[\omega]$ l'ensemble des \'el\'ements de volume proportionnels \`a
$\omega$, avec un coefficient de proportionnalit\'e positif et
$[- \omega]$ l'ensemble des \'el\'ements de volume proportionnels \`a
$\omega$, avec un coefficient de proportionnalit\'e n\'egatif. Une
vari\'et\'e orientable poss\`ede donc deux orientations possibles,
l'une quelconque d'entre elles \'etant caract\'eris\'ee par le
choix d'un \'el\'ement de volume appartenant \`a l'une des deux
classes possibles. Soient maintenant 
$M$ et 
$N$ deux vari\'et\'es diff\'erentiables de m\^eme dimension
$n$ et 
$f$ un diff\'eomorphisme de
$M$ dans
$N$~; on suppose 
$M$ et
$N$ orientables et orient\'ees par le choix des \'el\'ements de
volume 
$\omega_M$ et 
$\omega_N$. On dit que 
$f$ pr\'eserve l'orientation si et seulement si
$\vl f (\omega_N) \in [\omega_M]$ et renverse l'orientation si
$\vl f (\omega_N) \in [- \omega_M]$.

\subsection{Orientation -- Partition de l'unit\'e}

Notre but, dans ce paragraphe, est d'introduire la notion
d'int\'egration des formes diff\'eren\-tiel\-les. Comme d'habitude, on
va commencer par d\'efinir cette notion pour l'espace num\'erique
$\RR^n$, puis, gr\^ace \`a un syst\`eme de cartes, on va pouvoir
g\'en\'eraliser la construction au cas des vari\'et\'es. On suppose
le lecteur familier avec la notion d'int\'egrale (de Riemann) sur
$\RR^n$. Soit
$f$ une fonction (num\'erique) c'est-\`a-dire une fonction -- que
nous supposons diff\'erentiable -- de 
$\RR^n$ \`a valeurs r\'eelles. Nous supposons, de plus, que
$f$ est \`a support compact. Son int\'egrale est not\'ee
$\int_{\RR^n} f$ ou
$\int_{\RR^n} f(x) d^nx$, comme d'habitude. Choisissons maintenant une
orientation sur
$\RR^n$ et consid\'erons la 
$n$-forme
$$\omega  =  f(x) \,  d x^1 \land d x^2 \land \dots \land d x^n$$ o\`u
$d x^1 \land d x^2 \land \dots \land d x^n$ est une 
$n$-forme positive pour l'orientation choisie. On pose simplement
$$\int \omega  =  \int_{\RR^n} f d^n x$$ Notons que la d\'efinition du
membre de gauche d\'epend de l'orientation choisie~; en d'autres
termes, on peut identifier les deux notations et concepts en posant
$$d^n x = d x^1 \land d x^2 \land \dots \land d x^n$$ mais il faut
bien noter que l'identification des notations d\'epend du choix
d'une orientation car l'int\'egrale d'une 
$n$-forme d\'epend de l'ordre
$x^1, x^2, \dots, x^n$ alors que l'int\'egrale de Riemann d'une
fonction
$f$ n'en d\'epend pas. Soit
$T$ un diff\'eomorphisme de 
$\RR^n$, c'est-\`a-dire un changement de variables
$x^\mu \stackrel{T}{\to} y^\mu$. 

 Notre \'etude g\'en\'erale des formes
diff\'erentielles implique en particulier
$$
d x^1 \land d x^2 \land \dots \land d x^n 
= J (T) \, d y^1 \land d y^2 \land \dots \land d y^n
$$
o\`u
$J (T) = det({\partial x^\mu / \partial y^\nu})$ est le jacobien (le 
d\'eterminant de la
matrice jacobienne) de l'application
$T$. On a donc
$$
\int f (x) d x^1 \land \dots \land d x^n
= \int f (T^{- 1} (y)) J (T) 
d y^1 \land \dots \land d y^n
$$
Mais on sait bien que
$$
\int f (x) d^n x
= \int f (T^{- 1} (y)) | J (T) | d^n y
$$
Donc si 
$\vl T \omega$ d\'esigne l'image r\'eciproque de 
$\omega$, on voit que 
$\vl T \omega = \pm \int \omega$ suivant que 
$T$ pr\'eserve ou non l'orientation~: l'int\'egrale d'une 
$n$-forme est invariante sous le groupe des diff\'eomorphismes qui
pr\'eservent l'orientation.

Passons maintenant au cas des vari\'et\'es. Soit
$M$ une vari\'et\'e de dimension
$n$ et
$\omega$ une 
$n$-forme \`a support compact. Supposant la vari\'et\'e orientable,
on choisit une orientation
$[M]$ et une {\sl partition de l'unit\'e \/}\index{partition de l'unit\'e}
$\{ \rho_\alpha \}_{\alpha \in I}$ subordonn\'ee \`a un atlas
$\{ (U_\alpha, \varphi_\alpha) \}_{\alpha \in I}$, c'est-\`a-dire
qu'on se donne une famille de fonctions diff\'erentielles non
n\'egatives 
$\rho_\alpha$ telles que le support de
$\rho_\alpha$ soit contenu dans 
$U_\alpha$ et telles que
$\sum \rho_\alpha = 1$ (chaque point de 
$M$ doit poss\'eder un voisinage dans lequel la somme pr\'ec\'edente
est une somme finie). L'existence d'une telle partition de
l'unit\'e, pour une vari\'et\'e diff\'erentiable, est un
th\'eor\`eme (que nous ne d\'emontrons pas) qui permet, dans de
nombreux cas, de passer des r\'esultats locaux (valables dans une
carte) aux r\'esultats globaux (valables pour toute la vari\'et\'e
$M$). On d\'efinit l'int\'egrale de 
$\omega$ sur
$[M]$ par l'\'egalit\'e
$$
\int_{[M]} \omega
= \sum_\alpha \int_{U_\alpha} \rho_\alpha \omega
$$
o\`u la quantit\'e
$\int_{U_\alpha} \rho_\alpha \omega$ signifie en fait
$\int_{\RR^n} (\varphi_\alpha^{- 1})^* (\rho_\alpha \omega)$ pour
une trivialisation locale
$\varphi_\alpha : U_\alpha \to \RR^n$ pr\'eservant l'orientation. On
se ram\`ene ainsi au cas de
$\RR^n$.

L'orientation \'etant choisie une fois pour toutes, on note
$\int_M$ et non plus
$\int_{[M]}$ l'int\'egrale correspondante. Il reste alors \`a
d\'emontrer que la d\'efinition adopt\'ee ne d\'epend pas des cartes
choisies\dots

On appelle {\sl \'el\'ement de volume \/}\index{element de volume} sur 
$M$ (de dimension
$n$) ou {\sl forme volume \/}\index{forme volume} un \'el\'ement quelconque
$\epsilon$ de
$\Omega^n M$. Le {\sl volume \/}\index{volume} de 
$M$, suppos\'ee compacte,  est alors \'egal, par d\'efinition, \`a
$\int_M \epsilon$. Il faut bien noter que sur une vari\'et\'e
quelconque (orient\'ee), on int\`egre des 
$n$-formes, et non  des fonctions, \`a moins, pr\'ecis\'ement, d'avoir
choisi un \'el\'ement de volume
$\epsilon$ une fois pour toutes, auquel cas on peut \'evidemment
poser $\int_M f$  $ =  \int_M f \epsilon$ o\`u
$f \in C^\infty (M)$. Un cas particuli\`erement important \`a
consid\'erer est celui o\`u la forme volume est associ\'ee
canoniquement au choix d'une structure riemannienne (voir
section~1.11) sur la vari\'et\'e en question.


%%%%%%%%%%%%%%%%%%%%%%%%%%%%%%%%%%%%%
%%%%%%%%%%%%%%%%%%%%%%%%%%%%%%%%%%%%%
\section{Vari\'et\'es riemanniennes (propri\'et\'es \'el\'ementaires)}

En toute logique, cette section ne devrait pas se trouver
dans ce premier chapitre consacr\'e aux vari\'et\'es
diff\'erentielles. En effet, la d\'efinition de la structure de
vari\'et\'e riemannienne est li\'ee \`a un cas particulier de
restriction d'espace fibr\'e (les espaces fibr\'es font l'objet du
chapitre~4).
 Cela dit, pour des raisons \`a la fois historiques et
p\'edagogiques, il est sans doute pr\'ef\'erable que le lecteur se
familiarise d'ores et d\'ej\`a avec certaines propri\'et\'es des
vari\'et\'es riemanniennes.

En g\'eom\'etrie \'el\'ementaire, on \'etudie d'abord les
propri\'et\'es lin\'eaires et affines et on passe, ensuite, aux
notions m\'etriques. Il en va de m\^eme dans l'\'etude des
vari\'et\'es. Une vari\'et\'e diff\'erentiable est encore un objet
flasque et mou\dots\ la donn\'ee d'une m\'etrique rigidifie l'espace
consid\'er\'e et permet, d'une part, de parler de norme des vecteurs
tangents et, d'autre part, de parler de distances entre points. La
d\'efinition \'el\'ementaire d'une m\'etrique
$g$, sur une vari\'et\'e diff\'erentiable 
$M$, est la suivante~: c'est un champ de tenseurs covariants
sym\'etriques de degr\'e deux (en g\'en\'eral on impose \'egalement une condition 
de non d\'eg\'en\'erescence). Si
$\{ x^\mu \}$ d\'esigne un syst\`eme de coordonn\'ees locales, on \'ecrira
$$
\mbox{\fbox{$
g = g_{\mu \nu} \, d x^\mu \otimes d x^\nu
$}}
$$
La m\'etrique  $g$ est quelquefois not\'ee
$d s^2 = g_{\mu \nu} d x^\mu d x^\nu$ et appel\'ee  ``\'el\'ement de
longueur'' ou {\sl tenseur m\'etrique\/}\index{m\'etrique}. Si
$\{ e^\alpha \}$ est un co-rep\`ere mobile, on pourra \'ecrire
\'egalement
$g = g_{\alpha \beta} e^\alpha \otimes e^\beta$. En tout point
$P$ de 
$M$ on a donc un produit scalaire 
$g_P$ d\'efini sur 
$T_P M$ et permettant de calculer le produit scalaire
$g_P (v, w)$ de deux vecteurs quelconques 
$v$ et
$w$ appartenant \`a l'espace tangent
en ce point. On a d\'ej\`a dit que
$g$ devait \^etre sym\'etrique (c'est-\`a-dire 
$g_{\mu v} = g_{\nu \mu}$, ou
$g_P (v, w) = g_P (w, v)$). Dans la plupart des cas, on impose \'egalement \`a
$g$ d'\^etre non d\'eg\'en\'er\'ee~: le d\'eterminant de la matrice
$(g_{\mu v})$ est suppos\'e non nul en tout point
$P$ et on peut donc inverser cette matrice. On obtient ainsi $(g^{\mu\nu})$
avec $g^{\mu\nu}g_{\nu\rho}=\delta^\mu_\rho$. Cela d\'efinit
 un produit
scalaire sur l'espace cotangent qu'on pourra noter   $ \sharp g$
(et quelquefois $g^{-1}$) et m\^eme parfois $g$ s'il n'y
a pas d'ambigu\"it\'e):
$$
\mbox{\fbox{$
g = g^{\mu \nu} {\partial \over \partial x^\mu} 
\otimes {\partial \over \partial x^\nu}
$}}$$

Une vari\'et\'e diff\'erentiable munie d'une m\'etrique (non
d\'eg\'en\'er\'ee) est, par d\'efinition, une
 {\sl vari\'et\'e riemannienne\/}\index{vari\'et\'e riemannienne}.
Nous n'avons pas impos\'e au produit scalaire d\'efini par
$(g_{\mu v})$ d'\^etre positif et nous ne l'imposerons pas. En
g\'en\'eral, une forme bilin\'eaire sym\'etrique est
caract\'eris\'ee par sa signature
$(p, q)$ -- le nombre de signes
$+ (p)$ et de signes 
$- (q)$ obtenus lorsqu'on la diagonalise.
Si on tient \`a pr\'eciser que la signature est de type
$(p, 0)$ ou $(0,p)$, on dira que la vari\'et\'e est proprement riemannienne. Si
on tient \`a pr\'eciser que la signature est de type
$(p, 1)$ ou $(1,p)$, on dira que la vari\'et\'e est lorentzienne (on dit 
aussi, dans ce dernier cas, que la signature est hyperbolique). Les
 cas riemanniens et lorentziens sont particuli\`erement importants en 
physique mais
nous n'avons pas besoin de nous restreindre \`a ce cadre pour
l'essentiel de ce qui suit.

\smallskip
$\bullet$ Pour une vari\'et\'e riemannienne 
$(M, g)$ orient\'ee, on peut d\'efinir une {\sl forme de volume canonique\/}
\index{forme volume}\label{sec:forme volume}
de la fa\c con suivante. Soit
$\{ e_{\widehat \alpha} \}$ un rep\`ere mobile orthonormal, c'est-\`a-dire
$g (e_{\widehat \alpha}, e_{\widehat \beta}) 
= \eta_{\widehat \alpha \widehat \beta}$, avec
$\eta_{\widehat \alpha \widehat \beta} 
= \pm \delta_{\widehat \alpha \widehat \beta}$, le signe $\pm$ d\'ependant de la signature de la m\'etrique. Pour l'instant nons pouvons supposer que l'espace est proprement riemannien, et orient\'e, mais
\`a la fin de cette section, nous verrons comment compl\'eter les propri\'et\'es qui suivent lorsque la signature de la m\'etrique est quelconque, et plus particuli\`erement lorsqu'elle est hyperbolique.
 D\'esignant par
$\{ e^{\widehat \alpha} \}$ le co-rep\`ere dual du rep\`ere mobile choisi, l'\'el\'ement de volume riemannien est
$$
\mbox{\fbox{$
\epsilon 
= e^{\widehat 1} \land e^{\widehat 2} 
\land \dots \land e^{\widehat n}
$}}
$$

Soit
$\{ \sigma_\alpha \}$ un autre rep\`ere, non n\'ecessairement
orthonormal, et 
$\Lambda_\alpha^{\widehat \alpha}$ la matrice de passage, c'est-\`a-dire
$\sigma_\alpha = \Lambda_\alpha^{\widehat \alpha} e_{\widehat \alpha}$.
Alors
$g_{\alpha \beta} 
= \Lambda_\alpha^{\widehat \alpha} \, \Lambda_\beta^{\widehat \beta} \,
g_{\widehat \alpha \widehat \beta}$, ce qui implique 
$$
\det (g_{\alpha \beta})
= 
\Bigl[ 
\det (\Lambda_\alpha^{\widehat \alpha}) 
\Bigr]^2 
\det (g_{\widehat \alpha \widehat \beta})\ \mbox{et donc}\ 
\det (\Lambda_\alpha^{\widehat \alpha})
= 
\biggl|
{
\det g_{\alpha \beta} 
\over 
\det g_{\widehat \alpha \widehat \beta}
}
\biggr|^{1 / 2}
$$
mais 
$|\det (g_{\widehat \alpha \widehat \beta})| = 1$ puisqu'on a
suppos\'e la base
$\{ e_{\widehat \alpha} \}$ orthonormale et donc
$$
\det (\Lambda_\alpha^{\widehat \alpha}) 
= |\det (g_{\alpha \beta})|^{1 / 2}
=
\Bigl[
\det ((\Lambda^{- 1})_{\widehat \alpha}^\alpha)
\Bigr]^{- 1}
$$
Or
\begin{eqnarray*}
 \epsilon
& = & e^{\widehat 1} \land e^{\widehat 2} 
\land \dots \land e^{\widehat n}
= \Lambda_{\alpha_1}^{\widehat 1} 
\Lambda_{\alpha_2}^{\widehat 2}
\dots \Lambda_{\alpha_n}^{\widehat n} \, 
\sigma^{\alpha_1} \land \sigma^{\alpha_2} 
\land \dots \land \sigma^{\alpha_n}            \nonumber \\[1mm]
 \epsilon & = & \det (\Lambda_\alpha^{\widehat \alpha}) 
\cdot \sigma^1 \land \sigma^2 
\land \dots \land \sigma^n                     \nonumber \\[1mm]
 \epsilon
& = & \sqrt{|\det (g_{\alpha \beta})|} \: 
\sigma^1 \land \sigma^2 
\land \dots \land \sigma^n 
= \epsilon_{\alpha_1 \alpha_2 \dots \alpha_n}
\sigma^{\alpha_1} \land \sigma^{\alpha_2} 
\land \dots \land \sigma^{\alpha_n}
\end{eqnarray*}
Ainsi
\footnote{L'espace des $n$-formes \'etant de dimension $1$, deux 
$n$-formes quelconques sont obligatoirement proportionelles.
Dans les \'equations qui pr\'ec\`edent,
le passage de la premi\`ere \`a la seconde ligne peut donc \^etre 
consid\'er\'e comme une d\'efinition du d\'eterminant de la matrice $\Lambda$.}
$$
\mbox{\fbox{$
\epsilon_{\alpha_1 \alpha_2 \dots \alpha_n}
= \sqrt{|\det (g_{\alpha \beta})|} \: 
\delta_{\alpha_1 \, \alpha_2 \dots \alpha_n}^{1 2 \dots n}
$}}
$$
o\`u le symbole  $\delta_{\alpha_1 \, \alpha_2 \dots \alpha_n}^{1 2 \dots 
n}$ d\'esigne
$\pm$ suivant que ${\alpha_1 \, \alpha_2 \dots \alpha_n}$ est une 
permutation paire ou impaire de $1,2,3\ldots n$ et il est \'egal \`a 
z\'ero dans les autres cas.

En particulier, si
$\{ \sigma_\alpha \}$ d\'esigne un rep\`ere naturel
$\Bigl\{
{\partial \over \partial x^\mu}
\Bigr\}$, on a
$g = g_{\mu \nu} d x^\mu \land d x^\nu$ et
$$
\mbox{\fbox{$\epsilon
= \sqrt{|\det (g_{\mu v})|} \: d x^1 \land \dots \land d x^n $}}$$
Notons \'egalement que, dans un rep\`ere orthonorm\'e, la racine carr\'ee
pr\'ec\'edente vaut $1$ et donc
 \fbox{$\epsilon_{\alpha_1 \, \alpha_2 \dots \alpha_n}
=\delta_{\alpha_1 \, \alpha_2 \dots \alpha_n}^{1 2 \dots n}$}
En particulier $\epsilon_{12\ldots n}=1$.

\smallskip
$\bullet$ La m\'etrique
$g$ permet \'egalement d'\'etablir un isomorphisme canonique entre
l'espace tangent
$T M$ et l'espace cotangent 
$T^* M$. Cette propri\'et\'e est \'evidente puisqu'en chaque point,
l'existence d'un produit scalaire permet d'identifier
les espaces vectoriels $T_P M$ avec
$T^*_P M$. En d'autres termes, on peut ``monter'' et ``descendre'' les
indices \`a l'aide de la m\'etrique~: au vecteur
$v = v^\mu e_\mu$ on associe la 1-forme
${}^b v = v_\mu e^\mu$ d\'efinie par 
\fbox{$v_\mu = g_{\mu \nu} v^\nu$} (
$\{ e^\mu \}$ d\'esignant la base duale de 
$\{ e_\mu \}$). Inversement, \`a la 1-forme
$\sigma = \sigma_\mu e^\mu$ on associe le vecteur
$\sharp\sigma = \sigma^\mu e_\mu$ avec
\fbox{$\sigma^\mu = g^{\mu v} \sigma_\nu$}. On peut \'ecrire
$\langle {}^b v_1, v_2 \rangle = g (v_1, v_2)$ et 
$\langle \sigma,\ , \sharp\sigma_2 \rangle 
= (\sigma_1, \sigma_2)$. Les isomorphismes 
$\flat$ et
$\sharp$ sont appel\'es {\sl isomorphismes musicaux \/}(pour des raisons
\'evidentes~!)\index{isomorphismes musicaux}.Cela dit, les physiciens choisissent en g\'en\'eral
une m\'etrique une fois pour toutes et d\'ecident donc de passer
sous silence ces isomorphismes musicaux. En d'autres termes, ils
identifient 
$v$ et 
$\flat v$ ainsi que
$\sigma$ et
$\sharp \sigma$ et \'ecrivent tout simplement
$v = v^\mu e_\mu = v_\mu e^\mu$ ou
$\sigma = \sigma_\mu e^\mu = \sigma^\mu e_\mu$. Pour des raisons
analogues ils \'ecrivent
$g = g_{\mu v} e^\mu \otimes e^\nu = g^{\mu v} e_\mu \otimes e_v$
(mais il faut bien entendu se rappeler que
$(g_{\mu v})$ est la matrice inverse de
$(g^{\mu v})$. 
 Les isomorphismes
musicaux permettent, de la m\^eme fa\c con, d'identifier les
tenseurs covariants et contravariants de m\^eme rang~. Attention, lorsqu'on 
n'utilise pas de m\'etrique pour monter ou baisser les indices, il n'y a pas
de raison de faire attention \`a la position relative des indices 
covariants et
contravariants (par exemple, on peut parler de $T^\mu_{\nu\rho}$ sans dire
s'il s'agit de ${T^\mu}_{\nu\rho}$, de ${T_{\nu\rho}}^\mu$ ou de
${{T_\nu}^\mu}_\rho$). Les trois types de composantes correspondent 
d'ailleurs \`a des objets diff\'erents puisque on travaille, suivant les 
cas, dans
 $TM\otimes T^*M \otimes T^*M$,  $T^*M\otimes T^*M \otimes TM$ ou  
$T^*M\otimes TM \otimes T^*M$. Par contre, si on utilise la m\'etrique 
pour proc\'eder \`a des identifications, il faut faire attention aux 
positions relatives des indices haut et bas! Ainsi, par exemple, 
$T_{\mu\nu\rho}$ d\'esignera $g_{\mu \mu'}{T^{\mu'}}_{\nu\rho}$. Tout ceci 
est assez trivial, mais peut \^etre fallait-il le dire une fois ?

{\it Lorsqu'on a choisi une m\'etrique\/},  on \'ecrira
donc abusivement (sans utiliser la notation
$\flat$ et
$\sharp$), par exemple
$T = T_{\mu v \rho} e^\mu \otimes e^v \otimes e^\rho 
={T_\mu}^{\nu \rho} e^\mu \otimes e_v \otimes e_\rho
= T^{\mu v \rho} e_\mu \otimes e_v \otimes e_\rho
= \dots$ 

Les isomorphismes musicaux sont quelquefois simplement d\'esign\'es
par le m\^eme symbole
$g$ que la m\'etrique elle m\^eme, le nombre d'arguments permettant
de d\'ecider si on parle des isomorphismes en question ou de la
m\'etrique. Par exemple, on peut noter
$g (e_\mu) = g_{\mu v} e^\nu$ et
$g (e^\mu) = g^{\mu v} e_\nu$. Ceci est en accord avec les notations
pr\'ec\'edentes puisque par exemple
$g (v) = g (v^\mu e_\mu) = v^\mu g (e_\mu) 
= v^\mu g_{\mu v} e^\nu = v_v e^\nu$.

Avec ces notations, nous avons alors 
$g (v, w) = \langle g (v), w \rangle = \langle g (w), v \rangle
= g (w,v)$.

\smallskip
$\bullet$
L'existence d'une m\'etrique permet non seulement de calculer le
produit scalaire de deux vecteurs (ou de deux $1$-formes)
mais de contracter
n'importe quel tenseur d'ordre
$k$ covariant, contravariant ou partiellement covariant et
contravariant avec n'importe quel autre tenseur d'ordre
$k$. On utilisera encore la notation $\langle {\,}\vert {\,} \rangle$ pour 
\'ecrire 
ces contractions de type assez g\'en\'eral. 
Plut\^ot que de d\'ecrire les diff\'erents cas, il suffit de dire,
en termes imag\'es, qu'on ``monte'' tous les indices du premier \`a
l'aide de la m\'etrique, qu'on ``descend'' tous les indices du second
et qu'on contracte compl\`etement les objets obtenus. Par exemple, si
$S = {S_{\mu \nu}}^\rho e^\mu \otimes e^\nu \otimes  e_\rho$ et
$T ={ T_\mu}^{\nu \rho} e^\mu \otimes  e_\nu \otimes  e_\rho$ on fabrique
$S_{\mu v \rho'} = g_{\rho'\rho} {S_{\mu v}}^\rho$ et
$T^{\mu'\nu \rho} = g^{\mu'\mu} {T_\mu}^{\nu \rho}$. On peut alors calculer
$\langle S, T \rangle = S^{\mu \nu \rho} T_{\mu \nu \rho}$. En
particulier, si
$\{ \varphi^1, \varphi^2, \dots, \varphi^k \}$ d\'esignent une
famille de 1-formes et
$\{ \psi^1, \psi^2, \dots, \psi^k \}$ en d\'esigne une autre, on
peut v\'erifier que cette d\'efinition conduit \`a
$$
\langle
\varphi^1 \land \varphi^2 \land \dots \land \varphi^k,\ 
\psi^1 \land \psi^2 \land \dots \land \psi^k 
\rangle
= k ! \det (\langle \varphi^i, \psi^i \rangle)
$$
Dans le sous-cas particulier o\`u ces deux familles co\"incident et
o\`u on suppose la famille 
$\{ \varphi^k \}$ orthonorm\'ee, il vient
$$
\langle
\varphi^{\mu_1} \land \dots \land \varphi^{\mu_k},\ 
\varphi^{\nu_1} \land \dots \land \varphi^{\nu_k} 
\rangle
= k ! \delta_{\nu_1 \nu_2 \dots \nu_k}^{\mu_1 \mu_2 \dots \mu_k}
$$


\smallskip
$\bullet$ Lorsque la signature de la m\'etrique n'est pas proprement riemannienne, c'est \`a dire lorsqu'elle est de type $(p,q)$, il faut supposer que la vari\'et\'e admet une orientation temporelle et qu'elle est temporellement orient\'ee.
Dans le cas usuel de l'espace-temps de la physique (signature $(3,1)$), on utilisera des indices $0,1,2,3$, comme c'est l'usage, plut\^ot que $1,2,3,4$; on posera alors
$$
\epsilon = e^{\widehat 0} \land e^{\widehat 1} \land e^{\widehat 2}
 \land e^{\widehat 3}
$$
La diff\'erence avec le cas proprement euclidien vient du fait qu'il faut faire attention aux signes lorsqu'on ``monte'' les indices. Par ailleurs, 
$det(g_{{\widehat \alpha}{\widehat \beta}})=(-1)^q$.
Les formules pr\'ec\'edentes restent donc valables et on aura toujours, par exemple,
$$\epsilon_{{\widehat 0}{\widehat 1}{\widehat 2}{\widehat 3}} = 1$$
mais par contre
$$\epsilon^{{\widehat 0}{\widehat 1}{\widehat 2}{\widehat 3}} = -1$$
Plus g\'en\'eralement, nous avons vu que
$\epsilon_{\alpha_1 \alpha_2 \dots \alpha_n}
= \sqrt{|\det g_{\alpha \beta}|} \: 
\delta_{\alpha_1 \alpha_2 \dots \alpha_n}^{1 2 \dots n}$, mais si on
``monte'' ces indices (\`a l'aide de la m\'etrique) le r\'esultat va
d\'ependre de la signature, plus particuli\`erement du nombre $q$  de
signes 
``$-$'' dans la m\'etrique. On obtient donc
$$
\epsilon^{\alpha_1 \alpha_2 \dots \alpha_n}
= {(- 1)^q \over \sqrt{|\det g_{\alpha \beta}|}}
\delta_{1 2 \dots n}^{\alpha_1 \alpha_2 \dots \alpha_n}
= (- 1)^q \sqrt{|\det g^{\alpha \beta}|} \:
\delta_{1 2 \dots n}^{\alpha_1 \alpha_2 \dots \alpha_n}
$$
Citons quelques contractions utiles:
$$
q! \; \delta^{\alpha_1 \ldots \alpha_p \mu_1 \ldots 
\mu_q}_{\gamma_1\ldots \gamma_{p+q}} = 
\delta^{\alpha_1\ldots\alpha_p \beta_1\ldots 
\beta_q}_{\gamma_1\ldots\gamma_{p+q}}\delta^{\mu_1 \ldots \mu_q}_{\beta_1\ldots \beta_q}
$$
Lorsque la vari\'et\'e est lorentzienne, avec une signature de type $(3,1)$, on obtient en particulier
$$
\delta^{\alpha \beta \gamma}_{\mu\nu\lambda}=- \epsilon^{\alpha\beta\gamma\rho} \epsilon_{\mu\nu\lambda\rho}
$$

$$
\delta^{\alpha \beta}_{\mu\nu}={1 \over 2} \delta^{\alpha \beta \gamma}_{\mu\nu\gamma}= -{1 \over 2}\epsilon^{\alpha\beta\lambda\rho} \epsilon_{\mu\nu\lambda\rho}
$$

$$
\delta^{\alpha}_{\mu} = {1\over 3}\delta^{\alpha \beta}_{\mu\beta} = 
{1\over 6}\delta^{\alpha \beta \lambda}_{\mu\beta\lambda}=-{1\over 6}
 \epsilon^{\alpha\beta\lambda\rho} \epsilon_{\mu\beta\lambda\rho}
$$



\smallskip
$\bullet$
Pour terminer, notons que l'existence d'une m\'etrique permet d'associer
\`a la diff\'erentielle $df$ d'une fonction $f$, un champ de vecteurs, le
 {\sl gradient\/}\index{gradient} de $f$ d\'efini par $grad \, f = \sharp df$.
Ainsi, dans un rep\`ere naturel, on \'ecrira
$$ grad \, f = g^{\mu\nu} ({\partial f \over \partial x^\mu}){\partial 
\over \partial x^\nu}
$$

\bigskip
Encore une fois, la pr\'esente section consacr\'ee aux
 vari\'et\'es riemanniennes n'est destin\'ee qu'\`a introduire certaines  
notations utiles et quelques notions \'el\'ementaires. Nous reviendrons
plus en d\'etail sur les vari\'et\'es riemanniennes \`a la fin du chapitre consacr\'e aux connexions.



\section{Divers}

Nous regroupons dans ce paragraphe un certain nombre de notions et
de commentaires qui peuvent \^etre consid\'er\'es comme un peu moins
\'el\'ementaires que ce qui pr\'ec\`ede~; cela ne signifie pas
qu'ils sont moins ``importants'' mais simplement que nous utiliserons
peu ou pas ces concepts dans la suite de l'ouvrage. On se contente
donc ici de pr\'esenter quelques d\'efinitions de fa\c con \`a
sugg\'erer au lecteur des lectures plus approfondies et \`a donner
quelques id\'ees intuitives.

\subsection{ Compl\'ements sur les d\'erivations d'alg\`ebre}
\par
 L'ensemble des d\'erivations de l'alg\`ebre associative  ${\cal
A}$ se note $Der {\cal A}$.
Rappelons qu'une d\'erivation
est  une application lin\'eaire $v$ de $A$
dans $A$ telle que $v[f g] = v[f] g + f v[g]$ avec $f,g \in
{\cal A}$.
 Il est facile de voir que la
somme de deux d\'erivations est une d\'erivation, par contre
le produit de deux d\'erivations n'est pas une d\'erivation
(dans les cas o\`u on peut le d\'efinir, c'est un op\'erateur du
``second ordre'' ).  Il
est int\'eressant de savoir si l'op\'erateur $w$ d\'efini par $ w  =  h 
v$ avec $h \in
{\cal A}$ et $v \in Der{\cal A}$ est, ou non, une d\'erivation.
En d'autres termes, on veut savoir si l'espace $Der{\cal A}$ est stable
lorsqu'on multiplie (\`a gauche) ses \'el\'ements par des \'el\'ements
de ${\cal A}$. Dans l'affirmative, on dit que $Der{\cal A}$ est un {\sl 
module \/} \index{volume} sur ${\cal A}$.
 La
r\'eponse est oui, mais seulement dans le cas o\`u ${\cal
A}$ est commutative. En effet, prenons $f$ et $g$ dans ${\cal A}$. Alors,
$ (hv)[fg] =
h(v[fg]) = h(v[f] g + f v[g]) = h v[f] g + h f v[g]$ mais
$(hv)[f] g + f (hv)[g] = h v[f] g + f h v[g]$. Ces deux
expressions ne co\"incident que si $h f = f h$.


Conclusion: L'ensemble des d\'erivations d'une alg\`ebre
associative n'est pas, en g\'en\'eral, un module sur cette
alg\`ebre, sauf si cette cette derni\`ere est commutative.
Il est facile de voir que $Der{\cal A}$ est un module sur
le centre de ${\cal A}$, centre qui peut \^etre assez
petit...

 Par contre, l'ensemble des d\'erivations est
toujours une alg\`ebre de Lie~: on peut y
d\'efinir une loi de composition interne (not\'ee $[,]$) non
associative et anti-commutative ($[u,v]=-[v,u]$), qui
v\'erifie l'identit\'e suivante (l'identit\'e de Jacobi): $$
[u,[v,w]] + [v,[w,u]] + [w,[u,v]] = 0 $$
\par
Rappelons que l'ensemble des champs de vecteurs sur une vari\'et\'e
n'est autre que l'alg\`ebre de Lie $Der C^\infty(M)$.
\index{champs de vecteurs}
\par
 Le fait que l'ensemble des d\'erivations de ${\cal A}$
soit un module sur ${\cal A}$ lorsque ${\cal A}$ est
commutative admet une g\'en\'eralisation
{\sl supersym\'etrique\/}\index{supersym\'etrique}. Supposons que $A$ soit une alg\`ebre
$\ZZ_2$-gradu\'ee. Chaque \'el\'ement $a$ de ${\cal A}$ peut donc
s'\'ecrire comme somme d'un \'el\'ement pair ($\#a =0$) et d'un
\'el\'ement impair ($\#a =1$). On d\'efinit les
{\sl d\'erivations gradu\'ees \/} (ou {\sl super-d\'erivations \/}) 
\index{d\'erivations gradu\'ees} des
alg\`ebres $\ZZ_2$-gradu\'ees comme les d\'erivations usuelles, mais en
introduisant un signe. On dit qu'une super-d\'erivation
est paire si c'est une d\'erivation, au sens usuel du
terme. On dit qu'une super-d\'erivation
est impaire si c'est  une
application lin\'eaire de $A$ dans $A$ telle que $v[f g] =
v[f] g + (-1)^{\#f} f v[g]$. On introduit donc alors une
$\ZZ_2$ graduation pour les super-d\'erivations et on r\'eunit
les deux types de formules de la fa\c con suivante~:
 $$
 \mbox{\fbox{$ 
 v[f g] = v[f] g + (-1)^{\#v \#f} f v[g]
 $}}$$ avec  $f,g \in
{\cal A}$. En pratique, il suffit d'utiliser la r\`egle
dite ``R\`egle de Milnor'' disant qu'il faut introduire un signe ``$-$''
 chaque fois qu'on doit commuter deux \'el\'ements impairs.\par

 L'ensemble des super-d\'erivations d'une alg\`ebre {\cal A} ne constitue pas,
en g\'en\'eral, un module sur ${\cal A}$, sauf lorsque ${\cal A}$ est
{\sl commutative gradu\'ee \/}\index{alg\`ebre commutative gradu\'ee} (on dit aussi
{\sl super-commutative \/}), c'est \`a dire lorsque $f g =
(-1)^{\#f \#g} g f$. Par contre l'ensemble des
d\'erivations gradu\'ees constitue toujours un module sur le
{\sl super-centre \/} de ${\cal A}$ (l'ensemble des \'el\'ements de
${\cal A}$ qui commute -- au signe pr\`es -- avec ${\cal A}$)
et il  constitue \'egalement une {\sl super-alg\`ebre de Lie 
\/}\index{superalg\`ebre de Lie}, c'est
\`a dire que les d\'erivations gradu\'ees super-anticommutent~:
$$
\mbox{\fbox{$
[v,w]= - (-)^{\#v \#w} [w,v]
$}}
$$
et v\'erifient l'identit\'e de Jacobi gradu\'ee
$$
\mbox{\fbox{$
(-)^{\#u \#w}[u,[v,w]] +
(-)^{\#w \#v}[w,[u,v]] +
(-)^{\#v \#u}[v,[w,u]] = 0 
$}}
$$ 

%%% D\'efinition alg\'ebrique comme d\'erivations,
%%% op\'erateurs diff\'erentiels
%%% d'ordre $1$ agissant sur les fonctions, etc

\subsection{Cohomologie de De Rham}\label{sec: cohomologie de De Rham}

 Nous avons vu que l'op\'erateur 
$d$ satisfait
$d^2 = 0$ et envoie
$\Omega^k M$ dans
$\Omega^{k + 1} M$. Soit
$Z^k$ le noyau de
$d$, c'est-\`a-dire 
$Z^k = \{ \omega \in \Omega^k M \ t q \ d \omega = 0 \}$. Les
\'el\'ements de
$Z^k$ sont appel\'es {\sl cocycles de De Rham \/}\index{cocycles} de degr\'e 
$k$ (ou {\sl formes ferm\'ees}\index{formes ferm\'ees}). Soit
$B^k$ l'image par
$d$ de 
$\Omega^{k - 1} M$ dans 
$\Omega^k M$, c'est-\`a-dire 
$B^k = 
\{ 
\omega \in \Omega^k M \ t q \ \exists \tau \in \Omega^{k - 1}$ avec
$\omega = d \tau
\}$. Les \'el\'ements de
$B^k$ sont les {\sl cobords de De Rham \/}\index{cobords} de degr\'e
$k$ ou {\sl formes exactes \/}\index{formes exactes} . Le fait que 
$d^2 = 0$ implique l'inclusion
$B^k \subset Z^k$.

Il r\'esulte de la lin\'earit\'e de
$d$ que
$Z^k$ et 
$B^k$ sont stables par addition, ce sont donc des groupes
ab\'eliens~; on peut alors consid\'erer le groupe quotient
$H^k = Z^k / B^k$ qu'on appelle groupe de cohomologie (de De Rham)
de degr\'e
$k$. On peut calculer, pour toute vari\'et\'e, les groupes
$H^0, H^1, \dots, H^n$. Ces groupes fournissent, en quelque sorte
une ``mesure'' de la non-trivialit\'e de la topologie de la
vari\'et\'e
$M$. En effet, tous ces groupes sont triviaux (se r\'eduisent \`a
l'\'el\'ement neutre 0) dans le cas de l'espace num\'erique 
$\RR^n$, ce que le lecteur sait d\'ej\`a puisque, dans un autre
contexte, celui de la th\'eorie des \'equations diff\'erentielles
sur 
$\RR^n$, on montre de fa\c con \'el\'ementaire que, pour r\'esoudre
une \'equation 
$d f = 0$, il faut poser
$f = d g$ (Lemme de Poincar\'e).

\subsection{Homologie de De Rham}

 La d\'efinition de l'homologie de De Rham est
plus d\'elicate que celle de la cohomologie. De fa\c con \`a en
donner une image intuitive, disons qu'on s'int\'eresse \`a des
``morceaux'' de la vari\'et\'e
$M$ (compt\'es possiblement avec multiplicit\'e). Un tel morceau 
$C$ (techniquement une {\sl cha\^{\i}ne \/}) peut avoir un bord (le bord
d'un disque est un cercle) ou pas de bord (le bord d'un cercle est
nul). On peut formellement additionner les cha\^{\i}nes (avec des coefficients
r\'eels, dans le cas pr\'esent).
 On d\'efinit alors un {\sl op\'erateur bord\/}\index{bord}
$\partial$, de carr\'e nul lui aussi 
($\partial^2 = 0$, le bord d'un bord est nul) et on peut
consid\'erer les {\sl cycles \/} \index{cycle} (cha\^{\i}nes
$C$ dont le bord 
$\partial C$ est nul) et les {\sl bords\/} (cha\^{\i}nes 
$C$ qui sont le bord de quelque chose 
$C = \partial D$).

Tous les bords \'etant des cycles, on peut l\`a aussi consid\'erer
les cycles
$Z_k$ de dimension
$k$ modulo les bords
$B_k$ et d\'efinir les groupes d'homologie
$H_k = Z_k / B_k$. De fa\c con g\'en\'erale, on parle de
{\sl cohomologie \/}\index{cohomologie} lorsqu'on a un op\'erateur de carr\'e nul (tel 
$d$) dont l'action sur un espace vectoriel
$\ZZ$-gradu\'e fait cro\^{\i}tre le degr\'e d'une unit\'e et
 d'{\sl homologie\/}\index{homologie} lorsqu'on
a un op\'erateur de carr\'e nul 
(tel~$\partial$) dont l'action fait d\'ecro\^{\i}tre le degr\'e.

Paradoxalement, la d\'efinition de 
$d$ est plus simple que celle de
$\partial$ (nous avons pass\'e cette derni\`ere sous silence) alors
que l'action de 
$\partial$ est plus intuitive, plus ``visuelle'' que celle de 
$d$. Le lien entre les deux est fournit par \index{th\'eor\`eme de Stokes} le {\sl th\'eor\`eme de
Stokes\/}~: de fa\c con g\'en\'erale on peut int\'egrer les 
$k$-formes sur les 
$k$-cha\^{\i}nes et on a la propri\'et\'e
$$
\mbox{\fbox{$
\int_{\partial C} \omega
= \int_C d \omega
$}}
$$
qui g\'en\'eralise la relation bien connue des physiciens de premi\`ere
ann\'ee de nos universit\'es 
$\int_\Sigma \vr E. d \vr s = \int_V d i v \vr E \, d \tau$ o\`u la
surface 
$\Sigma$ est le bord du volume
$V$ et o\`u l'int\'egrale repr\'esente le ``flux sortant'' du
champ \'electrique
$\vr E$.

La dualit\'e entre homologie et cohomologie s'\'ecrit tr\`es
simplement dans le cas des vari\'et\'es compactes~; dans ce cas, on
d\'emontre que
$H^k$ est isomorphe \`a $ H_{n - k}$ o\`u 
$n$ est la dimension de la vari\'et\'e. Le support visuel intuitif
suffit, en dimension~2, pour calculer l'homologie (et donc la
cohomologie) de quelques vari\'et\'es tr\`es simples. C'est ainsi
que, pour la sph\`ere 
$S^2$ on a 
$H_0 (S^2) = H_2 (S^2) = \RR$ et
$H_1 (S^2) = 0$ (tout cercle trac\'e sur la sph\`ere est le bord de
quelque chose), alors que pour le tore
$T^2$, on a 
$H_0 (T^2) = H_2 (T^2) = \RR$ mais 
$H_1 (T^2) = \RR \oplus \RR$~: les deux g\'en\'erateurs de
$H_1 (T^2)$ correspondent respectivement aux deux types de cercles
qu'on peut tracer sur un tore et qui ne ``bordent'' rien,
c'est-\`a-dire ``ceux qui font un tour''. On appelle {\sl nombres de
Betti \/} \index{nombres de Betti} de la vari\'et\'e
$M$, la dimension 
$b_p$ de
$H_p (M)$ consid\'er\'e comme espace vectoriel.


\subsection{Espace des $p$-vecteurs}

Nous avons choisi de d\'evelopper la notion de produit ext\'erieur
en partant du fibr\'e cotangent~, c'est-\`a-dire que nous avons
consid\'er\'e des produits tensoriels compl\`etement
antisym\'etriques de vecteurs covariants. Ceci nous a amen\'e au
concept de forme diff\'erentielle. Nous aurions pu faire de m\^eme
en partant des vecteurs contravariants. Le formalisme est tr\`es
semblable et les objets contravariants $\Omega_p(M)$ correspondant aux 
formes
diff\'erentielles $\Omega^p(M)$ sont simplement baptis\'ees 
``$p$-vecteurs''. On peut alors bien entendu \'evaluer une 
$p$-forme sur un 
$p$-vecteur, le r\'esultat \'etant une fonction sur 
$M$.


\subsection{Espace des courants de De Rham}\label{sec: courants de De Rham}

Le lecteur est sans doute d\'ej\`a familier avec la notion de
distribution. Pour les fonctions num\'eriques sur un compact de $\RR^n$
 les distributions
sont d\'efinies comme dual des fonctions infiniment diff\'erentiables.
 Cet espace
contient d'une part des \'el\'ements ``r\'eguliers'' mais aussi
toutes les mesures (en particulier la mesure de Dirac) et m\^eme des
objets encore plus singuliers (les d\'eriv\'ees de la distribution
de Dirac par exemple). On peut g\'en\'eraliser la th\'eorie des
distributions aux formes diff\'erentielles~ de degr\'e quelconque sur une
vari\'et\'e; on d\'efinit ce qu'on
appelle l'espace des {\sl courants de De Rham \/}\index{courants de De 
Rham} comme dual (sur 
$\RR$) des formes diff\'erentielles. L'\'evaluation d'un courant 
$C$ sur une forme 
$\omega$ est donc un nombre
$\langle C, \omega \rangle$. Si la vari\'et\'e
$M$ est compacte et si 
$\omega$ est une 
$k$-forme,
un \'el\'ement ``r\'egulier'' peut \^etre
repr\'esent\'e par une 
$n - k$ forme
$\sigma$ puisque l'\'evaluation de l'int\'egrale
$\int_M \sigma \land \omega$ est bien une fonctionnelle lin\'eaire.
L'int\'egration d'une forme sur une cha\^{\i}ne (th\'eorie de
l'homologie), l'\'evaluation d'un 
$p$-vecteur sur une 
$p$-forme suivie de l'int\'egration sur
$M$ de la fonction obtenue,
fournissent aussi des exemples de
courants de De Rham. La th\'eorie de l'homologie de De Rham
(op\'erateur 
$\partial$) se g\'en\'eralise d'ailleurs au cadre des courants
 et le th\'eor\`eme de Stokes s'\'ecrit dans ce cas
$\langle \partial C,\omega \rangle = \langle C, d\omega \rangle$.

\subsection{Les alg\`ebres de Fr\"olicher -- Nijenhuis et de Nijenhuis--Richardson}

Nous savons que l'alg\`ebre de De Rham $\Omega(M)$, munie du produit 
ext\'erieur, est une alg\`ebre commutative gradu\'ee.

Nous savons aussi que l'ensemble des d\'erivations gradu\'ees d'une 
alg\`ebre commutative gradu\'ee constitue une super-alg\`ebre de Lie pour 
laquelle le
 crochet de Lie  est donn\'ee par le commutateur (gradu\'e) que nous 
noterons
simplement $[.,.]$.

En cons\'equence $Der(\Omega(M))$ est une alg\`ebre de Lie gradu\'ee. 
Reste \`a
identifier explicitement les \'el\'ements de cette alg\`ebre.

Tout d'abord, puisque $\Omega(M)$ est $\ZZ$-gradu\'ee, on dira qu'une 
d\'erivation est de degr\'e $p$ (qui peut \^etre positif, n\'egatif ou 
nul) si elle fait
passer de $\Omega^k(M)$ \`a $\Omega^{k+p}(M)$. On notera 
$Der_p(\Omega(M))$ l'espace des d\'erivations de degr\'e $p$. La 
d\'eriv\'ee ext\'erieure est elle-m\^eme un \'el\'ement de 
$Der_1(\Omega(M)$.

Soit $\Omega(M,TM)$ l'espace des formes diff\'erentielles sur $M$ {\it \`a 
valeurs\/} dans le fibr\'e tangent, c'est \`a dire $\Omega^k(M,TM)  = 
 \Gamma(\Lambda^kT^*M\otimes TM)$. Une $k$-forme $K$ \`a valeurs 
vectorielles s'\'ecrira, dans un rep\`ere naturel,
$$K = K_{\mu_1 \mu_2 \ldots \mu_k}^\nu dx^{\mu_1}\land dx^{\mu_2}\ldots \land dx^{\mu_k}
\otimes {\partial \over \partial x^\nu}$$


Un r\'esultat du \`a Richardson et Nijenhuis montre que l'alg\`ebre de Lie 
gradu\'ee des d\'erivations (gradu\'ees) de l'alg\`ebre de De Rham 
$\Omega(M)$ peut 
s'identifier \`a deux copies de  $\Omega(M,TM)$ munies de deux crochets 
diff\'erents, connus respectivement sous le nom de {\sl crochet de 
Nijenhuis-Richardson\/} \index{crochet de Nijenhuis-Richardson} et 
{\sl crochet de Fr\"olicher-Nijenhuis\/}\index{crochet de 
Fr\"olicher-Nijenhuis}. Plus pr\'ecis\'ement, pour tout 
toute d\'erivation $D$, de degr\'e $k$ de l'alg\`ebre $\Omega(M)$ on peut 
trouver un unique $K \in  \Omega^k(M,TM)$ et un unique $L \in  
\Omega^{k+1}(M,TM)$ tels que $$\mbox{\fbox{$D = {\cal L}_K + i_L$}}$$ o\`u $ {\cal L}_K$ et 
$ i_L$ d\'efinissent des d\'erivations que nous allons caract\'eriser un 
peu plus loin.
Nous ne d\'emontrerons pas le th\'eor\`eme de Richardson et Nijenhuis mais
 d\'efinirons seulement  les d\'erivations dont il vient d'\^etre question 
(voir \cite{MDV-PM}.

Il se trouve que les \'el\'ements de $\Omega(M,TM)$ peuvent en effet agir 
par d\'erivation sur $\Omega(M)$, et ce, de deux fa\c cons distinctes.

\smallskip
La premi\`ere consiste en une g\'en\'eralisation du produit int\'erieur.
Au lieu de consid\'erer le produit int\'erieur d'une forme par un vecteur, 
on remplace le vecteur par une $k$-forme \`a valeurs vectorielles.
 En effet, soit $K \in \Omega^k(M,TM)$, $L \in \Omega^l(M,TM)$ et $\omega$ 
une forme diff\'erentielle de degr\'e $q$ sur $M$. On va d\'efinir $i_K\omega$, qui 
sera une forme diff\'erentielle de degr\'e $k+(q-1)$ (la partie ``champ de 
vecteurs'' 
pr\'esente dans $K$ fait passer de $q$ \`a $q-1$ mais les $k$ indices de 
forme demeurent). Soient $X_i$ $i\in \{1,2,\ldots,k+(q-1)$ des champs de 
vecteurs.
On pose
$$
i_K\omega(X_1,X_2,\ldots,X_{k+q-1}) =
{1\over k!(q-1)!} \sum_\sigma \epsilon_\sigma
\omega(K(X_{\sigma 1},\ldots,X_{\sigma k}),X_{\sigma(k+1)},\ldots)
$$
Notons que, agissant sur une fonction (un \'el\'ement de $\Omega^0(M)$), $i_K$ 
donne z\'ero.
On peut v\'erifier que $i_K$ d\'efini bien une d\'erivation. Celle-ci est
d'ailleurs de degr\'e $k-1$ ; ainsi $i_K \in Der_{k-1}\Omega(M)$. On peut 
d\'emontrer que toute d\'erivation de l'alg\`ebre de De Rham dont la 
restriction aux fonctions est nulle est de cette forme. Le commutateur 
gradu\'e (dans $Der(\Omega(M))$ de deux d\'erivations de ce type est une 
d\'erivation du m\^eme type. Plus pr\'ecis\'ement,
$$ \mbox{\fbox{$ [i_K,i_L] = i_{[K,L]_{NR}} $}} $$
o\`u la forme \`a valeur vectorielle $[K,L]_{NR}$ est \'egale \`a
$$
[K,L]_{NR}  =  i_K \, L -(-1)^{(k-1)(l-1)}\,  i_L \, K
$$
et o\`u on g\'en\'eralise l'action de $i_K$ sur $\Omega(M)$ \`a une action
sur $\Omega(M,TM)$ 
 en posant
$i_K(\alpha \otimes X)  =  i_K(\alpha) \otimes X$, avec $\alpha \in 
\Omega(M)$ et $X$ un champ de vecteurs. Le crochet $[,.,]$ porte le nom de 
{\sl crochet de Nijenhuis-Richardson\/}.
\smallskip

La deuxi\`eme fa\c con d'agir consiste en une g\'en\'eralisation de la 
d\'eriv\'ee de Lie. Soit encore $K \in \Omega^k(M,TM)$. On d\'efinit 
${\cal L}_K$ par
$${\cal L}_K  =  [i_K,d] =  i_K \, d - (-1)^{k-1} d \, i_K
$$
On peut v\'erifier que cet op\'erateur fournit bien une d\'erivation de 
l'alg\`ebre $\Omega(M)$. Cette d\'erivation est de degr\'e $k$ : ${\cal 
L}_K \in Der_k(\Omega(M)$ (dans le cas particulier $k=0$ on retrouve un 
r\'esultat connu).
On peut d\'emontrer que
$$
\mbox{\fbox{$
[{\cal L}_K,{\cal L}_L] = {\cal L}_{[K,L]_{FN}}
$}}$$
pour une forme \`a valeurs vectorielles bien d\'etermin\'ee not\'ee 
$[K,L]_{FN}$ qu'on appelle {\sl crochet de Fr\"olicher-Nijenhuis\/}. Pour 
des \'el\'ements d\'ecompos\'es, on a la formule de Michor
\begin{eqnarray*}
[\alpha\otimes X,\beta \otimes Y]_{FN} & = & 
\alpha \land \beta \otimes [X,Y] \, + \,
\alpha \land {\cal L}_X \beta \otimes Y \\ {} & {} & - \, 
{\cal L}_Y \alpha \land \beta \otimes X \, + \, 
(-1)^{\#\alpha}(d\alpha\land i_X \beta \otimes Y \, + \, 
               i_Y \alpha \land d\beta \otimes X)
\end{eqnarray*}
Avant de conclure ce paragraphe, il est utile de d\'efinir la notion 
suivante.
 Soit $J \in \Omega^1(M,TM)$, alors le carr\'e gradu\'e de $J$, pour le 
crochet de Fr\"olicher-Nijenhuis, est un \'el\'ement
$[J,J]_{FN}$ de $\Omega^2(M,TM)$ appel\'e  {\sl torsion de Nijenjuis}
\index{torsion de Nijenhuis} du 
vecteur-1-forme $J$. Pour justifier l'int\'er\^et port\'e \`a cette 
notion, citons seulement le r\'esultat suivant (nous n'\'etudierons pas 
les vari\'et\'es complexes dans cet ouvrage) : lorsque $J$, qui peut 
s'interpr\'eter g\'eom\'etriquement comme un champ d'endomorphismes du 
fibr\'e tangent, est une structure presque complexe ($J^2=-1$), 
l'annulation de sa torsion de Nijenhuis fourni une condition n\'ecessaire et 
suffisante pour l'int\'egrabilit\'e de cette structure (c'est \`a dire que, 
dans ce cas, la structure presque-complexe est en fait, complexe).

%%%%%%%%%%%%%%%%%%%%%%%%%%%%%%%%%%%%%%%%%%%%%%%%%%%%
%%%%%%%%%%%%%%%%%%%%%%%%%%%%%%%%%%%%%%%%%%%%%%%%%%%%%
