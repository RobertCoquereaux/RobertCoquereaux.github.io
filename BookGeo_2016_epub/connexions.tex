\chapter{Connexions}

\section{Connexions dans un fibr\'e principal}\label{sec:connexions-principales}

\subsection{Motivations}

On veut donner un sens \`a l'id\'ee de vouloir ``garder un rep\`ere
fixe'' ou de ``transporter son rep\`ere avec soi''; il s'agit l\`a d'une 
notion intuitive qui
n'a, {\it a priori}, pas de sens lorsqu'on se d\'eplace sur une vari\'et\'e 
diff\'erentiable
quelconque munie de sa seule structure de vari\'et\'e. Intuitivement, on 
souhaite disposer d'un
moyen d'assujettir le d\'eplacement d'un rep\`ere choisi (d\'eplacement qui a 
donc lieu dans
l'ensemble des rep\`eres) lorsqu'on d\'eplace l'origine de ce rep\`ere dans 
l'espace qui nous
int\'eresse. Puisque nous avons maintenant \`a notre disposition la notion 
d'espace fibr\'e, nous
nous pla\c cons dans le fibr\'e des rep\`eres correspondant \`a une vari\'et\'e $M$ (la 
base du fibr\'e en
question) et nous souhaitons donc pouvoir disposer d'une m\'ethode nous 
permettant d'associer,
\`a tout chemin allant du point $\cal P$ au point $\cal Q$ sur la base, et \`a 
tout rep\`ere au
point $\cal P$, un certain chemin dans l'espace des rep\`eres. Choisir d'une 
telle m\'ethode
revient pr\'ecis\'ement \`a choisir ce qu'on appelle une connexion dans le fibr\'e 
principal des
rep\`eres lin\'eaires. Le mot ``connexion'' -- en anglais ``connection''-- est 
bien choisi
puisqu'il nous permet effectivement de connecter (de comparer) des 
vecteurs (plus
g\'en\'eralement des \'el\'ements d'un fibr\'e associ\'e) situ\'es en des points 
diff\'erents de la
vari\'et\'e. Le cas de l'espace affine $R^n$ est tr\`es particulier puisqu'on 
peut disposer l\`a
de rep\`eres globaux permettant de comparer des vecteurs situ\'es en des 
points diff\'erents; il
existe d'autres vari\'et\'es pour lesquelles cette propri\'et\'e est \'egalement 
valable et o\`u le
choix d'un rep\`ere mobile global est possible : ce sont les vari\'et\'es
parall\`elisables d\'ej\`a mentionn\'ees dans le chapitre pr\'ec\'edent.
Nous allons, dans un premier temps, d\'efinir la notion de 
connexion de fa\c con
infinit\'esimale, comme \'etant un moyen d'associer un d\'eplacement 
infinit\'esimal dans l'espace
des rep\`eres \`a un d\'eplacement infinit\'esimal du point de base (c'est \`a dire 
du point o\`u le
rep\`ere est situ\'e). En fait, la seule structure utilis\'ee dans la d\'efinition 
de la notion de
connexion est celle de fibr\'e principal et tout ce qu'on \'ecrira aura 
encore un sens
si on remplace le fibr\'e des rep\`eres par un fibr\'e principal quelconque 
(cela dit, il est bien
commode de visualiser les choses en utilisant des rep\`eres, nous nous 
permettrons donc
d'utiliser le mot ``rep\`ere'' pour d\'esigner un \'el\'ement d'un espace fibr\'e 
principal
quelconque).

\subsection{Distributions horizontales \'equivariantes}

Soit $P$ le fibr\'e principal des rep\`eres lin\'eaires de la vari\'et\'e $M$, avec 
groupe structural
$GL(n)$. Soit $e \in P$ un rep\`ere au point ${\cal P} \in M$
(attention $e$ ne d\'esigne pas l'\'el\'ement neutre de $G$ !).
 L'espace 
tangent $T(P,e)$,
tangent \`a $P$ en $e$ est, intuitivement, l'ensemble des d\'eplacements 
infinit\'esimaux de
rep\`eres, issus du rep\`ere $e$. Nous savons d\'ej\`a nous d\'eplacer dans la 
direction
verticale puisque cela correspond \`a un d\'eplacement infinit\'esimal induit 
par l'action du
groupe $GL(n)$ : on fait ``tourner'' le  rep\`ere $e$ sans
faire bouger le point $\cal P$. Le sous espace vertical $V(P,e)$ est donc 
d\'ej\`a bien d\'efini.
Une connexion sera caract\'eris\'ee par le choix d'un sous-espace 
suppl\'ementaire \`a $V(P,e)$
dans $T(P,e)$, sous-espace qui sera bien \'evidemment qualifi\'e 
d'``horizontal'' et not\'e
$H(P,e)$. Par ailleurs, on veut que ce choix puisse \^etre effectu\'e, de 
fa\c con continue et
diff\'erentiable, pour tout rep\`ere, c'est \`a dire en tout point $e$ de $P$. 
Enfin, on veut que
ce choix soit \'egalement \'equivariant sous l'action du groupe structural: 
nous savons que
$GL(n)$ op\`ere transitivement sur les fibres (par exemple sur les rep\`eres 
au point $\cal
P$), l'image  du rep\`ere $e$ sous l'action d' un \'el\'ement $g$ du groupe 
structural est 
un rep\`ere $e.g$ de la m\^eme
fibre ($R_g (e) = e.g$) et l'application tangente $dR_g \equiv (R_g)_*$ en $e$ envoie 
donc l'espace
tangent $T(P,e)$ dans l'espace tangent $T(P,e.g).$
Il est commode de noter simplement
$T(P,e)g  =  (R_g)_* T(P,e) \subset T(P,eg)$ puisque, formellement
$\partial (ge)/\partial e = g$.
La propri\'et\'e d'\'equivariance requise signifie simplement ceci:
 on veut que le choix de l'espace horizontal $H(P,e.g)$ en $e.g$
puisse \'egalement \^etre obtenu en utilisant l'action du groupe structural 
sur les fibres, en
d'autres termes on impose $H(P,e.g) = H(P,e).g$\par
Noter que la discussion qui pr\'ec\`ede ne d\'epend pas du type particulier de 
fibr\'e principal
consid\'er\'e et le lecteur est invit\'e \`a remplacer partout le groupe $GL(n)$ 
par un groupe de
Lie quelconque (et le mot ``rep\`ere'' par les mots ``\'el\'ement du fibr\'e 
principal $P$''). Le
choix, en tout point $e$ d'un fibr\'e principal $P$, d'un tel espace
vectoriel $H(P,e)$ suppl\'ementaire \`a $V(P,e)$, c'est \`a dire,  $T(P,e) = 
V(P,e) \oplus H(P,e)$, est
d\'esign\'e sous le nom de {\sl distribution horizontale} {\index{distribution horizontale}}; noter que le sens 
de ce mot
``distribution'' n'a ici aucun rapport avec celui utilis\'e en th\'eorie de la 
mesure (la
th\'eorie des distributions!). Notre premi\`ere d\'efinition d'une {\sl 
connexion principale}{\index{connexion principale}} est
donc la suivante: c 'est la donn\'ee, dans un fibr\'e principal $P$, d'une 
{\sl distribution
horizontale \'equivariante}{\index{distribution
horizontale \'equivariante}} sous l'action du groupe structural.

\subsection{Rel\`evement horizontal}

Soit $P=P(M,G)$ un fibr\'e principal, on dispose donc d'une projection $\pi: 
P \rightarrow M$
et donc \'egalement de son application tangente $\pi_*$. Cette application 
lin\'eaire envoie
l'espace tangent $T(P,e)$ sur l'espace tangent $T(M,\pi(e))$; en d'autres 
termes, elle nous
permet d'associer, \`a tout d\'eplacement infinit\'esimal d'un rep\`ere $e$ dans 
l'espace des
rep\`eres, le d\'eplacement infinit\'esimal correspondant du point $x = \pi(e)$ 
dans la vari\'et\'e
$M$ ($x$ est le point o\`u le rep\`ere est centr\'e). Comme nous l'avons 
remarqu\'e \`a plusieurs
reprises, \'etant donn\'ee une application  d'une vari\'et\'e dans une autre, nous 
pouvons faire
``voyager'' les vecteurs dans la m\^eme direction --il s'agit d'un
 ``{\it push-forward\/}''--  et les
formes dans la direction oppos\'ee (``{\it pull-back\/}'')\par
De fa\c con g\'en\'erale, soit $\gamma(t)$ une courbe dans $M$, on dira qu'une 
courbe
$\Gamma(t)$ dans $P$ est un {\sl rel\`evement}{\index{rel\`evement}} de $\gamma(t)$ si $\Gamma$ 
 se projette sur $\gamma$ (intuitivement ``on'' --un 
voyageur qui se
prom\`ene sur $M$-- s'est choisi un rep\`ere mobile quelconque en tout point 
du chemin qu'il
suit). Supposons maintenant que nous nous sommes donn\'es une connexion (au 
sens donn\'e dans la sous section pr\'ec\'edente) dans le fibr\'e
principal $P$, nous savons donc d\'efinir l'horizontalit\'e des vecteurs de 
$TP$; on dira qu'un
rel\`evement est horizontal si les vecteurs tangents au rel\`evement sont
horizontaux. Consid\'erons une courbe $\gamma(t)$ dans $M$ allant de 
${\cal P}_0 = \gamma(0)$ \`a
${\cal P}_1 = \gamma(1)$ et son rel\`evement horizontal issu de $e_0 \in P$, 
c'est \`a dire la
courbe $\Gamma(t)$ dans $P$. En utilisant les th\'eor\`emes habituels
concernant
les \'equations diff\'erentielles, on montre qu'il y a unicit\'e du 
rel\`evement $\Gamma$ de $\gamma$ issus de $e_0$. On dira que $e_1  =  
\Gamma(1)$ est le {\sl transport\'e par
parall\'elisme}
 {\index{transport par
parall\'elisme}} de $e_0$ le long de $\gamma$ par rapport \`a la connexion 
choisie.


 Nous
d\'efinirons l'{\sl application de rel\`evement horizontal}{\index{application de rel\`evement horizontal}} comme suit: soit 
$x$ un point de
$M$ et $e$ un rep\`ere quelconque en $x$ (c'est \`a dire un \'el\'ement de la 
fibre au dessus de
$x$), soit $v_x$ un vecteur appartenant \`a l'espace tangent \`a $M$ en $x$, on 
d\'esignera par
$\lambda_e(v_x)$ l'\'el\'ement de l'espace tangent $T(P,e)$ qui, d'une part, est 
horizontal et
qui, d'autre part, se projette sur $v_x$ gr\^ace \`a l'application $\pi_*$. Il 
est bien \'evident
que ce vecteur est unique: Il suffit de choisir un relev\'e quelconque de 
$v_x$ en $e$ (c'est
\`a dire un vecteur quelconque $V_e$ de $T(P,e)$ qui se projette sur $v_x$, 
puis de le
d\'ecomposer en une partie verticale $V^v_e$ et une partie horizontale 
$V^h_e =
\lambda_e(v_x)$ (c'est le vecteur cherch\'e) en utilisant la d\'ecomposition de 
$T(P,e)$ en deux
sous-espaces suppl\'ementaires.\par

La distribution horizontale n'\'etant pas quelconque (elle est \'equivariante !), l'application de rel\`evement horizontal n'est pas quelconque non plus : le
relev\'e du vecteur $v$ en $eg$ doit co\"incider avec l'image, par l'application
tangente  du relev\'e de $v$ en $e$. En clair,
 $\lambda_{eg}(v)=(R_g)_* (\lambda_e(v)),$ qu'on peut \'ecrire plus
simplement
$$\lambda_{eg}(v)= (\lambda_e(v) g)$$
On vient de d\'ecrire la fa\c con dont on peut, gr\^ace \`a la donn\'ee d'une 
connexion, relever les
vecteurs tangents de $M$ \`a $P$. Bien entendu, \`a condition de travailler de fa\c con 
duale, on peut
faire quelque chose d'analogue avec les formes diff\'erentielles. Plus 
pr\'ecis\'ement, le ``pull-back'' de
la projection $\pi$ est d\'efini sur les vecteurs cotangents \`a $M$ et \`a 
valeurs dans le fibr\'e
cotangent de $P$. La donn\'ee d'une connexion (et donc de l'application de 
rel\`evement
horizontal) permet, \`a l'inverse, de projeter les formes de $T^*P$ sur les formes de
$T^*M$.\par
Toute cette discussion peut \^etre r\'esum\'ee \`a l'aide de la figure ci-dessous. 

\begin{figure}[htbp]
 %\def\epsfsize#1#2{0.5#1}
\epsfxsize=8cm
$$
    \epsfbox{connexion.eps}
$$
\caption{Rel\`evement horizontal et connexion}
\label{fig:connexion}
\end{figure}

\subsection{Forme de connexion}

Il existe plusieurs fa\c cons de d\'efinir la notion de
connexion et nous venons d'en mentionner deux
(distribution horizontale \'equivariante et application de
rel\`evement horizontal) dont l'interpr\'etation g\'eom\'etrique
est intuitive; en pratique, cependant, c'est une troisi\`eme
m\'ethode qui va nous permettre de traduire cette
notion sous forme analytique. Nous supposons donc donn\'ee
sur l'espace fibr\'e principal $P=P(M,G)$ une distribution
horizontale \'equivariante. La {\sl forme de connexion\/}
 {\index{forme de connexion}} $\omega$
est une forme sur $P$ et \`a valeurs dans l'alg\`ebre de Lie
${\frak g}$ du groupe structural (nous verrons que cette forme doit, en outre, \^etre \'equivariante). 
Elle est d\'efinie comme
suit. Soit $e$ un \'el\'ement de $P$ (un rep\`ere de $M$), et
$X_\alpha$ une base de $Lie(G)$. Nous d\'ecomposons l'espace
tangent $T(P,e)$ en un sous espace vertical $T^v(P,e)=V(P,e)$
engendr\'e par les champs fondamentaux $\hat X_\alpha$ et un sous espace
horizontal $T^h(P,e)=H(P,E)$ d\'efini par la donn\'ee de la
distribution horizontale. On pose

\begin{eqnarray*}
\omega_e(\hat X_\alpha(e)) &  =  & X_\alpha \\
Ker \, \omega_e & = & T^h(P,e)
\end{eqnarray*}

Ainsi donc, on peut d\'ecomposer tout vecteur $V \in T(P,e)$, comme suit : 
$V=V^\alpha \hat
X_\alpha(e) + V^h$ avec $V^h \in H(P,e)$ et
$\omega(V)=V^\alpha X_\alpha$. La forme $\omega$ est donc
bien d\'efinie. Sa signification est claire :
si on effectue un d\'eplacement infinit\'esimal du rep\`ere $e$
(sans changer l'origine), on associe \`a ce d\'eplacement la
rotation infinit\'esimale $V^\alpha X_\alpha$
correspondante; si, au contraire, on effectue un
d\'eplacement infinit\'esimal du rep\`ere $e$ en d\'epla\c cant
l'origine mais sans faire ``tourner'' le rep\`ere, on obtient
$\omega(V) = 0$. Bien \'evidemment, on peut analyser les
choses diff\'eremment en d\'ecidant que la connexion est
d\'efinie par la forme de connexion $\omega$, la rotation du
rep\`ere $e$ lors d'un d\'eplacement infinit\'esimal $V$ \'etant
pr\'ecis\'ement mesur\'ee par la rotation infinit\'esimale
$\omega(V)$ et ceci donne alors un sens au verbe ``tourner''.
 La distribution horizontale \'etant
\'equivariante, il s'ensuit que les noyaux de $\omega$
en $e$ et en $eg, g\in G$ sont reli\'es par $$
Ker \, \omega_{eg} = (Ker \, \omega_e) g
$$
Ici encore, l'application tangente est simplement not\'ee $g$.

Nous reviendrons un peu plus loin (en \ref{sec:connexions-principales-2}) sur la th\'eorie des 
connexions dans les fibr\'es principaux mais nous allons, pour des raisons
aussi bien p\'edagogiques (c'est plus simple!) que pratiques (on m\`ene en 
g\'en\'eral les calculs sur la base), passer \`a une expression locale de la 
forme de connexion, puis, \`a partir de l\`a, d\'evelopper la th\'eorie dans les 
fibr\'es vectoriels associ\'es.

\subsection{ Ecriture locale de la forme de connexion: le potentiel de  jauge}

 Le fait que tout
fibr\'e principal soit localement trivial va nous permettre
- moyennant le choix d'une section locale - d'\'ecrire la
forme de connexion  $\omega$, non pas comme une forme sur
$P$ \`a valeurs dans ${\frak g}$ mais comme une forme sur $M$ \`a
valeurs dans ${\frak g}$. Choisissons donc une section locale
$M \stackrel{e}{\mapsto} P$ et notons $de: TM \mapsto
TP$ l'application lin\'eaire tangente. Rappelons que, si
$P$ est un fibr\'e de rep\`eres, la section locale $e(x)$ n'est
autre qu'un rep\`ere mobile $e(x)=(e_\mu(x))$ d\'efini dans le
domaine d'un certain ouvert et que, dans le cas o\`u $P$
d\'esigne le fibr\'e des rep\`eres d'un certain ``espace interne''
(terminologie utilis\'ee en physique des particules) on
dira plut\^ot que $e(x)$ est un choix de jauge. Supposons
qu'on s'est fix\'e une connexion caract\'eris\'ee par la forme de
connexion $\omega$, on d\'efini le {\sl potentiel de jauge\/}
 {\index{potentiel de jauge}} $A$
$$
\mbox{\fbox{$ A(v) = \omega_e(e_*(v)), \quad  v \in TM $}}
$$
o\`u $e_* \equiv de$ d\'esigne l'application tangente \`a la section
locale $e$.


$A$ est donc une $1$-forme sur $M$ \`a valeurs dans ${\frak g}$.
Soit $\{X_\alpha\}$ une base de ${\frak g}$ et $\{\partial
/ \partial x^\mu \}$ une carte locale sur $M$, on pourra
\'ecrire :
 $$
\mbox{\fbox{$ A = A_\mu^\alpha X_\alpha dx^\mu$}}
$$

En fait, il faudrait noter cet objet  ${}^eA$ et
non pas $A$ de fa\c con \`a se rappeler du fait que la d\'efinition de
$A$ d\'epend du choix de la section locale $e$ mais on omet
g\'en\'eralement d'y faire r\'ef\'erence. Par contre, il est tr\`es
important de savoir comment se transforme $A$ lorsqu'on
effectue un choix diff\'erent. Posant $A  =  {}^eA$ et 
$A^\prime  =  {}^{e^\prime}A$, et supposant que
 $e^\prime(x) = e(x).g(x)$ avec $g(.): M \mapsto G$, nous verrons un peu
plus loin que
la propri\'et\'e d'\'equivariance de $\omega$ se traduit
pour $A$, par la propri\'et\'e suivante :
$$\mbox{\fbox{$A^\prime = g^{-1} A g  + g^{-1} d g$}}$$

Pour finir, notons qu'en pratique les calculs sont
effectu\'es sur la base $M$ et non sur le fibr\'e $P$; en
d'autres termes on pr\'ef\`ere utiliser
le potentiel de jauge $A$ plut\^ot que la forme de connexion
$\omega$, quitte \`a devoir recoller les morceaux puisque le
fibr\'e $P$ consid\'er\'e n'est pas en g\'en\'eral trivial et qu'il
faut donc d\'efinir $A$ sur toutes les cartes d'un atlas de
$M$. Le potentiel de jauge $A$ est quelquefois d\'esign\'e sous
le nom ``pull back de la forme de connexion'' et il poss\`ede
des interpr\'etations physiques vari\'ees, d\'ependant, bien
s\^ur, de la nature du fibr\'e principal $P$ (et donc du
groupe structural $G$).
Citons quelques unes de ces interpr\'etations ainsi que la
terminologie correspondantes :

$G=U(1)$, $A$ est le potentiel \'electromagn\'etique (ou champ
de photons)

$G=SU(3)$, $A$ est le potentiel chromodynamique (ou champ
de gluons)

$G=SU(2) \times U(1)$, $A$ est le potentiel du champ
\'electro-faible (champ des bosons $\gamma$, $W^\pm$ et $Z$).

Dans le cas des th\'eories gravitationnelles o\`u $G$ d\'esigne
$GL(n)$ et plus g\'en\'eralement un groupe de changement de
rep\`eres d'une vari\'et\'e de dimension $n$, on parle plut\^ot de
``symboles de Christoffel'' pour d\'esigner les composantes de $A$ ; nous y 
reviendrons dans la
section consacr\'ee aux connexions lin\'eaires.


Le lecteur aura compris (chose qu'il sait sans doute
depuis longtemps) que les potentiels de jauge -et la
th\'eorie des connexions en g\'en\'eral - permettent de
repr\'esenter math\'ematiquement toutes les forces 
fondamentales de la physique.



\section{Connexions dans les  fibr\'es vectoriels
associ\'es}

\subsection{Matrice de connexion $A^i_j$, coefficients de
connexion $A^i_{j\mu}$}\label{sec:matrice-de-connexion}

Soit $A=A_\mu^\alpha X_\alpha dx^\mu$ le potentiel de jauge
d\'efinissant une forme de connexion $\omega$ sur le fibr\'e
principal $P=P(M,G)$ et soit $E=P\times_\rho V$ un fibr\'e
vectoriel associ\'e \`a $P$ via la repr\'esentation $\rho$ sur
l'espace vectoriel $V$ de dimension $p$. En utilisant la repr\'esentation
$\rho$ nous allons repr\'esenter le potentiel de jauge
lui-m\^eme et obtenir ainsi une notion de connexion pour tout
fibr\'e vectoriel associ\'e.  Soit
$\{X_\alpha\}_{\alpha\in\{1,\,dim(G)\}},$
une base de ${\frak g}$. Nous d\'esignerons
encore par $\rho$ la repr\'esentation de ${\frak g}$
correspondant \`a celle de $G$; ainsi
$\rho(X_\alpha)=(\rho(X_\alpha)^i_j)$ est une matrice
$p \times p$ d\'ecrivant un endomorphisme de l'espace
vectoriel $V$.

 Dans ce paragraphe
nous noterons :

 
$\mu, \nu, \rho \ldots$ les indices de base (vari\'et\'e $M$
de dimension $d$)

$i,j,k \ldots$ les indices de fibre (espace vectoriel $V$
de dimension $p$) 

$\alpha, \beta, \gamma \ldots$ les indices d'alg\`ebre de Lie ($
Lie \, G$ de dimension $n$) 

\noindent
L'image $\rho(A)$ du potentiel de jauge
par la repr\'esentation $\rho$ s'appelle la {\sl matrice de
connexion} {\index{matrice de
connexion}}
 $$
\mbox{\fbox{$
\rho(A) = A_\mu^\alpha \, \rho(X_\alpha)dx^\mu
$}}
$$
 et ses \'el\'ements sont les quantit\'es $$\rho(A)^i_j
 =  A^\alpha (\rho(X_\alpha))^i_j$$ avec $$A^\alpha = A^\alpha_\mu
dx^\mu$$

La repr\'esentation $\rho$ \'etant choisie une fois pour toutes (le fibr\'e
vectoriel $E$ \'etant choisi) on peut omettre le symbole
$\rho$ lui-m\^eme, et si nous posons
$$
\mbox{\fbox{$
T_\alpha  =  \rho(X_\alpha)
$}}
$$
alors
 $$A^i_j = A^\alpha (T_\alpha)^i_j$$

Ses \'el\'ements de matrice sont des $1$-forme puisque 
$$
\mbox{\fbox{$
A^i_j
= A^i_{j \mu} dx^\mu \quad {\mbox{avec}} \quad  A^i_{j \mu} =
A^\alpha_\mu(T_\alpha)^i_j
$}}
$$

 Les  nombres $A^i_{j \mu}$ sont les {\sl coefficients
de connexion} {\index{coefficients
de connexion}}.
 Revenons une fois de plus sur ces probl\`emes 
de notations, de terminologie et d'habitudes: les
physiciens des particules utilisent les $A_\mu^\alpha$, les
sp\'ecialistes de la gravitation pr\'ef\`erent les $A^i_{j
\mu}$ (la relation pr\'ec\'edente permettant de faire le lien
entre les deux); ces derniers ont d'ailleurs 
l'habitude  de noter $\Gamma$ (plut\^ot que $A$) les
coefficients de connexion, ils utilisent donc des
$\Gamma^i_{j \mu}$, les indices ${}^i_j$ \'etant les indices
de fibre et $\mu$ l'indice de forme. L\`a o\`u les choses se
compliquent, c'est que le fibr\'e $E$ peut d\'esigner le fibr\'e
tangent $TM$ et que, dans ce cas, les indices de fibre
peuvent appartenir au m\^eme jeu d'indices que les
indices de forme (on a donc des objets
$\Gamma^\nu_{\rho \mu}$) mais il faut toujours se rappeler qui est
qui. Les conventions, comme d'habitude, n'\'etant pas
universelles, le lecteur est pri\'e de se rappeler que, pour
nous, l'indice de base c'est \`a dire encore l'indice de forme (not\'e $\mu$
ci-dessus) est situ\'e en bas, et en derni\`ere position.
 Nous reviendrons un peu plus loin sur le cas particulier
$E=TM$ (connexions lin\'eaires). En attendant, il convient
de se rappeler que, dans la notation $A^i_{j\mu},$ l'indice
$\mu$ se r\'ef\`ere au choix d'un rep\`ere naturel associ\'e \`a une
carte ($e_\mu = {\partial \over \partial x^\mu}$) ou d'un
rep\`ere mobile quelconque ($[e_\mu,e_\nu]=f_{\mu\nu}^\rho
e_\rho$) et que les indices $i,j$ se r\'ef\`erent au choix
d'une base $(e_i(x))$ dans la fibre de $E$ situ\'ee au point
$x$ de $M$.

\subsection{Diff\'erentielle covariante des sections $\nabla$, d\'eriv\'ee covariante $\nabla_\mu$, parall\'elisme}\label{sec:differentielle-covariante}


La {\sl diff\'erentielle covariante\/} {\index{diff\'erentielle covariante}}
 $$
\mbox{\fbox{$
\nabla: \Gamma(E)
\rightarrow \Omega^1(M,E) =  \Gamma(E) \otimes \Omega^1(M)
$}}
$$
 transforme les
sections de $E$ en sections--$1$--formes et v\'erifie, par
d\'efinition, la propri\'et\'e
$$
\mbox{\fbox{$
\nabla(vf) = (\nabla v) f + v \otimes df
$}}
$$ 
pour toute section $v$ de $E$ et toute fonction $f$ d\'efinie\footnote{Bien \'evidemment,
la notation $v f$ d\'esigne simplement le produit de la fonction $f$ par la section $v$, et donc $v f = f v$, avec $(v f)(x)=v(x)f(x)$;
nous \'ecrirons n\'eanmoins souvent les fonctions 
``\`a droite'' pour des raisons li\'ees au d\'esir de pr\'esenter \`a la fin
de cet ouvrage certaines g\'en\'eralisations non commutatives de la g\'eom\'etrie.}  sur $M$. 
L'op\'erateur $\nabla$ sera donc connu d\`es qu'on conna\^itra la valeur de
$\nabla e_i$ o\`u les $e_i: x\in M \rightarrow e_i(x) \in E$
d\'esignent $p$ sections ind\'ependantes dans un voisinage de
$x$ ; l'ensemble $\{e_i(x)\}$ est donc une base de
l'espace vectoriel $E_x$. On pose (voir aussi la discussion 
\ref{sec:produit-tensoriel-et-formes})
$$
\mbox{\fbox{$
\nabla e_i = e_j   A^j_i
$}}
$$ o\`u les $A^j_i$ sont les \'el\'ements de la matrice de
connexion (par rapport au m\^eme choix de base). Soit $v$
une section quelconque de $E$, ainsi $v = e_j v^j$ o\`u les
$v^j$ --les composantes de $v$ suivant la base $\{e_j\}$--
sont des fonctions sur $M$. On obtient donc 

\begin{eqnarray*}
\nabla  v & = &\nabla(e_j v^j) = (\nabla e_j) v^j + e_j   dv^j \\
         & = & e_i   A^i_j v^j + e_i  dv^i =
 e_i   (A^i_j v^j+dv^i) \\
        &  = & e_i   (A^i_{j\mu} v^j + \partial_\mu v^i) dx^\mu
\end{eqnarray*}

Une section $v$ est dite {\sl parall\`ele\/} {\index{parall\`ele}}
 (ou transport\'ee par
parall\'elisme) lorsque $\nabla v = 0$

 Puisque $\nabla v$ poss\`ede
un indice de forme (c'est un \'el\'ement de
$\Gamma(E)\otimes\Omega^1(M)$), on peut l'\'evaluer sur les
vecteurs tangents. Soit $\xi = \xi^\mu \partial_\mu$
un vecteur tangent. La {\sl d\'eriv\'ee covariante\/}
 {\index{d\'eriv\'ee covariante}} de $v$ dans la
direction $\xi$ se note $\nabla_\xi v$ et s'obtient en
\'evaluant $\nabla v$ sur $\xi$:

\begin{eqnarray*}
\nabla_\xi v & = & \langle\nabla v,\xi\rangle =
\langle e_i   (A^i_{j\mu}v^j + v^i_{,\mu})dx^\mu,\xi^\nu
\partial_\nu \rangle \\
{} & =  & e_i(A^i_{j\mu}v^j + v^i_{,\mu})\xi^\mu
\end{eqnarray*}

Notons que $\nabla_\xi v$, ainsi que $v$,
est une section de $E$ alors que $\nabla v$ est une
section-$1$-forme. On note souvent $v^i_{;\mu}$ les
composantes de $\nabla_\mu v  =  \nabla_{{\partial
\over \partial x^\mu}}v$ par rapport au rep\`ere $\{e_i\}$ de
la fibre :
 $$
\mbox{\fbox{$
\nabla_\mu v = e_i v^i_{;\mu} = \langle
\nabla v, \partial_\mu \rangle {\mbox{\rm \ avec \ }}
v^i_{;\mu} =
v^i_{,\mu} + A^i_{j\mu} v^j
$}}
$$
Remarque: l'op\'erateur que nous appelons ``diff\'erentielle covariante'' est parfois d\'esign\'e, dans certains ouvrages, sous le nom d' ``op\'erateur de d\'eriv\'ee covariante''.

\subsection{Remarques concernant les notations}\label{sec:notations}

\subsubsection{Mise en garde concernant la notation $\nabla_\mu$}
Nous attirons l'attention du lecteur sur le fait que la moiti\'e des 
physiciens n'utilisent que des objets ``index\'es''. Ces derniers ne 
consid\`erent jamais ni $v$ ni $\nabla_\mu v$ mais seulement leurs 
composantes $v^i$ et $v^i_{;\mu}$. Cette habitude n'entra\^\i ne
g\'en\'eralement aucune 
confusion. Par contre, cela devient un probl\`eme si on d\'ecide, 
simultan\'ement, d'utiliser \'egalement la notation $\nabla_\mu$ et de 
d\'ecider que $\nabla_\mu v^i$ est un synonyme de $v^i_{;\mu}$. Nous 
estimons qu'il s'agit l\`a d'un abus de notations particuli�rement dangereux pouvant 
facilement conduire \`a des erreurs. En effet, $v^i$ est une composante de 
$v$, c'est \`a dire une fonction de $x$ ; sa d\'eriv\'ee covariante existe 
bien, mais, puisqu'il s'agit d'une simple fonction, elle est \'egale \`a sa 
d\'eriv\'ee ordinaire, ainsi $\nabla_\mu v^i = \partial_\mu v^i = 
v^i_{,\mu}$, ce qui n'est pas du tout \'egal \`a $v^i_{;\mu}$. Si le lecteur 
ne souhaite pas manipuler des objets comme $v$ ou $\nabla_\mu v$ mais seulement leurs 
composantes, nous lui sugg\'erons tr\`es fortement de se contenter de la 
notation ``point-virgule''.

\subsubsection{Remarque sur les produits tensoriels}
\label{sec:produit-tensoriel-et-formes}

Lors de la d\'efinition de $\nabla e_i = e_j A^j_i$ en 
\ref{sec:differentielle-covariante}, le lecteur a pu \^etre
surpris de l'absence du signe de produit tensoriel entre le vecteur $e_j$ 
qui est une section locale du fibr\'e vectoriel consid\'er\'e et la $1$-forme 
$A^j_i$ qui est une section locale du fibr\'e cotangent $T^*M$ (puisque 
$A^j_i = A^j_{i\mu} dx^\mu$).

Un peu plus loin, le lecteur a pu \^etre de nouveau surpris, lors du 
calcul de $\nabla v$ et de $\nabla_\xi v$, car nous avons all\`egrement 
commut\'e les formes sur $M$ (par exemple $e^\mu = dx^\mu$) et les sections 
de $E$ (les $e_i$).

Mis \`a part le fait qu'il existe un isomorphisme canonique entre $\Gamma(E)\otimes 
\Omega^1(M)$ et $\Omega^1(M)\otimes \Gamma(E)$ et que le produit 
tensoriel utilis\'e est un produit au dessus de $C^\infty(M)$ et non au 
dessus de $\RR$, les manipulations 
pr\'ec\'edentes, sur lesquelles nous ne nous \'etendrons pas, sont
justifi\'ees par le fait qu'il est possible de remplacer, dans la plupart 
des calculs de g\'eom\'etrie diff\'erentielle, l'alg\`ebre commutative 
$C^\infty(M)$ par l'alg\`ebre commutative gradu\'ee $\Omega(M)$.

Les espaces $E$ et $T^*M$ \'etant, en g\'en\'eral bien distincts, nous d\'ecidons 
d'identifier par exemple
 $dx^\mu e_i \otimes e_j \ldots$ et  $ e_i \otimes e_j \ldots dx^\mu $
 et nous n'utiliserons pas de symbole de produit tensoriel entre les 
 $p$-formes sur $M$ et les sections de $E$, traitant ainsi les $dx^\mu$ comme des 
 fonctions. Par ailleurs nous \'ecrirons toujours les formes ``en 
 dernier''.
 
 Cette identification nous permet d'\'ecrire aussi bien
 $$
 \nabla(v\otimes w) = \nabla(v)\otimes w + v \otimes \nabla(w)
 $$
 que
 $$
 \nabla_\mu(v\otimes w) = \nabla_\mu(v)\otimes w + v \otimes \nabla_\mu(w)
 $$
 Les auteurs s'interdisant d'effectuer cette identification (pourtant 
 sans danger !) imposent la r\`egle de Leibniz pour $\nabla_\mu$ mais ne 
 peuvent pas l'imposer pour $\nabla$.
 
 Il existe cependant un cas particulier o\`u il existe une confusion possible 
 entre les fibr\'es $E$ et $T^*M$, c'est pr\'ecis\'ement le cas o\`u on choisit $E = T^*M.$
 Dans ce cas, il faut se rappeler ``qui est qui'', c'est \`a dire quels 
 sont les indices de forme et quels sont les indices de fibre. Cette 
 possible identification permet en fait d'enrichir la th\'eorie.
 Nous y reviendrons.


\subsection{Loi de transformation des coefficients de
connexion}\label{sec:transformation-connexion}

Soient $e=(e_i)$ et $e^\prime=(e^\prime_{i^\prime})$ deux
rep\`eres dans les fibres. L'un s'obtient \`a partir de l'autre
par une transformation lin\'eaire $\Lambda$ : $$e^\prime_{i^\prime} = e_j
\Lambda^j_{i^\prime}$$ o\`u $\Lambda^j_{i^\prime}$ est la
matrice de changement de rep\`ere (on \'ecrira simplement
$e^\prime = e . \Lambda$) et soient $A^i_j$ et
${A^\prime}^{i^\prime}_{j^\prime}$ les matrices de
connexion correspondantes. On a $$
\nabla e^\prime = e^\prime A^\prime = e. \Lambda A^\prime $$
c'est \`a dire
$$
\nabla(e . \Lambda) = (\nabla
e).\Lambda + e. d\Lambda = e A \Lambda + e d\Lambda
$$ et donc
$$
A^\prime = \Lambda^{-1} A \Lambda + \Lambda^{-1} d \Lambda
$$
Si on \'ecrit explicitement les indices, on a 
$$
{A^\prime}^{i^\prime}_{j^\prime \mu} = 
(\Lambda^{-1}){}^{i^\prime}_k A^k_{l\mu}
\Lambda^l_{j^\prime} + (\Lambda^{-1}){}^{i^\prime}_k
\partial_\mu \Lambda^k_{j^\prime} 
$$
 Cette relation traduit simplement l'\'equivariance de la
forme de connexion $\omega$.
 On dit quelquefois
(terminologie un tantinet archa\"\i que) que $A^i_{j\mu}$ ``ne 
se transforme pas comme  un tenseur''$\ldots$ les indices $i,j$
d'une part et $\mu$ d'autre part \'etant de nature
diff\'erente, cette remarque n'est pas trop surprenante!
Notons que, dans la relation pr\'ec\'edente, nous avons
transform\'e le rep\`ere ``interne'' $e_i$ (base de la fibre)
mais pas le rep\`ere ``externe'' $e_\mu={\partial \over \partial x^\mu}$
 (base de $T(M,x)$) ; on
pourrait {\it aussi} changer de base dans $T(M,x)$ et poser 
$e_{\mu^\prime} = e_\mu L^\mu_{\mu \prime}$.

On obtient alors imm\'ediatement
$$
{A^\prime}^{i^\prime}_{j^\prime \mu^\prime}=
L^\mu_{\mu^\prime}(\Lambda^{-1})^{i^\prime}_k A^k_{l \mu} 
\Lambda^l_{j^\prime} + 
L^\mu_{\mu^\prime}(\Lambda^{-1})^{i^\prime}_k \partial_\mu 
\Lambda^k_{j^\prime}
$$

\subsection{D\'erivation covariante des sections du fibr\'e dual}

Apr\`es avoir d\'efini $\nabla$ sur les sections de $E$ gr\^ace \`a la
relation $\nabla e_i = e_j A^j_i$, $\{e_i(x)\}$   d\'esignant une base de la 
fibre au
point $x$, nous consid\'erons le fibr\'e dual $E^*$ et l'action
de $\nabla$ sur ses sections. D\'esignons respectivement par
$\sigma$ et $v$  une forme diff\'erentielle et un champ de vecteurs, on aura:
$\nabla \langle \sigma, v \rangle = \langle \nabla \sigma, v \rangle  + \langle \sigma, \nabla v \rangle$.
Notons $\{e^i(x)\}$ la base duale au point
$x$ ; alors $\langle e^j, e_i \rangle = \delta^i_j.$

La quantit\'e $\delta^i_j$ est une fonction constante, et donc $\nabla 
\delta^i_j = 0$.
On impose 
$$ 
\langle \nabla e^j, e_i \rangle + \langle e^j, \nabla e_i \rangle = 
\nabla \delta^i_j = 0
$$
Cette relation conduit \`a poser
$$
\mbox{\fbox{$
\nabla e^j = -  A^i_k e^k
$}}
$$
V\'erifions que c'est bien le cas
$$
\langle -  A^j_k e^k, e_i \rangle + \langle e^j,e_k   
A^k_i \rangle =
-A^j_k \delta^k_i + \delta^j_k A^k_i = - A^j_i + A^j_i = 0     
$$
On aura donc un signe $-$ lorsqu'on ``corrige'' un indice covariant.
 
\subsection{D\'erivation  covariante dans les 
puissances tensorielles d'un fibr\'e vectoriel}

Notre but est ici d'\'etudier la d\'erivation covariante des \'el\'ements de
 $E^{\otimes p} \otimes  (E^*)^{\otimes q}$.

Consid\'erons le fibr\'e vectoriel $E \otimes E$, c'est un fibr\'e associ\'e
comme un autre... Ses \'el\'ements peuvent s'\'ecrire
$$t = e_i \otimes e_j \,  t^{ij} $$
Si $X \in Lie \, G$ agit via la repr\'esentation $\rho: X \in Lie \, G 
\mapsto \rho(x) \in End \, V$ sur l'espace
vectoriel $V$, il agit via $\rho(x) \otimes 1 + 1 \otimes \rho(x) $ sur 
l'espace vectoriel
$V  \otimes V$, d'o\`u il s'ensuit que l'op\'erateur $\nabla$ v\'erifie la
propri\'et\'e $$\nabla(v_1 \otimes v_2) = \nabla(v_1) \otimes v_2 + v_1 
\otimes (\nabla v_2)$$ Le lecteur ayant quelques notions sur les 
alg\`ebres de Hopf aura reconnu la relation directe existant entre la 
r\`egle de Leibniz pour $\nabla$ et le coproduit standard sur l'alg\`ebre enveloppante de $Lie G$.

On \'ecrira par exemple, pour $t = e_i \otimes e_j \, t^{ij}$, 

\begin{eqnarray*}
\nabla t & = & (\nabla e_i) \otimes e_j \, \, t^{ij} + e_i \otimes (\nabla 
e_j) 
\, t^{ij} + e_i \otimes e_j dt^{ij} \\
& = & A_i^k e_k\otimes e_j \, t^{ij} + e_i \otimes A_j^k e_k \, t^{ij}
+ e_i \otimes e_j dt^{ij} \\
& = & e_i \otimes e_j \, t^{ij}_{;\mu} \, dx^\mu  
\end{eqnarray*}
o\`u $$t^{ij}_{;\mu}=t^{ij}_{,\mu} + A^i_{k\mu} \, t^{kj} + A^j_{k\mu}\, 
t^{ik}$$
Le lecteur n'aura aucun mal \`a g\'en\'eraliser cette derni\`ere
formule \`a des situations plus g\'en\'erales ;
mn\'emotechniquement la d\'erivation covariante dans la
direction $\mu$ s'obtient en rajoutant \`a la d\'eriv\'ee ordinaire
(le premier terme dans l'expression ci-dessus) un terme
de type " $A.t$ " pour chaque indice de fibre (on ``corrige'' chaque 
indice en
le rempla\c cant par un indice muet sur lequel on somme). Attention : si les
indices de fibre sont en bas, il faut utiliser un signe moins (voir
l'explication dans la sous-section pr\'ec\'edente).

Voici un dernier exemple : Soit $t\in E\otimes 
E\otimes
E^* $,
 $$
\mbox{\fbox{$
t=e_i\otimes e_j \otimes e^k \, t^{ij}_k
$}}
 $$
 alors $\nabla t \in 
E\otimes E\otimes
E^* \otimes \Omega^1(M) $,
 $$
\nabla t=e_i\otimes e_j \otimes e^k \, \, 
  t^{ij}_{k;\mu} \,
dx^\mu
 $$ avec 
$$
\mbox{\fbox{$
t^{ij}_{k;\mu}= t^{ij}_{k,\mu} + A^i_{l\mu} \, t^{lj}_k + A^j_{l\mu} \, 
t^{il}_k - t^{ij}_l A^l_{k\mu}
$}}
$$ 
Le premier terme $t^{ij}_{k,\mu}$ est d\'efini comme ${\partial \over
\partial x^\mu}[t^{ij}_{k}]$ si on travaille dans rep\`ere naturel, mais il 
doit
\^etre compris comme $e_\mu [t^{ij}_{k}]$ si on utilise un rep\`ere mobile
quelconque, et, dans ce cas, on a
  $$
\mbox{\fbox{$
\nabla t=e_i\otimes e_j \otimes e_k   \;
t^{ij}_{k;\mu} \;
e^\mu
$}}
$$
 o\`u $\{e^\mu\}$ d\'esigne le co-rep\`ere mobile
dual.

La d\'eriv\'ee covariante d'un tenseur quelconque $t = e_i\otimes e_j \ldots 
e^k\otimes e^l\ldots\; t^{ij\ldots}_{kl\ldots}$ dans la direction du champ 
de vecteurs $e_\mu$ est d\'efinie par
$$
\mbox{\fbox{$
\nabla_\mu t(\ldots) = \nabla t(\ldots,e_\mu)=e_i\otimes e_j \ldots 
e^k\otimes e^l\ldots \; t^{ij\ldots}_{kl\ldots\mu}
$}}
$$

\subsection{L'op\'erateur D}\label{sec:operateur-D}

On introduit le symbole $D$ sur l'exemple suivant : prenons $$t =e_i\otimes 
e_j \otimes e^k \quad t^{ij}_k$$ On pose
$$
\mbox{\fbox{$
\nabla t = e_i\otimes e_j \otimes e^k   \quad Dt^{ij}_k
$}}
$$ ainsi 
$$
\mbox{\fbox{$
Dt^{ij}_k = t^{ij}_{k;\mu} e^\mu
$}}
$$

Il faut bien voir que $t^{ij}_k$ est ici consid\'er\'e
 comme une z\'ero-forme sur le fibr\'e principal $P$ \`a valeurs dans un espace vectoriel (la 
 fibre type du fibr\'e vectoriel appropri\'e).
 
 Il est facile de g\'en\'eraliser cette notation $Dt^{ij}_k$ au cas o\`u 
 l'objet consid\'er\'e n'est pas une $0$-forme \`a valeurs dans un espace 
 vectoriel mais une $p$-forme \`a valeurs dans un espace 
 vectoriel.
 
Il n'existe pas de de convention d'\'ecriture qui soit universelle, pour d\'esigner cet op\'erateur.
Certains auteurs, par exemple, le notent {\bf d}(gras). M\^eme remarque d'ailleurs pour l'op\'erateur
de diff\'erentielle ext\'erieure covariante $d^\nabla$
d\'efini ci-dessous, que certains auteurs notent souvent$\ldots$ $D$ !

\subsection{Diff\'erentielle ext\'erieure covariante $d^\nabla$}

\subsubsection{D\'efinition}

Soit $E=E(M,F)$ un fibr\'e vectoriel ; on a d\'efini $\nabla$ comme un
op\'erateur : $\Gamma E \mapsto \Gamma E\otimes \Omega^1(M)$ ;  on d\'efinit
maintenant $d^\nabla$ comme l'unique op\'erateur prolongeant $\nabla$ comme
d\'erivation gradu\'ee de l'alg\`ebre $\bigoplus_p E\otimes \Omega^p(M)$. Ainsi
donc
 $$
\mbox{\fbox{$
d^\nabla: \Gamma E \otimes \Omega^p(M) \mapsto \Gamma E\otimes 
\Omega^{p+1}
$}}
$$
satisfait la propri\'et\'e
 $$
\mbox{\fbox{$
d^\nabla(\psi \wedge \lambda) = 
d^\nabla(\psi)\wedge \lambda + (-1)^k \psi \wedge d\lambda
$}}
$$
 lorsque $\psi \in
\Gamma E \otimes \Omega^k(M)$ et $\lambda \in \Omega^s(M)$. Les \'el\'ements de
$\Omega^p(M,E)  =  \Gamma E \otimes \Omega^p(M)$ sont appel\'es
$p$-formes sur $M$ \`a valeurs dans le fibr\'e vectoriel $E$ ou, plus 
simplement
``tenseurs-$p$-formes''. Lorsque $p=0$, les deux op\'erateurs $d^\nabla$ 
et $\nabla$ co\"incident. 

Le
lecteur peut, \`a juste titre, se demander pourquoi nous distinguons les deux
notations $d^\nabla$ et $\nabla$. La raison en est la suivante : nous 
r\'eservons la notation $\nabla$ au cas o\`u l'on agit sur une $0$-forme. En 
effet, il est des cas o\`u un objet math\'ematique donn\'e peut \^etre consid\'er\'e 
soit comme une $p$-forme \`a valeurs dans un certain fibr\'e, soit comme une 
$0$-forme \`a valeurs dans un fibr\'e diff\'erent. Ecrire $\nabla \omega$ 
signifie (pour nous) ``{\it $\omega$ est une $0$-forme \`a valeurs dans un certain fibr\'e 
vectoriel (c'est \`a dire une section de ce dernier) et nous calculons sa 
diff\'erentielle ext\'erieure covariante, qui est donc une $1$-forme \`a 
valeurs dans le m\^eme fibr\'e\/}''. Cette distinction que nous op\'erons, au 
niveau des notations, permet d'\'eviter des confusions possibles, en 
particulier lorsque le fibr\'e $E$ peut d\'esigner le fibr\'e tangent, son 
dual, ou une puissance tensorielle de ces derniers.

\subsubsection{ Les op\'erateurs $D$ et $d^\nabla$} 
Compl\'ement concernant la notation 
$D$ introduite plus haut :

Si $V$ est, par exemple, un \'el\'ement de $\Gamma(E\otimes E^* \otimes
E^*)\otimes \Omega^2(M) =\Omega^2(M, E\otimes E^* \otimes E^*)$,
c'est \`a dire
$$
\mbox{\fbox{$
V=e_i\otimes e^j\otimes e^k \; {1\over
2!}\; V^i_{jk{}\mu\nu}e^\mu \wedge e^\nu
$}}
$$
 on \'ecrira simplement 
$$
\mbox{\fbox{$
V=e_i\otimes e^j\otimes e^k \, V^i_{jk}
$}}
$$
 en ne faisant pas appara\^itre explicitement les indices de forme. On utilisera
\'egalement la notation $D$ introduite pr\'ec\'edemment en \'ecrivant 
$$
\mbox{\fbox{$
d^\nabla V = e_i\otimes e^j\otimes e^k {D}V^i_{jk}
$}}
$$
 avec
 $$
 \mbox{\fbox{$
{D} V^i_{jk} = d V^i_{jk} + A^i_l\wedge V^l_{jk} - A^l_j \wedge V^i_{lk} -
A^l_k \wedge V^i_{jl} 
$}}$$

En effet : $$d^\nabla V = d^\nabla(e_i\otimes e^j\otimes e^k)\wedge V^i_{jk} + 
e_i\otimes e^j\otimes e^k d V^i_{jk}$$ o\`u

\begin{eqnarray*}
d^\nabla(e_i\otimes e^j\otimes e^k)
 & = &
\nabla(e_i)\otimes e^j\otimes e^k +  
e_i\otimes \nabla(e^j)\otimes e^k +
e_i\otimes e^j\otimes \nabla(e^k) \\
& = &
A^l_i e_l\otimes e^j\otimes e^k - e_i \otimes A^j_l e^l \otimes e^k
-e_i\otimes e^j \otimes A^k_l e^l
\end{eqnarray*}
On obtient donc l'expression donn\'ee plus haut pour les
composantes ${D}V^{ij}_k$ de $d^\nabla V$.
Le lecteur g\'en\'eralisera sans peine
l'expression donn\'ee pour ${D}V^i_{jk}$ \`a un tenseur-$p$-forme de rang
quelconque.

\subsubsection{ Autre exemple} \label{sec:d-ext-sigma}\

A titre d'exercice, calculons $d^\nabla u$ o\`u $u$ est un \'el\'ement
de $\Gamma E\otimes\Omega^1(M)$. On choisit un rep\`ere $(e_i)$ dans $\Gamma 
E$
et un rep\`ere $(e^\mu)$ dans $\Omega^1(M)$ il peut s'agir d'un corep\`ere
naturel par rapport \`a une carte ($e^\mu = dx^\mu$) ou d'un corep\`ere mobile
quelconque $e^\mu$ avec $de^\mu=-{1\over
2}f_{\nu\rho}{}^\mu e^\nu \wedge e^\rho$. On \'ecrit donc $u =
e_i {u^i}_\mu e^\mu$. L'ordre des symboles n'a pas trop d'importance 
---tout
au moins en g\'eom\'etrie commutative!--- mais nous sugg\'erons fortement au
lecteur d'adopter cette \'ecriture, c'est \`a dire l'ordre $1)2)3)$ avec $1)$, 
un
\'el\'ement de $\Gamma E$, $2)$, un coefficient, c'est \`a dire un \'el\'ement de
l'alg\`ebre (commutative) $C^\infty(M)$ des fonctions sur $M$, et $3)$, une
$p$-forme. Le lecteur aura sans doute \'egalement not\'e que, conform\'ement 
\`a nos habitudes, nous avons omis
d'\'ecrire explicitement le symbole $\otimes$ du produit tensoriel entre les
sections de $E$ (les $e_i$) et les $p$-formes (ici, les $e^\mu$).

La premi\`ere m\'ethode permettant de calculer $d^\nabla u$ est 
d'utiliser la
r\`egle de d\'erivation g\'en\'eralis\'ee qui d\'efinit l'op\'erateur $d^\nabla$:

\begin{eqnarray*}
d^\nabla u & = & d^\nabla(e_i u^i_\mu)\wedge e^\mu + e_i 
u^i_\mu
de^\mu \\
{} & = & (\nabla e_i) u^i_\mu \wedge e^\mu + e_i du^i_\mu\wedge 
e^\mu
+ e_i u^i_\mu de^\mu \\
{}  & = & A^j_{i\nu} e_j e^\nu u^i_\mu \wedge e^\mu + e_i 
u^i_{\mu ,
\nu} e^\nu \wedge e^\mu + e_i u^i_\mu de^\mu
 \end{eqnarray*}
Si $e^\mu=dx^\mu$, le troisi\`eme terme est nul. Par contre, si
$e^\mu$ est un corep\`ere mobile avec fonctions de structure
$f_{\nu\rho}{}^\mu$, il vient (rappelons que $u^i_{\mu,\nu}  = 
e_\nu[u^i_\mu]$),
 
$$
 \begin{array}{|rcl|}
\hline
{} & {} & {} \\ 
d^\nabla u & = & e_j(u^j_{\mu,\nu} + A^j_{i\nu}u^i_\mu -
u^j_\rho {1\over 2}f_{\nu\mu}{}^\rho)\; e^\nu \wedge e^\mu \\
{} & {} & {} \\ 
 {} & = & e_j (u^j_{\mu ; \nu}- u^j_\rho {1\over 2} f_{\nu\mu}{}^\rho)
\; e^\nu \wedge e^\mu \\
{} & {} & {} \\ 
\hline
 \end{array}
$$


La deuxi\`eme m\'ethode pour calculer $d^\nabla u$ n'est pas vraiment une 
m\'ethode
puisqu'elle revient \`a utiliser une formule g\'en\'erale. Posons $u = e_i
u^i$ en ne faisant pas appara\^itre explicitement l'indice de forme 
dans
l'expression de $u$, bien que $u^i = u^i_\mu e^\mu$.
On \'ecrit imm\'ediatement (voir l'expression ``mod\`ele''
de ${D}V^i_{jk}$ donn\'ee pr\'ec\'edemment) 
$$\mbox{\fbox{$d^\nabla u = e_j {D}u^j$}}$$
avec 
$$\mbox{\fbox{${D}u^j = du^j + A^j_l\wedge u^l$}}$$
Le passage d'une expression de $d^\nabla u$ \`a l'autre utilise les relations 
$A^j_l = A^j_{l\mu} e^\mu $ et $du^j = d(u^j_\mu
e^\mu) = (du^j_{\mu})\wedge e^\mu + u^j_\mu 
de^\mu$.

Une forme ext\'erieure de degr\'e $p$ \`a valeurs dans un fibr\'e vectoriel $E$
peut, bien sur, \^etre \'evalu\'ee sur $p$ vecteurs $v_1,v_2,\ldots, v_p$ 
tangents \`a
$M$. Par exemple, si $u \in \Omega^1(M,E)$, en \'evaluant $d^\nabla 
u
\in \Omega^2(M,E)$ sur deux vecteurs $v_1$ et $v_2$, on obtient
$$
\mbox{\fbox{$
d^\nabla u(v_1,v_2) = \nabla_{v_1}u(v_2) - \nabla_{v_2} 
u(v_1)
- u([v_1,v_2])
$}}
$$

en effet,

\begin{eqnarray*}
d^\nabla u(e_\sigma, e_\tau) & = & e_j(u^j_{\mu;\nu} - 
u^j_\rho
{1 \over 2} f_{\nu\rho}{}^\mu) e^\nu \wedge e^\mu (e_\sigma, 
e_\tau) \\
{} & = & e_j (u^j_{\mu;\nu} - u^j_\rho {1 \over 2} f_{\nu\rho}{}^\mu)
(\delta^\nu_\sigma \delta^\mu_\tau - \delta^\mu_\sigma \delta^\nu_\tau) \\
{} & = & e_j (u^j_{\tau;\sigma} -
u^j_{\sigma ; \tau}-e_ju^j_\rho{1\over 2}(f_{\sigma\tau}{}^\rho-
f_{\tau \sigma}{}^\rho))\\
{} & = & e_j((u^j_{\tau ; \sigma} -  u^j_{\sigma ; \tau}) -
u^j_\rho f_{\sigma \tau}{}^\rho) \\
 {} & = & \nabla_{e_\sigma} u(e_\tau) - 
\nabla_{e_\tau}u(e_\sigma) -
u([e_\sigma,e_\tau])
\end{eqnarray*}

Plus g\'en\'eralement, si $u \in \Omega^p(M,E)$,
\begin{eqnarray*}
d^\nabla u (v_1,v_2,\ldots,v_{p+1}) & = & \Sigma_{i=1}^{p+1} (-1)^{i+1} 
\nabla_{v_i}
u(v_1,\ldots,\hat{v_i},\ldots,v_{p+1}) + \\
{} & {} & 
\Sigma_{1\leq i \leq j \leq p+1} (-1)^{i+j} 
u([v_i,v_j],v_1,\ldots,\hat
v_i,\ldots,\hat v_j, \ldots, v_{p+1})
\end{eqnarray*}

Pour conclure cette section, nous attirons l'attention du lecteur sur le 
fait
que, malgr\'e la notation utilis\'ee, le carr\'e de l'op\'erateur $d^\nabla$ n'est 
pas
nul. C'est d'ailleurs la pr\'esence d'un second membre non nul dans 
l'\'equation
$(d^\nabla)^2 = F$ qui va nous permettre un peu plus loin de d\'efinir 
l'op\'erateur
de courbure.

\subsection{Diff\'erentielles et d\'eriv\'ees covariantes g\'en\'eralis\'ees~$\nabla$}
\label{sec:nabla-generalise}
\subsubsection{Remarques}

La diff\'erentielle covariante $\nabla \sigma$ d'un \'el\'ement de $\Gamma E$ 
est un
\'el\'ement de $\Gamma E\otimes \Omega^1(M)$. En g\'en\'eral, les fibr\'es $E$ et $TM$ 
sont
distincts (sauf dans le cas particulier \ref{sec:connexions-lineaires}).
 Il n'emp\^eche que $\Omega^1(M)$ peut
\^etre consid\'er\'e comme l'ensemble des sections de $T^*M$, et, qu'en 
cons\'equence,
cet \'el\'ement $\nabla \sigma$, au lieu d'\^etre consid\'er\'e comme une $1$-forme \`a
valeurs dans le fibr\'e vectoriel $E$ peut \^etre consid\'er\'e comme une 
$0$-forme \`a
valeurs dans le fibr\'e vectoriel $T^*M \otimes E$, c'est \`a dire un
\'el\'ement de $\Gamma(T^*M \otimes E)$. Nous d\'esignerons en g\'en\'eral par le 
symbole $\Omega^p(X,Y)$ l'ensemble des $p$-formes sur $X$ \`a valeurs dans 
$Y$. Avec ces notations, nous voyons qu'il est possible d'identifier 
$\Omega^1(M,E)$ et $\Omega^0(M,T^*M\otimes E)$.

Jusqu'\`a pr\'esent, la connexion 
(le potentiel
de jauge $A^i_j = A^i_{j\mu} dx^\mu$) avait trait au fibr\'e $E$, mais il peut 
se
faire$\ldots$(et g\'en\'eralement il se fait) que $T^*M$ lui-m\^eme soit muni
d'une connexion (un potentiel de jauge $\Gamma^\nu_\rho = \Gamma^\nu_{\rho
\mu} dx^\mu$). Nous \'etudierons plus loin, et en d\'etails, le cas particulier
des connexions lin\'eaires (du type $\Gamma^\nu_{\rho\mu}$), mais dans le
pr\'esent paragraphe, nous souhaitons seulement attirer l'attention du 
lecteur
sur l'existence d'une ambigu\"\i t\'e (le fait qu'une $1$-forme \`a valeurs dans un
fibr\'e puisse aussi \^etre consid\'er\'ee comme une $0$-forme \`a valeurs dans un 
fibr\'e
diff\'erent) et sur le fait que cette ambigu\"\i t\'e permet, en quelque sorte,
d'enrichir la th\'eorie. Reprenons l'exemple de $\sigma \in \Gamma(E)$ ; on 
a $u
 =  d^\nabla \sigma = \nabla \sigma \in \Omega^1(M) \otimes \Gamma(E)$. 
En
tant que $1$-forme \`a valeurs dans le fibr\'e vectoriel $E$, nous pouvons 
faire
appel \`a la th\'eorie des diff\'erentielles ext\'erieures covariantes et calculer
$d^\nabla u = (d^\nabla)^2 \sigma$, qui est un \'el\'ement de $\Gamma(E) 
\otimes
\Omega^2(M)$. Cependant, en consid\'erant $u$ comme une $0$-forme \`a valeurs dans 
le
fibr\'e $T^*M \otimes E$, nous pouvons --- {\em dans la mesure o\`u les deux fibr\'es 
$E$
{\em et} $T^*M$ sont \'equip\'es de connexions\/} --- calculer la diff\'erentielle
covariante $\nabla u = \nabla \nabla \sigma $. Lorsque $\sigma \in 
\Gamma(E)$,
le calcul de $\nabla \sigma$ ne fait appel qu'\`a la connexion $A^i_j$, mais
celui de $\nabla u$ (avec $u = \nabla \sigma$) fait appel simultan\'ement \`a 
la
connexion $A^i_j$ sur $E$ et \`a la connexion $\Gamma^\mu_\nu$ sur $T^*M$. 
Plus
g\'en\'eralement, si $u$ d\'esigne un objet ayant un certain nombre $(p,q)$
d'indices de type $E$ (ou $E^*$) et un certain nombre $(p',q')$ d'indices 
de
type $TM$ (ou de type $T^*M$), nous pouvons le consid\'erer comme une 
$0$-forme \`a
valeurs dans $TM^{p',q'}\otimes E^{p,q}$, et calculer $\nabla u$, qui sera 
un
\'el\'ement de $\Gamma(TM^{p',q'}\otimes E^{p,q})\otimes \Omega^1(M)$. Nous 
pouvons
aussi, dans le cas o\`u $u$ est antisym\'etrique en $s$ indices de forme ($s$
indices de type $T^*M$), consid\'erer $u$ comme un \'el\'ement de
$\Gamma(TM^{p',q'-s}\otimes E^{p,q})\otimes \Omega^s(M)$ et calculer sa
diff\'erentielle ext\'erieure covariante $d^\nabla u$ qui sera un \'el\'ement de
$\Gamma(TM^{p',q'-s}\otimes E^{p,q})\otimes \Omega^{s+1}(M)$. On imagine
ais\'ement la richesse des possibilit\'es$\ldots$
Au niveau des notations, nous adoptons la convention suivante : {\em si on \'ecrit
$\nabla u$, c'est que $u$ doit \^etre consid\'er\'e comme une $0$-forme dans un
fibr\'e appropri\'e (m\^eme si $u = \nabla \sigma$); par contre, si on \'ecrit 
$d^\nabla
u$, c'est que $u$ doit \^etre consid\'er\'e comme une $p$-forme sur $M$ et que
$d^\nabla$ est une diff\'erentielle ext\'erieure covariante\/}. En d'autres 
termes,
l'utilisation de $d^\nabla$ conduit toujours \`a des objets poss\'edant une
certaine antisym\'etrie, ce qui n'est pas le cas pour $\nabla$. Il est bien
\'evident que ces notations sont insuffisantes pour couvrir tous les cas
possibles, mais en g\'en\'eral le contexte devrait permettre de pr\'eciser.

\subsubsection{Exemple}

Soit $u = e_i u^i_\mu e^\mu \in \Gamma E \otimes \Omega^1(M)$. En tant que
tenseur-$1$-forme, on \'ecrira plut\^ot $u = e_i u^i$ sans faire appara\^itre
explicitement l'indice de forme ; on a d\'ej\`a calcul\'e sa diff\'erentielle
ext\'erieure covariante
$$
\mbox{\fbox{$
d^\nabla u = e_i {D}u^i \in \Gamma E \otimes \Omega^2(M)
$}}
$$
Par contre, en tant que $0$-forme \`a valeurs dans le fibr\'e $T^*M \otimes 
E$, on
fera explicitement appara\^itre tous les indices $u^i_\mu$, et, dans 
l'hypoth\`ese
o\`u le fibr\'e tangent est lui-aussi muni d'une connexion $\Gamma$, on pourra
calculer la diff\'erentielle covariante 
$$
\mbox{\fbox{$
\nabla u \in \Gamma(T^*M \otimes E)\otimes \Omega^1(M)
$}}
$$
Il vient 
$$ 
\mbox{\fbox{$
\nabla u = \nabla(e^\mu \otimes e_i \,  u^i_\mu) = e^\mu \otimes e_i \,  u^i_{\mu;\nu}\,  e^\nu
$}}
$$
o\`u 
$$
\mbox{\fbox{$
u^i_{\mu;\nu} = u^i_{\mu,\nu}+A^i_{k\nu}u^k_\mu - 
\Gamma^\rho_{\mu\nu}u^i_\rho
$}}
$$
et 
$u^i_{\mu,\nu}=e_\nu[u^i_\mu]$.
En effet
$$
\nabla u = e^\mu \otimes e_i . du^i_\mu + e^\mu \otimes (\nabla e_i).  u^i_\mu + 
(\nabla
e^\mu)\otimes e_i .  u^i_\mu
$$

Par ailleurs,
$$
\nabla u(e_\alpha,e_\beta) =  e^\mu(e_\alpha) e_i u^i_{\mu;\nu} 
e^\nu(e_\beta)
= \delta^\mu_\alpha e_i u^i_{\mu;\nu} \delta^\nu_\beta = e_i
u^i_{\alpha;\beta}
 $$
et donc $\nabla u = e^\mu \nabla u (e_\mu,e_\nu) e^\nu$. On voit que
$\nabla_\beta u  =  e^\mu e_i u^i_{\mu;\beta}$, par cons\'equent,
$\nabla_\beta u(e_\alpha) = \nabla u(e_\alpha, e_\beta)$. 
Plus g\'en\'eralement
 $$
\nabla u(e_{\alpha_1},e_{\alpha_2},\ldots,e_{\alpha_{p+1}}) =
\nabla_{\alpha_{p+1}} u(e_{\alpha_1},e_{\alpha_2},\ldots,e_{\alpha_{p}})
$$

Notons que, conform\'ement \`a nos conventions, nous avons \'ecrit les vecteurs 
de
base (les $e^\mu e_i \equiv e^\mu \otimes e_i$) \`a gauche des composantes 
(les
$u^i_\mu$) et la base des $1$-formes (les $e^\nu$) \`a droite des 
composantes.

\subsubsection{Comparaison entre $\nabla$ et $d^\nabla$}

On pourra utilement comparer l'expression de $\nabla u$ obtenue ci-dessus 
avec
celle de $d^\nabla u$ calcul\'ee auparavant :

\begin{eqnarray*}
\nabla u & = & 
e^\mu e_i (u^i_{\mu,\nu}+A^i_{k\nu}u^k_\mu -
\Gamma^\rho_{\mu\nu}u^i_\rho) e^\nu  \\
d^\nabla u & = & e_i (u^i_{\mu,\nu}+A^i_{k\nu}u^k_\mu - 0 - {1\over 2}
u^i_\rho f_{\mu\nu}{}^\rho) e^\nu \wedge e^\mu
\end{eqnarray*}

On voit que $d^\nabla u = - 2 Alt \; \nabla u$ ou $Alt$ d\'esigne l'op\'erateur
d'antisym\'etrisation.
 De fa\c con g\'en\'erale, si $u \in \Omega^p(M,E)$, on peut fabriquer 
$d^\nabla u
\in \Omega^{p+1}(M,E)$ ou $\nabla u \in \Gamma(T^{*p}M\otimes E)\otimes
\Omega^1(M) \equiv \Omega^1(M,T^{*p}M\otimes E)$ ; la relation entre les 
deux
est $$
\mbox{\fbox{$ 
d^\nabla u = (-1)^p (p+1) \, Alt \, \nabla u 
$}}
$$
o\`u l'op\'erateur d'antisym\'etrisation $Alt$ n'agit pas sur $E$.

Notons enfin que $d^\nabla$ ne fait explicitement intervenir que la 
connexion
$A$ sur $E$ alors que $\nabla$ fait appel \`a la fois \`a la connexion $A$ sur 
$E$
et \`a la connexion $\Gamma$ sur $TM$. Il existe un cas particuli\`erement
int\'eressant o\`u les fibr\'es $E$ et $TM$ co\i incident ; nous y reviendrons dans 
la
section consacr\'ee aux connexions lin\'eaires.

\section{Courbure}

\subsection{Lin\'earit\'e de $(d^\nabla)^2$}
Comme nous l'avons vu, l'op\'erateur $$\nabla = d^\nabla :
\Gamma(E)=\Omega^0(M,E) \mapsto \Gamma(E) \otimes 
\Omega^1(M)=\Omega^1(M,E)$$
s'\'etend \`a un op\'erateur $d^\nabla : \Omega^p(M,E) \mapsto 
\Omega^{p+1}(M,E)$ et
v\'erifie la propri\'et\'e $$
d^\nabla(\psi \wedge \lambda) = d^\nabla(\psi) \wedge \lambda + (-1)^k \psi
\wedge d\lambda
$$ lorsque  $\psi \in \Omega^k(M,E)$ et $\lambda \in \Omega^s (M)$.
On peut donc consid\'erer l'op\'erateur $(d^\nabla)^2$ agissant sur les 
sections de
$E$
$$
\mbox{\fbox{$
(d^\nabla)^2 : \Gamma(E) \mapsto \Gamma(E)\otimes \Omega^2(M) = 
\Omega^2(M,E)
$}}
$$
La propri\'et\'e fondamentale de cet op\'erateur est d'\^etre {\em lin\'eaire} par rapport 
\`a
l'alg\`ebre des fonctions sur $M$.
On se souvient que $\Gamma(E)$ est un module sur $C^\infty(M)$, mais
$d^\nabla$ n'est pas lin\'eaire par rapport aux fonctions (il l'est seulement
par rapport aux scalaires), en effet, lorsque $\sigma \in  \Gamma(E)$ et 
$f\in
C^\infty(M)$, on a $$
d^\nabla(\sigma.f) = (d^\nabla \sigma) f + \sigma df \neq (d^\nabla \sigma)f
$$
par contre 
\begin{eqnarray*}
(d^\nabla)^2(\sigma . f) & = & d^\nabla(d^\nabla \sigma . f + \sigma . df) \\
{} & = & ((d^\nabla)^2\sigma) . f - d^\nabla\sigma df + d^\nabla\sigma df +
\sigma d^2f
\end{eqnarray*}
et donc
$$
\mbox{\fbox{$
(d^\nabla)^2(\sigma . f) =  (d^\nabla)^2(\sigma) . f
$}}
$$

Cette propri\'et\'e de lin\'earit\'e est absolument fondamentale. C'est elle qui, 
en
d\'efinitive, est responsable du fait que la courbure est caract\'eris\'ee par un
tenseur. Par ailleurs, le fait que $(d^\nabla)^2$ envoie $\Gamma(E)$ dans $\Gamma(E)\otimes \Omega^2(M)$
montre que, si on ``g\`ele'' les indices de forme (correspondant \`a l'espace 
$\Omega^2(M)$), cet op\'erateur envoie ---lin\'eairement--- les sections de 
$E$ dans les sections de $E$. En d'autres termes $(d^\nabla)^2$ peut \^etre consid\'er\'e comme une
$2$-forme \`a valeurs dans le fibr\'e des endomorphismes de $E$.

 \subsection{Expression de l'op\'erateur de courbure
dans les fibr\'es vectoriels associes}

{\it Remarque: 
On peut directement d\'efinir la courbure sur un fibr\'e principal en 
restant
``dans le fibr\'e principal'' c'est \`a dire sans avoir besoin de se placer sur 
la
base et sans m\^eme consid\'erer les fibr\'es associ\'es. Une telle d\'efinition, 
dans la
lign\'ee de celle donn\'ee en section \ref{sec:matrice-de-connexion},
peut alors se  transporter
 au niveau des diff\'erents fibr\'es vectoriels associ\'es, gr\^ace au choix
de la repr\'esentation d\'efinissant le fibr\'e en question. Il s'agit l\`a d'une
m\'ethode \'el\'egante mais un peu abstraite en premi\`ere lecture $\dots$ Nous y
reviendrons un peu plus loin (cf. paragraphe \ref{sec:connexions-principales-2}).
 En attendant, nous pr\'ef\'erons d\'efinir la
courbure plus simplement, dans chaque fibr\'e associ\'e,  en utilisant les
r\'esultats d\'ej\`a obtenus pour l'\'ecriture de la d\'eriv\'ee covariante et de la
diff\'erentielle ext\'erieure covariante.}\par
\smallskip
Nous venons de voir que le carr\'e de la diff\'erentielle ext\'erieure 
covariante est
un op\'erateur (en g\'en\'eral non nul!) qui est lin\'eaire par rapport aux 
fonctions
sur la vari\'et\'e. Appliqu\'e a une section d'un fibr\'e vectoriel $E$, il lui 
fait
correspondre une 2-forme (sur la base $M$) \`a valeurs dans $E$; l'\'ecriture
locale de cet op\'erateur fera donc appel a deux jeux d'indices diff\'erents:
des indices $\mu$ et $\nu$ en position basse, parce qu'il s'agit d'une 
2-forme,
et deux indices $i$ et $j$, l'un en position haute et l'autre en position 
basse,
en effet, les indices de forme \'etant fix\'es, l'op\'erateur agit comme un
endomorphisme de la fibre (il transforme les sections en sections). 
Cet objet est, de surcro\^it, antisym\'etrique par rapport aux indices de forme
$\mu$ et $\nu$ puisqu'il s'agit... d'une 2-forme!  La donn\'ee de
cet op\'erateur fera donc intervenir $n(n-1)/2 + p^2$ nombres 
$F^i_{j\mu\nu}$ si
on suppose que $dim \, M = n$ et que la fibre type est un espace vectoriel de
dimension $p$.

L'objet fabriqu\'e peut \^etre consid\'er\'e de bien des fa\c cons. En effet, si on 
g\`ele
les indices de forme, on obtient $F_{\mu\nu}$ qui est une matrice $p\times 
p$
dont les \'el\'ements de matrice sont les  $F^i_j{}_{\mu\nu}$. Si par contre, 
on
g\`ele les indices de fibre, on obtient $F^i_j$ qui est une 2-forme sur la 
base.
Nous verrons m\^eme un peu plus loin encore d'autres fa\c cons de consid\'erer cet
objet. Certains auteurs r\'eservent des notations diff\'erentes pour ces 
objets qui
sont math\'ematiquement distincts. Nous ne le ferons pas car 
ne
pas vouloir commettre de tels abus de notations nuit, \`a notre avis, \`a
la compr\'ehension. Chaque fois qu'on a un tenseur appartenant \`a un espace
vectoriel $E\otimes F\otimes G$, on peut le consid\'erer comme un 
\'el\'ement
de  $Hom((E\otimes F\otimes G)^*,\CC)$ ou comme un \'el\'ement de  
$Hom(E^*,F\otimes G)$
ou comme $\ldots$ Cela ne nous semble pas raisonnable de vouloir, \`a 
chaque fois, changer 
la
notation d\'esignant l'objet en question ! 

Nous poserons donc simplement
$$
\mbox{\fbox{$ F  =  ({d ^\nabla})^2
$}}
$$
et nous ferons appara\^itre tel ou tel jeu d'indices suivant qu'on d\'ecidera 
de
saturer le tenseur $F$ avec tel ou tel indice, conform\'ement aux conventions
d\'ej\`a utilis\'ees maintes fois dans ce qui pr\'ec\`ede. 

Nous calculons maintenant l'expression de l'op\'erateur de courbure $F$ en 
terme de
la connexion $A$. Soit ${e_i}$ une famille de sections (locales ou non)
constituant, en chaque point $x$ d'un certain voisinage de la base $M$ une
base de la fibre $V_x$ au dessus de $x$. On sait que $({d ^\nabla})^2: 
\Gamma E =
\Omega^0(M,E) \rightarrow \Omega^2(M,E)$. On va calculer $$({d ^\nabla}^2) 
e_j = e_i F^i_j = e_i \, {1\over 2}\, F^i_{j\mu\nu} \, dx^\mu \wedge dx^\nu$$

Il vient :

\begin{eqnarray*}
 ({d ^\nabla})^2 e_i  & = & d^\nabla(\nabla e_i) = d^\nabla(e_j A^j_{i\nu} 
 dx^\nu) \\
 {} & = &
(d^\nabla e_j)\wedge(A^j_i) + e_j dA^j_i = e_k A^k_j \wedge A^j_i + e_j d A^j_i = 
e_j (dA^j_i + A^j_k \wedge A^k_i)
\end{eqnarray*}
et donc 
$$
\mbox{\fbox{$
F^j_i =  (dA^j_i + A^j_k \wedge A^k_i)
$}}
$$



\subsection{Equation de structure pour la courbure} \label{sec:eq-de-structure-R}

Il existe essentiellement deux fa\c cons d'exprimer l'op\'erateur de courbure. 
la
premi\`ere, en fonction des coefficients de connexion (potentiels de jauge) 
est
celle que nous venons de voir. La seconde, baptis\'ee ``\'equation de 
structure''
exprime directement la courbure en fonction de l'op\'erateur de d\'eriv\'ee
covariante. A titre d'exercice pr\'eliminaire, nous avons d\'ej\`a calcul\'e explicitement les
composantes de la diff\'erentielle ext\'erieure covariante $d^\nabla\sigma$ d'une
$1$-forme $\sigma$ \`a valeurs dans un fibr\'e vectoriel $E$. Soit
$\{e^\mu\}$ un co-rep\`ere mobile (base constitu\'ee de sections locales 
de
$T^*M$) et $\{e_i\}$ un rep\`ere local du fibr\'e $E$ (base constitu\'ee de 
sections
locales de $E$). Soit $\sigma \in \Omega^1(M,E)$. On peut donc \'ecrire
$\sigma = e_i \sigma^i_\mu e^\mu$. Nous avons d\'ej\`a vu que

$$
d^\nabla \sigma(e_\sigma,e_\tau) = \nabla_\sigma \sigma (e_\tau) - \nabla_\tau \sigma (e_\sigma) -
\sigma([e_\sigma, e_\tau])
$$

Supposons maintenant que la $1$-forme $\sigma$, \`a valeurs dans $E$ soit
elle-m\^eme obtenue comme la diff\'erentielle d'une section $v$ de $E$ : 
$\sigma =
\nabla v$.
Dans ce cas $(d^\nabla)^2v = d^\nabla \sigma$. Nous utilisons le calcul pr\'ec\'edent ; il
vient

\begin{eqnarray*}
(d^\nabla)^2 v(e_\mu,e_\nu) & = & d^\nabla \sigma(e_\mu,e_\nu) \\
{} & = &  \nabla_\mu \sigma(e_\nu) - \nabla_\nu \sigma (e_\mu) - 
\sigma([e_\mu,e_\nu])
\\
{} & = & \nabla_\mu \nabla_\nu v - \nabla_\nu \nabla_\mu v - 
\nabla_{[\mu,\nu]} v
\end{eqnarray*}

Mais, par d\'efinition, $$(d^\nabla)^2 v(e_\mu,e_\nu) = F_{\mu\nu} v$$  Nous obtenons 
donc
l'\'equation de structure pour la courbure $F$:
$$
\mbox{\fbox{$
F_{\mu\nu} = [\nabla_\mu,\nabla_\nu] - \nabla_{[\mu,\nu]}
$}}
$$
Comme d'habitude, nous avons pos\'e $\nabla_\mu  =  \nabla_{e_\mu}$ et 
m\^eme
$[\mu,\nu]  =  [e_\mu,e_\nu]$ pour all\'eger les notations.
Noter que le second terme de l'\'equation de structure pour l'op\'erateur de
courbure s'annule lorsque le rep\`ere choisi dans $TM$ est un rep\`ere naturel,
puisque, dans un tel cas, $[e_\mu,e_\nu]=0$.
Rappelons que, $\mu$ et $\nu$ \'etant fix\'es, $F_{\mu\nu}$ est un op\'erateur
(lin\'eaire), plus pr\'ecis\'ement un endomorphisme de la fibre au dessus du 
point
$x$ dont les \'el\'ements de matrice sont des nombres $F^i_j{}_{\mu\nu}$.
Rappelons \'egalement que nous  \'ecrivons toujours les indices de forme ``en dernier'',
c'est \`a dire $F^i_j{}_{\mu\nu}$ et {\em non} $F_{\mu\nu}{}^i_j$.

\subsection{Identit\'e de Bianchi pour la courbure} \label{sec:bianchi-F}

Cette identit\'e, d\'esign\'ee souvent sous le nom de {\sl deuxi\`eme identit\'e de
Bianchi}
 {\index{identit\'es de
Bianchi}} (nous verrons la ``premi\`ere'' dans le chapitre consacr\'e aux
connexions lin\'eaires) est une identit\'e satisfaite par l'op\'erateur de 
courbure.
Elle s'obtient en utilisant la d\'efinition de $F = dA + A\wedge A$ et les
propri\'et\'es \'el\'ementaires suivantes: $d^2=0$ et $A\wedge (A \wedge 
A) = (A\wedge A) \wedge A$.

Tout d'abord $F = dA + A\wedge A \Rightarrow dF = d^2 A + dA \wedge A - A 
\wedge dA$.
On utilise alors une deuxi\`eme fois la d\'efinition de $F$ en rempla\c cant $dA$ 
par
$F - A\wedge A$ dans la derni\`ere expression. Il vient
$ dF = 0 + (F-A\wedge A)\wedge A - A\wedge (F-A\wedge A)$, d'o\`u
$$
\mbox{\fbox{$
dF + A\wedge F = F\wedge A
$}}
$$

On peut \'ecrire explicitement les indices de forme: voir page
 \pageref{sec:YangMillsExplicit}.

Notons que, lorsque le groupe structural est ab\'elien, 
$A\wedge F - F\wedge A =
0$ puisque ces formes sont \`a valeurs r\'eelles. L'identit\'e de Bianchi s'\'ecrit
alors simplement $dF = 0$, ce qui nous donne la ``moiti\'e'' des \'equations de
Maxwell d\'ecrivant le champ \'electromagn\'etique (celles qui ne font pas 
intervenir
les sources). Dans le cas non ab\'elien, et dans le contexte de l'\'etude des
particules \'el\'ementaires (par exemple dans l'\'etude du champ 
chromodynamique), la
m\^eme \'equation de Bianchi nous donne la ``moiti\'e'' des \'equations de 
Yang-Mills (celles qui ne font pas intervenir les sources). 
\index{equations de Yang-Mills\/}
\index{equations de Maxwell}

\subsection{Transformation de jauge pour la courbure}

Soit $\omega$ une forme de connexion sur un fibr\'e principal $P$ et $\Omega 
= D
\omega$ la forme de courbure correspondante. Nous avons vu que la 
d\'efinition
du potentiel de jauge $A$ et de la courbure $F$ faisait appel au choix 
d'une
section locale $$\sigma: x\in M \longrightarrow \sigma(x) \in P$$ Nous 
allons
momentan\'ement d\'esigner par ${}^\sigma A$ et ${}^\sigma F$ le potentiel de 
jauge
et la courbure correspondante. Soit $$\tau : x\in M \longrightarrow 
\tau(x) \in
P$$ une autre section locale qu'on obtient \`a partir de $\sigma$ par une 
fonction de transition $$g_{\sigma \tau}: x \in M \longrightarrow
g_{\sigma \tau}(x) \in G $$ Rappelons que $$\tau(x) =\sigma(x)  g_{\sigma \tau}(x)$$

Nous avons \'etudi\'e au \ref{sec:transformation-connexion}
 comment se transformaient les potentiels de 
jauge
par changement de section, \`a savoir, 
$$
{}^\tau A = g_{\sigma \tau}^{-1}\, {}^\sigma A \, g_{\sigma \tau} + 
g_{\sigma \tau}^{-1} d g_{\sigma \tau}
$$
La courbure associ\'ee \`a ${}^\tau A$ est \'egale \`a ${}^\tau F = d\, {}^\tau A + 
{}^\tau
A \wedge {}^\tau A$. En rempla\c cant ${}^\tau A$ par son expression en 
fonction
de ${}^\sigma A$, nous allons d\'ecouvrir la loi de transformation pour la
courbure. Notons provisoirement $A = {}^\sigma A$ et $g =g_{\sigma \tau}$ ; il vient
 
 \begin{eqnarray*}
{}^\tau F & = & d(g^{-1} A g + g^{-1}d g) +(g^{-1} A g + g^{-1}d g)\wedge
(g^{-1} A g + g^{-1}d g) \\
{} & = & (-g^{-2}dg\wedge A g + g^{-1}dA g - g^{-1}A\wedge dg) + (-g^{(-2)} 
dg \wedge dg + g^{-1} d^2g) + \\
{} & {} & 
g^{-1}A\wedge A g + g^{-1}dg \wedge g^{-1}A g + g^{-1}A\wedge dg +
g^{-1} dg \wedge g^{-1} dg \\
{} & = &
(-g^{-2} dg\wedge Ag + g^{-1}(dA + A \wedge A)g - g^{-1}A \wedge dg \\
{} & {} & 
+ 0 + 0 + g^{-1}dg \wedge g^{-1}A g + (g^{-1}A g)\wedge(g^{-1}dg)  \\
{} & = & g^{-1}(dA + A \wedge A) g
\end{eqnarray*}
En utilisant les propri\'et\'es \'el\'ementaires de la diff\'erentielle $d$, nous 
voyons
que tous les termes se compensent \`a l'exception de deux, et, qu'en 
cons\'equence, 
$$
\mbox{\fbox{$
{}^\tau F = g_{\sigma\tau}^{-1} \;  {}^\sigma  F \; g_{\sigma\tau}
$}}
$$
Attention :  il n'y a aucune sommation sur $\sigma$ et $\tau$ (les indices ne d\'esignent pas ici des
composantes mais servent seulement \`a d\'esigner les objets eux-m\^emes).
Il faut remarquer la simplicit\'e de cette loi de transformation, lorsqu'on la
compare \`a celle du potentiel de jauge. Noter qu'en particulier, le terme
``d\'erivatif'' $g^{-1}dg$ a disparu.


\subsection {Forme de connexion, diff\'erentielle covariante et courbure 
dans les fibr\'es principaux (compl\'ements)}
\subsubsection{Remarque g\'en\'erale} \label{sec:connexions-principales-2}
Nous avons tout
d'abord d\'efini la forme de connexion $\omega$ au niveau de fibr\'e principal,
puis le potentiel de jauge $A^\alpha_\mu$ sur la base ($A$ n'est rien 
d'autre
qu'une \'ecriture locale de $\omega$), puis nous avons d\'efini la matrice de
connexion $A^i_{j\mu}$ dans chaque fibr\'e associ\'e. Il est possible 
d'adopter une
d\'emarche similaire pour la courbure associ\'ee \`a une connexion : il est 
possible de d\'efinir
une $2$-forme de courbure $\Omega$ au niveau du fibr\'e principal, puis son
\'ecriture locale $F^\alpha_{\mu\nu}$ sur la base, puis enfin le tenseur
$F^i_j{}_{\mu\nu} = F^\alpha_{\mu\nu} \, \rho(X_\alpha)^i_j$ dans chaque
fibr\'e associ\'e caract\'eris\'e par une repr\'esentation $\rho$ du groupe 
structural. 
{\it Bien entendu, l'objet ainsi obtenu doit co\"incider avec la courbure 
introduite et \'etudi\'ee dans les sous-sections pr\'ec\'edentes o\`u nous avons 
travaill\'e tout du long dans les fibr\'es vectoriels.}
Cela dit, nous avons pr\'ef\'er\'e, dans le cadre de cet ouvrage, pour des raisons 
d'ordre aussi bien p\'edagogiques que pratiques, d\'evelopper le formalisme de
la courbure dans les fibr\'es vectoriels. Nous allons n\'eanmoins \'enoncer 
quelques d\'efinitions et r\'esultats g\'en\'eraux, de fa\c con 
\`a ce que lecteur se fasse une id\'ee de ce qu'aurait pu \^etre une autre 
fa\c con de pr\'esenter le sujet. Nous laissons au lecteur le soin de 
d\'emontrer l'\'equivalence des diff\'erentes approches.

\subsubsection{Forme de connexion} Compl\'ements sans d\'emonstration.
Soit $P=P(M,G)$ un fibr\'e principal muni d'une forme de connexion $\omega$.
Nous savons d\'ej\`a que les noyaux de $\omega$
en $e$ et en $eg, g\in G$ sont reli\'es par $
Ker \, \omega_{eg} = (Ker \, \omega_e) g$. Par ailleurs, on sait que la 
compos\'ee de $\omega$ avec l'application $(R_z)_* : Lie \, G \mapsto 
T(z,P)$ qui, \`a tout \'el\'ement de l'alg\`ebre de Lie associe un vecteur 
tangent en $z$, 
doit co\"incider avec l'application identique. On en d\'eduit la loi de 
transformation suivante pour $\omega$ : 
$$
\mbox{\fbox{$
R_g^*\omega = Ad_g^{-1} \omega
$}}
$$ 
o\`u $R_g$ est d\'efinie par $z \in P \rightarrow zg \in P$.
Pour conclure ce paragraphe,
notons qu'on aurait pu d\'efinir la connexion comme la donn\'ee
d'une forme $\omega$ sur $P$, \`a valeurs dans $Lie \, G$, telle que
$\omega(\hat X_\alpha(z))=X_\alpha$ et
satisfaisant \`a la propri\'et\'e d'\'equivariance ci-dessus.


\subsubsection{Diff\'erentielle covariante}
La {\sl diff\'erentielle covariante sur un fibr\'e principal}
 {\index{diff\'erentielle covariante sur un fibr\'e principal}},
 que nous noterons 
$D$ et non $\nabla$ se d\'efinit comme suit. 
Si $\overrightarrow{\lambda}$ est une 
$p$-forme sur $P$ (\`a valeurs r\'eelles ou complexes,
\`a valeurs dans un espace vectoriel, ou encore \`a valeurs dans une alg\`ebre de Lie), 
on d\'efinit $D\overrightarrow{\lambda}$ comme la diff\'erentielle ext\'erieure (usuelle)  de
l'horizontalis\'ee de $\overrightarrow{\lambda}$. Cela signifie que pour calculer
$D\overrightarrow{\lambda}(w_1,w_2,\ldots,w_p,w_{p+1})$ on \'evalue la diff\'erentielle $d\overrightarrow{\lambda}$ sur
l'horizontalis\'e des vecteurs $w_1,w_2,\ldots,w_{p+1}$.
$$
\mbox{\fbox{$
D\overrightarrow{\lambda}(w_1,\ldots,w_{p+1})  =  d\overrightarrow{\lambda}(w_1^h,\ldots,w_{p+1}^h)
$}}
$$
Cette d\'efinition est g\'en\'erale et ne suppose ni que $\overrightarrow{\lambda}$ est horizontal,
ni qu'il est \'equivariant.
Lorsque $\rho$ est une repr\'esentation du groupe structural $G$ et que
 $\overrightarrow{\lambda}$ est horizontal et \'equivariant de type $\rho$, c'est \`a dire qu'il est
\`a valeurs dans l'espace vectoriel support de la repr\'esentation $\rho$ et que
${\overrightarrow{\lambda}}_{zg}(w_1,\ldots,w_p) 
= \rho(g^{-1}){\overrightarrow{\lambda}}_{z}(w_1,\ldots,w_p),$
on d\'emontre que
$$
\mbox{\fbox{$
D\overrightarrow{\lambda} = d \overrightarrow{\lambda} + \rho(\omega) \wedge \overrightarrow{\lambda}
$}}
$$
C'est volontairement que La notation $D$ utilis\'ee ici co\"incide avec 
celle introduite au paragraphe \ref{sec:operateur-D}.
Notons \'egalement que $D\overrightarrow{\lambda}$ s'annule d\`es que l'un de ses arguments est un 
vecteur vertical : $D\overrightarrow{\lambda}$ est donc une forme horizontale.

\subsubsection{L'op\'erateur de courbure dans les fibr\'es principaux}
\subparagraph{Aspect global}
La $2$-forme de
courbure sur $P$, que nous noterons $\Omega$ (et non $F$) est d\'efinie tr\`es simplement comme 
$$
\mbox{\fbox{$
\Omega = D \omega 
$}}
$$
$\Omega$ est donc une $2$-forme \`a valeurs dans $Lie \, G$. Par 
construction, c'est une forme horizontale. Noter que la forme de 
connexion, au contraire, est verticale puisque les vecteurs horizontaux 
constituent son noyau.
A l'aide de la d\'efinition g\'en\'erale de $D$, on montre alors que
$$
\mbox{\fbox{$
\Omega = d\omega + {1\over 2}[\omega \wedge \omega]
$}}
$$
Il n'y a aucune contradiction entre les expressions donn\'ees pr\'ec\'edemment pour
$D\overrightarrow{\lambda}$ et $D \omega $ (pr\'esence du facteur $1/2$) car la forme $\omega$ est \`a valeurs dans $Lie \, G$. Notons aussi que le r\'esultat obtenu pour $\overrightarrow{\lambda}$
supposait ce dernier horizontal, ce qui n'est manifestement jamais le cas 
pour $\omega$.

\subparagraph{Aspect local}
De la m\^eme fa\c con que nous sommes pass\'es de $\omega$, d\'efini sur $P$ \`a son
expression locale, le potentiel de jauge $A$ (la $1$-forme d\'efinie sur $M$ par 
$A(v) =  \omega(\sigma_*v)$), nous pouvons
passer de $\Omega$ (d\'efini sur $P$) \`a son expression locale, qui est une
$2$-forme $F$ d\'efinie sur $M$ et \`a valeurs dans $Lie \, G$.
Choisissons en effet une section locale $x\in M \longrightarrow \sigma(x) 
\in
P$, et $v_1,v_2$, deux vecteurs en $x$. On pose
$$
\mbox{\fbox{$
F(v_1,v_2)  =  \Omega(\sigma_* v_1,\sigma_*v_2)
$}}
$$
On transporte (``push forward'') donc d'abord $v_1$ et $v_2$ par 
l'application
lin\'eaire tangente \`a $\sigma$ puis on utilise $\Omega$ pour obtenir un
\'el\'ement de $Lie \, G$.

Si $v_1$ et $v_2$ sont deux vecteurs $e_\mu, e_\nu$ appartenant \`a un rep\`ere
mobile $\{e_\mu\}$, on pose $F_{\mu\nu}  =  F(e_\mu,e_\nu)$. Cette
expression appartient \`a $Lie \, G$. Si on choisit une base $\{X_\alpha \}$ 
de
cette alg\`ebre de Lie, on peut d\'ecomposer $F_{\mu\nu}$ sur cette base, et on
\'ecrit donc
$$
\mbox{\fbox{$
F_{\mu\nu} = F_{\mu\nu}^\alpha X_\alpha
$}}
$$
Les indices $\mu$ et $\nu$ peuvent prendre des valeurs allant de $1$ \`a $n =
dim \, M$ mais l'indice $\alpha$ va de $1$ \`a $dim \, G$.

\subsubsection{Passage aux fibr\'es associ\'es}

Soit maintenant $E=E(M,V)$ un fibr\'e vectoriel associ\'e \`a $P$ via une
repr\'esentation $\rho$ de $G$ sur un espace vectoriel $V$ de dimension $p$. 

\subparagraph{Aspect global}
La forme de connexion $\omega$, \`a valeurs dans ${\frak g}$ devient $\rho(\omega)$ et,
de la m\^eme fa\c con, la forme de courbure 
 $\Omega$, \`a valeurs dans ${\frak g}$ devient $\rho(\Omega)$ et est
\`a valeurs dans les endomorphismes de $V$. On a encore
$$
\rho(\Omega) = \rho d(\omega) + {1\over 2}[\rho(\omega) \wedge \rho(\omega)]
$$

\subparagraph{Aspect local}
Pour obtenir l'expression locale de l'op\'erateur de courbure, il existe deux possibilit\'es.
La premi\`ere est de
transporter l'expression de $F$, un \'el\'ement de $\Omega^2(M,{\frak g}),$ dans le fibr\'e vectoriel
caract\'eris\'e par la repr\'esentation $\rho$ (il s'agit donc en fait de  $\rho(F) \in 
\Omega^2(M,End(V))$ mais nous continuerons \`a le noter abusivement $F$) en posant
$$
\mbox{\fbox{$
F  =  {1 \over 2} F_{\mu\nu} dx^\mu \wedge dx^\nu 
\mbox{\ avec \ }
F_{\mu\nu} = F^\alpha_{\mu\nu}\, T_\alpha \mbox{\ et \ } T_\alpha = \rho(X_\alpha)
$}}
$$
Notons que $X_\alpha$ est un \'el\'ement de l'alg\`ebre de Lie de
$G$ mais que $T_\alpha = \rho(X_\alpha)$ est un endomorphisme de l'espace
vectoriel $V$. Relativement \`a un rep\`ere local ${e_i}$ du fibr\'e $E$ (c'est 
\`a dire le choix, en tout point $x$ appartenant \`a un ouvert $U\subset M$, d'une base de la
fibre $V_x$ au dessus de $x$) on pourra noter $F^i_j{}_{\mu\nu}$ les 
composantes de l'endomorphisme $F_{\mu\nu}$.


La deuxi\`eme possibilit\'e est de repr\'esenter localement la $2$-forme $\rho(\Omega)$.
Notons \`a ce propos
que
$$A^i_j \wedge A^j_k = {1\over 2} [A \wedge A]^i_k
$$
en effet
\begin{eqnarray*}
A^i_j \wedge A^j_k &=& 
A^\alpha(T_\alpha)^i_j\wedge A^\beta(T_\beta)^j_k = 
A^\alpha \wedge A^\beta (T_\alpha T_\beta)^i_k \\ {} &=& 
{1\over 2}( A^\alpha \wedge A^\beta (T_\alpha T_\beta)^i_k + 
 A^\beta \wedge A^\alpha (T_\beta T_\alpha)^i_k) \\ {} &=&
{1\over 2}(A^\alpha\wedge A^\beta [T_\alpha,T_\beta]^i_k)=
{1\over 2}[A\wedge A]^i_k
\end{eqnarray*}

Ainsi $$F = dA + {1\over 2}[A\wedge A] \Leftrightarrow F^i_j = dA^i_j + A^i_k\wedge A^k_j$$
chose que nous savions d\'ej\`a.


Rappelons une fois de plus que la courbure $\Omega$, dans $P$, ne d\'epend que du choix de la connexion, 
alors
que l'objet $F$ qui lui correspond sur la base (ou dans un fibr\'e associ\'e) 
d\'epend \'egalement du choix d'une section locale $\sigma$.
Nous notons encore $F$ cet op\'erateur de courbure, mais un ``puriste'' 
rajouterait
quelque part les symboles $\sigma$ et $\rho$ pour se rappeler que la
d\'efinition d\'epend de la section locale $\sigma$ (d\'ependance ``de jauge'') 
et
de la repr\'esentation $\rho$ choisie.

Encore une fois, nous laissons au lecteur le soin de d\'emontrer que cette d\'efinition de la
courbure au niveau des fibr\'es associ\'es co\"incide avec celle donn\'ee
dans les sections pr\'ec\'edentes.

\subparagraph{L'op\'erateur $D^2$}
Les expressions donn\'ees pour $D\overrightarrow{\lambda}$ et $D\omega$ conduisent
\`a l'identit\'e de Bianchi que nous avons d\'ej\`a \'etudi\'ee auparavant (mais 
\'ecrite \`a l'aide de de $F$)\index{identit\'es de Bianchi}
$$D\Omega = D^2 \omega = 0$$ ainsi
qu'\`a $$D^2 \overrightarrow{\lambda} = \rho(\Omega) \wedge \overrightarrow{\lambda}$$ lorsque
$\overrightarrow{\lambda}$ est horizontale et \'equivariante.

\subsubsection{Mise en garde concernant l'utilisation de la notation $D$}

La diff\'erentielle covariante $D$ agit sur les formes (n'importe quelles 
formes) d\'efinies sur le fibr\'e principal $P$. Certaines de ces formes sont 
verticales (c'est le cas de la forme de connexion $\omega$), et les 
autres sont horizontales (c'est le cas de la forme de courbure). Par 
ailleurs, on sait comment associer, \`a toute $p$-forme $\lambda$ sur $M$ \`a valeurs 
dans le fibr\'e vectoriel $E$, une $p$-forme $\overrightarrow{\lambda}$ sur $P$,
 \'equivariante, \`a 
valeurs dans la fibre type ; par construction, $\overrightarrow{\lambda}$ est aussi une forme 
horizontale. La diff\'erentielle covariante $D$ agit aussi bien sur $\omega$ 
que sur $\Omega$ ou $\overrightarrow{\lambda}$. Cependant, nous avons {\em d\'ej\`a} introduit un 
op\'erateur $D$ agissant sur les sections de fibr\'es associ\'es (ou plus 
g\'en\'eralement sur les sections-$p$-formes) ainsi que les r\`egles de calcul 
correspondantes. Pour un objet \'equivariant et horizontal, il est facile 
de voir que les notations sont compatibles en ce sens que, si 
$\lambda = classe((z,\overrightarrow{\lambda}))$ alors
$d^\nabla \lambda = classe((z.D\overrightarrow{\lambda}))$.
Par contre, il est dangereux 
 d'utiliser la notation $D$ en dehors de ce contexte ; c'est ainsi, par 
 exemple, que nous nous garderons d'appliquer $D$ (ou $d^\nabla$) au 
 potentiel de jauge $A$ lui-m\^eme (car il s'agit alors du pull-back, via une section locale, 
 d'une forme verticale).

\subsection {Ecritures diverses de la courbure $F$ (r\'ecapitulatif)}

Nous avons d\'ej\`a mentionn\'e les diverses fa\c cons de consid\'erer cet objet. Nous 
nous
contentons ici de rassembler les diverses notations et formules 
essentielles.
On pose $T_\alpha  =  \rho (X_\alpha)$.

\begin{eqnarray*}
F & = & (d^\nabla)^2 \\
F_{\mu\nu} & = & [\nabla_\mu,\nabla_\nu] - \nabla_{[\mu,\nu]}  \\
F^i_j &  = & dA^i_j + A^i_k \wedge  A^k_j \Leftrightarrow F = dA + A 
\wedge A \\
F & = & {1 \over 2}\, F_{\mu\nu} \, dx^\mu \wedge dx^\nu \\
F_{\mu\nu} & = & F^\alpha_{\mu\nu} T_\alpha \\
F^i_j{}_{\mu\nu} & = & F^\alpha_{\mu\nu} (T_\alpha)^i_j
\end{eqnarray*}

Les physiciens des particules ont l'habitude de d\'evelopper la courbure sur 
les
g\'en\'erateurs de l'alg\`ebre de Lie, et pour cette raison utilisent le plus
souvent les composantes $F^\alpha_{\mu\nu}$.
Par contre, les astrophysiciens, cosmologistes et autres sp\'ecialistes des
ph\'enom\`enes gravitationnels ont plut\^ot l'habitude d'utiliser les composantes
$F^i_j{}_{\mu\nu}$. Le lien entre les deux sortes de notations se fait 
gr\^ace
aux relations pr\'ec\'edentes (rappelons que $(T_\alpha)^i_j$ d\'esigne, dans 
le
rep\`ere $\{e_i\}$ du fibr\'e vectoriel, les \'el\'ements de matrice 
$(i,j)$ de
la matrice $T_\alpha$, image du g\'en\'erateur $X_\alpha$ de $Lie \, G$ dans la
repr\'esentation consid\'er\'ee.


\subsection{Connexions et op\'erations de Cartan}

Dans le chapitre pr\'ec\'edent, nous avons vu qu'un espace fibr\'e principal $P$
permettait automatiquement de d\'efinir une op\'eration de Cartan de l'alg\`ebre de Lie ${\frak g}$
sur l'alg\`ebre diff\'erentielle $\Omega(P)$ des formes diff\'erentielles sur $P$, c'est \`a dire
un produit int\'erieur $i_X$ et une d\'eriv\'ee de Lie $L_X$, index\'es par $X\in {\frak g}$ ob\'eissant
aux relations usuelles. La seule structure de fibr\'e principal nous a ainsi permis de d\'efinir et de
caract\'eriser les notions de formes invariantes, horizontales ou basiques.

Nous voulons maintenant g\'en\'eraliser ces op\'erateurs de fa\c con \`a ce qu'ils puissent agir sur
des formes sur $P$ \`a valeurs dans une alg\`ebre de Lie ${\frak g}$, c'est \`a dire sur les
\'el\'ements de $\Omega(P,{\frak g} )= {\frak g} \otimes \Omega(P)$.

Soit $\tau \in {\frak g}\otimes \Omega^s(P)$ et $X \in {\frak g}$, on peut d\'ecomposer
$\tau$ comme une somme de termes du type $Y \otimes \alpha$ avec $Y\in {\frak g}$ et $\alpha \in \Omega^s(P)$.
On d\'efinit :
$$
i_X(Y\otimes \alpha)  =  Y \otimes i_X\alpha  \in \Omega^{s-1}(P)
$$
$$
L_X(Y\otimes \alpha)  =  Y \otimes L_X\alpha \in  \Omega^s(P)
$$
$$
d(Y\otimes \alpha)  =  Y \otimes d \alpha \in  \Omega^{s+1}(P)
$$
ainsi que le crochet
$$
[Y_1\otimes \alpha \wedge Y_2 \otimes \beta]  =  [Y_1,Y_2] \otimes \alpha \wedge \beta
$$
avec  $Y_1, Y_2 \in {\frak g}$ et $\alpha, \beta \in \Omega(P)$.
Ces concepts se g\'en\'eralisent imm\'ediatement au cas o\`u l'alg\`ebre diff\'erentielle $\Omega(P)$
est remplac\'ee par une alg\`ebre diff\'erentielle $\ZZ$-gradu\'ee
 quelconque munie d'une op\'eration de Cartan.

Une connexion alg\'ebrique $\omega$ sur l'alg\`ebre diff\'erentielle $\Omega(P)$
est, par d\'efinition, la donn\'ee 
d'un \'el\'ement de $ {\frak g}\otimes \Omega^1(P)$
ob\'eissant aux deux conditions suivantes
\begin{eqnarray*}
\forall X \in {\frak g} \quad  i_X \omega = X \; {\mbox{(verticalit\'e)}} \\
\forall X \in {\frak g} \quad  L_X \omega = [\omega \wedge  X] \;  {\mbox{(\'equivariance)}}
\end{eqnarray*}


Cette d\'efinition est encore valable lorsqu'on remplace $\Omega(P)$ par une 
alg\`ebre diff\'erentielle $\ZZ$-gradu\'ee quelconque munie d'une op\'eration de Cartan.
Nous laissons au lecteur le soin de d\'emontrer que la notion de forme de connexion, telle
qu'elle a \'et\'e d\'efinie pr\'ec\'edemment co\"incide bien avec la notion de connexion alg\'ebrique
sur $\Omega(P)$.

De fa\c con g\'en\'erale, on appelle {\sl courbure d'une connexion alg\'ebrique\/}  la $2$-forme 
$\Omega$ \`a valeurs dans ${\frak g}$ d\'efinie par
$$\Omega  =  d \omega + {1\over 2} \, [\omega \wedge \omega]$$
On v\'erifie alors les deux propri\'et\'es suivantes :
\begin{eqnarray*}
\forall X \in {\frak g}\quad i_X \Omega = 0  \; {\mbox{(horizontalit\'e)}} \\
\forall X \in {\frak g}\quad L_X \Omega  = [\Omega \wedge X]  \;  {\mbox{(\'equivariance)}}
\end{eqnarray*}
 


\subsection{Groupe d'holonomie d'une connexion, fibr\'e d'holonomie} 

Il s'agit l\`a d'un important sujet mais nous ne ferons que donner quelques 
d\'efinitions g\'en\'erales et \'enoncer un th\'eor\`eme c\'el\`ebre.

Soit $z$ un point d'un fibr\'e principal $P = P(M,G)$ qu'on supposera muni 
d'une connexion $\omega$. Le {\sl groupe d'holonomie}
 {\index{groupe d'holonomie}} $H(z)$ de la connexion $\omega$ au point $z$ est 
d\'efini comme suit
$$ H(z) = \{g \in G \vert z \, \mbox{\rm et} \, zg \, \mbox{\rm  peuvent  \^etre  reli\'es  par 
une courbe  horizontale} \}
$$
Il est bien \'evident que $H(z)$ est un sous-groupe du groupe structural 
mais il faut bien noter que le sous-groupe en question peut \^etre, dans 
certains cas, assez ``petit''. Intuitivement, on d\'ecrit, sur la base, une 
courbe quelconque revenant \`a son point de d\'epart, et on d\'ecide de 
transporter un rep\`ere avec soi (le ``rep\`ere'' $z$) par parall\'elisme. 
Lorsqu'on revient \`a son point de d\'epart, le rep\`ere aura tourn\'e par un 
\'el\'ement $g$ de $G$ qui d\'epend, en g\'en\'eral, aussi bien du chemin suivi que 
du rep\`ere de d\'epart.

%%%%%%%%%%%%%
%%%%%%%%%%%%Variation de $H(z)$ avec $z$. $***TODO$
%%%%%%%%%%%%%

Nous venons de consid\'erer l'ensemble des points de $P$ qui, d'une part, 
sont situ\'es dans la m\^eme fibre que 
$z$ et qui, d'autre part, peuvent \^etre reli\'es \`a $z$ 
par une courbe horizontale. On peut, de fa\c con plus g\'en\'erale ne pas 
supposer que ces points sont dans la m\^eme fibre. On obtient ainsi la 
d\'efinition du {\sl fibr\'e d'holonomie}{\index{fibr\'e d'holonomie}} en $z$:
$$
Y(z) = \{ z' \in P \vert  z \, {\mbox{\rm et}} \, z' \mbox{\rm peuvent \^etre reli\'es par 
une courbe horizontale} \}
$$
On peut montrer que $Y(z)$ est effectivement un espace fibr\'e de groupe 
structural $H(z)$. Noter qu'on obtient ainsi un espace fibr\'e {\em pour chaque point \/} $z.$

Soient $Z_1$ et $Z_2$, deux champs de vecteurs horizontaux et $\Omega$ la 
$2$-forme de courbure associ\'ee \`a la connexion choisie. On d\'emontre 
alors (th\'eor\`eme d'Ambrose-Singer) que les \'el\'ements de $Lie \, G$ qui sont de la forme 
$\Omega_z(Z_1(z),Z_2(z))$ engendrent l'alg\`ebre de Lie du groupe 
d'holonomie au point $z$.

D'une certaine fa\c con, le groupe d'holonomie en $z$ fournit donc une 
estimation de la courbure en ce point. Ceci est d'ailleurs assez intuitif 
: si un transport par parallelisme le long d'une courbe ferm\'ee n'entra\^ine 
aucune ``rotation'' du rep\`ere, c'est que la r\'egion dans laquelle on se 
prom\`ene est assez plate $\ldots$ 


\subsection{R\'eduction des connexions}

Soit $P=P(M,G)$, un espace fibr\'e principal, $H$ un sous-groupe de $G$ et soit $Q=Q(M,H)$ une 
r\'eduction du fibr\'e $P$. Nous savons qu'une telle r\'eduction est associ\'ee 
au choix d'une section globale (que nous d\'esignerons par $\Phi$) dans le fibr\'e en espaces homog\`enes 
$E=E(M,G/H)$ qui est associ\'e \`a $P$ via l'action \`a gauche de $G$ sur $G/H$. On rappelle
que $Q = p^{-1}(\Phi(M))$ o\`u $p$ d\'esigne la projection de $P$ sur $E = P \; mod \; H$. 

Soit maintenant $\omega$ une forme de connexion d\'efinissant une forme de connexion principale dans $P$. On dira que 
cette connexion est r\'eductible et se r\'eduit \`a $Q$ si la restriction (au sens de la 
restriction des applications) de $\omega$ \`a $Q$ d\'efinit une connexion 
principale dans le fibr\'e $Q$. En particulier, $\omega$, restreinte \`a $Q$ 
doit \^etre \`a valeurs dans l'alg\`ebre de Lie du groupe $H$.

Au niveau des alg\`ebres de Lie, nous pouvons \'ecrire ${\frak g} = {\frak 
h} \oplus {\frak s}$ o\`u ${\frak g}$ et ${\frak h}$ d\'esignent respectivement 
les alg\`ebres de Lie de $G$ et de $H$, et o\`u
${\frak s}$ est un sous-espace vectoriel suppl\'ementaire de 
${\frak h}$ dans ${\frak g}$, espace vectoriel qui peut \^etre identifi\'e avec l'espace 
tangent \`a l'origine de $S =  G/H$. Supposer la connexion $\omega$ 
r\'eductible revient \`a supposer qu'elle ne poss\`ede pas de composantes le 
long de l'espace vectoriel ${\frak s}$.

Soit $\omega$ une forme de connexion quelconque dans le fibr\'e $P$ et $\omega_Q$ sa restriction
au sous-fibr\'e $Q$. On peut toujours d\'ecomposer
$$\omega = \omega_\Phi + \theta_\Phi $$
avec $\omega_\Phi(V) \in {\frak h}$ et  $\theta_\Phi(V) \in {\frak s}$ lorsque $V \in TP$.
Il est facile de voir que  $\omega_\Phi$ est une forme de connexion sur le fibr\'e principal $Q$ et
que, plus g\'en\'eralement, il existe une bijection entre l'ensemble des connexions $\omega$ sur $P$
et l'ensemble des couples $(\omega_\Phi,\theta_\Phi)$ o\`u $\omega_Q$ est une connexion sur $Q$ et o\`u
$\theta_\Phi$ est une $1$-forme \`a valeurs dans ${\frak s}$. Supposer $\omega$ r\'eductible revient
d\`es lors \`a supposer que $\theta_\Phi = 0$.

Soit $\Omega = d\omega + {1\over 2} [\omega \wedge \omega]$ la courbure de $\omega$ et $\Omega_Q$ sa restriction
\`a $Q$. Il vient imm\'ediatement :
$$
\Omega_Q = \Omega_\Phi + D\theta_\Phi + {1\over 2} [\theta_\Phi \wedge \theta_\Phi]
$$
o\`u
$\Omega_\Phi = d\omega_\Phi + {1\over 2} [\omega_\Phi \wedge \omega_\Phi]$ et
$D\theta_\Phi = d\theta_\Phi + [\omega_\Phi \wedge \theta_\Phi].$


Cette d\'ecomposition de la forme de connexion $\omega$ et de la courbure correspondante $\Omega$
peut \^etre effectu\'ee chaque fois qu'on a une r\'eduction du fibr\'e $P$ (chaque fois qu'on a
une section globale $\Phi$ de $P \; mod \; H$). La forme de connexion elle-m\^eme  n'est  pas
n\'ecessairement  r\'eductible :  en g\'en\'eral  $\theta_\Phi \neq 0$.

Citons trois exemples particuli\`erement importants de ce type de construction. Nous n'aurons pas
le loisir de beaucoup les discuter plus avant
 mais nous les proposons n\'eanmoins au lecteur, comme th\`emes de r\'eflexion.
\begin{itemize}
\item
En physique, cette d\'ecomposition de la forme de connexion appara\^\i t lorsqu'on analyse
le ph\'enom\`ene de {\sl brisure de sym\'etrie\/} : la forme de connexion $\omega_\Phi$ (o\`u
plut\^ot le potentiel de Yang-Mills associ\'e) est le champ de Yang-Mills survivant \`a la
brisure de sym\'etrie, et $\tau_\Phi$  (o\`u plut\^ot le potentiel associ\'e) d\'ecrit les bosons
vectoriels massifs.
\item
Il existe un fibr\'e principal dont nous n'avons jamais parl\'e  et qu'on peut construire
en \'elargissant le fibr\'e $FM$ des rep\`eres lin\'eaires : c'est le fibr\'e $AM$ des {\sl rep\`eres affines\/}.
On le construit par la m\'ethode g\'en\'erale d'\'elargissement des espaces fibr\'es, en
rempla\c cant le groupe de structure de $FM$ (\`a savoir le groupe lin\'eaire $GL(n)$) par son
produit semi-direct avec les translations, c'est \`a dire le groupe affine $GL(n) \circ \RR^n$.
Conform\'ement \`a la discussion qui pr\'ec\`ede, on peut, \`a l'aide d'une connexion lin\'eaire (une
connexion sur $FM$) et de la $1$-forme canonique $\theta$ sur $FM$ \`a valeurs dans $\RR^n$
 (un \'el\'ement particulier de $\Lambda^1(FM,\RR^n)$ sur lequel nous reviendrons en section
 \ref{sec:forme-canonique}),
fabriquer une forme de connexion sur le fibr\'e principal $AM$. Une telle connexion sur $AM$,
construite gr\^ace \`a la forme canonique $\theta$,
s'appelle {\sl connexion affine.\/} \index{connexion affine} 
(attention :  il existe des connexions sur $AM$ qui ne sont pas de ce type) et la
diff\'erentielle covariante $D\theta$ s'interpr\`ete, dans ce
cas, comme la $2$-forme de {\sl torsion\/}{\index{torsion}}. Nous y reviendrons.
Notons que, dans le cas g\'en\'eral  on peut d\'esigner $D\theta$ sous le nom de
forme de {\sl torsion g\'en\'eralis\'ee.\/}{\index{torsion g\'en\'eralis\'ee}}
\item
On peut aussi partir du fibr\'e principal $FM$ des rep\`eres lin\'eaires et, gr\^ace au choix
d'une m\'etrique (une section globale de $FM \; mod \; SO(n)$, dont la fibre type est $GL(n)/SO(N)$),
 le r\'eduire \`a un fibr\'e de rep\`eres orthonorm\'es not\'e $OM$. Une connexion arbitraire
sur $FM$ n'est pas n\'ecessairement r\'eductible. On dira qu'une connexion sur $FM$ est une
{\sl connexion m\'etrique}\index{connexion m\'etrique}, ou encore, qu'elle est compatible
avec la m\'etrique, si elle est r\'eductible. Bien entendu, on pourrait tout aussi bien choisir
directement  une connexion sur un fibr\'e $OM$ lui-m\^eme caract\'eris\'e par le choix d'une m\'etrique.
Nous y reviendrons \'egalement.

\end{itemize}

\section{Cas particulier des connexions lin\'eaires} \label{sec:connexions-lineaires}

\subsection{D\'efinition et g\'en\'eralit\'es}

{\it A priori}, un fibr\'e principal donn\'e n'est pas n\'ecessairement reli\'e au
fibr\'e des rep\`eres lin\'eaires d'une vari\'et\'e. Cela \'etant, il est certain que 
le
fibr\'e des rep\`eres lin\'eaires fournit un exemple particuli\`erement remarquable
d'espace fibr\'e. Il en va de m\^eme, plus g\'en\'eralement, pour n'importe quel 
fibr\'e
principal associ\'e au fibr\'e tangent d'une vari\'et\'e et pour lequel, donc, le
groupe structural est un sous-groupe de $GL(n,\RR)$.

On dira qu'une connexion est une {\sl connexion lin\'eaire \/}
 {\index{connexion lin\'eaire }} si elle est
d\'efinie dans le fibr\'e principal des rep\`eres lin\'eaires ou dans un 
sous-fibr\'e de
ce dernier. Ce qu'il y a de particulier dans ce cas est que les indices de
fibre (que nous avons not\'e $i,j,k\ldots$ dans les sections pr\'ec\'edentes) 
peuvent
co\"incider ---ou tout au moins \^etre canoniquement associ\'es --- avec les
indices de base (que nous avons not\'e $\mu,\nu,\rho,\ldots$ dans les 
sections
pr\'ec\'edentes).
Dans le paragraphe consacr\'e aux diff\'erentielles ext\'erieures covariantes, 
nous
nous sommes efforc\'es de bien \'etablir une distinction entre ces deux types
d'indices. Le fait de pouvoir les confondre, dans le cas des 
connexions
lin\'eaires, ouvre de nouvelles possibilit\'es (on peut ainsi, par exemple,
``contracter'' un indice de fibre avec un indice de base) mais est 
\'egalement \`a
l'origine de confusions dangereuses$\ldots$

Bien \'evidemment, les connexions dans des fibr\'es vectoriels 
associ\'es quelconques (non reli\'es au fibr\'e tangent) sont aussi 
``lin\'eaires'' que ``nos''   
connexions lin\'eaires mais il se trouve qu'une grande partie de la 
plan\`ete (en particulier la communaut\'e des physiciens 
th\'eoriciens) a adopt\'e cette terminologie, par ailleurs commode.

Nous venons de d\'efinir une connexion lin\'eaire comme connexion d\'efinie dans 
le
fibr\'e des rep\`eres lin\'eaires ou dans un sous-fibr\'e de ce dernier. Il y a l\`a 
une
subtilit\'e qu'il faut bien comprendre : il est certain qu'une forme de
connexion \`a valeurs dans l'alg\`ebre de Lie du groupe $H$, avec $H \subset 
G$, peut s'\'etendre \`a une forme de 
connexion \`a
valeurs dans l'alg\`ebre de Lie de $G$ puisque tout fibr\'e principal peut \^etre
\'elargi (relire \`a ce sujet la section  consacr\'ee au changement de 
groupe
structural dans les fibr\'es principaux) et qu'il suffit alors de mettre \`a z\'ero les
composantes suppl\'ementaires de la forme de connexion choisie.
Par contre, et m\^eme dans le cas o\`u le fibr\'e des rep\`eres lin\'eaires peut \^etre
r\'eduit (relire la m\^eme section), il n'est pas du tout \'evident que la forme 
de
connexion puisse l'\^etre. Nous reviendrons \`a ce probl\`eme dans la section
consacr\'ee \`a l'\'etude des connexion riemanniennes.

Il faut enfin attirer l'attention du lecteur sur le fait qu'il est {\it a
priori\/} possible de fabriquer, \`a partir d'une vari\'et\'e diff\'erentiable de
dimension $n$ donn\'ee, diff\'erents fibr\'es principaux ayant pour groupe
structural $GL(n,\RR)$ et ne co\"incidant pas entre eux. Par exemple, on peut
choisir une vari\'et\'e non parall\'elisable (comme la sph\`ere $S^4$) et 
construire
d'une part le fibr\'e principal (non trivial) des rep\`eres lin\'eaires $P$ ainsi que le fibr\'e principal
trivial $Q=S^4\times GL(4,\RR)$. 

\subsection{Potentiel de jauge et courbure des connexions lin\'eaires}
 \label{sec:courbure-con-lineaire}
 Une connexion lin\'eaire \'etant un cas particulier de connexion principale, 
tout
ce qui a \'et\'e \'ecrit pr\'ec\'edemment \`a ce sujet reste vrai. Nous nous 
contenterons
donc de re-\'ecrire les formules les plus utiles dans le contexte pr\'esent. 

\bigskip

{\em Pour
des raisons historiques, le potentiel de jauge se note plut\^ot $\Gamma$ (et
non $A$) et le tenseur de courbure se note plut\^ot $R$ (et non $F$).}

\begin{itemize}
\item
Soit $\{X_a\}$ une base de $Lie \, GL(n,\RR)$ et
$\{{\partial \over \partial x^\mu}\}$ le rep\`ere naturel associ\'e \`a une carte
locale sur $M$. On pourra \'ecrire $$\Gamma = \Gamma^a_\mu X_a dx^\mu$$
\item
En gardant la m\^eme notation $X_a$ pour les matrices qui repr\'esentent les
g\'en\'erateurs, matrices qui agissent donc sur les $n$-uplets de composantes 
des
vecteurs tangents, $\Gamma$ devient alors une {\sl matrice de connexion\/} 
 {\index{matrice de connexion}} 
dont
les \'el\'ements de matrice $\Gamma^\nu_\rho$ sont des $1$-formes, puisque
$$\Gamma^\nu_\rho = \Gamma^\nu_{\rho\mu} dx^\mu \, {\mbox{\rm avec}} \,
\Gamma^\nu_{\rho\mu}
= \Gamma^a_\mu (X_a)^\nu_\rho
$$
Les nombres $\Gamma^\nu_{\rho\mu}$ sont les {\sl coefficients de 
connexion\/}
 {\index{coefficients de 
connexion}}
qu'on d\'esigne souvent, dans ce cas, sous le nom de {\sl Symboles de
Christoffel\/} {\index{symboles de
Christoffel}}.
En fait, les symboles de Christoffel d\'esignent traditionnellement les coefficients de connexion
associ\'es \`a la connexion riemannienne (connexion de Levi-Civita) \'ecrits
 dans un rep\`ere naturel (voir plus loin).
\item
Attention, rien ne nous oblige \`a choisir la m\^eme base dans la fibre au 
point
$P$ (\ie dans $T(M,P)$) et dans l'espace tangent au point $P$ (encore 
$T(M,P)$
!). Ainsi donc, nous pouvons choisir sur la base un rep\`ere naturel
$\{\partial_\mu\}$ et, sur la fibre ---qui co\"incide avec la base--- un
rep\`ere mobile $\{e_\alpha = \Lambda^\mu_\alpha \partial_\mu\}$ ; dans ce 
cas,
les \'el\'ements de matrice se noteront $\Gamma^\alpha_\beta$ et ce seront 
\'evidement
toujours des $1$-formes $\Gamma^\alpha_\beta = \Gamma^\alpha_{\beta\mu} 
dx^\mu$.
\item
Par ailleurs, rien ne nous oblige, non plus, \`a choisir un rep\`ere naturel 
sur
la base$\ldots$ On pourrait, en effet, tr\`es bien choisir un autre rep\`ere mobile
$e_\mu$, avec co-rep\`ere dual $e^\mu$, auquel cas, on \'ecrirait
$\Gamma^\alpha_\beta = \Gamma^\alpha_{\beta\mu} e^\mu$.
\item
Dans le cas o\`u nous choisissons les indices de base diff\'erents des indices de
fibre, les diverses formules donn\'ees dans les sections pr\'ec\'edentes
 pour un fibr\'e vectoriel quelconque
 restent absolument identiques (remplacer seulement les indices de fibre ``$i,j$'' par les indices
appropri\'es).
\item
Dans le cas o\`u l'on d\'ecide d'utiliser un seul et unique rep\`ere local
(par exemple un rep\`ere naturel $e_\mu  =  \partial_\mu$ et co-rep\`ere 
correspondant $e^\mu  =  dx^\mu$), il faut faire tr\`es attention \`a la
position de l'indice de forme (il n'existe pas de conventions 
universelles!)
En g\'en\'eral, 
$$ 
\mbox{\fbox{$
\nabla e_\rho = e_\nu    \Gamma^\nu_\rho \, = e_\nu 
\Gamma^\nu_{\rho \mu} e^\mu
$}}
$$
Pour un vecteur $v = e_\nu (.) \, v^\nu \in TM$, on obtient

\begin{eqnarray*}
\nabla v & = & \nabla(e_\rho v^\rho) \\
{} & = &  (e_\nu    \Gamma^\nu_\rho)v^\rho + e_\rho    dv^\rho \\
{} & = &  (\Gamma^\nu_{\rho \mu} v^\rho + e_\mu[v^\nu]) e_\nu    e^\mu
\\
& = & \nabla_\mu v \, e^\mu
\end{eqnarray*}
Notons que $$\nabla_\mu v = \langle \nabla v,e_\mu \rangle =  
v^\nu_{;\mu} e_\nu 
$$
Pour les \'el\'ements du dual, la diff\'erentiation covariante introduit, comme
d'habitude, un signe ``moins'':
$$
\mbox{\fbox{$
\nabla e^\nu = -  e^\rho \Gamma^\nu_\rho   =  - e^\rho 
\Gamma^\nu_{\rho\mu} e^\mu
$}}    
$$
Attention : conform\'ement \`a nos conventions g\'en\'erales (que nous ne respectons pas toujours !)
nous avons \'ecrit les composantes {\em \`a droite\/} des vecteurs de base, mais, bien que
$e_nu$ soit une d\'erivation de l'alg\`ebre des fonctions, il ne faut pas
confondre le champ de vecteurs $v=e_\nu v^\nu$, qui signifie, en fait,
 $v[.] = e_\nu[.] v^\nu$ avec la fonction scalaire $e_\nu[v^\nu]$ !

Re-\'ecrivons, pour terminer, la loi de transformation des coefficients de
connexion (avec, par exemple $e^\prime_{\nu^\prime} =
\Lambda^\mu_{\nu^\prime} e_\mu$)
$$
\Gamma^\prime{}^{\nu^\prime}_{\rho^\prime \mu^\prime} = 
(\Lambda^{-1})^{\nu^\prime}_\sigma \Gamma^\sigma_{\tau\mu}
\Lambda^\tau_{\rho^\prime} \Lambda^\mu_{\mu^\prime}+ (\Lambda^{-1})^{\nu^\prime}_\sigma 
\partial_\mu
\Lambda^\sigma_{\rho^\prime} \Lambda^\mu_{\mu^\prime}$$

\item
En ce qui concerne le tenseur associ\'e \`a l'op\'erateur de courbure, il se 
nomme le
{\sl tenseur de Riemann \/}
{\index{tenseur de Riemann }} et ses composantes sont not\'ees
$R^\alpha_{\beta\mu\nu}$ (au lieu de $F^\alpha_{\beta\mu\nu}$). Si on 
choisit
la m\^eme base dans la fibre et dans l'espace tangent au point consid\'er\'e, il 
se
note alors $R^\rho_{\sigma\mu\nu}$ et il faut bien entendu se rappeler 
qu'il
est antisym\'etrique sur les indices $\mu$ et $\nu$ puisqu'il provient d'une
matrice $R = d \Gamma + \Gamma \wedge \Gamma$  dont les \'el\'ements de matrice
$R^\rho_\sigma$  sont des $2$-formes. 

Si le rep\`ere choisi  est un rep\`ere
naturel $\{e_\mu = {\partial \over \partial x^\mu}\}$, on a donc
$$R^\rho_\sigma = {1\over 2} R^\rho_{\sigma\mu\nu} dx^\mu \wedge dx^\nu $$

Si le rep\`ere choisi est  un rep\`ere mobile $\{e_\mu\}$ avec fonctions de
structure ${f_{\mu\nu}}^\rho$ d\'efinies par $[e_\mu,e_\nu]={f_{\mu\nu}}^\rho
e_\rho$, on a 
$$R^\rho_\sigma = {1\over 2} R^\rho_{\sigma\mu\nu} e^\mu \wedge 
e^\nu
$$
o\`u $e^\mu$ d\'esigne le corep\`ere mobile dual.

 L'expression $R = d\Gamma 
+
\Gamma \wedge \Gamma$, ou, plus simplement, l'equation de structure pour la
courbure, $$R(u,v) = [\nabla_u,\nabla_v] - \nabla_{[u,v]}$$ \'equation 
\'etablie en section
 \ref{sec:eq-de-structure-R}, \'equation qui est, bien s\^ur, encore valable dans le cas des
connexions lin\'eaires, nous permet de calculer  explicitement les 
composantes
de $R$ en fonction des coefficients de connexion et des fonctions de 
structure
du rep\`ere. On calcule simplement

$$
\begin{array}{|rcl|}
\hline
R^\alpha_{\beta\mu\nu} & = & \langle e^\alpha,
R(e_\mu,e_\nu)e_\beta\rangle \\
{} & = & ({\Gamma^\alpha}_{\beta \nu,\mu} - {\Gamma^\alpha}_{\beta 
\mu,\nu})
+({\Gamma^\alpha}_{\tau \mu}{\Gamma^\tau}_{\beta \nu} - 
{\Gamma^\alpha}_{\tau \nu}{\Gamma^\tau}_{\beta \mu} -
\Gamma^\alpha_{\beta\tau}{f_{\mu\nu}}^\tau)\\
\hline
\end{array}
$$

o\`u, comme d'habitude, on a not\'e $h_{,\mu}  =  e_\mu[h]$, pour toute
fonction $h$, que $\{e_\mu\}$ soit un rep\`ere mobile ou un rep\`ere naturel
(dans ce dernier cas, ${f_{\mu\nu}}^\tau = 0$).
L'alignement des indices haut et bas dans la formule pr\'ec\'edente est {\em pour l'instant\/}
sans importance, car on n'a pas encore de m\'etrique (de produit scalaire) pour identifier
un espace vectoriel avec son dual, c'est \`a dire pour ``monter'' ou ``descendre'' les indices.
Cela dit, c'est une bonne habitude d'\'ecrire ${\Gamma^\alpha}_{\beta\mu}$ et ${f_{\mu\nu}}^\rho$
plut\^ot que  $\Gamma^\alpha_{\beta\mu}$ et $f_{\mu\nu}^\rho$ car, dans la section suivante,
consacr\'ee aux connexions m\'etriques, nous poserons
 $\Gamma_{\alpha\beta\mu}  =  g_{\alpha \gamma}
{\Gamma^\gamma}_{\beta\mu}$ et
 $f_{\mu\nu\rho}  =  g_{\rho\sigma} {f_{\mu\nu}}^\sigma$
\end{itemize}

\subsection{Diff\'erentielle ext\'erieure covariante  (cas des connexions 
lin\'eaires)}

Nous avons d\'ej\`a d\'efini l'op\'erateur 
$d^{\nabla}: \Omega^p(M,{E})\mapsto \Omega^{p+1}(M,{E})$ agissant 
sur
les sections-$p$-formes d'un fibr\'e vectoriel quelconque. Lorsque ${E} = TM$
(ou une puissance tensorielle quelconque d'icelui), ce qui a \'et\'e
pr\'ec\'edemment \'ecrit reste vrai. La nouveaut\'e vient du fait que, par suite de
l'identification possible entre indices de base et indices de fibre, un 
seul
et m\^eme objet peut \^etre regard\'e de plusieurs fa\c cons diff\'erentes.
Nous allons
directement d\'efinir l'action de l'op\'erateur ${D}$, {\em agissant sur des objets
index\'es}  (par exemple ${B^\mu}_{\nu\rho}$) en d\'ecidant de {\it ne jamais faire
appara\^itre les indices de forme} : c'est ainsi que si nous nous int\'eressons \`a
une $2$-forme \`a valeurs dans le fibr\'e vectoriel $TM \otimes T^*M \otimes
T^*M$, objet dont la d\'ecomposition compl\`ete par rapport \`a un rep\`ere naturel 
s'\'ecrirait
$$
B = {1\over 2!} {B^\mu}_{\nu\rho\sigma\tau}
({\partial \over \partial x^\mu}  \otimes dx^\nu \otimes dx^\rho)
\bigotimes(dx^\sigma \wedge dx^\tau)
$$
ou, mieux encore, plus simplement (c'est \`a dire en passant le $\bigotimes$ sous silence)
$$
B = ({\partial \over \partial x^\mu}\otimes dx^\nu \otimes dx^\rho)
{1\over 2!} {B^\mu}_{\nu\rho\sigma\tau}
(dx^\sigma \wedge dx^\tau)
$$
Nous conviendrons de sous-entendre les indices de forme $\sigma$, $\tau$ et 
d'appliquer ${D}$ \`a l'objet ${B^\mu}_{\nu\rho}$ qui, \'evidemment, n'est plus 
une fonction mais une $2$-forme, puisque 
$$
\mbox{\fbox{$
{B^\mu}_{\nu\rho}={1\over 
2}{B^\mu}_{\nu\rho\sigma\tau} dx^\sigma \wedge dx^\tau
$}}
$$
C'est donc la notation elle-m\^eme qui d\'efinit le fibr\'e dans lequel on se 
place, puisque seuls apparaissent les indices de fibre. On voit donc que 
$$
\mbox{\fbox{$
B = ({\partial \over \partial x^\mu}\otimes dx^\nu \otimes 
dx^\rho)   {B^\mu}_{\nu\rho}
$}}
$$

Pour ce qui est de l'op\'erateur ${D}$ nous obtenons,
$$
\mbox{\fbox{$
{D}{B^\mu}_{\nu\rho} = d {B^\mu}_{\nu\rho} + \Gamma^\mu_\lambda \wedge 
{B^\lambda}_{\nu\rho} - \Gamma^\lambda_\nu \wedge {B^\mu}_{\lambda\rho}
- \Gamma^\lambda_\rho \wedge {B^\mu}_{\nu\lambda}
$}}
$$

Nous laissons au lecteur le soin de g\'en\'eraliser (de mani\`ere \'evidente) ces 
formules pour un objet $B$ quelconque ayant un nombre quelconque 
d'indices en haut et en bas. Dans le cas pr\'esent, l'objet obtenu  est 
donc une $3$-forme \`a valeurs dans $TM\otimes T^*M \otimes T^*M$ et 
on pourrait le noter, de fa\c con intrins\`eque, sous la forme $d^\nabla B$, 
avec, par cons\'equent
$$
\mbox{\fbox{$
d^\nabla B =
 ({\partial \over \partial x^\mu}\otimes dx^\nu \otimes dx^\rho)
 {D}{B^\mu}_{\nu\rho}
$}}
$$
L'inconv\'enient de la notation $d^\nabla B$ est qu'il faut se rappeler 
dans quel fibr\'e on se place ; en effet, rien ne nous interdit de 
consid\'erer $B$ comme une $0$-forme \`a valeurs dans $TM \otimes 
(T^*M)^{\otimes 4}$ ou m\^eme, comme une $1$-forme \`a valeurs dans $TM 
\otimes (T^*M)^{\otimes 3}\ldots$ le probl\`eme \'etant alors que les 
op\'erateurs $d^\nabla$ relatifs \`a ces diff\'erents fibr\'es sont diff\'erents et 
que donc, les objets $d^\nabla B$ obtenus sont \'egalement diff\'erents (et 
tous absolument intrins\`eques!). L'action de l'op\'erateur ${D}$, quant \`a 
elle, est bien d\'etermin\'ee, {\it \`a condition}, bien s\^ur, d'adopter la convention 
pr\'ec\'edemment d\'ecrite, \`a savoir le fait d'\'ecrire syst\'ematiquement 
les indices de fibre et de simultan\'ement masquer les indices de 
forme. Le lecteur saura donc imm\'ediatement calculer ${D}{B^\mu}_{\nu\rho}$
aussi bien que ${D}{B^\mu}_{\nu\rho\sigma\tau}$ ou que
${D}{B^\mu}_{\nu\rho\sigma}$.

Pour lever l'ambigu\"\i t\'e concernant la notation $d^\nabla$, il faudrait 
\'ecrire $d^\nabla_{(p,q)}$ pour la diff\'erentielle ext\'erieure covariante 
agissant sur les formes de degr\'e quelconque \`a valeurs dans $(TM)^{\otimes 
p} \otimes (T^*M)^{\otimes q}$. 

{\small
Pour illustrer notre propos, nous 
consid\'erons un premier exemple donn\'e par
un tenseur antisym\'etrique de rang $2$, not\'e $F = {1 \over 
2}F_{\mu\nu}\; dx^\mu\wedge dx^\nu =
F_{\mu\nu} dx^\mu\otimes dx^\nu$, sur la vari\'et\'e $M$.
Nous pouvons consid\'erer cet objet comme
\begin{itemize}
\item Une $2$-forme sur $M$ \`a valeurs r\'eelles. Dans ce cas, la 
diff\'erentielle ext\'erieure covariante, que nous devrions noter 
$d^\nabla_{0,0}$ co\"incide avec la diff\'erentielle ext\'erieure. En effet, 
puisqu'il n'y a aucun indice de fibre, l'op\'erateur ${D}$ agit sur $F$ et 
$${D}F = dF = d^\nabla_{0,0}F$$ Le r\'esultat est une $3$-forme.
\item Une $0$-forme sur $M$ \`a valeur dans le fibr\'e $T^*M \otimes T^*M$ 
(en l'occurrence, dans la partie antisym\'etrique de ce dernier). Dans ce 
cas l'op\'erateur ${D}$ agit sur $F_{\mu\nu}$ et 
$${D}F_{\mu\nu} = dF_{\mu\nu} - \Gamma^\sigma_\mu F_{\sigma\nu} - 
\Gamma^\sigma_\nu F_{\mu\sigma}
$$
Quant \`a la diff\'erentielle ext\'erieure covariante, que nous devrions noter $ 
d^\nabla_{0,2}$, elle est donn\'ee par
$$
d^\nabla_{0,2}F = (dx^\mu \otimes dx^\nu) {1\over 2} {D}F_{\mu\nu}
$$
et co\"incide donc tout simplement avec la diff\'erentielle covariante 
usuelle puisque $F$ est consid\'er\'ee comme $0$-forme : 
$\nabla F = {1\over 2} F_{\mu\nu;\rho}\, dx^\mu \otimes  dx^\nu \otimes 
dx^\rho$ avec $F_{\mu\nu;\rho} = F_{\mu\nu,\rho} - 
\Gamma^\sigma_{\mu\rho}F_{\sigma\nu}-\Gamma^\sigma_{\nu\rho}F_{\mu\sigma}.$
Comme d'habitude, la diff\'erentielle ext\'erieure covariante d'une $0$-forme 
co\"incide avec la diff\'erentielle covariante. Notons \'egalement que $\nabla 
F$ n'est pas compl\`etement antisym\'etrique : ce n'est pas une $3$-forme 
mais une $1$-forme \`a valeurs dans la partie antisym\'etrique du produit 
tensoriel $T^*M\otimes T^*M$.
\item 
Une $1$-forme sur $M$ \`a valeurs dans le fibr\'e vectoriel $T^*M$.
Conform\'ement \`a nos habitudes, nous faisons agir ${D}$ sur un objet dont on 
n'\'ecrit jamais les indices de forme. On pose donc $F = dx^\mu  
F_\mu$, ce qui d\'efinit en m\^eme temps la $1$-forme $F_\mu = F_{\mu\nu} 
dx^\nu$, et on calcule ${D}F_\mu = dF_\mu - \Gamma_\mu^\sigma F_\sigma$.
La diff\'erentielle ext\'erieure covariante $d^\nabla_{0,1}F = dx^\mu  
{D}F_\mu$ est donc une $2$-forme sur $M$ \`a valeurs dans $T^*M$.
\item
Les deux indices de $F_{\mu\nu}$ jouant des r\^oles semblables, on peut 
\'egalement d\'efinir la $1$-forme ${ F^\prime_\nu} = F_{\mu\nu} dx^\mu (= - 
F_\nu)$, calculer ${D}{ F^\prime_\nu} = d{ F^\prime_\nu} - \Gamma_\nu^\sigma 
{ F^\prime_\sigma}$ et poser ${ d^{'\nabla}_{0,1}}F = dx^\nu  
{D}{ F^\prime_\nu} (= - d^\nabla_{0,1} F)$
\end{itemize}
Notre deuxi\`eme exemple sera un tenseur sym\'etrique de rang $2$ not\'e $g = 
g_{\mu\nu} \, dx^\mu \otimes dx^\nu$. Cet objet peut \^etre consid\'er\'e comme 
une $0$-forme \`a valeurs dans le fibr\'e $T^*M \otimes T^*M$, en 
l'occurrence, dans la partie sym\'etrique de ce dernier, ou encore, de deux 
fa\c cons diff\'erentes, comme une $1$-forme \`a valeurs dans le fibr\'e $T^*M$. 
Pour \^etre en accord avec nos conventions d'\'ecriture, on devrait 
plut\^ot \'ecrire $g = (dx^\mu \otimes dx^\nu) \; g_{\mu\nu}$ lorsqu'on  veut 
consid\'erer $g$ comme $0$-forme. Dans ce cas, on posera 
$${D}g_{\mu\nu}= dg_{\mu\nu} - \Gamma^\sigma_\mu g_{\sigma \nu} - 
\Gamma^\sigma_\nu g_{\sigma \mu}$$ La diff\'erentielle ext\'erieure 
covariante $d^\nabla_{0,2}$ co\"incide alors avec la diff\'erentielle covariante 
$\nabla$ puisqu'elle est appliqu\'ee \`a une $0$-forme :
$$
d^\nabla_{0,2}\, g = \nabla g = dx^\mu \otimes dx^\nu {D}g_{\mu\nu}= 
g_{\mu\nu ; \rho} \, dx^\mu \otimes dx^\nu \otimes dx^\rho
$$
En tant que $1$-forme, on \'ecrira plut\^ot $g = dx^\mu g_\mu$, ce 
qui d\'efinit la $1$-forme $g_\mu = g_{\mu\nu} dx^\nu$. Dans ce cas, on 
posera $${D}g_\mu = dg_\mu - \Gamma^\sigma_\mu g_\sigma$$ et la 
diff\'erentielle ext\'erieure covariante $d^\nabla_{0,1}\; g$ sera donn\'ee par 
$$
d^\nabla_{0,1}\; g = dx^\mu {D}g_\mu
$$
Les deux indices de $g_{\mu\nu}$ jouant des r\^oles semblables, on peut 
\'egalement ``geler'' l'indice $\nu$ et d\'efinir un autre objet ${ 
d^{'\nabla}_{1,0}}\; g$, d'ailleurs \'egal \`a ${d_{0,1}}g$ puisque $g$ est sym\'etrique.
Nous reviendrons \`a ces diverses diff\'erentielles ext\'erieures
en donnant la d\'efinition du laplacien de Lichnerowicz, page \pageref{sec:Lichnerowicz}.
\label{sec:fancy-d}
}

\subsection{Forme canonique (ou forme de soudure)} \label{sec:forme-canonique}

Le lecteur est maintenant familiaris\'e avec ce qui fait l'originalit\'e des 
connexions lin\'eaires par rapport aux connexions principales en g\'en\'eral, \`a 
savoir la possibilit\'e d'identifier ``les indices de base'' avec ``les 
indices de fibre''. Il est donc largement temps d'examiner cette 
identification sous un angle un peu plus g\'eom\'etrique, ceci va nous 
conduire \`a d\'ecouvrir un nouvel objet : la torsion.

Soit $P = P(M,G)$ le fibr\'e principal des rep\`eres sur $M$, ou un 
sous-fibr\'e de ce dernier. Soit $e$ un \'el\'ement de $P$, c'est \`a dire, un 
rep\`ere de $M$. Consid\'erons un vecteur $u$ en $e$, c'est \`a dire un \'el\'ement 
de $T(P,e)$, c'est \`a dire encore, intuitivement, un ``petit d\'eplacement'' 
du rep\`ere $e$ dans l'espace des rep\`eres. Gr\^ace \`a la projection $\pi : P 
\mapsto M$ qui, \`a un rep\`ere, associe son origine, ou plut\^ot, gr\^ace \`a son 
application tangente $\pi_*$, nous pouvons prendre l'image $v = \pi_* u$ 
de $u$. Le vecteur $v$ est un vecteur tangent \`a $M$ situ\'e \`a l'origine du 
rep\`ere $e$ : $v \in T(M,\pi(e))$. Ce qu'il y a d'absolument unique dans 
le cas du fibr\'e des rep\`eres, c'est que nous pouvons maintenant d\'ecomposer 
$v$ sur l'\'el\'ement $e$ de $P$ d'o\`u nous sommes partis (puisque $e$ est un 
rep\`ere!): $v = e_\mu . v^\mu, {\hskip 0.2cm v^\mu \in \RR}.$
Nous avons donc construit une application $\theta$ qui, \`a tout vecteur $u$
tangent au fibr\'e principal $P$, associe un $n$-uplet de nombres (les 
composantes de $v$), c'est \`a dire un \'el\'ement de $\RR^n$. Cette 
application $\theta$ est donc une $1$-forme sur $P$ \`a valeurs dans 
$\RR^n$ et est d\'esign\'ee sous le nom de {\sl forme canonique\/}
 {\index{forme canonique}} (le mot 
``canonique'' faisant r\'ef\'erence au fait que sa d\'efinition ne d\'epend 
d'aucun 
choix de syst\`eme de coordonn\'ees) et quelquefois sous le nom de {\sl forme 
de soudure \/}
 {\index{forme 
de soudure }} puisqu'elle permet de ``souder'' la fibre type $\RR^n$ 
(consid\'er\'e comme espace de repr\'esentation pour le groupe lin\'eaire) avec
l'espace tangent \`a la base. Cette forme $\theta$ est \'evidemment 
\'equivariante puisque $v = e_\mu.v^\mu = e_\mu \Lambda.\Lambda^{-1}v^\mu$. 
Elle d\'efinit donc une $1$-forme sur $M$ \`a valeurs dans le fibr\'e tangent 
$TM = P \times_G \RR^n$. A ce propos,nous sugg\'erons au lecteur de relire la 
discussion g\'en\'erale, section \ref{sec:sections}, d\'ecrivant la correspondance 
bi-univoque existant entre sections de fibr\'es associ\'es ---ou $p$-formes \`a 
valeurs dans un fibr\'e associ\'e--- et les fonctions ---ou les $p$-formes--- 
\'equivariantes, d\'efinies sur le fibr\'e principal et \`a valeurs dans la 
fibre type. On identifie en g\'en\'eral :
$ \Omega^p_{eq} (P,\RR^n) \simeq \Omega^p(M,TM).$


La  $1$-forme obtenue sur $M$ se note encore $\theta$ et son 
expression, relativement au choix d'un
rep\`ere mobile $\{e_\mu\}$  et du corep\`ere mobile dual $\{\theta^\mu\}$,
est tout simplement $$\theta = 
e_\mu . \theta^\mu \in \Omega^1(M,TM)$$
o\`u, conform\'ement \`a la convention d\'ej\`a utilis\'ee pr\'ec\'edemment, nous avons 
omis de faire figurer le symbole du produit tensoriel entre les \'el\'ements 
pris comme base de la fibre (ici $e_\mu$) et ceux pris comme base de 
l'espace des formes (ici $\theta^\mu$). Nous avons aussi, conform\'ement 
\`a nos conventions, \'ecrit les formes \`a droite des vecteurs de la fibre. 
Notons enfin que $\theta$ est bien tel que $$\theta(v)=e_\mu . \theta^\mu 
(e_\nu v^\nu) = e_\mu . \theta^\mu(e_\nu) v^\nu = e_\mu . \delta^\mu_\nu 
v^\nu = e_\mu . v^\mu = v$$
$\theta$ n'est donc rien d'autre que l'application identique$\ldots$ mais 
consid\'er\'ee comme $1$-forme sur $M$ \`a valeurs dans le fibr\'e tangent, c'est 
\`a dire, comme un \'el\'ement de $\Omega^1(M,TM)$. Dans la litt\'erature physique, le rep\`ere mobile $\{e_{\mu}\}$ entrant dans 
l'expression de la forme de soudure est quelquefois 
appel� {\it vierbein\/} (``quatre pattes'').

\subsection{Torsion}

Le lecteur trouve peut-\^etre un peu longue (tordue ?) cette variation sur 
l'application identique$\ldots$ mais il se trouve que c'est elle qui fait 
la sp\'ecificit\'e du fibr\'e principal des rep\`eres. Lorsque ce dernier 
est muni d'une connexion, la diff\'erentielle covariante de $\theta$ n'est 
pas n\'ecessairement nulle et n'est autre que la torsion. 

Reprenons:

$$\theta = e_\mu \theta^\mu \in \Omega^1(M,TM)$$ est la $1$-forme canonique

D\'efinissons {\sl la $2$-forme de torsion\/} 
 {\index{forme de torsion} }
$$
\mbox{\fbox{$T= d^\nabla \theta \in 
\Omega^2(M,TM)
$}}
$$

$T$ est ainsi une $2$-forme \`a valeurs dans le fibr\'e tangent, on peut donc 
l'\'ecrire 
$$
\mbox{\fbox{$
T = e_\mu T^\mu
$}}
$$ o\`u $T^\mu$ est la $2$-forme, 
$$
\mbox{\fbox{$
T^\mu = 
D\theta^\mu = {1 \over 
2} {T^\mu}_{\nu\rho} \theta^\nu \wedge \theta^\rho
$}}
$$
 Bien entendu, on 
peut \'egalement consid\'erer la torsion comme un tenseur de rang $3$, de 
type $(1,2)$, antisym\'etrique sur les indices $\nu$ et $\rho$, et \'ecrire 
$$T =  {T^\mu}_{\nu\rho}\; \; e_\mu \otimes \theta^\nu \otimes \theta^\rho.$$
Calculons \`a pr\'esent la torsion \`a partir de sa d\'efinition:

\begin{eqnarray*}
  T  & = & d^\nabla \theta = d^\nabla(e_\mu . \theta^\mu) \\
 {}  & = & (\nabla e_\mu) \wedge \theta^\mu + e_\mu . d\theta^\mu \\
 {}  & = & e_\nu \Gamma^\nu_\mu \wedge \theta^\mu + e_\mu d\theta^\mu \\
 {}  & = & e_\mu (d\theta^\mu + \Gamma^\mu_\nu \wedge \theta^\nu)  = 
 e_\mu D\theta^\mu \equiv 
e_\mu . 
  T^\mu
\end{eqnarray*}
  
  Ainsi $$T^\mu = d \theta^\mu + \Gamma^\mu_\nu \wedge \theta^\nu $$
  et ses composantes sont ${T^\mu}_{\nu\rho}  =  T^\mu(e_\nu,e_\rho)$

La diff\'erentielle ext\'erieure ordinaire $d$ n'agissant que sur les indices 
de forme, on a $d\theta = d(e_\mu . \theta^\mu) = e_\mu . d\theta^\mu$ et 
on peut donc \'ecrire, encore plus simplement
$$
\mbox{\fbox{$
T = d\theta + \Gamma \wedge \theta
$}}
$$
  Si on utilise \'egalement la notation ${D}$ d\'ecrite avec force d\'etails en 
  section \ref{sec:operateur-D}, on voit que $$d^\nabla \theta = e_\mu . {D}\theta^\mu$$ 
  (puisque $\theta = e_\mu . \theta^\mu$), et donc 
$$
\mbox{\fbox{$
T^\mu = {D}\theta^\mu
$}}
$$
  On pourra \'egalement \'ecrire $T_{\nu\rho} =  T(e_\nu,e_\rho)$.
 
  
  \subsection{Equation de structure pour la torsion}
 \label{sec:torsion}

Nous avons \'etabli, en section \ref{sec:d-ext-sigma}, l'\'egalit\'e suivante, valable pour la 
diff\'erentielle ext\'erieure covariante d'une $1$-forme $\sigma$ quelconque, 
\`a valeurs dans 
un fibr\'e vectoriel $E$ : soit $\sigma \in \Omega^1(M,E)$, alors 
$$
d^\nabla \sigma (e_\mu,e_\nu) = \nabla_\mu \sigma(e_\nu) - \nabla_\nu 
\sigma(e_\mu) - \sigma([e_\mu,e_\nu])
$$
Dans le cas particulier o\`u $E = TM$ et o\`u $\sigma$ est \'egale \`a la forme 
canonique 
$\theta$, l'\'egalit\'e pr\'ec\'edente se simplifie consid\'erablement puisque 
$\theta$ n'est 
autre que l'identit\'e ($\theta(v) = v$). Il vient donc 
   $$
\mbox{\fbox{$
   T_{\mu\nu} = \nabla_\mu e_\nu - \nabla_\nu e_\mu - [e_\mu,e_\nu]
$}}
 $$
Cette derni\`ere \'egalit\'e, qui est quelquefois prise comme d\'efinition de la 
torsion, est l'\'equation de structure cherch\'ee. L'expression du 
tenseur de torsion en termes de 
coefficients de connexion et des fonctions de structure du rep\`ere mobile 
($[e_\mu,e_\nu] = {f_{\mu\nu}}^\rho e_\rho$) est donc la suivante
 $$
 \mbox{\fbox{$
 {{T^\rho}_{\mu\nu}} = {\Gamma^\rho}_{\nu\mu} - {\Gamma^\rho}_{\mu\nu} - 
{f_{\mu\nu}}^\rho
$}}
 $$
On retrouve, bien sur, la propri\'et\'e d'antisym\'etrie qu'on connaissait d\'ej\`a :
$$
 {{T^\rho}_{\mu\nu}} =  - {{T^\rho}_{\nu\mu}}
$$
Notons que, si on se place dans un rep\`ere naturel (${f_{\mu\nu}}^\rho = 
0$) l'expression du 
tenseur de torsion est simplement donn\'ee par la partie antisym\'etrique des 
coefficients de connexion. En cons\'equence, si, dans un rep\`ere 
naturel, la connexion est telle que ${\Gamma^\rho}_{\nu\mu} = 
{\Gamma^\rho}_{\mu\nu}$, la torsion est nulle.

\subsection{Identit\'es de Bianchi pour les connexions lin\'eaires}
\index{identit\'es de Bianchi}
Nous avons, en section \ref{sec:bianchi-F} \'etabli l'identit\'e de Bianchi relative \`a la 
courbure $F$ d'une connexion quelconque $A$. Rappelons qu'elle s'\'ecrit
$dF + A \wedge F = F \wedge A.$
Dans le cas des connexions 
lin\'eaires on obtient donc directement l'identit\'e 
$$
\mbox{\fbox{$
dR + \Gamma \wedge R  = 
R \wedge \Gamma .
$}}
$$
 Rappelons que cette identit\'e s'obtient en calculant la 
diff\'erentielle ext\'erieure de $R = d\Gamma + \Gamma \wedge \Gamma$, en 
substituant $d\Gamma$ par $R - \Gamma \wedge \Gamma$ dans le r\'esultat. 
Pour des raisons historiques cette identit\'e relative \`a la courbure
 est connue sous le nom de ``deuxi\`eme identit\'e de Bianchi''.
  Le qualificatif  ``deuxi\`eme'' vient du fait 
qu'il existe une ``premi\`ere identit\'e de Bianchi'' ; c'est  une identit\'e 
relative \`a la torsion, elle  n'a donc un sens que pour les connexions 
lin\'eaires. Elle s'obtient par une m\'ethode analogue \`a la pr\'ec\'edente, mais 
cette fois-ci en calculant la diff\'erentielle ext\'erieure de la 
torsion.

\begin{eqnarray*}
{}  T &=& d\theta + \Gamma\wedge\theta \\
\Longrightarrow dT &=& 0 + d\Gamma \wedge \theta - \Gamma \wedge d\theta \\
\Longrightarrow dT &=& (R - \Gamma \wedge \Gamma)\wedge \theta - \Gamma 
\wedge (T - \Gamma \wedge \theta) \\
\Longrightarrow dT &=& R \wedge \theta - \Gamma\wedge\Gamma\wedge\theta - 
\Gamma \wedge T + \Gamma\wedge\Gamma\wedge\theta
\end{eqnarray*}
D'o\`u
$$
\mbox{\fbox{$
dT + \Gamma \wedge T = R \wedge \theta
$}}
$$
\index{identit\'es de Bianchi}
Les deux identit\'es de Bianchi s'\'ecrivent, comme on vient de le voir, de 
fa\c con assez simple lorsqu'on utilise des notations suffisamment compactes.
On peut m\^eme ``compactifier'' davantage en \'ecrivant $d^\nabla T = dT + 
\nabla T = d T + \Gamma \wedge T$ et en utilisant le 
fait que $T = d^\nabla \theta$ ; l'identit\'e de Bianchi relative \`a la 
torsion s'\'ecrit 
donc 
$$
\mbox{\fbox{$
(d^\nabla)^2 \theta = R\wedge \theta
$}}
$$
 ce qui est d'ailleurs bien 
\'evident puisque le carr\'e de la diff\'erentielle ext\'erieure covariante n'est autre que 
l'op\'erateur de courbure. Par contre, si on veut absolument \'ecrire ces 
identit\'es avec tous les indices, les choses peuvent devenir assez 
compliqu\'ees... Pour appr\'ecier tout le sel de cette remarque, il n'est 
peut-\^etre pas inutile de nous vautrer, pour un court paragraphe, dans la 
``d\'ebauche des indices'', activit\'e qui fut tr\`es pris\'ee au d\'ebut du si\`ecle 
et qui reste encore presque indispensable lorsqu'on veut effectuer des calculs 
totalement explicites.

\subsubsection{Premi\`ere identit\'e (relative \`a la torsion)}

Le membre de droite de cette identit\'e s'\'ecrit explicitement

\begin{eqnarray*}
R^\mu_\tau \wedge \theta^\tau & = & {1 \over 2} {R^\mu}_{\tau\rho'\sigma'} 
\theta^{\rho'}\wedge\theta^{\sigma'}\wedge\theta^{\tau} \\
{} & \Rightarrow &
 \langle R^\mu_\tau \wedge \theta^\tau, e_\nu\otimes e_\rho \otimes 
 e_\sigma \rangle = {1 \over 2} 
 {R^\mu}_{\tau\rho'\sigma'}
 \langle \theta^{\rho'}\wedge\theta^{\sigma'}\wedge\theta^{\tau},
 e_\nu\otimes e_\rho \otimes 
 e_\sigma \rangle
\end{eqnarray*}
 Donc
 $$
 {1 \over 2!}{1 \over 3!} {R^\mu}_{\tau\rho'\sigma'} 
 \delta_{\nu\rho\sigma}^{\rho'\sigma'\tau'} = {1 \over 
 2}({R^\mu}_{\nu\rho\sigma}+{R^\mu}_{\rho\sigma\nu}+{R^\mu}_{\sigma\nu\rho})
$$
Le membre de gauche, quant \`a lui, $d T^\mu + \Gamma^\mu_\tau \wedge T^\tau$ 
peut \'egalement s'\'evaluer sur $e_\nu\otimes e_\rho \otimes  e_\sigma$.
Il est donc possible d'\'ecrire la premi\`ere identit\'e
de Bianchi de fa\c con 
telle que seuls les tensions de courbure et de torsion apparaissent 
explicitement :\index{identit\'es de Bianchi}
$$
\mbox{\fbox{$
{R^\mu}_{\nu\rho\sigma}+{R^\mu}_{\rho\sigma\nu}+{R^\mu}_{\sigma\nu\rho} =
{T^\mu}_{\nu\rho;\sigma}+{T^\mu}_{\rho\sigma;\nu}+{T^\mu}_{\sigma\nu;\rho} 
+
{T^\tau}_{\nu\rho}{T^\mu}_{\tau\sigma}+{T^\tau}_{\rho\sigma}{T^\mu}_{\tau\nu}+
{T^\tau}_{\sigma\nu}{T^\mu}_{\tau\rho}
$}}
$$
Si on introduit l'op\'erateur de cyclicit\'e $\Sigma_\lambda$ d\'efini pour tout 
tenseur $B$ de 
rang trois par 
$$
\Sigma_\lambda B(x,y,z)  =  B(x,y,z) + B(y,z,x)+ B(z,x,y) 
$$ cette identit\'e s'\'ecrit encore :
$$
\Sigma_\lambda\{R(x,y)z\} =\Sigma_\lambda\{(\nabla_X T)(y,z)\} + 
\Sigma_\lambda\{T(T(x,y),z)\} 
$$

\subsubsection{Deuxi\`eme identit\'e (relative \`a la courbure)}

On peut soumettre la deuxi\`eme identit\'e de Bianchi (celle relative \`a la 
courbure)  \`a un traitement similaire : le membre de gauche $dR + \Gamma 
\wedge R$ se 
transcrit imm\'ediatement en une somme cyclique de d\'eriv\'ees covariantes du 
type ${R^\mu}_{\nu\rho\sigma;\tau}$ et le membre de droite $R\wedge 
\Gamma$ peut se retranscrire en une somme de 
termes du type ${R^\mu}_{\nu\rho\sigma}{T^\rho}_{\tau \kappa}$ en 
utilisant la relation entre torsion et coefficients 
de connexion \'etablie en \ref{sec:torsion}. Il vient
$$
\mbox{\fbox{$
{R^\mu}_{\nu\rho\sigma;\tau} + 
{R^\mu}_{\nu\sigma\tau;\rho}+{R^\mu}_{\nu\tau\rho;\sigma} +
{R^\mu}_{\nu\kappa\rho} T^\kappa_{\sigma\tau} + {R^\mu}_{\nu\kappa\sigma} 
T^\kappa_{\tau\rho} + {R^\mu}_{\nu\kappa\tau} T^\kappa_{\rho\sigma}   = 0
$}}
$$
Cette identit\'e s'\'ecrit encore
$$
\Sigma_\lambda\{(\nabla_z R)(x,y,w)\} + 
\Sigma_\lambda\{R(T(x,y),z)w\} = 0
$$


\subsection{D\'eriv\'ees covariantes secondes, hessien et identit\'es de Ricci}
 \label{sec:ricci-id}

\begin{description}
\item[Commentaires concernant $D^2$.]
Tout d'abord, on sait que le carr\'e de l'op\'erateur de diff\'erentiation 
ext\'erieure covariante $d^\nabla$ n'est autre que la courbure. Retrouvons 
tout d'abord cette 
propri\'et\'e, \`a titre d'exercice, dans quelques cas particuliers, en 
utilisant l'op\'erateur $D$.

Soit $v \in \Omega^p(M,TM)$, on a $$v = e_\alpha . 
v^\alpha$$ o\`u $v^\alpha$ est une $p$-forme sur $M$.
\begin{eqnarray*}
d^\nabla v & = & e_\alpha . D v^\alpha = e_\alpha . (dv^\alpha + 
{\Gamma^\alpha}_\beta \wedge v^\beta) \in \Omega^{p+1}(M,TM) \\
(d^\nabla)^2 v & = & e_\alpha . D^2 v^\alpha = e_\alpha . (Ddv^\alpha + 
D({\Gamma^\alpha}_\beta\wedge v^\beta)) \\
& = &  e_\alpha . (d^2 v^\alpha + {\Gamma^\alpha}_\gamma\wedge dv^\gamma  + d({\Gamma^\alpha}_\beta\wedge v^\beta)
+ {\Gamma^\alpha}_\gamma \wedge {\Gamma^\gamma}_\beta \wedge v^\beta) \\
& = & e_\alpha . (0 + {\Gamma^\alpha}_\gamma\wedge dv^\gamma + 
(d{\Gamma^\alpha}_\beta\wedge v^\beta) - ({\Gamma^\alpha}_\beta\wedge 
dv^\beta) + {\Gamma^\alpha}_\gamma \wedge {\Gamma^\gamma}_\beta \wedge v^\beta) \\
& = & e_\alpha . ((d{\Gamma^\alpha}_\beta + {\Gamma^\alpha}_\gamma \wedge 
{\Gamma^\gamma}_\beta) \wedge v^\beta) \\
& = & e_\alpha . ({R^\alpha}_\beta \wedge v^\beta) \in \Omega^{p+2}(M,TM)
\end{eqnarray*}

Ainsi donc, $D^2 v^\alpha ={R^\alpha}_\beta\wedge v^\beta$, comme il se doit.

Soit maintenant $v \in \Omega^p(M,TM\otimes T^*M)$, donc 
$v=e_\alpha\otimes \theta^\beta \, v^\alpha_\beta$ o\`u $v^\alpha_\beta$ est une $p$-forme. 
Un calcul parfaitement 
analogue conduit \`a
$$(d^\nabla)^2 v =e_\alpha\otimes\theta^\beta \, D^2 V^\alpha_\beta $$
avec
$$
D^2 {V^\alpha}_\beta = {R^\alpha}_\rho \wedge {V^\rho}_\beta - 
{R^\rho}_\beta\wedge {V^\alpha}_\rho$$
Le fait d'obtenir une somme de deux termes faisant intervenir la $2$-forme 
de courbure ne doit pas surprendre : cela est fondamentalement li\'e \`a la 
fa\c con dont se repr\'esente l'alg\`ebre de Lie du groupe lin\'eaire dans la 
d\'efinition du fibr\'e vectoriel $TM\otimes T^*M$. 

La g\'en\'eralisation est \'evidente : si, par exemple, $v \in 
\Omega^p(M,TM\otimes T^*M\otimes T^*M)$ 
c'est \`a dire $v=e_\alpha\otimes\theta^\beta\otimes\theta^\gamma . 
{v^\alpha}_{\beta\gamma}$, avec ${v^\alpha}_{\beta\gamma} \in \Omega^p(M)$, alors 
$(d^\nabla)^2 v=e_\alpha \otimes\theta^\beta\otimes\theta^\gamma \, D^2 {v^\alpha}_{\beta\gamma}$ et 
$D^2 {v^\alpha}_{\beta\gamma} = {R^\alpha}_\rho \wedge 
{V^\rho}_{\beta\gamma} - 
{R^\rho}_\beta\wedge {V^\alpha}_{\rho\gamma} - {R^\rho}_\beta\wedge 
{V^\alpha}_{\beta \rho}$

\item[Op\'erateurs $\nabla\nabla\nabla\ldots$]

Soit $S$ est un tenseur quelconque de rang $(r,s)$, consid\'er\'e comme section de $E 
=TM^{\otimes r}\otimes T^*M^{\otimes s}$, c'est 
\`a dire $S\in \Omega^0(M,E)$, nous voulons donner un sens \`a $\nabla\nabla 
S$. Nous savons que $\nabla S$ est une 
$1$-forme \`a valeurs dans E, (ainsi $\nabla S=d^\nabla S \in 
\Omega^1(M,E)$, puisque que $\nabla$ et $d^\nabla$ co\"incident sur 
les $0$-formes), et que $(d^\nabla)^2 S\in \Omega^2(M,E)$, mais $\nabla$,
 par d\'efinition, n'agit que sur les 
$0$-formes (\`a valeurs dans n'importe quel fibr\'e). Pour pouvoir appliquer 
$\nabla$ sur $\nabla S$ il suffira donc de consid\'erer les $1$-formes \`a 
valeurs dans $E$ comme 
des $0$-formes \`a valeurs dans $E\otimes T^*M$.  Encore une fois, nous 
identifions $\Omega^1(M,E)$ avec $\Omega^0(M,E\otimes T^*M)$ (l'application 
identique sera not\'ee ${Id}$). 
Explicitement, si $v = e_I V^I \in \Omega^1(M,E)$, avec 
$V^I = V^I_\mu \theta^\mu \in \Omega^1(M)$ et si ``$I$'' d\'esigne un multi-indice relatif \`a une 
base de $E$, on \'ecrira simplement 
$v = e_I \theta^\mu \, V^I_\mu  \in \Omega^0(M,E\otimes T^*M)$,
 avec $V^I_\mu \in \Omega^0(M)$. Pour ne pas alourdir les 
notations, on note encore $v$ l'image de $v$ par l'application identique ! 
Cette application identique d\'eguis\'ee ${\mbox \cal Id}$ se g\'en\'eralise d'ailleurs de fa\c con 
\'evidente pour fournir un homomorphisme injectant l'espace vectoriel 
$\Omega^p(M,E)$ 
dans $\Omega^0(M,E\otimes(T^*M)^{\otimes p}))$. Il faut donc 
comprendre $\nabla \nabla S$ comme $$(\nabla \circ {\mbox \cal Id} \circ 
\nabla) \; S$$ mais, bien entendu, nous 
n'\'ecrivons jamais ${\mbox \cal Id}$ explicitement. 
Noter que rien n'interdit au lecteur 
un peu pervers de consid\'erer des objets comme $$\nabla \nabla \nabla  
d^\nabla \nabla  d^\nabla d^\nabla \nabla S$$ $\ldots$ qu'il faut comprendre 
comme la compos\'ee
\begin{eqnarray*}
\Omega^0(M,E) 
\stackrel{\nabla}{\mapsto}  \Omega^1(M,E) 
\stackrel{d^\nabla}{\mapsto}\Omega^2(M,E)
\stackrel{d^\nabla}{\mapsto}\Omega^3(M,E)\\
\stackrel{Id}{\mapsto}\Omega^0(M,E\otimes(T^*M)^{\otimes 3}) 
\stackrel{\nabla}{\mapsto} \Omega^1(M,E\otimes(T^*M)^{\otimes 3})
\stackrel{d^\nabla}{\mapsto}\Omega^2(M,E\otimes(T^*M)^{\otimes 3})\\
\stackrel{Id}{\mapsto}\Omega^0(M,E\otimes(T^*M)^{\otimes 5})
\stackrel{\nabla}{\mapsto}\Omega^1(M,E\otimes(T^*M)^{\otimes 5})
\stackrel{Id}{\mapsto} \Omega^0(M,E\otimes(T^*M)^{\otimes 6}) \\
\stackrel{\nabla}{\mapsto}  \Omega^1(M,E\otimes(T^*M)^{\otimes 6})
\stackrel{Id}{\mapsto}\Omega^0(M,E\otimes(T^*M)^{\otimes 7})
\stackrel{\nabla}{\mapsto}\Omega^1(M,E\otimes(T^*M)^{\otimes 7})\\
\simeq \Omega^0(M,E\otimes(T^*M)^{\otimes 8})
\end{eqnarray*}
et de comparer cet op\'erateur avec, par exemple 
$\nabla\nabla\nabla d^\nabla d^\nabla\nabla d^\nabla \nabla S$ !

\item[Hessien.]
{\small
 Nous nous 
contenterons d'examiner d'un peu plus pr\`es l'op\'erateur $Hess  =  \nabla\nabla.$
Prenons $S \in \Omega^p(M,E)$, $S = e_I S^I$ avec $S^I \in 
\Omega^p(M)$. Alors $\nabla S = e_I D S^I \in 
\Omega^{p+1}(M,E)$ qu'on peut consid\'erer (application $Id$) comme 
$\nabla S = e_I \theta^\beta \, S^I_{;\beta} \in 
\Omega^p(M,E\otimes T^*M)$
puisque $DS^I = \theta^\beta \, S^I_{;\beta}.$
Alors $\nabla\nabla S = e_I\otimes 
\theta^\beta \otimes(S^I_{;\beta})_{;\gamma} \theta^\gamma \in 
\Omega^1(M,E\otimes(T^*M))$ qu'on peut consid\'erer comme 
$\nabla\nabla S = e_I\otimes \theta^\beta \otimes \theta^\gamma \, 
(S^I_{;\beta\gamma}) \in \Omega^0(M,E\otimes(T^*M)^{\otimes 2}).$
On a not\'e $S^I_{;\beta\gamma} = (S^I_{;\beta})_{;\gamma}$ les composantes 
de $\nabla \nabla S$. On note \'egalement $\nabla_\beta S  =  \langle 
\nabla S , e_\beta \rangle = e_I . S^I_{;\beta} \in \Omega^0(M,E)$ la d\'eriv\'ee 
covariante de $S$ dans la direction $e_\beta$. La d\'eriv\'ee covariante de 
$\nabla S$ ---consid\'er\'ee comme \'el\'ement de $\Omega^0(M,E\otimes T^*M)$--- 
dans la direction $e_\gamma$ sera donc $\nabla_\gamma\nabla S = \langle 
\nabla (\nabla S), e_\gamma \rangle = e_I \otimes \theta^\beta 
(S^I_{;\beta})_{;\gamma} \in \Omega^0(M,E\otimes T^*M)$, 
par cons\'equent $e_I . S^I_{;\alpha\gamma} = \langle\langle \nabla \nabla 
S , e_\gamma \rangle , e_\alpha \rangle  = \langle \nabla \nabla S, 
e_\alpha \otimes e_\gamma \rangle .$
Attention : $\nabla_\beta S \in \Omega^0(M,E)$, $\beta$ \'etant fix\'e, est 
une brave section de $E$ et on peut donc 
aussi calculer $\nabla\nabla_\beta S$ qui est un \'el\'ement de 
$\Omega^1(M,E)$, mais alors, l'indice $\beta$ \'etant 
gel\'e, il n'y a pas \`a introduire de $\Gamma$ relatif \`a l'indice $\beta$ dans le calcul 
de $\nabla\nabla_\beta S$. La conclusion est alors que 
$\nabla_\alpha(\nabla_\beta S) = \langle \nabla \langle \nabla S, 
e_\beta\rangle, e_\alpha \rangle$ n'est absolument pas \'egal \`a 
$S_{;\alpha\beta} = \langle \langle \nabla \nabla S, e_\beta \rangle, 
e_\alpha \rangle$; il faut donc faire tr\`es attention ! En pratique, les choses sont assez simples 
car ce sont les composantes de $\nabla \nabla S$ (ou d'autres expressions 
d'ordre sup\'erieur du m\^eme type) qui 
sont int\'eressantes et non les composantes de $\nabla(\nabla_\beta S)$. La raison pour laquelle 
nous consacrons ces quelques lignes \`a attirer l'attention du lecteur sur 
ce sujet assez trivial, c'est que la confusion possible dont on vient de 
parler est \`a l'origine de bien des erreurs$\ldots$
A ce sujet, il est assez inexact 
de pr\'etendre (comme on l'entend parfois) que ``de toutes fa\c cons, un objet 
tel que $\nabla_\beta S$ n'est pas un objet covariant'', c'est faux. 
La situation que nous avons ici est parfaitement analogue \`a celle qu'on 
rencontre en relativit\'e restreinte : bien
 que ``l'\'energie d'une particule'' soit une quantit\'e dont la valeur 
d\'epende du rep\`ere (de l'observateur) choisi et donc une caract\'eristique 
non intrins\`eque de la particule,  ``l'\'energie d'une 
particule mesur\'ee par un observateur d\'etermin\'e'' est n\'eanmoins une 
quantit\'e digne d'int\'er\^et qu'on peut d'ailleurs calculer dans n'importe 
quel rep\`ere.}

\item[Hessien d'une fonction scalaire.]
 Abandonnons l\`a ces remarques semi-p\'edagogiques et illustrons 
les consid\'erations pr\'ec\'edentes avec un exemple tr\`es simple, le calcul de 
$\nabla \nabla h$ o\`u $h$ est une fonction sur la vari\'et\'e $M$.

Prenons $h \in \Omega^0(M,\RR)$ et donc $\nabla h \in \Omega^1(M,\RR) \stackrel 
{Id} {\simeq} \Omega^0(M,T^*M)$ avec $\nabla h = \theta^\alpha 
h_{;\alpha}$,
et $h_{;\alpha} = h_{,\alpha} = e_\alpha[h]$
Partant de $\nabla h \in \Omega^0(M,T^*M)$ on obtient $ \nabla \nabla h \in 
\Omega^1(M,T^*M) \stackrel {Id}{\simeq} \Omega^0(M,T^*M \otimes T^*M)$, 
ainsi
$$
\mbox{\fbox{$
Hess(h)  =  \nabla \nabla h =
\theta^\alpha \otimes \theta^\beta \; h_{;\alpha 
\beta}
$}}
$$ 
En vertu des r\`egles de calcul d\'ej\`a \'etablies, on obtient directement
$h_{;\alpha \beta} = e_\beta[h_{;\alpha}] - h_{; 
\rho}\Gamma^\rho_{\alpha\beta} = e_\beta[h_{,\alpha}] - h_{, 
\rho}\Gamma^\rho_{\alpha\beta} =
e_\beta[e_\alpha[h]] - 
e_\rho[h]\Gamma^\rho_{\alpha\beta}$. Ainsi
$$
\mbox{\fbox{$
h_{;\alpha \beta} = e_\beta[e_\alpha[h]] - e_\rho[h]\; \Gamma^\rho_{\alpha\beta}
$}}
$$
Notons que 
$$\nabla\nabla h = \nabla(\theta^\alpha \, h_{,\alpha})=\theta^\alpha Dh_{,\alpha} = 
\theta^\alpha \otimes \theta^\beta h_{,\alpha;\beta} = 
\theta^\alpha \otimes \theta^\beta h_{;\alpha\beta}$$
Dans un syst\`eme de coordonn\'ees locales $\{x^\mu\}$, on \'ecrira
$$Hess(h) = h_{;\mu\nu} \; dx^\mu\otimes dx^\nu $$
avec
$$
\mbox{\fbox{$
 h_{;\mu\nu} = {\partial^2 h\over \partial x^\mu \partial x^\nu} - 
\Gamma^\rho_{\mu\nu} {\partial h \over \partial x^\rho}
$}}
$$

\item[Hessien d'une $0$-forme \`a valeurs vectorielles.]
PLus g\'en\'eralement, soit $\xi \in \Omega^0(M,E)$, $\{e_i\}$ un rep\`ere dans les fibres
de $E$ et $\{\theta^\alpha \}$ un corep\`ere mobile sur $M$. Il vient imm\'ediatement :
\begin{eqnarray*}
Hess(\xi) &=& \nabla \nabla \xi = \nabla \nabla (e_i \; \xi^i) \\
{} &=& \nabla (e_i \; D\xi^i) = \nabla (e_i\; \theta^\alpha \xi^i_{;\alpha})\\
{} &=& e_i\,  \theta^\alpha \; D \xi^i_{;\alpha} = 
 e_i\, \theta^\alpha \otimes \theta^\beta \;  \xi^i_{;\alpha ;\beta} \\
{} &=&  e_i\, \theta^\alpha \otimes \theta^\beta \;  \xi^i_{;\alpha \beta} 
\end{eqnarray*}
avec
$$\xi^i_{;\alpha} = \xi^i_{,\alpha} + A^i_{j\alpha} \xi^j$$
et
$$\xi^i_{;\alpha\beta} = 
e_\beta[\xi^i_{;\alpha}] - \Gamma^\gamma_{\alpha\beta}\xi^i_{;\gamma} + A^i_{j\beta}\xi^j_{;\alpha}
$$
Noter que nous devons ``corriger'', dans le calcul des d\'eriv\'ees covariantes, aussi bien les
indices de type $TM$ (gr\^ace \`a la connexion $\Gamma$) 
que les indices de type $E$ (gr\^ace \`a la connexion $A$). Noter aussi que notre symbole $\nabla$ est
``global'' en ce sens que nous n'introduisons pas de symboles particuliers pour les diff\'erentes
connexions.

Nous reparlerons du  hessien dans la section consacr\'ee aux laplaciens (en page~\pageref{sec:laplaciens}).

\item[Non commutation des d\'eriv\'ees covariantes secondes.]

\begin{itemize}
\item
Si $\{e_\mu\}$ est un rep\`ere naturel $\{e_\mu = {\partial \over \partial 
x^\mu}\}$, il est bien \'evident que $\partial_\mu \partial_\nu h = 
\partial_\nu \partial_\mu h $, ce qu'on peut \'ecrire $h_{\mu\nu} = h_{\nu 
\mu}.$
\item 
Si $\{e_\alpha\}$ est un rep\`ere mobile ($[e_\alpha,e_\beta] = 
f_{\alpha\beta}^\gamma e_\gamma$), il faut d\'ej\`a remarquer le fait que 
$h_{,\alpha \beta} \neq h_{,\beta \alpha}$ puisque $h_{,\alpha\beta} = 
e_\beta[e_\alpha[h]]$ et $h_{,\beta\alpha} =  e_\alpha[e_\beta[h]]$, 
ainsi, 
$$
\mbox{\fbox{$
h_{,\alpha\beta} - h_{,\beta \alpha} = - f_{\alpha\beta}^\gamma 
e_\gamma [h]
$}}
$$
\item
Nous avons d\'ej\`a calcul\'e $\nabla \nabla h$.
\item
Calculons maintenant la diff\'erence
\begin{eqnarray*}
 h_{;\alpha\beta} - h_{;\beta\alpha} & = & [e_\beta e_\alpha - e_\alpha 
 e_\beta][h] - e_\rho[h](\Gamma^\rho_{\alpha\beta} - 
 \Gamma^\rho_{\beta\alpha}) \\
 & = & [\Gamma^\rho_{\beta\alpha} - \Gamma^\rho_{\alpha\beta} - 
 f^\rho_{\beta\alpha}]e_\rho[h] 
 \end{eqnarray*}
et donc, en utilisant l'expression du tenseur de torsion,
$$
\mbox{\fbox{$
 h_{;\alpha\beta} - h_{;\beta\alpha} =  T^\gamma_{\alpha\beta} \; e_\gamma[h]
$}}
$$
  Cette identit\'e est d\'esign\'ee sous le nom d'{\sl Identit\'e de Ricci}{\index{identit\'e de Ricci}}. Plus 
  g\'en\'eralement, on obtient des identit\'es de ce type lorsqu'on calcule des 
  diff\'erences telles que $S^I_{;\alpha\beta} - S^I_{;\beta\alpha}$, $S$ 
  d\'esignant un tenseur quelconque. Il faut donc remarquer le fait que 
  {\em les d\'eriv\'ees covariantes secondes ne commutent pas en g\'en\'eral}. 
    
  Finalement, notons que $h_{;\alpha}\sim h_{,\alpha}$, l'indice $\alpha$ \'etant fix\'e, 
  est une brave fonction sur la vari\'et\'e (pas un vecteur tangent!).
  En cons\'equence, $\nabla_\beta h_{,\alpha} = e_\beta[h_{,\alpha}] = 
  e_\beta[e_\alpha[h]]$ et cette quantit\'e, qu'on peut m\^eme noter 
  $\nabla_\beta\nabla_\alpha h$ {\bf n'est pas} \'egale \`a $h_{;\alpha\beta}$.
  A ce propos, relire la section \ref{sec:notations} consacr\'ee aux notations.

\item
{\small
  Afin de familiariser le lecteur avec la manipulation des indices 
  (activit\'e parfois fort utile), \'etablissons l'identit\'e de Ricci relative 
  aux tenseurs de type $(1,0)$. Prenons $V=V^\mu e_\mu$. Tout d'abord, voici le r\'esultat:
$$
{V^\mu}_{;\alpha\beta} - {V^\mu}_{;\beta\alpha} = T^\gamma_{\alpha\beta} 
{V^\mu}_{;\gamma} - {R^\mu}_{\gamma\alpha\beta}{V^\gamma}
$$
La d\'emonstration est imm\'ediate, il suffit de calculer ${V^\mu}_{\alpha\beta}$. Indications :
dans l'expression de ${V^\mu}_{;\alpha\beta} - {V^\mu}_{;\beta\alpha} $, 
on reconna\^it un terme ${R^\mu}_{\sigma\alpha\beta}{V^\sigma}$
lorsqu'on utilise l'\'ecriture explicite du tenseur de courbure ${R^\mu}_{\sigma\alpha\beta}$
donn\'ee en fin de section \ref{sec:courbure-con-lineaire} ainsi qu'un terme 
$T^\sigma_{\alpha\beta} (V^\mu_{,\sigma} + \Gamma^\mu_{\rho\sigma}V^\rho)=
T^\sigma_{\alpha\beta} {V^\mu}_{;\sigma}$
lorsqu'on utilise l'expression explicite du tenseur de torsion donn\'ee en 
fin de section \ref{sec:torsion}. Les autres termes se compensent (noter en 
particulier que ${V^\mu}_{;\alpha\beta} - 
{V^\mu}_{;\beta\alpha}=[e_\beta,e_\alpha](V^\mu)={f_{\alpha\beta}}^\rho V^\mu_{,\rho}$
et que ce terme se compense avec le terme du m\^eme type apparaissant 
lorsqu'on \'ecrit 
$-\Gamma^\sigma_{\alpha\beta}V^\mu_{,\sigma} + 
\Gamma^\sigma_{\beta\alpha} V^\mu_{,\sigma} = 
T^\sigma_{\alpha\beta}V^\mu_{,\sigma} + 
{f_{\alpha \beta}}^\sigma V^\mu_{,\sigma}.$


Le lecteur pourra g\'en\'eraliser sans peine les identit\'es de Ricci relatives 
\`a des tenseurs $S$ d'ordre quelconque : en plus d'un unique terme de type 
``$T \, S \, ; $'', on voit appara\^itre, pour chaque indice du tenseur 
consid\'er\'e, une contribution ---sign\'ee--- de type `$R\, S$'', au membre de 
droite de l'identit\'e de Ricci. Par exemple, prenons
$
S = e_\mu \otimes e_\nu \otimes \theta^\rho \, S^{\mu \nu}_\rho
$,
il vient
$$
S^{\mu \nu}_{\rho ; \alpha \beta} - S^{\mu \nu}_{\rho ; \beta \alpha} =
T^\gamma_{\alpha\beta} S^{\mu\nu}_{\rho ; \gamma}
- 
R^\mu_{\gamma\alpha\beta} S^{\gamma\nu}_{\rho}
-
R^\nu_{\gamma\alpha\beta} S^{\mu\gamma}_{\rho}
+
R^\gamma_{\rho\alpha\beta} S^{\mu\nu}_{\gamma}
$$
}
\end{itemize}
\end{description}

\subsection{Tenseur de Ricci} 
Le tenseur de courbure $F$ pour une connexion principale quelconque 
poss\`ede des composantes ${F^i_j}_{\mu\nu}$ et il est impossible de 
contracter l'indice $i$ avec l'indice $\mu$ puisque ces indices sont de 
nature diff\'erente : l'un est un indice de fibre et l'autre un indice de 
base. Par contre, pour une connexion lin\'eaire, on peut choisir le m\^eme 
rep\`ere dans les fibres et sur la base, il devient donc possible de 
contracter un indice de fibre avec un indice de base : \`a partir du tenseur de courbure 
$R \equiv F$ de composantes ${R^\rho_\sigma}_{\mu\nu}$, on peut fabriquer 
un tenseur covariant de rang $2$, le {\sl tenseur de Ricci} {\index{tenseur de Ricci}}, que nous noterons 
$\rho$ et qui est donc d\'efini par l'\'egalit\'e
$$
\rho_{\sigma \nu}  = {R^\mu_\sigma}_{\mu\nu}
$$
Notons que nous n'avons pas eu besoin de m\'etrique pour d\'efinir ce 
tenseur alors que l'utilisation d'une m\'etrique est n\'ecessaire, comme 
nous le verrons, pour d\'efinir la courbure scalaire.
Nous reviendrons au tenseur de Ricci dans le cadre de l'\'etude des connexions 
m\'etriques.
\subsection{Courbes autoparall\`eles} 
Un champ de vecteurs $v$ est dit {\sl parall\`ele\/}
 {\index{parall\`ele}} ou {\sl transport\'e par 
parall\'elisme\/}
 {\index{transport par 
parall\`elisme}} le long d'un arc de courbe 
${\cal C}: \tau \in \RR \rightarrow {\cal C}(\tau)$ lorsque sa d\'eriv\'ee 
covariante est nulle dans la direction du vecteur tangent $u  =  {d 
\over d \tau}$ \`a ${\cal C}$. Ce vecteur tangent poss\`ede des composantes 
$u^\alpha = {d {\cal C}^\alpha \over d\tau}$ dans un rep\`ere donn\'e. La loi 
du transport parall\`ele s'\'ecrit donc
\begin{eqnarray*}
 \nabla_u v  =  0  & \Leftrightarrow & u[v^\mu] + {\Gamma^\mu}_{\sigma 
\alpha}v^\sigma u^\alpha = 0 \\
{} & \Leftrightarrow &  {d v^\mu \over d \tau} +  {\Gamma^\mu}_{\sigma 
\alpha}v^\sigma {d {\cal C}^\alpha \over d \tau} = 0
\end{eqnarray*}

La courbe ${\cal C}(\tau)$ est dite {\sl courbe autoparall\`ele}
 {\index{courbe autoparall\`ele}} si son 
vecteur tangent $u$ est lui-m\^eme transport\'e par parall\'elisme le long de 
${\cal C}$.  Ainsi 
\begin{eqnarray*}
 {\cal C}(\tau)\; {\mbox {\rm est autoparall\`ele}} & \Leftrightarrow &
 \nabla_u u  =  0 \, {\mbox{, \rm avec \ }}\,  u = {d \over d\tau} \\
 {} & \Leftrightarrow & 
 {d^2 {\cal C}^\alpha \over d \tau^2} +  {\Gamma^\alpha}_{\beta 
\gamma}{d {\cal C}^\beta \over d \tau}{d {\cal C}^\gamma \over d \tau} = 0
\end{eqnarray*}
Nous verrons, dans la section consacr\'ee aux connexions riemanniennes, 
comment ces courbes autoparall\`eles sont reli\'ees aux g\'eod\'esiques. De fait, 
les autoparall\`eles d'une connexion donn\'ee sont quelquefois d\'esign\'ees sous 
le nom de ``g\'eod\'esiques de la connexion'', mais nous pr\'ef\'erons r\'eserver cette 
terminologie au cas de connexions tr\`es particuli\`eres.

%%%%%%%%%%%%%%%%%%%%%%%%%%%%
%%%%%%%%%%%\subsection{Champs de Jacobi} $TODO ***$
%%%%%%%%%%%%%%%%%%%%%%%%%%%%

\subsection{Connexions lin\'eaires sur les groupes et espaces homog\`enes} 
Soit $G$ un groupe, que nous supposons ici compact, et $G/H$ un espace homog\`ene. Au niveau des alg\`ebres de 
Lie, gr\^ace au choix d'un produit scalaire dans $Lie \, G$
on peut  \'ecrire $Lie \, G = Lie \, H \oplus {\frak s}$ o\`u ${\frak s}$ 
peut s'identifier avec l'espace tangent \`a $G/H$ en l'origine.
 La forme de Maurer-Cartan sur $G$ est \`a valeurs dans $Lie \, G$ et on peut consid\'erer sa 
projection sur $Lie \, H$. On montre qu'on obtient ainsi une forme de 
connexion (dite canonique) sur le fibr\'e principal $G=G(G/H,H)$. Par ailleurs
nous supposons (cas 
usuel) que $[Lie \, H, {\frak s}] \subset {\frak s}$ et m\^eme que  $\forall 
h \in H, \, h{\frak s}h^{-1} \subset {\frak s}$. En d'autres termes, 
l'espace vectoriel ${\frak s}$ est le support d'une repr\'esentation 
lin\'eaire du groupe $H$ (c'est la repr\'esentation $ad_H$). En cons\'equence, 
on obtient un homomorphisme de $H$ dans $End \, {\frak s}$. Cet 
homomorphisme permet d'\'etendre le fibr\'e principal $G=G(G/H,H)$ au fibr\'e 
des rep\`eres lin\'eaire au dessus de $G/H$ (c'est un fibr\'e de base $G/H$ et 
de fibre type $GL(s)$ avec $s = dim({\frak s})$). La connexion canonique 
donne ainsi naissance \`a une connexion lin\'eaire sur l'espace homog\`ene $G/H$. 

Ce type de construction 
et les 
g\'eom\'etries qui lui sont associ\'ees constituent un vaste chapitre de la 
g\'eom\'etrie diff\'erentielle et nous renvoyons le lecteur \`a un ouvrage tel 
que \cite{Kobayashi-Nomizu} pour plus de d\'etails. Notons pour finir qu'on 
obtient ainsi \'egalement une connexion lin\'eaire sur $G$ lui-m\^eme lorsqu'on 
consid\`ere la vari\'et\'e sous-jacente comme quotient de $G \times G$ par son sous-groupe diagonal 
(isomorphe \`a $G$).


\section{Connexions m\'etriques}

REMARQUE : 
La th\'eorie des connexions m\'etriques en g\'en\'eral et 
la g\'eom\'etrie riemannienne en particulier m\'eriterait \`a elle seule 
plusieurs volumes$\ldots$ Malgr\'e l'importance du sujet, la pr\'esente 
section est relativement br\`eve et doit \^etre consid\'er\'ee par le lecteur 
comme un simple survol permettant d'illustrer certaines 
des constructions g\'en\'erales pr\'ec\'edentes.
Nous voulons consid\'erer la notion de connexion m\'etrique comme un cas 
particulier de la notion de connexion r\'eductible, et l'ensemble des rep\`eres 
orthonorm\'es associ\'es \`a une m\'etrique comme un cas particulier de r\'eduction d'espace 
fibr\'e principal. Nous engageons le lecteur \`a discuter de fa\c con analogue 
d'autres types de g\'eom\'etries (par exemple les structures presque complexes).


\subsection{La m\'etrique} 
Soit $FM=FM(M,GL(n,\RR))$ le fibr\'e principal des rep\`eres sur la vari\'et\'e 
$M$. On consid\`ere une r\'eduction de $FM$ \`a un fibr\'e principal de rep\`eres 
orthonorm\'es
$OM=OM(M,SO(n))$. Cette r\'eduction est caract\'eris\'ee par une m\'etrique $g$ 
qui, en chaque point, est donn\'ee par une forme bilin\'eaire sym\'etrique non 
d\'eg\'en\'er\'ee sur l'espace tangent. On sait que $g$ peut \^etre consid\'er\'ee 
comme une section globale du fibr\'e en espaces homog\`enes 
$E=E(M,GL(n,\RR)/SO(n))$. 
Le lecteur aura not\'e que nous consid\'erons une r\'eduction \`a $SO(n)$ et non 
\`a $O(n)$, en d'autres termes, la vari\'et\'e est suppos\'ee orientable et 
orient\'ee.
Localement, c'est \`a dire relativement au choix 
d'un rep\`ere mobile $(e_\alpha)$ ou d'un rep\`ere naturel $({\partial \over 
\partial x^\mu})$, $g$ est donn\'ee par
$$
\mbox{\fbox{$
 g = g_{\alpha \beta}\, e^\alpha \otimes e^\beta  
= g_{\mu \nu}\, dx^\mu \otimes dx^\nu
$}} $$
L'\'ecriture  $g_{\mu\nu}\, dx^\mu\otimes dx^\nu$ justifie la notation 
$ds^2 = g_{\mu\nu}\,dx^\mu dx^\nu$
des vieux manuels et l'appellation {\it \'el\'ement de longueur}.
Lorsque le rep\`ere mobile choisi est orthonorm\'e, on aura simplement
$$
g_{\alpha \beta} = \delta_{\alpha \beta}
$$
Ceci suppose, d'ailleurs, que nous utilisons $SO(n)$ plut\^ot que $SO(p,q)$ 
et que $g$ est d\'efinie positive. Dans ce cas, 
on dit que la vari\'et\'e $M$ est une vari\'et\'e riemannienne. En fait, toutes 
les consid\'erations qui suivent restent valables lorsque 
la signature de $g$ est $(p,q)$, avec $p+q = n = dim\, M$ c'est \`a dire 
lorsque $M$ est une vari\'et\'e pseudo-riemannienne.
Dans le cas pseudo-riemannien, lorsque le rep\`ere choisi est orthonorm\'e, 
on a  $$g_{\alpha \beta} = 
\eta_{\alpha \beta}$$ o\`u $\eta$ d\'esigne la matrice diagonale 
$diag(1,1,\ldots,-1,\ldots,-1)$ avec $p$ signes $+$ et $q$ signes $-$.

La matrice $g_{\mu\nu}$ \'etant non d\'eg\'en\'er\'ee en chaque point, nous 
pouvons consid\'erer son inverse, not\'ee $(g^{\mu\nu})$. Ainsi $g^{\mu\nu} 
g_{\nu\rho} = \delta^\mu_\rho$. On obtient donc ainsi une forme 
bilin\'eaire sym\'etrique sur l'espace cotangent. Cette forme est le plus 
souvent \'egalement not\'ee $g$ (mais il peut arriver de la noter $g^{-1}$ !).

$$ g = g^{\alpha \beta} e_\alpha \otimes e_\beta $$
$$ g = g^{\mu \nu} {\partial \over \partial x^\mu} \otimes {\partial 
\over \partial x^\nu} $$
{\sl
Le lecteur est \'egalement invit\'e \`a relire la derni\`ere section du chapitre
consacr\'e aux vari\'et\'es diff\'erentiables, section dans laquelle nous avons 
d\'ej\`a introduit la notion de m\'etrique de fa\c con \'el\'ementaire et o\`u nous 
avons \'etabli un certain nombre de propri\'et\'es g\'en\'erales pour les vari\'et\'es 
riemanniennes}.

\subsection{Compatibilit\'e avec la m\'etrique}

On se donne une connexion lin\'eaire dans $FM$. Une m\'etrique \'etant donn\'ee, 
on suppose que cette connexion 
est compatible avec la r\'eduction de $FM$ \`a $OM$. On peut aussi
 supposer, de mani\`ere \'equivalente,
  qu'on se donne directement une connexion principale dans le fibr\'e $OM$.
On dit alors qu'on s'est donn\'e une connexion m\'etrique. Par construction, la 
m\'etrique est orthonorm\'ee lorsqu'on l'\'ecrit \`a l'aide des rep\`eres de $OM.$
Soit $\{e_a\}$ un rep\`ere mobile orthonorm\'e, alors $g =e_a \otimes e_b \;  \eta^{ab}$
ou $\eta$ est diagonal et constant (composantes $\pm 1$).
Lorsqu''on consid\'er\'e la m\'etrique comme $0$-forme  \`a valeurs dans le fibr\'e des 
tenseurs sym\'etriques de rang $2$, qui est inclus dans $\Omega^0(M,TM\otimes TM)$,
on voit imm\'ediatement que
$$ 
\mbox{\fbox{$
\nabla g = 0
$}}
$$
puisque $\nabla g = e_a \otimes e_b \; D\eta^{ab}$ et que $D\eta^{ab}=0$
(on utilise l'op\'erateur $D$ introduit au \ref{sec:operateur-D}.

La diff\'erentielle covariante de $g$ est donc nulle, mais ceci peut alors s'\'ecrire
dans n'importe quel rep\`ere, orthonorm\'e ou non $\{e_\alpha \}$.
Ainsi, $ \nabla g =  (e^\alpha \otimes 
e^\beta) D\, g_{\alpha \beta}$ et
$$
\mbox{\fbox{$
 D \, g_{\alpha \beta} = 0
$}}
$$
Explicitement, et en notant, comme d'habitude, ${\Gamma^\alpha}_\beta$ 
les \'el\'ements de la matrice de connexion, on obtient
$$
\mbox{\fbox{$
 d \, g_{\alpha \beta} - {\Gamma^\gamma}_\alpha \; g_{\gamma \beta} - 
{\Gamma^\gamma}_\beta \; g_{\alpha \gamma} = 0
$}}
$$

L'existence d'un produit scalaire non d\'eg\'en\'er\'e $g$ permet d'identifier un espace
vectoriel avec son dual et de
 ``monter'' ou ``descendre'' les indices \`a volont\'e. 
On pose donc
$$
\Gamma_{\alpha \beta}  =  {\Gamma^\delta}_\beta \; g_{\alpha \delta}
$$

Il ne devrait pas y avoir de confusion possible au niveau des notations : 
lorsqu'on \'ecrit $\Gamma_{\alpha \beta}$, c'est que cet objet est une
$1$-forme ($\Gamma_{\alpha \beta} = \Gamma_{\alpha \beta 
\gamma}\, e^\gamma$) et qu'on a utilis\'e la 
m\'etrique pour abaisser le premier indice de fibre. Si on \'ecrit, au contraire, 
$\Gamma_\gamma$, c'est qu'on parle alors du potentiel de jauge et qu'on a 
simplement fait appara\^itre l'indice de forme (qui est toujours en 
troisi\`eme --- et derni\`ere--- position).

L'\'equation de compatibilit\'e (condition de m\'etricit\'e) s'\'ecrit alors
$$  
d \, g_{\alpha \beta} - \Gamma_{\alpha \beta} - \Gamma_{\beta \alpha} = 0
$$

{\it Si on se place dans un rep\`ere orthonorm\'e\/}, les fonctions $g_{\alpha 
\beta}$ sont des constantes. Il en va de m\^eme dans un ``rep\`ere mobile de 
forme invariable'' (pr\'ecis\'ement d\'efini en imposant la constance des  $g_{\alpha 
\beta}$). Dans ce(s) cas, $d \, g_{\alpha \beta} = 0$ et la condition de 
m\'etricit\'e s'\'ecrit simplement
$$
\Gamma_{\alpha \beta} + \Gamma_{\beta \alpha} = 0
$$
On sait que les \'el\'ements de la matrice de connexion 
$(\Gamma^\gamma_\alpha)$ sont les $1$-formes
$\Gamma^\gamma_\alpha = \Gamma^\gamma_{\alpha \beta}\; e^\beta .$
En ``abaissant'' comme pr\'ec\'edemment le premier indice avec la m\'etrique, 
on pose
$$ 
\Gamma_{\alpha \beta \gamma}  =  \Gamma^\delta_{\beta \gamma}\; g_{\alpha 
\delta}
$$
{\it En rep\`ere orthonorm\'e (ou de forme fixe)\/} la condition de m\'etricit\'e se 
traduit par l'antisym\'etrie des $\Gamma_{\alpha \beta \gamma}$ sur les 
deux premiers indices.
$$
\Gamma_{\alpha \beta \gamma} = - \Gamma_{ \beta \alpha \gamma}
$$

Une autre fa\c con de retrouver de fa\c con tr\`es naturelle (et sans calcul!) 
cette propri\'et\'e d'antisym\'etrie est de se rappeler que les $1-formes$
${\Gamma^{\alpha}}_\beta$ ne sont autres que les \'el\'ements de matrice 
${\Gamma^{\alpha}}_\beta  =  \Gamma^a (X_a)^{\alpha}_\beta$
o\`u les $X_a$ constituent une repr\'esentation de l'alg\`ebre de Lie du groupe 
structural, en l'occurrence 
$SO(n)$, et ce sont donc, automatiquement, des matrices antisym\'etriques.

Nous verrons un peu plus loin que, pour une connexion bien particuli\`ere 
(la connexion de Levi-Civita) et dans des rep\`eres particuliers (les 
rep\`eres naturels), les coefficients de connexion poss\`edent \'egalement une propri\'et\'e
de sym\'etrie sur les deux derniers indices. 


Si, contrairement \`a ce que nous supposions ci-dessus, on se donne s\'epar\'ement une 
m\'etrique $g$ et une connexion lin\'eaire 
$\Gamma$ {\it quelconque}, il n'y a aucune raison pour que la connexion soit 
r\'eductible. En d'autres termes, il n'y a aucune raison de supposer que 
cette connexion soit compatible avec la m\'etrique. Si $\nabla$ d\'esigne la 
diff\'erentielle covariante associ\'ee \`a $\Gamma$, on d\'efinit alors le 
{\sl tenseur de non-m\'etricit\'e}{\index{tenseur de non-m\'etricit\'e}}
$$
\mbox{\fbox{$
Q  =  \nabla g
$}}
$$
C'est \`a dire, $Q(u,v,w) = \nabla g(u,v,w) = \nabla_w g(u,v).$
Dans un (co)-rep\`ere $e^\mu$, on a $Q = Q_{\mu\nu\rho} e^\mu \otimes e^\nu 
\otimes e^\rho$ avec
$$
Q_{\nu\rho\mu} = Q(e_\nu,e_\rho,e_\mu) = \nabla_\mu g_{\nu\rho} = 
g_{\nu\rho ;\mu}
$$

\subsection{Calcul des coefficients de connexion} 

Etant donn\'ee une m\'etrique et une connexion lin\'eaire compatible (qui, {\it 
a priori}, poss\`ede de la torsion), nous 
allons voir qu'il est possible d'exprimer les coefficients de connexion 
en terme de la m\'etrique, de ses d\'eriv\'ees premi\`eres, du tenseur de 
torsion et des fonctions de structure du rep\`ere mobile choisi.

Utilisant la m\'etrique, on d\'efinit
\begin{eqnarray*}
\Gamma_{\alpha \beta \gamma} & = &  g_{\alpha \alpha'}{\Gamma^{\alpha'}}_{\beta \gamma}\\
T_{\alpha \beta \gamma} & = &  g_{\alpha \alpha'}{T^{\alpha'}}_{\beta \gamma}\\
f_{\beta \gamma \alpha} & = &  g_{\alpha \alpha'}{f_{\beta \gamma}}^{\alpha'}
\end{eqnarray*}
Le tenseur de torsion (avec ses trois indices covariants) s'\'ecrit alors 
(voir section \ref{sec:torsion})
$$
T_{\alpha \beta \gamma} = \Gamma_{\alpha \gamma \beta} - \Gamma_{\alpha \beta \gamma} - f_{\beta \gamma \alpha}
$$

Nous avons vu que la condition de m\'etricit\'e s'\'ecrivait  $\nabla g = 0$, 
c'est \`a dire, en terme de composantes $g_{\alpha\beta ; \gamma} = 0$. 
Ceci implique
\begin{eqnarray*}
g_{\alpha\beta ; \gamma} = 0 & \Rightarrow & g_{\alpha\beta , \gamma} =  
\Gamma_{\beta\alpha\gamma} + \Gamma_{\alpha\beta\gamma} \\
g_{\beta\gamma ; \alpha} = 0 & \Rightarrow & g_{\beta\gamma , \alpha} =  
\Gamma_{\gamma\beta\alpha} + \Gamma_{\beta\gamma\alpha} \\
-g_{\gamma\alpha ; \beta} = 0 & \Rightarrow & -g_{\gamma\alpha , \beta} =  
-\Gamma_{\alpha\gamma\beta} - \Gamma_{\gamma\alpha\beta} \\
\end{eqnarray*}
En cons\'equence
$$
g_{\alpha\beta , \gamma} + g_{\beta\gamma , \alpha} - g_{\gamma\alpha , \beta}  =  
(\Gamma_{\beta\alpha\gamma} + \Gamma_{\beta\gamma\alpha}) + (\Gamma_{\alpha\beta\gamma} - 
\Gamma_{\alpha\gamma\beta})+ (\Gamma_{\gamma\beta\alpha} - \Gamma_{\gamma\alpha\beta})
$$
En utilisant l'expression du tenseur de torsion, il vient
$$
g_{\alpha\beta , \gamma} + g_{\beta\gamma , \alpha} - g_{\gamma\alpha , \beta}  = 
(\Gamma_{\beta\gamma\alpha} + f_{\gamma\alpha\beta}+ 
T_{\beta\gamma\alpha}+\Gamma_{\beta\gamma\alpha})+(-T_{\alpha\beta\gamma}-f_{\beta\gamma\alpha})+(-T_{\gamma\beta\alpha}-f_{\beta\alpha\gamma})
$$
En utilisant la sym\'etrie de $g$ et en renommant les indices, on obtient finalement

$$
\begin{array}{|rcl|}
\hline
2 \; \Gamma_{\alpha\beta\gamma} & = & 
(g_{\alpha\beta , \gamma} + g_{\gamma\alpha , \beta} - g_{\beta\gamma , \alpha}) \;  + \\
{} & {} & (-T_{\alpha\beta  \gamma} + T_{\gamma\alpha  \beta} - T_{\beta\gamma  \alpha}) \;  + \\
{} & {} & (f_{\alpha\beta  \gamma} - f_{\gamma\alpha  \beta} - f_{\beta\gamma  \alpha})  \\
\hline
\end{array}
$$

Les trois lignes ci-dessus sont quasiment identiques mais attention 
\`a notre fa\c con de ``descendre'' les indices (comparer la d\'efinition de 
$T_{\alpha\beta  \gamma}$ et celle de $f_{\alpha\beta  \gamma}$). Nos conventions 
concernant la position des indices ne sont pas universelles$\ldots$

L'expression ci-dessus des coefficients de connexion 
$\Gamma_{\alpha\beta\gamma}$ a \'et\'e calcul\'ee en travaillant dans un {\it rep\`ere 
quelconque\/} (ainsi, par exemple, $g_{\alpha\beta , \gamma}  =  e_\gamma 
[g_{\alpha\beta}]$). 
En travaillant dans un rep\`ere naturel on a, plus simplement, 
$g_{\mu\nu , \rho}  =  {\partial \over \partial x^\rho} 
[g_{\mu\nu}]$ et $f_{\mu\nu \rho} = 0$.

L'expression 
$$
\mbox{\fbox{$
 \{{}^{\,\mu}_{\nu \rho}\}  =  1/2 (g_{\mu\nu , \rho} + g_{\rho\mu , \nu} - g_{\nu\rho , \mu})
$}}
$$
est souvent d\'esign\'ee sous le nom de {\sl Symbole de Christoffel}
 {\index{symboles de Christoffel}}.

Le tenseur
$$
S_{\alpha\beta\gamma}  =  (T_{\alpha\beta  \gamma} - T_{\gamma\alpha  \beta} + T_{\beta\gamma  
\alpha})
$$
est souvent d\'esign\'e sous le nom de {\sl tenseur de contorsion}
 {\index{tenseur de contorsion}} (si! si!). Ce tenseur peut \^etre 
 nul sans que $T$ le soit; lorsque $S$ est nul, la torsion dispara\^it de 
 l'expression des coefficients de connexion 
 $\Gamma_{\alpha\beta\gamma}.$

Le calcul pr\'ec\'edent montre donc qu'une connexion lin\'eaire compatible avec 
une m\'etrique donn\'ee est enti\`erement caract\'eris\'ee par cette derni\`ere et 
par la torsion de cette connexion (on peut toujours faire dispara\^itre les fonctions de 
structure du rep\`ere mobile en se pla\c cant dans un rep\`ere naturel).

Le calcul pr\'ec\'edent montre aussi que, \'etant donn\'ee une m\'etrique, {\it si on impose une torsion 
nulle}, on obtient une unique connexion compatible qui s'exprime enti\`erement en terme de la 
m\'etrique. Cette (unique) connexion s'appelle {\sl la connexion de 
Levi-Civita}
 {\index{connexion de 
Levi-Civita}} ou encore {\it la} connexion riemannienne.
Re-exprimons ce r\'esultat sous la forme suivante:

{\em Soit $OM$ le fibr\'e principal des rep\`eres orthonorm\'es associ\'e \`a une 
m\'etrique donn\'ee. Alors, parmi toutes les connexions possibles sur le 
fibr\'e $OM$, il en existe une et une seule qui soit sans torsion. On 
l'appelle la connexion de Levi-Civita\/}.

Pour cette connexion, et en rep\`ere naturel, on a donc simplement
$$
 \Gamma_{\mu\nu\rho}  =  \{{}^{\,\mu}_{\nu \rho}\} =
{1 \over 2}(g_{\mu\nu , \rho} + g_{\rho\mu , \nu} - g_{\nu\rho , \mu})
$$

De fa\c con g\'en\'erale, l'expression du tenseur de torsion en terme des coefficients 
de connexion et des fonctions de structure du rep\`ere mobile (expression 
rappel\'ee ci-dessus) montre que, {\it si on utilise la connexion de Levi-Civita ($T_{\mu\nu\rho}=0$)
dans un rep\`ere naturel\/} ($f_{\mu\nu\rho}=0$), les coefficients de 
connexions poss\`edent la sym\'etrie suivante:
$${\Gamma^\mu}_{\nu\rho} = {\Gamma^\mu}_{\rho\nu}$$
ou encore
$$\Gamma_{\mu\nu\rho} = \Gamma_{\mu\rho\nu}$$
Il faut se garder de confondre cette sym\'etrie sur les deux derniers 
indices avec l'antisym\'etrie sur 
les deux premiers indices (voir plus haut), antisym\'etrie qui est valable 
pour {\it toute} connexion compatible avec la m\'etrique,{\it \`a condition\/} 
toutefois de 
travailler dans un rep\`ere orthonorm\'e (ce qui est rarement le cas des 
rep\`eres naturels).


Il n'est pas difficile de d\'emontrer que, dans le cas o\`u la connexion 
n'est pas compatible avec la m\'etrique, il faut rajouter, au second membre 
de l'\'equation donnant l'expression de $ 2 \Gamma_{\mu\nu\rho}$, la quantit\'e
$$ - (Q_{\mu\nu  \rho} + Q_{\rho\mu  \nu} - Q_{\nu\rho  \mu})$$
o\`u $Q$ est le tenseur de non-m\'etricit\'e.


\subsection{Compl\'ements sur le tenseur de 
Riemann. Propri\'et\'es de sym\'etrie.}

Nous avons d\'ej\`a analys\'e en d\'etail les propri\'et\'es du tenseur de courbure, 
pour une connexion lin\'eaire quelconque, et il n'y a pas grand chose \`a 
rajouter lorsqu'on suppose que cette connexion est une connexion 
m\'etrique, si ce n'est la propri\'et\'e d'antisym\'etrie suivante : en posant
$$ R_{\alpha \beta \gamma \delta}  =  g_{\alpha \alpha'} 
{R^{\alpha'}}_{\beta \gamma \delta}$$
on s'aper\c coit que non seulement ce tenseur est antisym\'etrique sur ses 
deux derniers indices (comme toujours, puisqu'il provient de la $2$-forme 
de courbure) mais que, en outre, il est \'egalement antisym\'etrique sur ces 
deux premiers indices (utiliser l'expression de la courbure en terme de 
la connexion et l'antisym\'etrie d\'ej\`a discut\'ee des 
\'el\'ements de la matrice de connexion, ou, plus directement, se rappeler 
que les g\'en\'erateurs de $SO(n)$ sont des matrices antisym\'etriques).



En r\'esum\'e, pour une connexion quelconque :
$$ R_{\alpha \beta \gamma \delta} =  - R_{\alpha \beta \delta  \gamma}$$
Pour une connexion m\'etrique : 
$$ R_{\alpha \beta \gamma \delta} = - R_{ \beta \alpha \gamma \delta} $$

Pour une connexion m\'etrique, sans torsion, c'est � dire pour la connexion de Levi-Civita, 
les deux identit\'es pr\'ecedentes impliquent une propri\'et\'e de sym\'etrie par \'echange de paires
(utiliser les identit\'es de Bianchi) : 
$$ R_{\alpha \beta \gamma \delta} =  R_{ \gamma \delta \alpha \beta}$$



\smallskip
\noindent
Par ailleurs, les expressions des identit\'es de Bianchi se simplifient 
pour la connexion de Levi-Civita (puisque la torsion dispara\^it). Il vient : 

\index{identit\'es de Bianchi}
Bianchi-1 sans torsion ($R \wedge \theta = 0$) : 
$$
\mbox{\fbox{$
{R^{\alpha}}_{\beta \gamma \delta} + {R^{\alpha}}_{ \gamma \delta \beta} + 
{R^{\alpha}}_{ \delta \beta \gamma } = 0
$}}
$$

Bianchi-2 sans torsion : 
$$
\mbox{\fbox{$
{R^{\alpha}}_{\beta \gamma \delta ; \epsilon} +
 {R^{\alpha}}_{\beta \delta \epsilon; \gamma } + 
 {R^{\alpha}}_{\beta \epsilon \gamma ; \delta}  = 0
$}}
$$

 
\subsection{Equation des g\'eod\'esiques} 
Nous avons d\'ej\`a d\'efini la notion de courbe autoparall\`ele dans la section 
pr\'ec\'edente. Il s'agissait l\`a d'une notion li\'ee \`a la donn\'ee d'une 
connexion lin\'eaire. Dans le cas o\`u cette connexion est {\it la} connexion 
de Levi-Civita d\'etermin\'ee par une m\'etrique, une telle courbe est, par 
d\'efinition, une g\'eod\'esique de la vari\'et\'e riemannienne. En fait, la 
terminologie  (``le plus court chemin'') vient du fait qu'une telle courbe est \'egalement solution 
d'un probl\`eme variationnel ($\delta \int(g_{\mu\nu}dx^\mu 
dx^\nu)^{1/2}$) mais nous 
n'\'etudierons pas cet aspect de la question. Notons que, dans le cas o\`u 
nous nous donnons s\'epar\'ement une m\'etrique (et donc la connexion de 
Levi-Civita correspondante), ainsi qu'une seconde connexion compatible avec la 
m\'etrique donn\'ee, mais poss\'edant de la torsion, il faut distinguer entre les
courbes autoparall\`eles associ\'ees \`a cette derni\`ere et les g\'eod\'esiques. Une 
telle distinction est physiquement importante lorsqu'on traite du 
mouvement des particules \`a spin en Relativit\'e G\'en\'erale : les particules 
scalaires suivent en effet les g\'eod\'esiques du champ gravitationnel mais 
les particules ayant un moment cin\'etique intrins\`eque suivent les 
autoparall\`eles d'une connexion avec torsion.

Pla\c cons nous dans un rep\`ere naturel. Nous avons d\'ej\`a \'etabli l'\'equation des 
autoparall\`eles dans la section consacr\'ee aux connexions lin\'eaires. 
L'\'equation des g\'eod\'esiques est donc identique, \`a ceci pr\`es que les 
coefficients de connexion ${\Gamma^\mu}_{\nu\rho}$ doivent \^etre donn\'es par les symboles de 
Christoffel $ \{{}^{\,\mu}_{\nu \rho}\}.$ On voit donc que pour que 
g\'eod\'esiques et autoparall\`eles co\"incident, il faut et il suffit que le 
tenseur de contorsion $S_{\alpha\beta\gamma}$ soit nul. Ce sera 
\'evidemment le cas si la torsion est nulle mais cette condition n'est pas 
n\'ecessaire.

De fa\c con plus g\'en\'erale, le lecteur pourra se convaincre que si, $\nabla$ 
et $\nabla'$ d\'esignent deux connexions, une condition n\'ecessaire et 
suffisante pour que les autoparall\`eles des deux connexions co\"incident 
est que la partie sym\'etrique $s(x,y)  =  (c(x,y) + c(y,x))/2$ du tenseur 
diff\'erence $c(x,y)  =  \nabla_x y - {\nabla'}_x y$ s'annule (noter que 
$c(x,y)$ est bien un tenseur).



\subsection{Tenseur de Ricci, courbure scalaire et tenseur d'Einstein, 
courbures sectionelles} 

\begin {itemize}
\item Tenseur de Ricci.
Nous avons d\'ej\`a d\'efini le tenseur de Ricci $\rho$ de composantes 
$$
\mbox{\fbox{$
\rho_{\sigma \nu}  = {R^\mu_\sigma}_{\mu\nu}
$}}
$$
 dans la section consacr\'ee aux 
connexions lin\'eaires puisque l'utilisation d'une m\'etrique n'\'etait pas 
n\'ecessaire. Pour la connexion de Levi-Civita la torsion est nulle et
 on sait que, dans ce cas,  le tenseur de Riemann 
de composantes $R_{\alpha \beta \mu \nu}$ est invariant lorsqu'on permute 
les deux premiers indices avec les deux derniers. On en d\'eduit 
imm\'ediatement que, pour cette connexion,  le tenseur de 
Ricci est sym\'etrique ($\rho_{\mu \nu} = \rho_{\nu\mu}$) et d\'efinit donc 
une forme bilin\'eaire sym\'etrique ainsi qu'une forme quadratique $r(u) 
 =  \rho(u,u)$ o\`u $u \in TM$. L'application polaire associ\'ee est 
l'application lin\'eaire $\hat r(u)  =  \Sigma_\mu \rho(u,e_\mu) e_\mu$ 
et les valeurs propres de $\hat r$ sont les {\sl courbures principales de 
Ricci\/} {\index{courbures principales de 
Ricci}}. 



\item Courbure scalaire.{\index{courbure scalaire}}
La courbure scalaire $\tau $ est d\'efinie par l'\'egalit\'e
$$
\mbox{\fbox{$
\tau  =  g^{\mu \nu} \rho_{\mu\nu}
$}}
$$

\item Tenseur d'Einstein.{\index{tenseur d'Einstein}}
Pour la connexion de Levi-Civita, le tenseur d'Einstein $G$ est d\'efini par l'\'egalit\'e
$$
\mbox{\fbox{$
G_{\mu \nu}  =  \rho_{\mu\nu} - {1\over 2} g_{\mu\nu} \tau
$}}
$$
 ce nouveau tenseur, comme $\rho$ et $g$, est manifestement sym\'etrique.
Son nom provient bien entendu de la th\'eorie de la Relativit\'e G\'en\'erale, 
puisque les \'equations d'Einstein {\index{equations d'Einstein}}
d\'ecrivant la r\'eaction de la g\'eom\'etrie (d\'ecrite par 
la m\'etrique $g$) \`a la mati\`ere (d\'ecrite par un tenseur d'\'energie-impulsion 
not\'e traditionnellement
$T$ --- ce n'est pas la torsion !---) s'\'ecrivent simplement 
$$
\mbox{\fbox{$
G = 8 \; \pi \;  T
$}}
$$
L'\'etude de ces \'equations et de la physique qui leur correspond sort du 
cadre de cet ouvrage. Notons simplement que ces \'equations sont des 
\'equations diff\'erentielles du second ordre par rapport \`a la m\'etrique 
puisque le tenseur de courbure s'\'ecrit en terme des d\'eriv\'ees des 
coefficients de connexion et que ces derniers (rappelons que nous supposons choisie la connexion de Levi Civita) s'expriment en 
terme des d\'eriv\'ees de la m\'etrique. 

Par ailleurs mentionnons qu'une vari\'et\'e pour laquelle le tenseur 
d'Einstein est proportionnel \`a la m\'etrique est, par d\'efinition, une
 {\sl vari\'et\'e d'Einstein\/} {\index{vari\'et\'e 
d'Einstein}}.
L'\'etude des vari\'et\'es d'Einstein ---un sujet fascinant de la g\'eom\'etrie 
Riemannienne--- sort \'egalement du cadre de cet ouvrage et nous renvoyons le 
lecteur aux articles sp\'ecialis\'es ainsi qu'au beau livre d'Arthur Besse 
(autre math\'ematicien imaginaire, comme Bourbaki) intitul\'e ``Einstein 
Manifolds'' \cite{Besse}.

Remarque: si la connexion, suppos\'ee m\'etrique, poss\`ede une torsion, la propri\'et\'e pour $R$ d'\^etre invariant par \'echange de paires n'est plus v\'erifi\'ee, 
le tenseur de Ricci n'est plus sym\'etrique et, dans ce cadre, on ne connait pas de g\'en\'eralisation naturelle du tenseur d'Einstein.

\item Courbures sectionelles.
On d\'efinit tout d'abord la fonction bi-quadratique de courbure
$$
K(u,v)  =  g(R(v,u) u,v)
$$
Cette fonction d\'efinie dans le fibr\'e tangent associe un r\'eel \`a tout 
couple de vecteurs. Cette fonction est 1) sym\'etrique (puisque $R$ est 
sym\'etrique par \'echange de paires), 2) bi-quadratique, 3) telle que 
$K(u,u) = 0$. R\'eciproquement (d\'emonstration non triviale que nous 
omettrons), toute fonction $K(u,v)$ v\'erifiant ces trois propri\'et\'es peut 
\^etre consid\'er\'ee comme la fonction bi-quadratique de courbure associ\'ee \`a 
une certaine m\'etrique. L'ensemble des fonctions sym\'etriques, 
bi-quadratique avec $K(u,u)=0$ forme un espace vectoriel de dimension 
$n^2(n^2-1)/12$ ; en d'autres termes, il faut pr\'eciser  $n^2(n^2-1)/12$ 
nombres r\'eels pour d\'ecrire la courbure Riemannienne d'une vari\'et\'e de 
dimension $n$ en {\it un} point.

La fonction $K$ associe un nombre \`a tout sous-espace de 
dimension $2$ de l'espace tangent. Soit ${u,v}$ une base orthonormale 
d'un $2$-plan, on d\'efinit alors la {\sl courbure sectionelle\/} {\index{courbure sectionelle}} de ce 
sous-espace de dimension $2$
par $K = K(u,v)$ et on d\'emontre que $K$ est ind\'ependant du choix de $u$ 
et $v$ dans ce sous-espace. On d\'emontre que, g\'eom\'etriquement, $K$ n'est autre que la 
courbure Gaussienne de la $2$-surface engendr\'ee par les g\'eod\'esiques ayant 
des vecteurs tangents de la forme $\lambda u + \mu v$. Nous renvoyons le 
lecteur \`a un trait\'e \'el\'ementaire de g\'eom\'etrie des surfaces pour les 
notions relatives aux courbures de Gauss etc.
\end {itemize}

%%%%%%%%%%%%%%%%%%%%%%%%%%%%
%%%%%%%%%%%%%%%%\subsection{Aspects conformes. Tenseur de Weyl}  
%%%%%%%%%%%%%%%%$TODO***$
%%%%%%%%%%%%%%%%%%%%%%%%%%%%
\subsection{Dualit\'e de Hodge et laplaciens}

 Nous nous contenterons
de rassembler ici  quelques d\'efinitions et r\'esultats g\'en\'eraux.
Cette section m\'eriterait \'egalement un  traitement plus approfondi. Nous sugg\'erons au lecteur de relire la section \ref{sec:forme volume} consacr\'e \`a la d\'efinition de l'\'el\'ement de volume $\epsilon$. 

\subsubsection{Appariement entre $\Omega^k(M)$ et $\Omega^{n-k}(M)$}
Soit $\epsilon$ la forme volume canonique d'une vari\'et\'e riemannienne 
orient\'ee. On d\'efinit une op\'eration $\{\, , \, \}$ associant un nombre 
r\'eel \`a la donn\'ee d'une $k$-forme et d'une $n-k$-forme.
$$
(\omega , \alpha) \in \Omega^k(M) \times \Omega^{n-k}(M) \rightarrow 
\omega \wedge \alpha = \lambda \epsilon \in \Omega^n(M) \rightarrow 
\{\omega ,\alpha\}  =  \lambda \in \RR
$$
Le r\'eel $\lambda$ est bien d\'etermin\'e puisque l'espace des $n$-formes est 
de dimension $1$. On a donc
$$\omega \wedge \alpha = \{\omega ,\alpha\} \epsilon$$

\subsubsection{L'isomorphisme $\star$ : La dualit\'e de Hodge}{\index{dualit\'e 
de Hodge}}
On d\'efinit un isomorphisme $ \star \, : \,  \Omega^k(M) \rightarrow \Omega^{n-k}(M)$ 
comme la composition de l'isomorphisme $\omega \in \Omega^k(M) \rightarrow  \{\omega , . 
\} \in (\Omega^{n-k}(M))^*$ et de l'isomorphisme musical entre 
$\Omega^{n-k}(M)$  et son dual, induit par la m\'etrique.

Explicitement, soit $\{\epsilon_\mu\}$ une base orthonorm\'ee de sens 
direct et $\{\omega^\mu\}$ sa base duale. Soit $\mu_1 \leq \mu_2 \ldots 
\leq \mu_k \subset \{1,2,\ldots n\}$ et soit $\nu_1 \leq \nu_2 \ldots 
\leq \nu_{n-k}$ la suite compl\'ementaire. Alors
$$
\mbox{\fbox{$
\star(\omega^{\mu_1}\wedge \omega^{\mu_2}\wedge\ldots \omega^{\mu_k}) = 
\pm \omega^{\nu_1}\wedge \omega^{\nu_2}\wedge\ldots \omega^{\nu_{n-k}}
$}}
$$
o\`u $\pm$ d\'esigne la signature de $[\mu_1 \mu_2 \ldots \mu_k \nu_1 \nu_2 \ldots  \nu_{n-k}]$.
Ainsi
$$ 
\mbox{\fbox{$
\star \,1 = \epsilon
$}}
$$

Soit $\{\epsilon_\mu\}$ une base orthonorm\'ee de sens 
direct et $\{\omega^\mu\}$ sa base duale, alors 
$$
\omega^{\mu_1}\wedge \omega^{\mu_2}\wedge\ldots \omega^{\mu_k}\wedge \star \, 
(\omega^{\nu_1}\wedge \omega^{\nu_2}\wedge\ldots \omega^{\nu_{n-k}}) = 
\delta^{\mu_1 \mu_2 \ldots \mu_k}_{\nu_1 \nu_2 \ldots \nu_k} . \epsilon
$$

Soit $S \in \Omega^k(M)$, alors, $\star \, S$ est le seul \'el\'ement 
de $\Omega^{n-k}(M)$ tel que
$$
\forall v_1, v_2, \ldots, v_{n-k} \in TM, \; \star\, S(v_1, v_2, \ldots, 
v_{n-k}) \epsilon = S \wedge v_1^\flat \wedge v_2^\flat \wedge \ldots \wedge 
v_{n-k}^\flat$$
o\`u $v^\flat$ d\'esigne la forme lin\'eaire correspondant au vecteur $v$ via 
l'isomorphisme musical induit par la m\'etrique.


Explicitement, soit $S$ une $k$-forme de composantes $S_{\mu_1\mu_2\ldots\mu_k}$.
Son dual est la $(n-k)$-forme $\star S$ de composantes
$$
\mbox{\fbox{$
{\star S}_{\mu_{k+1}\ldots\mu_n}={1\over k!}\,
S^{\mu_1\mu_2\ldots\mu_k}\, \epsilon_{\mu_1\mu_2\ldots\mu_k\mu_{k+1}\ldots\mu_n}
$}}
$$
Lorsque la m\'etrique poss\`ede une signature $(p,q)$ 
(avec $q$ signes $-$ et $p+q=n$),
on voit que $\star \epsilon = 1/n!\, \epsilon^{\mu_1\ldots\mu_n}\, \epsilon_{\mu_1\ldots\mu_n} = (-1)^q$ puisque $\epsilon^{\mu_1\ldots\mu_n} \epsilon_{\mu_1\ldots\mu_n}= (-1)^q n!$.
$$ 
\mbox{\fbox{$
\star \, \epsilon  = (-1)^q
$}}
$$

Notons que si $\omega$ est une $n$-forme, et qu'il existe donc un nombre r\'eel $\lambda$ tel que $\omega = \lambda
\epsilon$, alors $\star \, \omega = (-1)^q \lambda$.


On voit que $\star\star 1 = (-1)^q 1$ et que $\star \star \epsilon = (-1)^q \epsilon$. Plus g\'en\'eralement, pour une $k$-forme $S$, 
$$
\mbox{\fbox{$
\star\star S = (-1)^{k(n-k)}\,  (-1)^q \, S
$}}
$$
L'op\'erateur $\star$, agissant sur  $\Omega^k(M)$ est donc inversible, avec $\star^{-1}$, agissant sur $\Omega^{n-k}(M)$, donn\'e par
$\star^{-1} = (-1)^{k(n-k) + q}\, \star$.

Notons que, dans le cas d'une signature purement euclidienne $(n,0)$, il n'y a pas \`a ce pr\'eoccuper du  signe suppl\'ementaire $(-1)^q$.

Pour des applications \`a la physique de l'espace-temps (signature $(3,1)$),
les formules suivantes sont particuli\`erement utiles :

pour une $1$-forme $J$, 
$\star J_{\alpha\beta\gamma} = J^\mu \epsilon_{\mu\alpha\beta\gamma}$ et
$\star \star J = J$.

pour une $2$-forme $F$, 
$\star F_{\alpha\beta} = {1\over 2!} F^{\mu\nu} \epsilon_{\mu\nu\alpha\beta}$ et
$\star \star F = -F$.

pour une $3$-forme $B$, 
$\star B_{\alpha} = {1\over 3!} B^{\lambda\mu\nu} \epsilon_{\lambda\mu\nu\alpha}$ et
$\star \star B = B$.





\subsubsection{Le produit $<<,>>$ dans l'espace des $k$-formes}
Soient $\omega, \eta \in \Omega^k(M)$. On pose
$$\omega \wedge \star \, \eta = \eta \wedge \star \, \omega = <<\omega , \eta>>. 
\epsilon
$$ 
Notons que $<<\omega , \eta>>$ est une fonction sur $M$ et que $<<\omega 
, \eta>> = \{\omega,\alpha\}$ o\`u $\alpha = \star \, \eta$.

On en d\'eduit un produit scalaire global dans $\Omega^k(M)$. 
Nous supposons maintenant que la vari\'et\'e $M$ est compacte.
Soient $\omega, \eta \in \Omega^k(M)$. On pose
$$
(\omega,\eta)  =  \int_M \omega \wedge \star \, \eta = \int_M 
<<\omega , \eta>> \epsilon
$$
Cette forme est bilin\'eaire sym\'etrique et non d\'eg\'en\'er\'ee sur $\Omega^k(M)$. 

On d\'emontre que $\star$ est unitaire pour $(.,.)$ : 
$$ (\star \omega, \star \eta) = (\omega,\eta) $$
	
\subsubsection{La codiff\'erentielle $\delta$ sur les 
formes}{\index{codiff\'erentielle}}
$\delta : \Omega^k(M) \mapsto \Omega^{k-1}(M)$ est d\'efini comme l'adjoint de 
$d$ pour le produit scalaire global, c'est \`a dire
$$
\mbox{\fbox{$
(\alpha_k, d\beta_{k-1}) = (\delta\alpha_k,\beta_{k-1})
$}}
$$
En \'ecrivant la d\'efinition et en int\'egrant par parties, on obtient l'expression de $\delta$ agissant sur une $k$-forme  $\alpha_k \in \Omega^k(M)$.
%$$
%\mbox{\fbox{$
%\delta = (-1)^k \,  \star^{-1} \,  d  \, \star
%$}}
%$$
Pour une vari\'et\'e de dimension $n$ munie d'une m\'etrique de signature $(p,q)$, il vient : 
$$
\delta \alpha_k = (-1)^{nk+n+1} \, (-1)^q \,  \star \,  d  \, \star  \; \alpha_k \in \Omega^{k-1}(M).
$$
On en d\'eduit imm\'ediatement que
$$
\delta^2 = 0
$$
Par ailleurs $\delta$ est nul sur les fonctions (sur $\Omega^0(M)$).

Notons \'egalement que
$$
\star \delta \omega_k = (-1)^k d \star \omega_k
$$ et que
$$\delta \star \omega_k = (-1)^{k+1} \star d \omega_k
$$

Notons enfin qu'on peut d\'efinir $\delta$ m\^eme si $M$ n'est pas 
orientable. En effet, la d\'efinition est locale et changer l'orientation 
revient \`a changer $\star$ en $-\star$, de sorte que $\delta$ est inchang\'e.

Le lecteur pourra v\'erifier que, relativement \`a un syst\`eme de coordonn\'ees 
locales ${x^\mu}$, la codiff\'erentielle d'une $k$-forme 
$$
\mbox{\fbox{$
\omega = 
 {\sum}_{\mu_1 \leq \mu_2\leq  \ldots \mu_k}
\omega_{\mu_1 \mu_2 \ldots \mu_k} dx^{\mu_1} \wedge dx ^{\mu_2} \wedge 
\ldots \wedge dx^{\mu_k}
$}}
$$  s'\'ecrit
$$
\mbox{\fbox{$
\delta \omega = - g^{\rho\mu} 
{\sum}_{\mu_1 \leq \mu_2\leq  \ldots \mu_{k-1}}
\omega_{\mu_1 \mu_2 \ldots \mu_{k-1} \rho , \mu} 
dx^{\mu_1} \wedge dx ^{\mu_2} \wedge 
\ldots \wedge dx^{\mu_{k-1}}
$}}
$$

Le lecteur pourra \'egalement v\'erifier qu'on peut remplacer les d\'eriv\'ees 
ordinaires par des d\'eriv\'ees covariantes dans la formule ci-dessus 
(remplacer la virgule par le point virgule) car, bien que $\omega_{\rho\mu_1 \mu_2 \ldots \mu_{p-1} , \mu}$
soit en g\'en\'eral diff\'erent de $\omega_{\rho\mu_1 \mu_2 \ldots \mu_{k-1} ; 
\mu}$, tous les termes d\'ependant de la connexion disparaissent dans la 
somme.

On peut consid\'erer une $k$-forme comme un tenseur covariant (particulier) 
c'est \`a dire comme une $0$-forme \`a valeur dans le 
fibr\'e $(T^*M)^{\otimes k}$. On peut donc consid\'erer la diff\'erentielle 
covariante $\nabla \omega$ ainsi que la d\'eriv\'ee covariante $\nabla_v 
\omega$ dans la direction d'un vecteur $v$. Soit $\{e_\mu\}$ un rep\`ere 
mobile orthonorm\'ee et $\nabla$ la connexion de Levi-Civita, la formule 
pr\'ec\'edente, donnant l'expression de $\delta \omega$ peut se re-\'ecrire 
sous la forme
\begin{eqnarray*}
\delta \omega (e_{\mu_1},e_{\mu_2}, \ldots ,e_{\mu_{k-1}}) & = & 
- \Sigma_\mu \nabla_\mu \omega(e_{\mu_1},e_{\mu_2},\ldots ,e_{\mu_{k-1}}, 
e_\mu) \\
{} & = & 
- \Sigma_\mu \nabla \omega(e_{\mu_1},e_{\mu_2},\ldots ,e_{\mu_{k-1}}, 
e_\mu, e_\mu)
\end{eqnarray*}

Il est parfois commode d'utiliser un symbole $Tr$ (trace) qui contracte 
les deux derniers indices d'un tenseur donn\'e \`a l'aide de la m\'etrique (on 
suppose que ces deux derniers indices sont tous deux covariants ou tous 
deux contravariants, sinon, on n'a pas besoin de m\'etrique). La 
relation pr\'ec\'edente s'\'ecrit alors
$$
\delta \omega = - Tr \; \nabla \omega
$$


\subsubsection{La divergence $div$ sur les tenseurs quelconques}
La codiff\'erentielle $\delta$ n'\'etait d\'efinie que sur les formes 
ext\'erieures, mais la formule $\delta \omega = - Tr \nabla \omega$ a un sens 
pour des tenseurs quelconques. Soit $T$ un tenseur de rang quelconque. On 
pose
$$
div \, T = Tr \; \nabla T
$$
En particulier, pour une forme, on a $div \, \omega = - \delta \omega$.


\subsubsection{Le laplacien de De Rham sur les formes}{\index{laplacien 
de De Rham}}
On d\'efinit le laplacien de De Rham (encore appel\'e laplacien de Hodge ou 
op\'erateur de Beltrami) comme
$$
\Delta_{DR}\, = (d+\delta)^2=d\delta + \delta d
$$
\noindent
Voici quelques unes de ses propri\'et\'es :

$\Delta_{DR}\,$ est self adjoint pour $(\,.\,)$ : $(\omega, \Delta_{DR}\, \eta) = 
(\Delta_{DR}\, \omega, \eta)$.

$\Delta_{DR}\,$ est un op\'erateur positif (on suppose maintenant que $M$ est 
proprement riemannienne) : $(\omega, \Delta_{DR}\, \omega) \geq 0$.

On dit que $\omega$ est {\sl harmonique\/} {\index{harmonique}}
 si et seulement si $\Delta_{DR}\, 
\omega = 0$ c'est \`a dire si et seulement si $\omega$ est \`a la fois ferm\'ee 
($d \omega = 0$) et coferm\'ee ($\delta\omega = 0$).
\smallskip \noindent
{\sl Th\'eor\`eme de Hodge\/} ($M$ est suppos\'ee compacte) :
$$
\forall \omega_p \in \Omega^p(M) \; \exists ! 
(\alpha_{p-1},\alpha_{p},\alpha_{p+1}) \in \Omega^{p-1}(M)\times 
\Omega^{p}(M)\times \Omega^{p+1}(M) \, \vert \, \omega_p = d\alpha_{p-1} + \alpha_{p} 
+ \delta \alpha_{p+1}
$$
o\`u $\alpha_{p}$ est harmonique. Il y a unicit\'e du triplet 
$(\alpha_{p-1},\alpha_{p},\alpha_{p+1})$.

Toute classe de cohomologie de De Rham $H^p$ contient un unique 
repr\'esentant harmonique.

Nous d\'efinissons un peu plus bas le laplacien usuel (na\"\i f)
 qu'on notera $\Delta$ sur les fonctions 
($C^\infty(M) = \Omega^0(M))$. Attention, il s'av\`ere que le laplacien de 
De Rham sur les fonctions et le laplacien na\"\i f diff\`erent par un signe :  $\Delta f = - \Delta_{DR}\, f$.

Au lieu de consid\'erer des formes diff\'erentielles (\'el\'ements de 
$\Omega^p(M)$) on peut consid\'erer des formes diff\'erentielles \`a valeur 
dans un fibr\'e vectoriel (\'el\'ements de $\Omega^p(M,E)$). L'existence d'une 
m\'etrique sur la base ainsi que d'un produit scalaire dans les fibres
permet l\`a encore, avec les m\^emes hypoth\`eses, de d\'efinir un produit scalaire global. On 
d\'efinit alors une codiff\'erentielle covariante $\delta^\nabla$ comme 
l'adjoint de la diff\'erentielle ext\'erieure covariante $d^\nabla$. On 
d\'efinit ensuite un laplacien de De Rham  
$$
\mbox{\fbox{$
\Delta^\nabla_{DR}\,  =  d^\nabla 
\delta^\nabla + \delta^\nabla d^\nabla
$}}
$$
 et la th\'eorie se g\'en\'eralise$\ldots$


{\small 
Mentionnons l'existence du {\sl laplacien de Lichnerowicz\/}\label{sec:Lichnerowicz}\index{laplacien de Lichnerowicz} associ\'e
\`a une m\'etrique $g$ et agissant
sur les tenseurs sym\'etriques $2\times 2$:
$$
\Delta_L \; h_{\mu\nu}  =  \Delta \; h_{\mu\nu} + \rho^\lambda_\mu \; h_{\lambda\nu}
+ \rho^\lambda_\nu h_{\mu\lambda} + 
2  g^{\lambda \lambda'}g^{\sigma \sigma'}g_{\nu\nu'} R^{\nu'}_{\sigma'\mu\lambda'} h_{\lambda\sigma}
$$
A titre d'exercice, on peut voir que ce dernier op\'erateur peut s'obtenir comme somme pond\'er\'ee
$\Delta_L = \Delta_{02} - \Delta_{01} - \Delta'_{01}$ o\`u les  laplaciens g\'en\'eralis\'es
$\Delta_{ij}$ sont eux-m\^emes associ\'es aux diff\'erentielles covariantes g\'en\'eralis\'ees introduites
en page \pageref{sec:fancy-d}.
}


\subsubsection{Hessiens et Laplaciens na\"\i fs}\label{sec:laplaciens}
Nous avons consacr\'e une section aux diff\'erentielles covariantes 
g\'en\'eralis\'ees (section \ref{sec:nabla-generalise}). Nous avons vu, en particulier, que si 
$\xi$ est une $0$-forme \`a valeurs dans un fibr\'e vectoriel $E$, on 
pouvait tout d'abord consid\'erer $\nabla \xi$ qui est un \'el\'ement de 
$\Omega^1(M,E)$ puis identifier $\Omega^1(M,E)$ avec $\Omega^0(M,E\otimes 
T^*M)$, ce qui nous autorise, dans la mesure o\`u les fibr\'es $E$ et $T^*M$ 
sont tous deux \'equip\'es de connexion, \`a consid\'erer l'objet $\nabla \nabla \xi$, 
qui sera donc un \'el\'ement de $\Omega^1(M,E\otimes 
T^*M)$ qu'on peut identifier avec $\Omega^0(M,E\otimes T^*M \otimes T^*M)$.
De fa\c con g\'en\'erale, nous avons  d\'efini le {\sl hessien\/} {\index{Hessien}}
 de $\xi$ comme
$
Hess (\xi)  =  \nabla \nabla \xi
$
Dans le cas le plus simple o\`u $\xi$ est une fonction scalaire $f$ et 
si nous choisissons la connexion de Levi-Civita, la torsion 
est nulle, auquel cas les d\'eriv\'ees covariantes secondes commutent (voir 
section \ref{sec:ricci-id}) et $Hess(f)$ est une forme bilin\'eaire sym\'etrique sur le 
fibr\'e tangent.

Plus g\'en\'eralement, soit $\xi \in \Omega^0(M,E)$, $e_i$ un rep\`ere local 
dans les fibres de $E$ et $\{\theta^\alpha\}$ un corep\`ere mobile sur $M$.
Nous avons calcul\'e explicitement $Hess(\xi)$ en section \ref{sec:ricci-id}.
Le laplacien na\"\i f sur $\Omega^0(M,E)$ (on dit souvent ``laplacien 
brut'') est d\'efini par l'\'egalit\'e
$$
\Delta(\xi)  =  Tr \, Hess (\xi)
$$
Explicitement, on a
$$
\Delta(\xi)  = e_i \; g^{\alpha\beta}\; \xi^i_{;\alpha\beta}
$$
C'est encore un \'el\'ement de $\Omega^0(M,E)$.

Dans le cas particulier d'une fonction $f$, et en utilisant un rep\`ere 
naturel, on obtient simplement
$$\Delta(f) = - \Delta_{DR}\, f = g^{\mu\nu}f_{;\mu\nu}= g^{\mu\nu}[{\partial^2 f \over \partial 
x^\mu \partial x^\nu} - \Gamma^\rho_{\mu\nu} {\partial f \over \partial 
x^\rho}]$$

En jouant avec les diff\'erentes sortes de diff\'erentielles covariantes 
g\'en\'eralis\'ees, on peut d\'efinir plusieurs autres sortes de Laplaciens. On 
peut aussi trouver les formules reliant ces diff\'erents Laplaciens$\ldots$

\subsubsection{Equations de Maxwell et \'equations de Yang-Mills} \label{sec:YangMillsExplicit}
\index{champ electromagn\'etique\/}
\index{equations de Maxwell}
Les \'equations de Maxwell d\'ecrivent la physique de l'electromagn\'etisme.
Nous avons vu comment \'ecrire la moiti\'e de ces \'equations,
 en l'occurence, les \'equations ``sans terme de source'', $d F = 0$, qui r\'esultent directement de la d\'efinition $F = dA$ du champ electromagn\'etique $F$ en terme du potentiel electromagn\'etique $A$.
Ces \'equations, qui sont en quelque sorte des \'equations structurelles,  n'utilisent pas la notion de m\'etrique. A ce propos, si on souhaite \'ecrire explicitement ces \'equations ``avec des indices'', dans un rep\`ere naturel, on obtiendra indiff\'eremment 
	$$F_{\mu\nu,\rho} + F_{\rho \mu,\nu} + F_{\nu\rho,\mu} = 
        F_{\mu\nu;\rho} + F_{\rho \mu;\nu} + F_{\nu\rho;\mu} =0 $$
	en effet, les coefficients de la connexion lin\'eaire se compensent dans la seconde \'equation, ce qui est {\it a priori\/} bien \'evident, du fait de la d\'efinition de $F$ en termes de $A$, laquelle ne fait aucunement appel \`a la notion de connexion lin\'eaire. De la m\^eme fa\c con, dans un rep\`ere naturel,
	$$ F_{\mu\nu} = A_{\nu,\mu} - A_{\mu,\nu} =  A_{\nu;\mu} - A_{\mu;\nu}$$
	Attention, dans un rep\`ere mobile $\{\theta^\rho\}$, les valeurs de $d\theta^\rho$ ne sont pas nulles puisque les fonctions de structure du rep\`ere ne sont pas nulles (voir section \ref{sec:mobileMaurerCartan}): les coefficients de la connexion
lin\'eaire se compensent comme auparavant, mais il faut tenir compte des fonctions de structure du rep\`ere (cela n'a rien \`a voir avec l'existence, ou non, d'un champ gravitationnel d\'ecrit par la m\'etrique $g$). Puisque $F = dA = 
{1\over 2} F_{\mu\nu} \theta^\mu \wedge \theta^\nu$, on obtient
	$$F_{\mu\nu} =  (A_{\nu,\mu} - A_{\mu,\nu}) - A_\rho {f_{\mu\nu}}^\rho$$

 Les charges \'electriques, quant \`a elles, sont les ``sources'' du champ et ces charges sont d\'ecrites par une  $1$-forme $J = J_\mu dx^\mu$ (le ``vecteur courant''). Le couplage entre champ et charges doit \^etre tel que toutes les \'equations de Maxwell soient v\'erifi\'ees. En dimension $4 = 3+1$ (physique de l'Espace-Temps) le dual de Hodge de $J$ est une $3$-forme et les \'equations de Maxwell ``avec sources'' s'\'ecrivent $$d\star F = \star J $$ Ici, par contre, la m\'etrique intervient explicitement (via l'op\'eration $\star$). De mani\`ere explicite, ces \'equations s'\'ecrivent $${F^{\mu\nu}}_{;\nu}=  J^\mu$$
Notons qu'\`a l'aide de la codiff\'erentielle $\delta$, les \'equations de Maxwell avec sources $d\star F = \star J$ s'\'ecrivent simplement $\delta F = J$.

Le lecteur pourra bien entendu \'ecrire les deux \'equations ``quadridimensionelles''  \`a l'aide des composantes $\overrightarrow E$ et $\overrightarrow B$ de $F$ (les champs \'electriques et magn\'etiques) et \`a l'aide des composantes $\rho$ (la charge) et $\overrightarrow j$ (le courant electrique tridimensionnel) de $J$ et red\'ecouvrir ainsi les quatre \'equations de Maxwell habituelles. On rappelle que
	$$
	F_{\mu\nu} = \begin{pmatrix}  
	0 & -E_x & -E_y & -E_z \\
 	E_x & 0 & B_z & -B_y \\
	E_y & -B_z & 0 & B_x \\
	E_z & B_y & -B_x & 0
	\end{pmatrix} 
	$$
	

\index{equations de Yang-Mills\/}
En ce qui concerne les th\'eories de jauge non ab\'eliennes (par exemple la chromodynamique d\'ecrivant les interactions fortes \'elementaires entre quarks), le potentiel de jauge $A$ et la courbure $F$ sont maintenant des formes \`a valeurs matricielles (l'alg\`ebre de Lie de $SU(3)$ dans le cas de la chromodynamique). Il en est de m\^eme du vecteur courant $J$. Les \'equations de Yang-Mills ``sans sources'' s'\'ecrivent
$$
\mbox{\fbox{$ 
D F = dF - F \wedge A + A\wedge F = 0
$}}$$
Ce sont donc simplement les identit\'es de Bianchi.  Dans un rep\`ere naturel, et en utilisant des indices, notons que
	\begin{eqnarray*}
	F\wedge A - A\wedge F&=&{1\over 2} [F_{\mu\nu},A_\rho] \, dx^\rho \wedge dx^\mu \wedge dx^\nu \cr {} &=& 
	{1\over 3!}( [F_{\nu\rho},A_\mu] +  [F_{\mu\nu},A_\rho]+ [F_{\rho\mu},A_\nu])   \, dx^\rho \wedge dx^\mu \wedge dx^\nu
	\end{eqnarray*}
	et donc
	$$
\mbox{\fbox{$
 \nabla_\mu F_{\nu\rho}+\nabla_\nu F_{\rho\mu} + \nabla_\rho F_{\mu\nu} = 0
$}}$$
	avec 
$$
 \nabla_\rho F_{\mu\nu} = \partial_\rho F_{\mu\nu}+[A_\rho,F_{\mu\nu}]
	$$
 	

	Les \'equations de Yang-Mills ``avec sources'' s'\'ecrivent
 $$
\mbox{\fbox{$
D \star F = \star J
$}} $$
 c'est \`a dire encore, si on utilise les indices, 
$$
\mbox{\fbox{$
\partial_\nu F^{\mu\nu} + [A_\nu,F^{\mu\nu}] = J^\mu
$}}$$

	Notons que, m\^eme lorsque $J = 0$, c'est \`a dire dans ``le vide'',  l'ensemble des  \'equations de Yang-Mills $D F=0$ et $D \star F = 0$ constitue un syst\`eme d'\'equations diff\'erentielles hautement non trivial dont la discussion g\'en\'erale sort du cadre de cet ouvrage.

	
\subsection{Connexions spinorielles et op\'erateur de Dirac}
\subsubsection{Compl\'ement sur l'alg\`ebre de Lie du groupe $Spin$}
Soit $P = OFM$ le fibr\'e des rep\`eres orthonorm\'es correspondant \`a une certaine
m\'etrique $g$ sur une vari\'et\'e orient\'ee $M$, 
$\widehat P = \widehat OFM$, le fibr\'e des rep\`eres spinoriels correspondant (on suppose
qu'il existe --- c'est \`a dire que la vari\'et\'e est spinorielle --- et qu'il
est unique --- c'est \`a dire que la vari\'et\'e poss\`ede une seule structure
spinorielle). 

Nos consid\'erations sont valables pour une signature quelconque $(p,q)$ et on d\'esignera
par $\eta$ la forme diagonale de la m\'etrique $g$, c'est \`a dire
la matrice donnant l'expression de la m\'etrique $g$ dans un rep\`ere orthonorm\'e,  
ainsi,  $\eta_{ab}=\pm 1$. Pour
simplifier on posera $Spin(\eta)=Spin(p,q)$. Rappelons que nous avons \'egalement
une inclusion $Spin^\uparrow(\eta) \subset Spin(\eta)$ et que l'existence d'une structure
$spin^\uparrow$ sur une vari\'et\'e implique non seulement que $M$ est orient\'ee mais
encore orient\'ee temporellement (le ``temps'' pouvant avoir plusieurs directions).

Soit $V$ un espace vectoriel r\'eel, complexe (ou m\^eme quaternionique) sur 
lequel est donn\'ee une rep\'esentation $\rho$  de l'alg\`ebre de Clifford
$Cliff(\eta)$.
On sait (voir la section consacr\'ee aux groupes $Spin$) que $Spin(\eta)$ est un sous-ensemble de
$Cliff(\eta)$, on a donc ainsi automatiquement une repr\'esentation, encore not\'ee $\rho$, du
groupe $Spin(\eta)$. Nous pouvons \'egalement, dans le cas pair,
parler de spineurs de Weyl (demi-spineurs) mais il n'est pas utile d'\'etablir ici une
distinction entre la droite et la gauche ni d'ailleurs de mentionner les 
propri\'et\'es de r\'ealit\'e des spineurs consid\'er\'es. 
Nous travaillerons donc avec des spineurs d\'efinis comme section
d'un fibr\'e vectoriel $SM = \widehat P \times_{Spin(\eta)} V$.

On choisit un ensemble de g\'en\'erateurs $\{\gamma_a\}$ de $Cliff(n)$, avec
$$
\mbox{\fbox{$
\gamma_a \gamma_b + \gamma_b \gamma_a = 2\, \eta_{ab}
$}}
$$
Soit $g \in Spin(n)$, on sait que $\rho(g)\gamma_a \rho(g^{-1}) = \gamma_b \Lambda(g)^b_a,$
o\`u $\rho(g)$ est la repr\'esentation du groupe $Spin(\eta)$ et $\Lambda$ est celle
du groupe orthogonal $SO(\eta)$ correspondant (le groupe de Lorentz).
Nous avons besoin de savoir \'ecrire explicitement la repr\'esentation $\rho$ pour
 l'alg\`ebre $Lie(SO(\eta))=Lie(Spin(\eta))$. Soit $M^{ab}$ l'ensemble des matrices
antisym\'etriques en dimension $n$, c'est \`a dire $M^{ab} + M^{ba} = 0$.
On sait que $Lie(SO(\eta)$ est engendr\'ee par les matrices $M$ du type
$M_{ab} = \eta_{ac}\, M^{cb}.$
Notons encore 
$$
\rho(M)  =  {1\over 8} M^{ab}[\gamma_a,\gamma_b] = {1\over 2} M^{ab}\Sigma_{ab}
$$
avec $$\Sigma_{ab}  =  {1/4} [\gamma_a,\gamma_b]$$
On v\'erifie imm\'ediatement que
$[\rho(M1),\rho(M2)]=\rho([M1,M2])$ et que $[\rho(M),\gamma_a]=\gamma_b M^b_a.$
On voit donc que $\rho$, d\'efini comme ci-dessus, nous fournit la repr\'esentation de
$Lie(Spin(\eta)$ cherch\'ee.

\subsubsection{Connexion spinorielle}
Une {\sl connexion spinorielle\/} est une connexion principale sur le fibr\'e $\widehat P$.
Il existe une bijection entre l'ensemble des connexions sur le fibr\'e $P$ des rep\`eres
orthonorm\'es et l'ensemble des connexion sur le fibr\'e des rep\`eres spinoriels. En effet,
si $\omega$ est une connexion sur $P$, si $\lambda$ d\'esigne l'homomorphisme d'espaces fibr\'es 
$\widehat e \in \widehat P \rightarrow e \in P,$ et si  $V_{\widehat e} \in T(\widehat P, \widehat e)$, on 
obtient une connexion $\widehat \omega$ sur $\widehat P$ en posant
$$\widehat \omega(V_{\widehat e})  =  \omega(V_e) \; \hbox{ \ avec \ } 
V_e  =  \lambda_*(V_{\hat e})$$
La connexion m\'etrique $\omega$ s'\'ecrivait localement \`a l'aide de la forme $\Gamma$,
c'est \`a dire  explicitement \`a l'aide de la matrice
de connexion ${\Gamma^a}_b = {\Gamma^a}_{b\mu} dx^\mu$. En utilisant la repr\'esentation
trouv\'ee pr\'ec\'edemment pour $Lie(Spin(\eta))$, nous obtenons 
$$\widehat \Gamma = \rho(\Gamma) = {1\over 8} \Gamma^{ab}\; [\gamma_a,\gamma_b]
$$
Posons \fbox{$\gamma^a  =  \eta^{ab} \gamma_b$}, nous obtenons,
$$
\mbox{\fbox{$
\widehat \Gamma = {1\over 8} \Gamma_{ab} [\gamma^a,\gamma^b] 
= {1\over 4} \Gamma_{ab} \gamma^a\gamma^b 
$}}
$$
L'op\'erateur de courbure peut \^etre obtenu de la m\^eme fa\c con. On part de l'op\'erateur
de courbure \'ecrit comme endomorphisme $R = ({R^a}_b)$ et on obtient :
$$
\mbox{\fbox{$
\widehat R =  {1\over 4} R_{ab} \gamma^a\gamma^b
$}}
$$
avec 
$$
R_{ab} = {1\over 2}R_{ab\mu\nu} dx^\mu\wedge dx^\nu
$$
Si on d\'ecide de n'utiliser qu'un seul rep\`ere mobile orthonorm\'e $e_a$, ainsi
que le co-rep\`ere mobile correspondant $e^a$, il vient
$$
R_{ab} = {1\over 2}R_{abcd} e^c e^d
$$
Enfin, mentionnons les deux formules bien utiles suivantes, donnant
le tenseur de Ricci $\rho$ et la courbure scalaire $\tau$. Elles 
s'obtiennent en utilisant simplement les relations de commutations des 
g\'en\'erateurs de l'alg\`ebre de Clifford :
$$R_{abcd}  \gamma^d \gamma^a  \gamma^b = 2 \rho_{ce}  \gamma^e$$
$$R_{abcd}  \gamma^c \gamma^d \gamma^a  \gamma^b = -2 \tau $$

\subsubsection{D\'eriv\'ee covariante sur les champs de spineurs}
En vertu de l'\'etude g\'en\'erale sur les diff\'erentielles et d\'eriv\'ees 
covariantes, on voit imm\'ediatement que si $\Psi$ est un champ de spineurs 
(une section du fibr\'e vectoriel $SM$), sa diff\'erentielle covariante est 
donn\'ee par
$$
\mbox{\fbox{$
\nabla \Psi = d \Psi + \widehat \Gamma \Psi = d \Psi + {1 \over 4} 
\Gamma_{ab}\gamma^a \gamma^b \Psi
$}}
$$
La quantit\'e $\nabla \Psi$ est un spineur-$1$-forme, et son \'evaluation, sur 
un vecteur tangent $v$ \`a $M$, c'est \`a dire la d\'eriv\'ee covariante de $\Psi$ dans 
la direction $v$ est le champ de spineurs $\nabla_v \Psi  =  \nabla \Psi (v)$.
Un spineur pour lequel $\nabla \Psi = 0$ est un {\sl spineur parall\`ele}.

\subsubsection{L'op\'erateur de Dirac}
Soit $(e_a)$ un rep\`ere mobile orthonorm\'e (en g\'en\'eral local), on peut 
construire les d\'eriv\'ees covariantes $\nabla_a \Psi  =  \nabla_{e_a} \Psi$
et on d\'efinit l'{\sl op\'erateur de Dirac\/} ${\cal D}$ par :
$$
\mbox{\fbox{$
{\cal D}\Psi  =  \gamma^a \nabla_a \Psi
$}}
$$
On voit que l'op\'erateur de Dirac transforme \'egalement les champs de spineurs en 
d'autres champ de spineurs. Ceux pour lesquels ${\cal D}\Psi = 0$ sont 
appel\'es {\sl spineurs harmoniques \/}.
Par ailleurs, le carr\'e $\widehat \Delta = {\cal D}^2$ s'appelle le {\sl 
laplacien spinoriel\/} et on peut v\'erifier que sa relation avec le laplacien na\"\i f 
$-\eta^{ab}\nabla_a\nabla_b$ est donn\'ee par la formule
$$
\mbox{\fbox{$
\widehat { \Delta } = -\eta^{ab}\nabla_a\nabla_b + {1\over 4}\, \tau
$}}
$$
$\tau$ \'etant la courbure scalaire.

 L'op\'erateur de Dirac est sans doute l'op\'erateur diff\'erentiel le plus
important de la g\'eom\'etrie (et de la physique !) Ne pouvant y consacrer 
plus de place ici, nous renvoyons le lecteur \`a des ouvrages (ou articles) 
sp\'ecialis\'es (le r\'esum\'e qui pr\'ec\`ede s'inspire de \cite{RC-AJ}).

\subsubsection{Le fibr\'e de Clifford}
On part du fibr\'e principal $\widehat P$ des rep\`eres spinoriels et on utilise 
l'action adjointe du groupe structural $Spin(\eta)$ sur l'alg\`ebre de 
Clifford $Cliff(\eta)$ (voir plus haut) pour construire le fibr\'e associ\'e
en alg\`ebres de Clifford $Cliff(M)  =  \widehat P \times_{Spin(\eta)} Cliff(\eta).$
On peut d\'efinir une inclusion $\gamma$ du fibr\'e tangent $TM$ dans le 
fibr\'e de Clifford $Cliff(M)$ et on note $\gamma_\mu  =  
\gamma({\partial \over  \partial x^\mu}).$ Plus simplement, on peut 
consid\'erer les $\gamma_\mu$ comme des ``matrices gamma'' d\'ependant du 
point $x \in M$. Soit $\{e_a\}$ un rep\`ere mobile orthonorm\'e et 
$\partial_\mu$ un rep\`ere naturel, alors $\partial_\mu = e^a_\mu e_a$.
On peut alors utiliser la matrice $e^a_\mu$  (qui d\'epend de $x$) pour 
d\'efinir $\gamma_\mu(x) = e^a_\mu(x) \gamma_a.$ En posant 
$\gamma^\mu = g^{\mu\nu} \gamma_\nu$, on v\'erifie imm\'ediatement que
$\gamma^\mu \gamma^\nu  +  \gamma^\nu \gamma^\mu = 2 g^{\mu \nu}$ et que 
l'op\'erateur de Dirac s'\'ecrit ${\cal D} = \gamma^\mu \nabla_\mu.$
 
 Remarque : Il existe un sous-fibr\'e de $Cliff(M)$ dont les fibres sont des
 groupes spinoriels, puisque $Spin(\eta) \subset Cliff(\eta)$. Ce 
 sous-fibr\'e ne co\"incide pas avec le fibr\'e principal $\widehat P$ des 
 rep\`eres spinoriels puisque l'action du groupe $Spin(\eta)$ sur lui-m\^eme, 
 d\'efinie par restriction, est l'action adjointe. On en d\'eduit que le 
 sous-fibr\'e en question n'est autre que le fibr\'e adjoint $Ad \widehat P$.
 C'est un fibr\'e en groupes, mais il n'est pas principal. On sait que
 ses sections ne sont autres que les transformations de jauge du fibr\'e 
 principal  $\widehat P$ dont on est parti.


%%%%%%%%%%%%%%%%%
%%%%%%%%%\subsection{Les pseudo tenseurs}(Remarques)
%%%%%%%%%$TODO***$
%%%%%%%%%\subsection{Connexions induites par des applications harmoniques (mod\`eles 
%%%%%%%%%%$\sigma$)} (Remarques)
%%%%%%%%%$TODO***$

\subsection{M\'etriques sur les groupes et espaces homog\`enes}
Il n'entre pas dans nos intentions d'\'etudier ici ce vaste chapitre de la 
g\'eom\'etrie riemannienne. Nous voulons simplement attirer l'attention du 
lecteur sur le fait suivant : un groupe de Lie donn\'e poss\`ede en g\'en\'eral 
une infinit\'e de m\'etriques. Si, pour simplifier,
le groupe est simple et compact, il existe une m\'etrique particuli\`erement ``sym\'etrique''  
d\'esign\'ee g\'en\'eralement sous le nom de {\sl m\'etrique de Killing\/}
 {\index{m\'etrique de Killing}}. Comme 
tous ses multiples, elle est 
bi-invariante, en ce sens que son groupe d'isom\'etries est $G \times G$. 
Par contre il existe des m\'etriques invariantes par $G\times {1}$, des 
m\'etriques invariantes par ${1}\times G$, des m\'etriques invariantes par $H 
\times K$ o\`u $H$ et $K$ sont des sous-groupes de $G$, mais aussi des 
m\'etriques n'ayant aucune isom\'etrie particuli\`ere$\ldots$
Par ailleurs, il est int\'eressant de comparer la connexion riemannienne
(sans torsion) induite par la m\'etrique de Killing sur un groupe de Lie avec la connexion 
canonique d\'efinie \`a l'aide de la forme de 
Maurer Cartan (celle de gauche ou celle de droite)
qui g\'en\'eralement a de la torsion mais est de courbure nulle (revoir
l'\'equation de Maurer-Cartan sur les groupes de Lie). L'\'etude des 
m\'etriques invariantes sur les espaces homog\`enes est \'egalement un chapitre 
important de la g\'eom\'etrie riemannienne, et l\`a encore, nous renvoyons au 
trait\'e \cite{Kobayashi-Nomizu}.
%%%%%%%%%%%%%%%%%%%%%%%%%%%%%%%%%%%%%%%%%%%%%%%
%%%%%\subsection{Fibr\'es d'ordre sup\'erieur, jets, et diff\'erentielles de Leibniz}
%%%%%%%%%%%%%%%%%%$TODO***$
%%%%%%%%%%%%%%%%%%%%%%%%%%%%%%%%%%%%%%%%%%%%%%%%%%%%%%%%%%%%%%%
