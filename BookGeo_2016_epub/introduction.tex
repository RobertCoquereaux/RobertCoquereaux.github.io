
\chapter*{Introduction}
\addcontentsline{toc}{chapter}{Introduction}
\section*{\sl Mod\`eles math\'ematiques et r\'ealit\'e}
\addcontentsline{toc}{section}{Mod\`eles math\'ematiques et r\'ealit\'e}

Qu'est ce que la ``R\'ealit\'e''? Existe-t-elle seulement? Que
signifie le verbe ``exister'' de la proposition
interrogative pr\'ec\'edente? Que le lecteur allergique aux
discussions philosophiques se rassure, nous n'allons pas
continuer longtemps dans cette direction. Cependant, pour
ne pas nous enliser dans de faux probl\`emes s\'emantiques et
pour bien appr\'ecier en quel sens nous comprenons ou
pr\'etendons comprendre les ph\'enom\`enes naturels (y en a-t-il
qui ne le soient pas?) il nous faut apporter une r\'eponse
pragmatique aux questions pr\'ec\'edentes et tenter de d\'efinir
les mots eux-m\^emes que nous utilisons.\par
Le point de vue adopt\'e par l'auteur est le
suivant:\par

$\bullet$ Il est impossible de donner une signification
quelconque \`a la phrase suivante: La R\'ealit\'e est.
L'auteur croit cependant en l'existence d'une r\'ealit\'e
objective dont la nature est ind\'ependante de l'analyse
qui peut en \^etre faite. Malheureusement, il s'av\`ere
\'egalement impossible de donner un sens raisonnable \`a
l'assertion pr\'ec\'edente. La croyance de l'auteur est donc
un acte de foi au sens m\'etaphysique du terme. On pourra
donc utiliser le mot ``ph\'enom\`ene'' comme
synonyme du mot ``r\'ealit\'e'', le vocable en question \'etant
lui-m\^eme non d\'efini. \par

$\bullet$ La description d'un ph\'enom\`ene, quel qu'il soit,
fait toujours appel aux math\'ematiques, m\^eme si le spectateur n'en est
pas conscient. Ainsi, d\'eclarer que deux individus font
partie de la m\^eme lign\'ee (au sens h\'er\'editaire du terme)
signifie qu'on assimile --peut \^etre inconsciemment-- les
individus en question aux \'el\'ements d'un ensemble sur
lequel on a d\'efini une  relation d'ordre partiel.
De la m\^eme fa\c con, la travers\'ee d'un terrain par un ballon
de foot-ball est un ph\'enom\`ene admettant une description
(en fait plusieurs) dont la nature est essentiellement
math\'ematique. Par exemple, on peut consid\'erer la
trajectoire d'un point traversant un rectangle en ligne
droite. Il existe cependant une description du m\^eme
ph\'enom\`ene ou le ballon n'est plus un point mais une sph\`ere
et ou le terrain n'est plus assimil\'e \`a un rectangle mais
une figure g\'eom\'etrique plus complexe (coins plus ou moins
arrondis, c\^ot\'es plus ou moins parall\`eles etc.) On peut
d'ailleurs continuer dans ce sens et tenir en compte
l'existence de creux et de bosses sur la surface du
ballon, de la couleur etc. Les humains n'ont pas besoin
de suivre des cours de math\'ematiques sup\'erieures pour
appr\'ecier un match de foot-ball, mais il est important de
constater l'aptitude de l'esprit \`a cr\'eer
inconsciemment des mod\`eles math\'ematiques relativement
\'elabor\'es pour analyser l'exp\'erience quotidienne. Notons
enfin qu'un ph\'enom\`ene donn\'e poss\`ede d'ordinaire plusieurs
descriptions math\'ematiques (et m\^eme une infinit\'e).\par

$\bullet$ La croyance en l'existence d'une r\'ealit\'e objective
n'a aucune importance pratique; seule compte
l'ensemble de ses descriptions math\'ematiques.
 En effet, lors de l'analyse d'un
ph\'enom\`ene (la travers\'ee de la cour par un ballon de
foot-ball), nous pouvons adopter les deux points de vue
suivants. 1) La travers\'ee de la dite cour par le ballon en
question est un ph\'enom\`ene ``r\'eel'' dont nous pouvons
donner une quantit\'e de descriptions math\'ematiques
compatibles, et il est d'ailleurs possible de pr\'eciser la notion de
compatibilit\'e des descriptions. 2) La travers\'ee de la
dite cour par le ballon en question est en fait {\sl
d\'efinie} par un ensemble (infini) de descriptions
math\'ematiques compatibles. Peu importe que nous adoptions
l'un ou l'autre de ces deux points de vue, car si un
aspect d'un ph\'enom\`ene n'est pas math\'ematiquement
mod\`elisable, cet aspect rel\`eve --presque par d\'efinition-- de la
m\'etaphysique et il n'est pas clair qu'on puisse y
attribuer un sens (m\^eme si on a envie de croire sans
comprendre). On peut se convaincre du fait que l'exercice classique
de m\'editation sur le th\`eme de la chaise (Quelle est
cette chaise? Quelle est sa fonction? Quelle est sa nature?
Quelle est son histoire? etc.) est compl\`etement
mod\`elisable en termes math\'ematiques...\par


Pour nous, un ph\'enom\`ene est donc d\'efini par l'ensemble de ses
descriptions math\'ematiques. Du point de vue linguistique, on
devrait peut-\^etre distinguer en g\'en\'eral le ph\'enom\`ene lui-m\^eme
(concept assez flou) de sa description math\'ematique -- ou plut\^ot,
de ses descriptions math\'ematiques. On peut alors parler de
mod\'elisation du ph\'enom\`ene, mais il faut bien voir que c'est la
mod\'elisation elle-m\^eme qui rend le ph\'enom\`ene accessible \`a
l'analyse. Le mod\`ele math\'ematique, qu'il soit choisi consciemment
(par un physicien, par exemple) ou inconsciemment (par exemple, par un
spectateur du match) apporte avec lui son propre langage, c'est \`a
dire les mots qui permettent \`a l'observateur de se poser des
questions \`a propos du ph\'enom\`ene qu'il contemple. Chacun de ces
mots est cens\'e \^etre susceptible d'une traduction math\'ematique
pr\'ecise dans un cadre formel --- que l'observateur ne d\'efini pas
n\'ecessairement --- faute de quoi, les mots en question sont simplement
vides de sens. Il faut bien \^etre conscient du fait que la phrase
``mais que se passe-t-il vraiment?'' pos\'ee par le profane
repose sur la croyance en une r\'ealit\'e objective, r\'ealit\'e qui,
de notre point de vue, \'echappe \`a toute analyse scientifique.\par
Qu'en est-il donc de la distinction entre physique et
math\'ematiques? Pour nous, dire qu'une figure dessin\'ee sur une
feuille de papier est un triangle, c'est ``faire de la physique'': le
triangle est une notion abstraite appartenant au
monde des math\'ematiques, associer cette notion au dessin qu'on
a sous les yeux est un travail de physicien. Dans un genre
diff\'erent, supposons qu'on fabrique des ``choses'' avec un canon
\`a \'electrons$\ldots$ qu'est ce donc qu'un \'electron? On peut dire
que c'est une petite boule, on peut dire que c'est une fonction
(complexe) --une onde~!--, on peut dire que c'est une section d'un
certain espace fibr\'e vectoriel (un ``champ de Dirac'') ou que c'est un
\'el\'ement d'un module projectif de type fini sur une alg\`ebre non
n\'ecessairement commutative$\ldots$ Toutes ces descriptions sont
math\'ematiques et la premi\`ere (la boule) est la plus simple du
point de vue du bagage math\'ematique utilis\'e mais toutes ces
descriptions sont \'egalement ``vraies'' et apportent avec elles
leur propre langage. Il y a des questions qu'on ne peut poser
qu'apr\`es avoir {\it choisi} une certaine description. C'est ainsi
que les math\'ematiques sont n\'ecessaires \`a la description de ce
que nous appelons les ph\'enom\`enes naturels (cons\'equence
imm\'ediate: si vous avez des difficult\'es en physique, c'est que
vous n'avez pas proprement assimil\'e les math\'ematiques
n\'ecessaires!). La physique consiste essentiellement \`a habiller le
ph\'enom\`ene de notre choix avec des math\'ematiques appropri\'ees et
c'est cet habillage qui rend les choses accessibles au discours.
C'est l\`a quelque chose qu'il ne faut pas oublier mais il faut
avouer qu'il est n\'eanmoins commode de vivre en faisant ``comme si'' on
croyait \`a l'existence d'une r\'ealit\'e objective! On pourrait
aussi passer au cran sup\'erieur et se demander si les
math\'ematiques elles-m\^emes ``existent''. Il n'est pas clair que la
phrase ait un sens mais il est certain que, de la m\^eme fa\c con
qu'il est commode de croire en l'existence d'une r\'ealit\'e physique
objective, il est \'egalement commode de  croire en l'existence d'une
r\'ealit\'e math\'ematique qu'il s'agit pour nous de d\'ecouvrir
(comme un explorateur dans la jungle ou comme un physicien
exp\'erimentateur).
Les chapitres qui suivent pr\'esentent des concepts math\'ematiques.
Ind\'ependamment de la beaut\'e ou de l'\'el\'egance intrins\`eque
des concepts en question, nous voulons attirer l'attention du
lecteur (m\^eme s'il n'est pas physicien) sur le fait que ces
concepts jouent un r\^ole majeur dans l'``habillage'' contemporain des
th\'eories physiques,  et que, dans de nombreux cas, ces concepts
sont eux-m\^emes issus de consid\'erations relevant de la physique
th\'eorique.

\section*{Du classique au quantique: math\'ematiques commutatives et non commutatives}
\addcontentsline{toc}{section}{Du classique au quantique}

 Avant d'arr\^eter l\`a ces consid\'erations
\'epist\'emologiques pour passer \`a notre premier chapitre
consacr\'e \`a l'\'etude des vari\'et\'es diff\'erentiables, nous
voulons dire un mot sur la distinction entre physique classique et
physique quantique, en parall\`ele avec la distinction entre 
``math\'ematiques commutatives'' et ``math\'ematiques non
commutatives''. Cette remarque risque de n'\^etre comprise que par
les lecteurs ayant d\'ej\`a une certaine familiarit\'e avec les
sujets mentionn\'es mais le lecteur int\'eress\'e pourra peut-\^etre
relire ce commentaire en y revenant un peu plus tard.
Les math\'ematiques commutatives (la g\'eom\'etrie commutative en
particulier) s'occupe des propri\'et\'es math\'ematiques des
``espaces'' (th\'eorie de la mesure, espaces topologiques,
diff\'erentiables, riemanniens, homog\`enes, poss\'edant une
structure de groupe$\ldots$) Pour le physicien, ces espaces
fournissent un mod\`ele math\'ematique concernant le syst\`eme qu'il a
choisi d'\'etudier et toutes les quantit\'es qui l'int\'eressent
peuvent \^etre d\'ecrites \`a l'aide d'une classe appropri\'ee de
fonctions d\'efinies sur de tels espaces. Il se trouve que les
propri\'et\'es des espaces en question peuvent elles-m\^emes \^etre
cod\'ees en termes des propri\'et\'es de ces alg\`ebres de
fonctions; il s'agit l\`a d'un r\'esultat profond dont l'expression
pr\'ecise est due \`a Gelfand (voir chapitre 6). Le vocable ``math\'ematiques
commutative'' vient du fait que toutes ces alg\`ebres sont des
alg\`ebres commutatives pour les lois d'addition et de multiplication
des fonctions. Attention, de ce point de vue, la th\'eorie des
groupes de Lie (voir plus loin) --groupes qui ne sont pas, en
g\'en\'eral, commutatifs-- fait partie des ``math\'ematiques
commutatives'' car l'alg\`ebre des fonctions (\`a valeurs r\'eelles
ou complexes) d\'efinie sur un groupe est une alg\`ebre commutative !
Les ``math\'ematiques non commutatives'', au contraire, s'occupent des
propri\'et\'es d'alg\`ebres qui ne sont pas commutatives et des
objets qui g\'en\'eralisent les constructions usuelles lorsqu'on
remplace les alg\`ebres de fonctions (et les ``espaces'' eux-m\^emes)
par des alg\`ebres d'op\'erateurs. Les quantit\'es qui int\'eressent
le physicien ne sont plus alors cod\'ees par des fonctions
num\'eriques mais, typiquement, par des op\'erateurs agissant dans
des espaces hilbertiens. Il est inutile d'en dire plus \`a ce niveau
mais nous effectuerons deux remarques. La premi\`ere est
terminologique: un physicien dit qu'il fait de la physique classique
lorsqu'il utilise des math\'ematiques commutatives pour d\'ecrire un
ph\'enom\`ene (ce qui, philosophiquement, revient \`a le d\'efinir !
Voir la discussion pr\'ec\'edente)  et de la physique quantique
lorsqu'il utilise des math\'ematiques non commutatives (idem).
 La seconde remarque a trait au contenu de cet ouvrage: il
traite de g\'eom\'etrie, et la plupart du temps de g\'eom\'etrie
utilis\'ee en physique fondamentale, cependant il s'agira presque
toujours de g\'eom\'etrie commutative, vocable englobant d'ailleurs
toute la g\'eom\'etrie, au sens usuel du terme, qu'elle soit
euclidienne ou non. Du point de vue de la physique, nos
constructions correspondront donc \`a des constructions de th\'eorie
classique des champs (m\^eme s'il nous arrive de parler de quarks ou
d'\'electrons de Dirac) et non de th\'eorie quantique des champs.
\par
Le dernier chapitre est une introduction aux ``math\'ematiques non 
commutatives'' (un point de vue assez particulier sur la th\'eorie 
des alg\`ebres associatives) et pr\'esente  
quelques notions fondamentales relevant de le g\'eom\'etrie 
diff\'erentielle non commutative. Ce dernier chapitre pourrait donc 
aussi s'intituler : Introduction \`a la {\sl g\'eom\'etrie quantique\/}.
\vfill
\eject

\section*{Guide de lecture, autocritique et perspectives}
\addcontentsline{toc}{section}{Guide de lecture, autocritique et perspectives}

Le lecteur ne connaissant rien au sujet et d\'esirant ``se faire une id\'ee'',
peut dans un premier temps,
parcourir les sections 1.1, 1.2, 2.1, 2.2.1, ainsi que
3.1, (3.2.1 $\rightarrow$ 
3.2.5), 3.3.1, 3.3.2,  (4.1.1 $\rightarrow$ 4.1.4)  et (4.4.1, 5.6.1, 6.1, 6.2.1, 
6.3.1) dont le contenu, 
\`a peu pr\`es exempt de 
formules, fait appel \`a 
l'intuition et ne suppose que tr\`es peu de connaissances 
pr\'ealables.

\bigskip
Les autres sections sont assez in\'egales ; certaines pr\'esentent un mat\'eriel 
qui fait ou devrait faire partie du
bagage math\'ematique standard de tout math\'ematicien ou physicien th\'eoricien,
certaines autres sont d'un niveau plus avanc\'e et peuvent contenir des 
informations qui ne sont pas n\'ecessairement disponibles ailleurs (sauf 
peut-\^etre dans quelques articles sp\'ecialis\'es).
En fait, comme le titre l'indique, le but initial de ce travail \'etait de
fournir une pr\'esentation  --- si possible p\'edagogique --- des espaces
fibr\'es et de la th\'eorie des connexions. Il se trouve que certains lecteurs
potentiellement int\'eress\'es, en particulier les \'etudiants de
troisi\`eme cycle de physique th\'eorique, n'ont souvent pas, au d\'epart, les 
bases math\'ematiques n\'ecessaires pour attaquer, de front, un cours 
relativement complet sur les espaces fibr\'es : il leur manque souvent un
cours pr\'ealable de calcul diff\'erentiel sur les vari\'et\'es et un cours sur
les groupes de Lie. C'est la raison d'\^etre des parties $1$ et $2$ de cet
ouvrage. On a essay\'e d'y pr\'esenter les notions indispensables \`a la
lecture des chapitres $3$ et $4$ consacr\'es aux espaces fibr\'es et 
\`a la th\'eorie des connections. Nous sugg\'erons donc \`a ceux qui ont d\'ej\`a
acquis une formation raisonnable en ce qui concerne les vari\'et\'es 
diff\'erentiables (par exemple en lisant
le premier volume de \cite{Spivak}
et les groupes de Lie, de jeter d'abord un coup d'\oe{il} au
sommaire, puis de sauter les deux premiers chapitres --- qui
ne leur apprendront sans doute pas grand chose --- et 
d'entamer directement la lecture de cet ouvrage au chapitre $3$.
Pour les autres$\ldots$ il vaudrait peut-\^etre mieux 
s'astreindre \`a lire les diff\'erentes parties dans l'ordre.
Comme nous l'avons mentionn\'e dans la pr\'eface,
l'ensemble de l'ouvrage devrait
\^etre lisible par  quelqu'un ne disposant pas d'un bagage math\'ematique
sup\'erieur \`a celui qu'on acquiert d'ordinaire, ou qu'on est sens\'e acqu\'erir,  en premier cycle.
Son contenu, n\'eanmoins, serait plut\^ot d'un niveau $3^{\grave{e}me} 
\, cycle$. 

Le plan et la structure de ce livre r\'epond \`a la pr\'eoccupation suivante : 
faire du lecteur un ``honn\^ete homme'' en g\'eom\'etrie diff\'erentielle 
classique en pr\'esentant un certain nombre de notions qui sont 
fr\'equemment utilis\'ees en physique th\'eorique ou en math\'ematiques. Savoir si le but sera atteint est une 
autre histoire$\ldots$ Enfin, et au risque de faire hurler certains
math\'ematiciens, il nous semble plus important, tout au moins dans un premier temps, de se familiariser avec les 
id\'ees fondamentales ainsi qu'avec de nombreux exemples, que de conna\^itre
le d\'etail de toutes les d\'emonstrations relatives aux propositions et 
th\'eor\`emes cit\'es. 


Le style adopt\'e dans ce livre \'etant volontairement informel, il 
peut \^etre parfois difficile au lecteur de retrouver la d\'efinition 
pr\'ecise de tel ou tel concept. Pour cette raison, il peut \^etre 
utile de consulter l'index situ\'e en fin d'ouvrage, et, bien 
entendu, la table des mati\`eres.

Notre pr\'esentation est bien, sur, incompl\`ete. Certains aspects ne sont qu'effleur\'es,
d'autres sont totalement absents et bien qu'il ne s'agisse pas 
ici, loin s'en faut, d'une tentative encyclop\'edique, voici
quelques t\^etes de chapitres dont
on pourra d\'eplorer l'absence$\ldots$ :
compl\'ements de g\'eom\'etrie diff\'erentielle \'el\'ementaire en 
basse dimension (la liste serait longue),
g\'eom\'etrie symplectique et m\'ecanique, 
op\'erateurs diff\'erentiels, pseudo-diff\'erentiels, symboles etc.,
\'etude des \'equations de Yang-Mills, instantons etc.,
classification des espaces fibr\'es, fibr\'es universels et espaces classifiants,
K-th\'eorie,
classes caract\'eristiques (et classes caract\'eristiques secondaires),
g\'eom\'etries sur les groupes de Lie et les espaces 
homog\`enes,
applications harmoniques,
aspects conformes,
m\'etriques et connexions invariantes (sym\'etries, isom\'etries),
vari\'et\'es complexes, hypercomplexes etc.,
g\'eom\'etrie de l'espace des orbites des connexions,
g\'eom\'etrie de l'espace des m\'etriques,
etc.


Par ailleurs, l'auteur aurait aim\'e ins\'erer, \`a la fin de chaque 
chapitre, une section consacr\'ee aux g\'en\'eralisations des id\'ees
rencontr\'ees, lorsqu'on passe de la g\'eom\'etrie commutative \`a la g\'eom\'etrie
non commutative, c'est \`a dire lorsqu'on passe du classique au quantique\footnote{Mentionnons
\`a ce propos le trait\'e \cite{ACbook} qui restera sans
doute pour longtemps l'ouvrage de r\'ef\'erence en g\'eom\'etrie non
commutative.}.
Il est sans doute dommage de devoir parler au conditionnel pass\'e$\ldots$
mais il fallait bien mettre fin \`a la r\'edaction ! 
De fait, faisant suite \`a une premi\`ere version de cet ouvrage, rendue 
disponible sur Internet, en format html, en mai 1997, la derni\`ere 
section (section 6), consacr\'ee \`a une pr\'esentation g\'en\'erale 
des math\'ematiques non commutatives et au calcul diff\'erentiel sur 
les alg\`ebres non commutatives, a \'et\'e rajout\'ee  en mars 1998. Ce 
rajout r\'epond donc, en partie, \`a la pr\'eoccupation mentionn\'ee plus haut.

 Bien entendu, toutes les remarques permettant d'am\'eliorer ce 
 document, voire de corriger certaines sections si besoin est, sont les 
 bienvenues : envoyer un courrier \`a l'auteur ou un courriel 
\`a coque at cpt.univ-mrs.fr.

On aura compris que ce livre a \'et\'e r\'edig\'e en fran\c cais. 
Certes, il eut \'et\'e pr\'ef\'erable, pour rassembler un plus large lectorat, de r\'ediger directement l'ouvrage en anglais.
L'usage de la langue anglaise, et en particulier la lecture de l'anglais, sont devenus obligatoires dans notre soci\'et\'e, et il est certain que l'enseignement de cette langue a fait des progr\`es consid\'erables en France; il n'en demeure pas moins que la lecture de textes en anglais, m\^eme de textes scientifiques, pose toujours \`a nos \'etudiants ainsi qu'\`a certains de leurs a\^in\'es, des difficult\'es.
Il en va d'ailleurs de m\^eme pour une vaste partie du monde francophone, o\`u le fran\c cais n'est parfois qu'une deuxi\`eme langue (l'anglais venant en troisi\`ieme position).
Il n'est pas certain que le pr\'esent ouvrage devienne un livre de chevet (!) mais pour faciliter sa lecture sans rajouter une difficult\'e linguistique, l'auteur a d\'ecid\'e de r\'ediger ce livre directement en fran\c cais. 
Ces notes sont donc  d\'edi\'ees aux \'etudiants et aux chercheurs de la francophonie, et \`a tous les esprits curieux qui souhaitent acqu\'erir un certain nombre de notions g\'eom\'etriques des math\'ematiques contemporaines avant de s'embarquer eux-m\^emes dans l'aventure de la recherche, que ce soit en Physique ou en Math\'ematique, ou qui souhaitent tout simplement satisfaire leur curiosit\'e intellectuelle.
Une version anglaise serait la bienvenue mais l'auteur n'a pas eu,  jusqu'\`a pr\'esent, le courage de s'atteler \`a cette t\^ache.


\section*{\sl Disponibilit\'e}

L'ensemble du document (sa derni\`ere version) est accessible,
via internet, en version html ou pdf sur \url{http://www.cpt.univ-mrs.fr/~coque/}.
% via internet, en version html ou postscript sur $http://www.cpt.univ-mrs.fr/\tilde{}coque/$.
% et est \'egalement  disponible, en version papier, 
% sous forme d'un rapport CNRS reli\'e (contacter Mme M. Rossignol, 
% CPT-CNRS, case 907, Luminy, 13288, Marseille, ou, par e-mail
% michele at cpt.univ-mrs.fr).

\vfill
\eject

\section*{\sl Bibliographie}

On trouvera assez peu de r\'ef\'erences mentionn\'ees dans
cet ouvrage. Il existait \'evidemment la tentation de citer tous les
livres traitant, de pr\`es ou de loin, de g\'eom\'etrie diff\'erentielle,
d'espaces fibr\'es, de connexions, de  g\'eom\'etrie riemannienne \etc.
Un tel effort bibliographique semble \'evidemment, d\`es le d\'epart, vou\'e \`a
l'\'echec. Une autre solution e\^ut \'et\'e de ne citer que les ouvrages
\'el\'ementaires. Malheureusement, les ouvrages en question
 ne recouvrent pas
 n\'ecessairement tous les sujets qui sont abord\'es ici.
 Enfin, on rappelle que la premi\`ere r\'edaction de ces notes, avant leur mise \`a disposition sur internet, date de 1996; 
 plusieurs ouvrages d'enseignement sur des sujets voisins sont apparus depuis.
 L'attitude que nous avons choisi d'adopter est de ne citer que les livres
 et autres travaux pour lesquels l'auteur a conscience d'avoir subi
 une influence possible ou certaine. 
 Les documents en question sont assez vari\'es :
 certains sont des ouvrages de r\'ef\'erence, d'autres sont des 
 monographies sp\'ecialis\'ees, d'autres encore, des articles de recherche.
 L'auteur n'a pas cherch\'e \`a suivre tel ou tel trait\'e et a essay\'e de r\'ediger
 ces notes de fa\c con originale$\ldots$ certains pourront peut-\^etre s'en plaindre !
 Tout ceci explique la raison du petit nombre de r\'ef\'erences, que voici. 

\vfill\eject



